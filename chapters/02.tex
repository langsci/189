\documentclass[output=paper]{langscibook} 
\ChapterDOI{10.5281/zenodo.2545511}

\author{Tatiana Bondarenko\affiliation{Massachusetts Institute of Technology}}
\title{Russian datives again:\newlineCover On the (im)possibility of the small clause analysis} 
\abstract{In this paper I use the interpretation of the repetitive adverb \textit{opjat’} ‘again’ in Russian to argue that ditransitive structures in this language do not involve a small clause structure (\citealt{Kayne1984,Beck-Johnson2004}; a.o.). Under the syntactic approach to the semantics of repetitives that I adopt (\citealt{vonStechow1996,Beck2005}; a.o.), the interpretation of repetitives is determined by their attachment in the syntactic representation. I show that in Russian ditransitives, unlike in English ones \citep{Beck-Johnson2004}, only the repetitive reading of ‘again’ is possible, and argue that no reason other than a difference in the syntactic structures of ditransitives in two languages can account for that. I also observe that unlike datives that are found in ditransitives, “higher” dative arguments and locative applicatives in Russian can occur in constructions where there is a syntactic constituent denoting the resultant state, and thus the restitutive reading of repetitives is available.

\keywords{ditransitives, repetitives, datives, small clauses, Russian}
}


 
% \IfFileExists{../localcommands.tex}{%hack to check whether this is being compiled as part of a collection or standalone
%   \usepackage{amsmath}
%\usepackage{bbding}% add all extra packages you need to load to this file  
\usepackage{csquotes}
% \usepackage{draftwatermark}
%\usepackage{draftwatermark}
\usepackage[main=english,
%                  czech, %% Check for newer Version of [czech] in babel
                 russian,
                 ngerman,
                 polish,
            ]{babel}
\usepackage{eurosym}
\usepackage{fixltx2e}
\usepackage{float}
% \usepackage[german,english]{babel}
\usepackage{hhline}
\usepackage{./langsci/styles/jambox}
\usepackage{langsci/styles/langsci-cgloss}
\usepackage{./langsci/styles/langsci-lgr}
% \usepackage{langsci-linguex}
\usepackage{./langsci/styles/langsci-optional}
\usepackage[linguistics]{forest}
\usepackage{longtable}
% % \usepackage{marvosym} % incompatible
\usepackage{multicol}
\usepackage{multirow}
\usepackage[normalem]{ulem}
\usepackage{pifont} %for checkmark and cross
% \usepackage[polish,czech,english]{babel}
% \usepackage[russian,english]{babel}
\usepackage{slantsc} %needed for slanted smallcaps
\usepackage{stmaryrd} %defines \llbracket and \rrbracket, needed for semantic interpretation brackets [[.]]
\usepackage{subfigure}
\usepackage{tabto}
\usepackage{tabularx} 
\usepackage{qtree}
\usepackage{tikz-qtree}
\usepackage{tikz-qtree-compat}
\usetikzlibrary{arrows,arrows.meta,decorations.markings,shapes,calc,fit}
\usepackage{url}
\usepackage{vwcol}
\usepackage{wasysym}%symbols
\usepackage{siunitx}
\makeatletter
\let\pgfmathModX=\pgfmathMod@
\usepackage{pgfplots,pgfplotstable}%
\let\pgfmathMod@=\pgfmathModX
\makeatother
\usepgfplotslibrary{colorbrewer,groupplots} 
% % \usepackage{MdSymbol} % \ngg, \gg
\usepackage{langsci-gb4e}

%    
\makeatletter
\let\thetitle\@title
\let\theauthor\@author 
\makeatother

\newcommand{\togglepaper}{ 
  \bibliography{../localbibliography}
  \papernote{\scriptsize\normalfont
    \theauthor.
    \thetitle. 
    To appear in: 
    Radek et al ...
    Formal ....
    Berlin: Language Science Press. [preliminary page numbering]
  }
  \pagenumbering{roman}
}
 
%   \togglepaper
% }{}

\begin{document}

\maketitle 
\shorttitlerunninghead{\ili{Russian} datives again: On the (im)possibility of the \isi{small clause} analysis} 

\section{{Introduction}}

In this paper I will discuss applicability of the \isi{small clause} analysis (\citealt{Kayne1984,Harley1996,Beck-Johnson2004,Pylkkänen2008}, among others) that has been proposed for the \ili{English} \isi{double object construction} \REF{ex:bondarenko:1} {to constructions with \isi{dative} arguments in \ili{Russian} \REF{ex:bondarenko:2}.}\footnote{All examples in this paper are either in \ili{English} or in \ili{Russian}, unless explicitly indicated otherwise.}


 \ea John gave Mary a letter.\label{ex:bondarenko:1}\z

 \ea\label{ex:bondarenko:2}
\gll Vasja otdal \{\hspace{-2pt} Maše pis’mo / pis’mo Maše\}.\\
     Vasja gave {} Masha.\textsc{dat} letter.\textsc{acc} {} letter.\textsc{acc} Masha.\textsc{dat}\\
\glt `Vasja gave Masha a letter.'
\z

\largerpage
\noindent{The \isi{small clause} analysis involves the idea that in \isi{ditransitive} constructions a \isi{direct object} and an indirect object are merged together forming a \isi{small clause} excluding the verb. This idea is shared by a variety of approaches (\citealt{Kayne1984,Pesetsky1995,Harley1996,Harley2002,Cuervo2003,Beck-Johnson2004,Jung-Miyagawa2004,McIntyre2006,Pylkkänen2008,Schäfer2008,Lomashvili2010,HarleyJung2015}, among others), which diverge on the exact nature of this formation (\isi{small clause}\slash low \isi{applicative}\slash PP\slash HaveP) and a few other details of the derivation. The tree in \figref{fig:bondarenko:1} (adapted from \citealt{Harley2002}) illustrates a version of this analysis for the \ili{English} \isi{double object construction} in \REF{ex:bondarenko:1}: the \isi{direct object} (}{\textit{a} \textit{letter}}{) and the indirect object (}{\textit{Mary}}{) are combined with the help of a special P}{\textsubscript{HAVE}}{, and the resulting PP becomes a complement of the verb.}

\begin{figure}
\begin{forest}for tree={l=0}
%  [\textit{v}P
%   [\hspace*{2cm}]
  [\textit{v}P
    [\textit{v} 
      [CAUSE]
    ]
    [PP
      [DP
	[\textit{Mary}, roof]
      ]
      [P'
	[P
	  [P\textsubscript{HAVE}]
	]
	[DP
	  [\textit{a letter}, roof]
	]
      ]
    ]
  ]
%  ]
\end{forest}

\caption{\label{fig:bondarenko:1} Double object construction (adapted from \citealt[4]{Harley2002})}
\end{figure}



{The \isi{small clause} analysis makes use of \isi{lexical decomposition} in syntax: different subevents of a predicate are represented by different projections in syntax (}{\textit{v}}{\textsubscript{DO/CAUS}}{P for a causing subevent, SC\slash ResultP\slash HaveP\slash PP for a result state subevent, among some others). Under such approach to the \isi{syntax-\isi{semantics} interface}, indirect objects differ with respect to where they are introduced in the syntactically represented \isi{lexical decomposition} of a given verb (\citealt{Cuervo2003,Schäfer2008}; among others). Their positions account for different interpretations and different syntactic properties. Indirect objects in the \ili{English} \isi{double object construction} are participants of the result state subevent under the \isi{small clause} analysis.}



The aim of this paper is to argue that \ili{Russian} \isi{ditransitive} verbs like \textit{otdavat’} ‘give’ in \REF{ex:bondarenko:2} should not be analyzed as involving a \isi{small clause} structure. While \ili{English} might decompose \isi{ditransitive} verbs in syntax (\textit{give} as CAUSE to HAVE), \ili{Russian} does not exhibit the decomposition of this sort. My argumentation employs the idea that \isi{repetitive} morphemes like \textit{again} single out subevents in the \isi{semantics} of a predicate, and thus, are able to detect the exact placement of indirect objects in syntactic structures with lexically decomposed verbs. If an indirect object denotes a participant of some subevent $e_1$, then it should be in the scope of a \isi{repetitive} \isi{adverb} that singles out that subevent $e_1$. I will try to show that \ili{Russian} has constructions where a \isi{dative} argument is a participant of a \isi{stative} subevent of a predicate, but \isi{ditransitive} sentences are not among such constructions. 


\largerpage[-1]
The crucial observation for my proposal is that the \isi{restitutive} reading of \textsc{again} is available in \ili{English} \isi{ditransitive} sentences -- in both the \textsc{\isi{double object} construction}, see \REF{ex:bondarenko:3}, and the \textsc{to-PP construction}, see \REF{ex:bondarenko:4}, but not in \ili{Russian}, no matter if the \isi{dative} argument precedes the \isi{accusative} one, as in \REF{ex:bondarenko:5}, or conversely, see \REF{ex:bondarenko:6}.\footnote{I do not want to imply that \REF{ex:bondarenko:5} and \REF{ex:bondarenko:6} are equivalents of \ili{English} \isi{double object construction} and \textit{to-}PP construction correspondingly. The sentences in \REF{ex:bondarenko:5}-\REF{ex:bondarenko:6} just show that the availability of the \isi{restitutive} reading does not depend on the relative \isi{word order} of \isi{dative} and \isi{accusative} arguments in \ili{Russian}.}$^,$\footnote{I use \textsc{again} to refer to this kind of \isi{repetitive} adverbs generally and words in italics (\ili{English} \textit{again,} \ili{Russian} \textit{opjat’}) to refer to concrete lexical items of languages.}


 \ea\label{ex:bondarenko:3}{Thilo gave Satoshi the map again.}\hfill\textsc{\isi{double object} construction}
\ea \textbf{Repetitive}: Available\\`Thilo gave Satoshi the map, and that had happened before.'
\ex \textbf{Restitutive}: Available\\`Thilo gave Satoshi the map, and Satoshi had had the map before.'\\
\hfill(\citealt{Beck-Johnson2004}: 113)
\z
\z

 \ea\label{ex:bondarenko:4}{Thilo gave the map to Satoshi again.}\hfill\textsc{to-PP construction}
\ea \textbf{Repetitive}: Available\\`Thilo gave Satoshi the map, and that had happened before.'
\ex \textbf{Restitutive}: Available\\`Thilo gave Satoshi the map, and Satoshi had had the map  before.'\\
\hfill (\citealt{Beck-Johnson2004}: 116)
\z
\z

 \ea\label{ex:bondarenko:5}
\gll Maša opjat’ otdala Vase knigu.\\
     Masha again gave Vasja.\textsc{dat} book.\textsc{acc}\\\hfill\textsc{dat} > \textsc{acc}
\ea \textbf{Repetitive}: Available\\`Masha gave Vasja the book, and that had happened before.'
\ex \textbf{Restitutive}: Unavailable\\`Masha gave Vasja the book, and Vasja had had the book   before.'
\z
\z

 \ea\label{ex:bondarenko:6}
\gll Maša opjat’ otdala knigu Vase.\\
     Masha again gave book.\textsc{acc} Vasja.\textsc{dat}\\\hfill\textsc{acc} > \textsc{dat}
\ea \textbf{Repetitive}: Available\\`Masha gave Vasja the book, and that had happened before.'
\ex \textbf{Restitutive}: Unavailable\\`Masha gave Vasja the book, and Vasja had had the book   before.'
\z
\z

\largerpage[-3]
\noindent{Under the \isi{restitutive} reading, the subevent that is singled out by} \textsc{again}{ is the state of possession between the indirect object and the \isi{direct object}. For example, in \REF{ex:bondarenko:3} and \REF{ex:bondarenko:4} it is the reading when a state of Satoshi having the map is being repeated.}\footnote{An anonymous reviewer asks whether the presence of the \isi{restitutive} reading entails the \isi{small clause} analysis for the PP datives, given the logic of \citet{Beck-Johnson2004}. While the analysis for the PP datives is not spelled out in detail in \citet{Beck-Johnson2004}, one can infer from the discussion therein that the authors propose distinct syntactic structures for the \isi{double object construction} and the \textit{to-}PP construction, both of which include a \isi{small clause}. Given the logic of \citet{Beck-Johnson2004}, the \isi{double object construction} includes a \isi{small clause} that consists of the two objects merging with the help of a functional projection (XP), which is then combined with the verb. The \textit{to-}PP construction under their view presents a subcase of a more general NP + PP pattern. In sentences of this sort V merges directly with a PP and takes an NP as its \isi{specifier}. The PP under consideration contains a null PRO as its subject that corefers with the NP that is the \isi{specifier} of the verb. Thus, as the authors themselves put it, the PP becomes in effect a \isi{small clause} (\citealt{Beck-Johnson2004}: 118). In other words, the presence of the \isi{restitutive} reading in \REF{ex:bondarenko:4} under the logic of \citet{Beck-Johnson2004} does entail the presence of a \isi{small clause} in the syntactic structure but does not necessarily entail that the syntactic structures of the \isi{double object construction} and the \textit{to-}PP construction are identical.}{ This reading is impossible for \ili{Russian} ditransitives: in \REF{ex:bondarenko:5} and \REF{ex:bondarenko:6}} \textsc{again}{ cannot single out the state of Vasja having the book. The example in \REF{ex:bondarenko:7} illustrates that providing more context does not increase the availability of the \isi{restitutive} reading in \ili{Russian} ditransitives.}


 \ea\label{ex:bondarenko:7}\textit{Context:} Vasja had always had the book \textit{Two captains} by Kaverin; he had never given it to anyone. One day he accidentally left the book at Masha’s place\dots
\ea[\#]{\gll I togda Maša opjat’ \{\hspace{-2pt} otdala / otpravila / vernula\} Vase knigu.\\
 and then Masha again {} gave {} sent {} returned Vasja.\textsc{dat} book.\textsc{acc}\\
\glt Intended: `And then Masha gave / sent / returned Vasja the book, and Vasja had had the book before.'}
\ex[\#]{\gll I togda Maša opjat’ \{\hspace{-2pt} otdala / otpravila / vernula\} knigu Vase.\\
 and then Masha again {} gave / sent / returned book.\textsc{acc} Vasja.\textsc{dat}\\
\glt Intended: `And then Masha gave / sent / returned the book to Vasja, and Vasja had had the book before.'}
\z
\z


\noindent{Why does \ili{Russian} differ from \ili{English} with respect to the availability of the \isi{restitutive} reading in ditransitives? Does this difference reflect different syntactic structures of \isi{ditransitive} sentences in these languages? Does \ili{Russian} have constructions with \isi{dative} arguments where} \textsc{again}{ is able to single out the \isi{stative} subevent of a predicate? These questions will be central to the forthcoming discussion.}



This paper is structured as follows. In \sectref{s2} I will introduce the syntactic approach to the meaning of \textsc{again} and discuss how the availability of the \isi{restitutive} reading in \ili{English} ditransitives argues for the \isi{small clause} analysis. In \sectref{s3} I will argue against \ili{Russian} ditransitives involving a \isi{small clause} structure. I will consider different potential reasons for the unavailability of the \isi{restitutive} reading in \ili{Russian} \isi{ditransitive} sentences and conclude that it has a syntactic explanation. In \sectref{s4} I will discuss constructions with higher \isi{dative} arguments and show that in these sentences the \isi{stative} subevent can be singled out, but the \isi{dative} argument is not a participant of it. In \sectref{s5} I will provide evidence that \isi{dative} arguments in \ili{Russian} can in principle be participants of the \isi{stative} subevent of a predicate and that a construction with \isi{locative} applicatives exemplifies such a case. \sectref{s6} concludes the paper.


\section{The small clause analysis of ditransitives: Evidence from \textsc{again}}\label{s2}

In this paper I will assume the syntactic approach to the ambiguity of \isi{repetitive} adverbs (\citealt{vonStechow1996,Beck-Johnson2004,Beck2005,Alexiadou-etal2014,Lechner-etal2015}; among others), according to which different readings of \textsc{again} are attributed to different attachments of \textsc{again} in the syntactic representation. Under this approach the \isi{semantics} of \textsc{again} is taken to be always the same and involve repetition of some event:\footnote{There is a competing semantic approach to the ambiguity of repetitives (\citealt{FabriciusHansen2001,Jäger-Blutner2000}; among others), according to which different readings of \textsc{again} emerge due to the lexical ambiguity of \isi{repetitive} morphemes. In this paper I will not discuss the applicability of the \isi{semantics} approach to the data under consideration.}


 \ea\label{ex:bondarenko:8}
\sx{again}$(e)(P)$
\ea $=  1$ iff          $P(e) \wedge \exists e' [e'<$\uncnst{t}$\;e\wedge P(e')]$
\ex $= 0$ iff $\neg P(e)\wedge {\exists} e'[e'<$\uncnst{t}$\;e\wedge P(e')]$
\ex undefined otherwise
\z\z


\noindent The \isi{semantics} in \REF{ex:bondarenko:8} states that \textsc{again} takes an event $e$ and a property of events $P$ as its arguments and returns $1$ if the property is true of the event and $0$ if the property is not true of the event. The crucial part of \textsc{again}’s meaning is a \isi{presupposition} that there is another event that temporally precedes ($<$\uncnst{t}) the event under consideration of which the property is true. If the \isi{presupposition} is not met, the meaning of \textsc{again} is undefined. Under the syntactic approach different readings of \textsc{again} arise due to its modification of different subevents in the syntactically represented \isi{lexical decomposition}: the subevent that is modified by \textsc{again} is understood as being repeated.



\citet{Beck-Johnson2004} claimed that the presence of the two readings of \textit{again} with the \isi{double object construction} provides support for the \isi{small clause} analysis of \ili{English} ditransitives. If \isi{ditransitive} verbs such as \textit{give} are lexically decomposed into the subevent denoting the action undertaken by an agent (represented in syntax by \textit{v}) and the \isi{stative} subevent (represented in syntax by a \isi{small clause} – HaveP), then \textit{again} should be able to attach to both \textit{v}P and HaveP and modify the respective subevents, giving rise to the repetitive-\isi{restitutive} ambiguity. This expectation is borne out, as we have observed in \REF{ex:bondarenko:3} (repeated here as \REF{ex:bondarenko:9}). The fact that indirect objects are understood as participants of \isi{stative} subevents of \isi{ditransitive} verbs suggests that they are inside a \isi{small clause} that represents a given \isi{stative} subevent syntactically. The analysis that \citet{Beck-Johnson2004} propose for sentences like \REF{ex:bondarenko:9} is sketched out in \REF{ex:bondarenko:10} and \REF{ex:bondarenko:11} (for the \isi{repetitive} and the \isi{restitutive} reading, respectively).\footnote{Smallcaps in semantic formulas indicate metalinguistic translations of object language. For instance, \sx{Satoshi} = \textsc{Satoshi}. This means that \textsc{again} in semantic formulas equals \sx{again} (the meaning of the word \textit{again}) and not the cover term for \ili{English} \textit{again} and \ili{Russian} \textit{opjat'}, used elsewhere in the body of the paper.}


 \ea\label{ex:bondarenko:9}{Thilo gave Satoshi the map again.}
\ea \textbf{Repetitive}\\`Thilo gave Satoshi the map, and that had happened before.'
\ex \textbf{Restitutive}\\`Thilo gave Satoshi the map, and Satoshi had had the map before.'\\
\hfill(\citealt{Beck-Johnson2004}: 113)
\z
\z

\ea\label{ex:bondarenko:10}\textbf{Repetitive reading}
\ea \textbf{[\textit{\textsubscript{v}}}\textbf{\textsubscript{P}}\textsubscript{}  [\textit{\textsubscript{v}}\textsubscript{P} Thilo [give [BECOME [\textsubscript{HaveP} Satoshi HAVE the map]]]] \textbf{again]}
\ex $\lambda e\,.\;$\fbox{\textsc{again}}$\,\big(e\big)\big(\lambda e_1\,.\,\textsc{give}(e_1)(\textsc{Thilo})$\\
\tabto{1cm}${}\wedge\exists e_2[$\cnst{become}$(e_2)(\lambda e_3\,.\,$\cnst{have}$(e_3)(\textsc{the map})(\textsc{Satoshi}))$
\tabto{1cm}${}\wedge{}$\cnst{cause}$(e_2)(e_1)]\big)$
\ex     `Once more, a giving by Thilo caused Satoshi to come to have   the map.'\\
\hfill(\citealt{Beck-Johnson2004}: 114)
\z
\z


 \ea\label{ex:bondarenko:11}\textbf{Restitutive reading}
\ea Thilo [give [BECOME \textbf{[\textsubscript{HaveP}} [\textsubscript{HaveP} Satoshi HAVE the map] \textbf{again]}]]
\ex $\lambda e\,.\,\textsc{give}(e)(\textsc{Thilo})\wedge\exists e_1[$\cnst{become}$\big(e_1\big)$\\
\tabto{1cm}$\big(\lambda e_2\,.\;$\fbox{\textsc{again}}$\,(e_2)(\lambda e_3\,.\,$\cnst{have}$(e_3)(\textsc{the map})(\textsc{Satoshi}))\big)$\\
\tabto{1cm}${}\wedge{}$\cnst{cause}$(e_1)(e)]$
\ex     `A giving by Thilo caused Satoshi to come to once more have the   map.'\\
\hfill(\citealt{Beck-Johnson2004}: 114)
\z
\z

\noindent In \REF{ex:bondarenko:10} \textit{again} attaches to the \textit{v}P denoting the whole event of Thilo giving Satoshi the map, giving rise to the \isi{repetitive} interpretation. In \REF{ex:bondarenko:11} \textit{again} attaches to the \isi{small clause} that denotes the \isi{stative} event of Satoshi having the map, thus the \isi{restitutive} reading arises.



For \citet{Beck-Johnson2004} there are no elements CAUSE and BECOME in the syntactic representation of \isi{ditransitive} sentences. Syntax provides a verb that takes a \isi{small clause} as its complement, and it’s the semantic component that is responsible for introducing components like CAUSE and BECOME that are required for deriving the correct interpretations. It was proposed by \citet{vonStechow1995Lexicaldecompositionsyntax} (and further employed in \citealt{Beck-Johnson2004} and \citealt{Beck2005}) that the following special semantic principle is at work in structures with small clauses:


\ea \textbf{Principle R}\\
     If $\alpha ={}$ [\textsubscript{V} $\gamma$ [\textsubscript{SC} $\beta$]] and $\beta$ is of type $\semantictype{s,t}$ and $\gamma$ is of type $\semantictype{e,\dots\semantictype{e,\semantictype{s,t}}}$ (an $n$-place predicate), then\\\sx{$\alpha$}${}=\lambda x_1\dots\lambda x_n\lambda e\,.\,$ \sx{$\gamma $}$(e)(x_1)\dots(x_n)$\\
     \tabto{1cm}${}\wedge\exists e_1[$\cnst{become}$(e_1)($\sx{$\beta$}$)\wedge{}$\cnst{cause}$(e_1)(e)]$.\\
     \hfill (adapted from \citealt[7]{Beck2005})
     \z


\noindent This principle ensures that a verb (an $n$-place predicate) is properly “glued” with a \isi{small clause} (a property of events) by inserting CAUSE and BECOME components into the \isi{semantics} representation.

This line of reasoning (\citealt{Beck-Johnson2004}), which makes use of the syntactic decomposition of \isi{ditransitive} verbs into a verb and a \isi{small clause} and of the syntactic approach to the ambiguity of \isi{repetitive} morphemes, allows naturally to explain the possible interpretations of \ili{English} \textit{again} in the \isi{double object construction}.\footnote{There has been another attempt to explain the repetitive-\isi{restitutive} ambiguity of \textit{again} in the \ili{English} \isi{double object construction} by \citet{Bruening2010}, who argues for the asymmetrical \isi{applicative} analysis of \ili{English} ditransitives: a verb merges with a \isi{direct object} first, and then the VP combines with an \isi{applicative} head that introduces an indirect object as its \isi{specifier}. Unlike under a \isi{small clause} analysis, under this syntactic analysis the two interpretations of \textit{again} do not fall out for free: special assumptions about verb head movement, object movement and interpretation of copies are required in order to obtain both \isi{repetitive} and \isi{restitutive} readings in \isi{ditransitive} structures.} In the next section I will discuss why a similar logic is not applicable to the case of \ili{Russian} ditransitives.


\section{Russian ditransitives: Against the small clause analysis}\label{s3}

There could be potentially different reasons for why \isi{restitutive} readings are not available in \ili{Russian} \isi{ditransitive} clauses. The first hypothesis that I will explore is that the \ili{Russian} \isi{repetitive} \isi{adverb} \textit{opjat’} has different properties than \ili{English} \textit{again}. It has been observed that not all \isi{repetitive} morphemes across languages have the ability to access different subevents inside decomposition structures (\citealt{Rapp-vonStechow1999,Beck2005,Alexiadou-etal2014,Lechner-etal2015}). For example, the \ili{German} \isi{repetitive} \isi{adverb} \textit{erneut} ‘again’ cannot have \isi{restitutive} readings with lexical accomplishment verbs like \textit{öffnen} ‘open’, unlike another \isi{repetitive} \isi{adverb} \textit{wieder} ‘again’; see \REF{ex:bondarenko:13} and \REF{ex:bondarenko:14}.\footnote{Note that the unavailability of the \isi{restitutive} reading in \REF{ex:bondarenko:13} cannot be due to its verb form (which is different from the one in \REF{ex:bondarenko:14}), since the use of the same form as in \REF{ex:bondarenko:14} does not lead to the availability of the \isi{restitutive} reading:

\ea\label{ex:bondarenko:fn7exi}
\gll {\dots} dass Maria die Tür erneut öffnete.\\
{} that Maria the door again opened \\\hfill (\ili{German})
\ea     \textbf{Repetitive}: Available\\
`\dots that Maria opened the door, and that had happened before.'
\ex     \textbf{Restitutive}: Unavailable\\
`\dots that Maria opened the door, and the door had been open before.'
\z
\z
}


 \ea\label{ex:bondarenko:13}
\gll Maria hat die Tür erneut geöffnet.\\
     Maria has the door again opened\\
\ea \textbf{Repetitive}: Available\\
`Maria opened the door, and that had happened before.'
\ex \textbf{Restitutive}: Unavailable\\
`Maria opened the door, and the door had been open before.'\\
\hfill({German}; \citealt[12]{Beck2005})
\z
\z

 \ea\label{ex:bondarenko:14}
\gll {\dots} dass Ali Baba Sesam wieder öffnete\\
    {} that Ali Baba Sezam again opened\\
\ea \textbf{Repetitive}: Available\\
`\dots that Ali Baba opened Sezam, and that had happened before.'
\ex \textbf{Restitutive}: Unavailable\\
`\dots that Ali Baba opened Sezam, and Sezam had been open before.'\\
\hfill (\ili{German}; adapted from \citealt{vonStechow1996}: 3)
\z
\z

\noindent This variation with respect to the ability of adverbs to single out different subevents in the syntactically represented \isi{lexical decomposition} of predicates was captured by the Visibility Parameter (\citealt{Rapp-vonStechow1999,Beck2005}):


 \ea\label{ex:bondarenko:15}
\textbf{The Visibility Parameter for decomposition adverbs}\\
A D(ecomposition)-\isi{adverb} can\slash cannot attach to a phrase with a phonetically empty head.\\
\hfill(\citealt{Rapp-vonStechow1999} via \citealt[13]{Beck2005})
\z

\noindent Under the assumption that lexical accomplishments in \REF{ex:bondarenko:13} and \REF{ex:bondarenko:14} involve a \isi{small clause} with a null head that corresponds to the \isi{stative} subevent of the door\slash Sezam being open, the Visibility Parameter states that the difference between \ili{German} \textit{wieder} and \textit{erneut} is that the former, but not the latter can attach to a phrase with a phonetically null head, hence only the former can have the \isi{restitutive} reading in sentences with lexical accomplishments.



The following question can then be asked about \ili{Russian} \textit{opjat’}: Is it an \isi{adverb} that can attach to a phrase with a phonetically empty head? It turns out that \textit{opjat’} can single out the \isi{stative} subevent of lexical accomplishments, see \REF{ex:bondarenko:16} and \REF{ex:bondarenko:17}, thus classifying as a decomposition \isi{adverb} that can “look inside” the decomposition structure and modify subevents that are not expressed by overt phonetic material. \textit{Opjat’} is not different from \ili{German} \textit{wieder} or \ili{English} \textit{again} in this respect.\largerpage[-1]


 \ea\label{ex:bondarenko:16}
\gll Vasja opjat’ otkryl dver’.\\
     Vasja again opened door.\textsc{acc}\\
\ea \textbf{Repetitive}: Available\\
`Vasja opened the door, and that had happened before.'
\ex \textbf{Restitutive}: Available\\
`Vasja opened the door, and the door had been open before.'
\z
\z

 \ea\label{ex:bondarenko:17}
\gll Vasja opjat’ opustošil butylku.\\
     Vasja again emptied bottle.\textsc{acc}\\
\ea \textbf{Repetitive}: Available\\
`Vasja emptied the bottle, and that had happened before.'
\ex \textbf{Restitutive}: Available\\
`Vasja emptied the bottle, and the bottle had been empty   before.'
\z
\z

 \ea\label{ex:bondarenko:18}
{Ali Baba opened Sezam again.}
\ea \textbf{Repetitive}: Available\\
`Ali Baba opened Sezam, and that had happened before.'
\ex \textbf{Restitutive}: Available\\
`Ali Baba opened Sezam, and Sezam had been open before.'
\z
\z


\noindent Note that unlike \textit{wieder} and \textit{again}, \ili{Russian} \textit{opjat’} occurs preverbally, see \REF{ex:bondarenko:5}--\REF{ex:bondarenko:7}, \REF{ex:bondarenko:16}, and \REF{ex:bondarenko:17}, which does not prevent it from being able to have \isi{restitutive} readings see \REF{ex:bondarenko:16} and \REF{ex:bondarenko:17}.\footnote{The situation is different for \ili{English} and \ili{German}, where the pre-object position of \isi{repetitive} adverbs makes the \isi{restitutive} reading unavailable, see \REF{ex:bondarenko:fn8exi} and \REF{ex:bondarenko:fn8exii}.
  \ea\label{ex:bondarenko:fn8exi}
  {Ali Baba again opened Sezam.}
    \ea \textbf{Repetitive}: Available\\
    `Ali Baba opened Sezam, and that had happened before.'
    \ex \textbf{Restitutive}: Unavailable\\
    `Ali Baba opened Sezam, and Sezam had been open before.'
    \z
    \z
    
  \ea\label{ex:bondarenko:fn8exii}
  \gll {\dots} dass Ali Baba wieder Sesam öffnete.\\
  {} that Ali Baba again Sezam opened\\\hfill (\ili{German})
  \ea  \textbf{Repetitive}: Available\\
  `{\dots} that Ali Baba opened Sezam, and that had happened before.'
  \ex \textbf{Restitutive}: Unavailable\\
  `{\dots} that Ali Baba opened Sezam, and Sezam had been open before.'
  \z
  \z
  
\noindent Unlike \ili{English} \textit{again} and \ili{German} \textit{wieder}, \ili{Russian} \textit{opjat’} is generally not very good in a sentence-final position and is mostly used in the preverbal position.} The fact that \textit{opjat’} generally allows for \isi{restitutive} readings when it precedes the verb suggests that the \isi{word order} in \REF{ex:bondarenko:5}--\REF{ex:bondarenko:7} cannot be the reason for the unavailability of \isi{restitutive} readings in \isi{ditransitive} clauses. To sum up, it seems highly unlikely that the properties of \textit{opjat’} prevent \isi{restitutive} readings in \ili{Russian} ditransitives.



A second hypothesis that I will consider is that \isi{restitutive} readings are unavailable in \ili{Russian} ditransitives due to the absence of a \isi{stative} subevent in \isi{semantics} of \isi{ditransitive} verbs. I will argue that this hypothesis is also wrong: ditransitives have a \isi{stative} subevent in their \isi{semantics}, which can independently be detected by another \ili{Russian} \isi{adverb}, namely \textit{obratno} ‘back’/‘again’, and can be introduced into syntax with the help of an eventive goal PP. Crucially, I will argue that the \isi{stative} subevent is not represented in the syntactic decomposition of \isi{ditransitive} verbs that take just an \isi{accusative} argument and a \isi{dative} one.\largerpage[-4]



The \ili{Russian} \isi{adverb} \textit{obratno} ‘back’/‘again’ (glossed below simply as \textsc{obratno}), although similar in its meaning to \textit{opjat’}, has different \isi{semantics}, which involves a return to a state in which an entity had been before (as observed already by \citealt{Tatevosov2016}). As a consequence, it can modify only descriptions with a target state in the sense of \citep{Kratzer2000} and allows for \isi{restitutive} readings only \REF{ex:bondarenko:19}.


 \ea\label{ex:bondarenko:19}
\textit{Context} (after \citealt{Lechner-etal2015}): Three students -- Masha, Vasja, and Petja -- were studying in the library. They wanted the window in the library to be open, but the librarian wanted the window to be closed. Masha opened the window, but the librarian closed it. Vasja opened the window, but the librarian closed it. Petja opened the window, but the librarian closed it. Finally, Masha opened the window for the second time.
\ea[\#]{\gll Rovno odin student otkryl okno obratno.\\
     exactly one student opened window.\textsc{acc} \textsc{obratno}\\
\glt `Exactly one student opened the window again.'}
\ea \textbf{Repetitive reading}: Unavailable\\
`There exists a student that opened the window and had opened it before, and it is not true that other students opened the window and had opened it before.'\\\hfill{\small (exactly one $x$ > again > $x$ opened the window > the window was open)}
\ex \textbf{Restitutive reading}: False\\
`There exists a student that opened the window and no other student opened the window and the window had been open before.'\\\hfill {\small (exactly one $x$ > $x$ opened the window > again > the window was open)}
\z
\ex[]{\gll Rovno odin student opjat’ otkryl okno.\\
     exactly one student again opened window.\textsc{acc}\\
\glt `Exactly one student opened the window again.'}\label{ex:bondarenko:19b}
\ea \textbf{Repetitive reading}: True\label{ex:bondarenko:19c}\\
`There exists a student that opened the window and had opened it before, and it is not true that other students opened the window and had opened it before.'\\\hfill{\small (exactly one $x$ > again > $x$ opened the window > the window was open)}
\ex \textbf{Restitutive reading}: False\label{ex:bondarenko:19d}\\
`There exists a student that opened the window and no other student opened the window and the window had been open before.'\\\hfill {\small (exactly one $x$ > $x$ opened the window > again > the window was open)}\\
\hfill (adapted from \citealt[31]{Tatevosov2016})
\z
\z\z



\noindent \citet{Alexiadou-etal2014} and \citet{Lechner-etal2015} observed that the \isi{repetitive} and the \isi{restitutive} readings exhibit different truth conditions in contexts with non-monotone quantifiers like ‘exactly' or `only one student’. For the context in \REF{ex:bondarenko:19}, sentences with subjects that are non-monotone quantifiers are true only under the \isi{repetitive} reading of \textsc{again}, see \REF{ex:bondarenko:19c} vs. \REF{ex:bondarenko:19d}. While \textit{opjat’} can have \isi{repetitive} readings and thus \REF{ex:bondarenko:19b} is appropriate in the context provided, \textit{obratno} is illicit in this context because it cannot have \isi{repetitive} readings.



\textit{Obratno} “looks into” the \isi{semantics} of a verbal phrase with which it merges and searches for a target state in this semantic representation that it can modify. As the sentence in \REF{ex:bondarenko:20} shows, \textit{obratno} is able to find a target state in the semantic representation of \ili{Russian} ditransitives.


 \ea\label{ex:bondarenko:20}
\gll Maša \{\hspace{-2pt} otdala / otpravila / vernula\} Vase knigu obratno.\\
     Masha {} gave {} sent {} returned Vasja.\textsc{dat} book.\textsc{acc} \textsc{obratno}\\
\glt `Masha gave / sent / returned Vasja the book, and Vasja had had the book before.'
\z

\noindent Elaboration of the analysis of properties of \ili{Russian} \textit{obratno} is beyond the scope of this paper. What is important for us here is that \textit{obratno} can serve as a diagnostic for a \isi{stative} subevent: it shows us that a result state is present in \isi{semantics} of \isi{ditransitive} predicates.\footnote{There
  could be different plausible explanations for the unavailability of \isi{repetitive} readings with \textit{obratno}. For example, it could be the case that \textit{obratno} is actually not a VP-level \isi{adverb} but a PP modifier which in some cases signals the presence of a silent PP. Some support in favor of this hypothesis is provided by examples like \REF{ex:bondarenko:fn9i} and \REF{ex:bondarenko:fn9ii}, where \textit{obratno} seems to form a constituent with an overtly realized PP (the examples involve a movement of \textit{obratno} + PP -- scrambling and wh-movement, respectively):

  \ea\label{ex:bondarenko:fn9i}
  \gll [\hspace{-2pt} Obratno v Moskvu] Vasja rešil priexat’.\\
  {} \textsc{obratno} to Moscow Vasja decided come.\textsc{inf}\\
  \glt `Vasja decided to come back to Moscow.'
  \z

  \ea\label{ex:bondarenko:fn9ii}
    \gll [\hspace{-2pt} Obratno v kakoj gorod] oni otpravilis’?\\
    {} \textsc{obratno} in what city they went\\
    \glt `What city did they go back to?'
    \z
    
    \noindent If \textit{obratno} is a PP modifier, then it follows that it can have exclusively \isi{restitutive} readings. Under this hypothesis, \textit{obratno} signals the presence of a silent goal PP in \REF{ex:bondarenko:20}, which introduces the \isi{stative} subevent into the syntactic representation that was otherwise not present. I will not pursue this idea here, leaving it for the future research.
    }

Another piece of evidence that \ili{Russian} \isi{ditransitive} verbs have a \isi{stative} sub\-event in their \isi{semantics} comes from the comparison of \isi{ditransitive} constructions with a \isi{dative} and an \isi{accusative} argument with constructions with the same verbs that take an \isi{accusative} argument and a goal PP. Consider the following two sentences with the verb \textit{otpravlyat’} ‘send’:


 \ea\label{ex:bondarenko:21}
\gll Maša opjat’ otpravila \{\hspace{-2pt} Vase igrušku / igrušku Vase\}.\\
     Masha again sent {} Vasja.\textsc{dat} toy.\textsc{acc} {} toy.\textsc{acc} Vasja.\textsc{dat}\\
\ea  Available: `Masha sent Vasja the toy, and that had happened before.'
\ex Unavailable: `Masha sent Vasja the toy, and Vasja had had the toy before.'
\z
\z

 \ea\label{ex:bondarenko:22}
\gll Rukovoditel’ opjat’ otpravil sotrudnika v Moskvu.\\
     manager again sent employee.\textsc{acc} in Moscow\\
\ea     Available: `The manager sent the employee to Moscow, and that had   happened before.'
\ex     Available: `The manager sent the employee to Moscow, and the employee   had been in   Moscow before.'
\z
\z


\noindent When this verb takes an \isi{accusative} argument and a \isi{dative} one \REF{ex:bondarenko:21}, the \isi{restitutive} reading of \textit{opjat’} is unavailable. When, however, it takes an \isi{accusative} argument and a goal PP \REF{ex:bondarenko:22}, \textit{opjat’} is able to single out the subevent that denotes the state of the theme argument (the employee) being at the location specified by the goal PP (Moscow).


This difference can also be observed with PPs headed by \textit{k} ‘to’, which can take \isi{animate} \isi{noun} phrases as their complements. Sentences with \isi{ditransitive} verbs that take a \isi{direct object} and a \textit{k}{}-PP, see \REF{ex:bondarenko:24}, seem almost synonymous to those with \isi{ditransitive} verbs that take two objects, see \REF{ex:bondarenko:23}; but the \isi{restitutive} reading is available only in the former construction.


 \ea\label{ex:bondarenko:23}
\gll Maša opjat’ otpravila knigu Kate.\\
     Masha again sent book.\textsc{acc} Katja.\textsc{dat}\\
\ea \textbf{Repetitive}: Available\\
`Masha sent the book to Katja, and that had happened before.'
\ex \textbf{Restitutive}: Unavailable\\
`Masha sent the book to Katja, and Katja had had the book   before.'
\z
\z

 \ea\label{ex:bondarenko:24}
\gll Maša opjat’ otpravila knigu k Kate.\\
     Masha again sent book.\textsc{acc} to Katja.\textsc{dat}\\
\ea \textbf{Repetitive}: Available\\
`Masha sent the book to Katja, and that had happened before.'
\ex \textbf{Restitutive}: Available\\
`Masha sent the book to Katja, and Katja had had the book   before.'
\z
\z


\noindent If we assume that \isi{ditransitive} verbs like \textit{otpravljat}’ ‘send’ have uniform \isi{semantics} across their uses, then it follows that they should have a \isi{stative} subevent in their semantic representation, since it is visible in some clauses with these verbs.

Why does the presence of a goal PP make the \isi{restitutive} reading available in sentences with \isi{ditransitive} verbs? I would like to suggest that the reason for that is that PPs, unlike \isi{dative} arguments, can be eventive (see \citealt{McIntyre2006}) and introduce subevents that are present in the \isi{semantics} of a predicate into the syntactic representation. This difference between \isi{dative} arguments and goal PPs, as well as the fact that they can co-exist in the same \isi{clause}, see \REF{ex:bondarenko:25x} (cf. \ili{English} \REF{ex:bondarenko:27x}), suggests that PP ditransitives and ditransitives with \isi{dative} arguments cannot be derivationally related.


 \ea\label{ex:bondarenko:25x}\ea\label{ex:bondarenko:25}
\gll Oni otpravili \{\hspace{-2pt} ej vrača / vrača ej\} v školu.\\
     they sent {} her.\textsc{dat} doctor.\textsc{acc} {} doctor.\textsc{acc} her.\textsc{dat} in school\\
\glt `They sent a doctor into the school for her.'
 \ex\label{ex:bondarenko:26}
\gll Ja brosil \{\hspace{-2pt} Vasje mjač / mjač Vasje\} v ruki.\\
     I threw {} Vasja.\textsc{dat} ball.\textsc{acc} {} ball.\textsc{acc} Vasja.\textsc{dat} in hands\\
\glt `I threw a ball to Vasja, into his hands.'
\z\z

 \ea\label{ex:bondarenko:27x}\ea[*]{They sent her a doctor into the building.}\label{ex:bondarenko:27}
 \ex[*]{I threw Fred a ball into his hands.\label{ex:bondarenko:28}\hfill\citep{McIntyre2011}}
\z\z


\noindent To sum up, sentences with \ili{Russian} \isi{ditransitive} verbs can have \isi{restitutive} readings in two cases. First, the \isi{adverb} \textit{obratno} can access a target state in the semantic representation of a verbal phrase. Second, a goal PP can introduce a target state into the syntactic representation, making the \isi{restitutive} reading available even with the \isi{repetitive} \isi{adverb} \textit{opjat’}, which requires a syntactic constituent corresponding to the result state. This suggests that the unavailability of \isi{restitutive} readings with \isi{dative} arguments cannot be explained by the absence of a \isi{stative} subevent in the \isi{semantics} of \ili{Russian} ditransitives.



If \ili{Russian} \textit{opjat’} has the same properties as \ili{English} \textit{again} and \ili{Russian} ditransitives have a \isi{stative} subevent in their event structure, then we have to conclude that for some reason this \isi{stative} subevent is not represented in syntax. In other words, no \isi{small clause} (or HaveP\slash PP\slash LowApplP) is present in \ili{Russian} \isi{ditransitive} sentences with \isi{dative} arguments. Why is it the case that such a \isi{small clause} cannot be built? I will first explore a semantic hypothesis: the relevant structure can be built, but cannot be interpreted due to absence of the interpretation Principle R in \ili{Russian}.



It has been argued (\citealt{Snyder2001,Beck-Snyder2001,Beck2005}) that the interpretation Principle R is not universal: languages differ with respect to whether they have a principle allowing to successfully interpret the combination of a verb and a \isi{small clause}, and this variation is responsible for the (un)availability of a number of constructions, including resultatives, verb-particle constructions, \textit{put}{}-\isi{locative} constructions, \textit{make}{}-causative constructions and the \isi{double object construction}, among others. Could it be the case that \ili{Russian} is one of the languages that do not have the Principle R?



This hypothesis is dubious, since \ili{Russian} seems to require some version of this principle independently for interpreting other constructions.\footnote{As an anonymous reviewer points out, \ili{Russian} does have resultative constructions. For example, one type of \ili{Russian} resultatives is discussed in \citet{Tatevosov2010}. I am grateful to the anonymous reviewer for this observation, which provides an additional argument against the inaccessibility of Principle R in \ili{Russian}.} One example of a case where such a principle would be needed is sentences with verbs that take lexical prefixes.


 \ea\label{ex:bondarenko:29}
\gll Vasja za-brosil mjač v vorota.\\
     Vasja \textsc{pvb}-throw ball in goal\\
\glt `Vasja threw the ball into the goal.'
\z


\noindent \citet{Svenonius2004} has proposed that lexical prefixes in \ili{Russian}, such as \textit{za} in \REF{ex:bondarenko:29}, enter the derivation as heads of small clauses that are complements of verbs. Under this view, lexical prefixes head their own projections and take PPs as their complements and direct objects as their subjects (\figref{fig:bondarenko:2}).


\begin{figure}
\begin{forest}for tree={l=0}
[VP
  [V
    [\textit{brosil}]
  ]
  [RP
    [DP
	[\textit{mjač}, roof]
    ]
    [R'
      [R
	[\textit{za}]
      ]
      [PP
	[\textit{v vorota}, roof]
      ]
    ]
  ]
]
\end{forest}
\caption{Lexical prefixes as heads of small clauses}
\label{fig:bondarenko:2}
\end{figure}


This analysis receives additional support from the fact that \textit{opjat’} can have the \isi{restitutive} reading in sentences with verbs with lexical prefixes. Consider \REF{ex:bondarenko:30}:


 \ea\label{ex:bondarenko:30}
\textit{Context:} This ball was lying inside the goal for as long as we can remember. For the first time someone threw the ball out of the goal. But five minutes later\dots
\exi{}{\gll Vasja opjat’ za-brosil mjač v vorota.\\
     Vasja again \textsc{pvb}-throw ball in goal\\
\glt `Vasja threw the ball into the goal, and the ball had been in the goal before.'}
\z


\noindent \textit{Opjat’} in \REF{ex:bondarenko:30} has the interpretation under which an event that has occurred before is the event of the ball being inside the goal. Under the syntactic approach to the ambiguity of \textsc{again}, this suggests that there is a syntactic constituent -- a \isi{small clause}, which represents the \isi{stative} subevent of the predicate and to which \textit{opjat’} can attach (\figref{fig:bondarenko:3}).


\begin{figure}
\begin{forest}for tree={l=0}
[VP
  [V
    [\textit{brosil}]
  ]
  [RP,name=RP
    [\textit{opjat'}]
    [RP
      [DP,name=DP
	[\textit{mjač}, roof]
      ]
      [R'
	[R
	  [\textit{za}]
	]
	[PP
	  [\textit{v vorota}, roof]
	]
      ]
    ]
  ]
]
\node[right=of RP] (\isi{restitutive}) {\textbf{restitutive}};
\draw(DP.north west) to[bend left=30](\isi{restitutive}.south) ;
\end{forest}


\caption{The small clause analysis of Russian \textit{zabrosit’} ‘throw’}
\label{fig:bondarenko:3}
\end{figure}


If \ili{Russian} did not have means of interpreting the combination of a verb and a \isi{small clause} (the Principle R or its equivalent), then the sentence in \REF{ex:bondarenko:30} should be uninterpretable and thus lead to a derivation crash. This implies that uninterpretability cannot be the problem that prevents building a \isi{small clause} structure for sentences with \isi{ditransitive} verbs in \ili{Russian}.



This brings us to the conclusion that \isi{ditransitive} sentences with \isi{dative} arguments in \ili{Russian} do not contain a \isi{small clause} for syntactic reasons: the structure with SC\slash HaveP\slash LowApplP\slash particular kinds of null P/R cannot be built. As a consequence, under our assumption that the availability of the \isi{restitutive} reading entails \isi{lexical decomposition} in syntax,\footnote{An anonymous reviewer reasonably points out that that this assumption is not shared by everyone working on \isi{double object} constructions. The conclusions that I argue for in this paper follow only if this assumption is retained.} the syntax of \isi{ditransitive} clauses in \ili{Russian} significantly differs from the syntax of similar sentences in \ili{English}. If \ili{English} might decompose \textit{give} syntactically as CAUSE to HAVE, this sort of decomposition does not take place in \ili{Russian}. A more general consequence follows from this difference between the two languages: the \isi{lexical decomposition} for a given predicate cannot be universal; languages differ with respect to how they map event structures of similar predicates onto syntactic representations.


\section{Restitutive readings with Russian datives: Higher datives}\label{s4}

Dative arguments can differ with respect to how they are related to a result state of a given predicate. In this section I will show that \isi{restitutive} readings of \textit{opjat’} are available in sentences with higher, non-subcategorized \isi{dative} arguments, but that in these clauses \isi{dative} \isi{noun} phrases do not denote participants of \isi{stative} subevents singled out by \textit{opjat’}.

 
Clauses with non-subcategorized \isi{dative} arguments and predicates like \textit{otkryt}’ \textit{dver}’ ‘open the door’ do not exhibit the \isi{restitutive} reading when \isi{dative} arguments follow the verb \REF{ex:bondarenko:31}, but are able to escape the scope of \textsc{again} when they are scrambled to the left of it, in which case the \isi{restitutive} reading becomes available \REF{ex:bondarenko:32}:


 \ea\label{ex:bondarenko:31}
\gll Vasja opjat’ otkryl \{\hspace{-2pt} Maše dver’ / dver’ Maše\}.\\
     Vasja again opened {} Masha.\textsc{dat} door.\textsc{acc} {} door.\textsc{acc} Masha.\textsc{dat}\\
\ea \textbf{Repetitive}: Available\\
`Vasja opened the door for Masha, that had happened before.'
\ex  \textbf{Restitutive}: Unavailable\\
`Vasja opened the door for Masha, the door had been open   before.'
\z\z

 \ea\label{ex:bondarenko:32}
\gll Vasja Maše opjat’ otkryl dver’.\\
     Vasja Masha.\textsc{dat} again opened door.\textsc{acc}\\
\ea \textbf{Repetitive}: Available\\
`Vasja opened the door for Masha, and that had happened   before.'
\ex \textbf{Restitutive}: Available\label{ex:bondarenko:32b}\\
`Vasja opened the door for Masha, and the door had been open   before.'
\z\z


\noindent As can be seen from the \isi{restitutive} reading of \REF{ex:bondarenko:32}, the \isi{dative} argument is not interpreted as a participant of the \isi{stative} subevent of the predicate \textit{otkryt}’ \textit{dver}’ ‘open the door’. The interpretation in \REF{ex:bondarenko:32b} states that Vasja did some activity for Masha that resulted in the repeated state of the door being open. This suggests that non-subcategorized datives are introduced higher than the syntactically represented \isi{stative} subevents.



Note that scrambling of \isi{dative} arguments to the left of \textit{opjat’} in \isi{ditransitive} sentences does not feed the \isi{restitutive} reading:


 \ea\label{ex:bondarenko:33}
\textit{Context:} Vasja had always had the book \textit{Two captains} by Kaverin; he had never given it to anyone. One day he accidentally left the book at Masha’s place\dots
\exi{}[\#]{\gll I togda Maša Vase opjat’ \{\hspace{-2pt} otdala / otpravila / vernula\} knigu.\\
     and then Masha Vasja.\textsc{dat} again {} gave {} sent {} returned book.\textsc{acc}\\
\glt Intended: `And then Masha gave / sent / returned Vasja the book, and Vasja had had the book before.'}
\z

\noindent This means that \isi{stative} subevents are not represented in the syntax of ditransitives with \isi{dative} arguments. If they were present in the syntactic representation, they could be singled out at least in cases when datives are scrambled.


\largerpage[2]
The fact that the \isi{restitutive} reading of \textit{opjat’} is available in sentences with non-subcategorized datives, in contrast to \isi{ditransitive} sentences with datives, is concordant with the proposal that non-subcategorized \isi{dative} arguments are introduced higher than VPs (\citealt{Boneh-Nash2017}). One piece of evidence for this comes from the fact that sentences with non-subcategorized datives show asymmetrical binding: only the \isi{dative} argument can bind the \isi{accusative} one, but not the other way around:\largerpage


 \ea\label{ex:bondarenko:34}
 \judgewidth{??}
\ea[*]{\gll Šaman zakoldoval oxotnikov drug drugu.\\
  shaman jinxed hunters.\textsc{acc} each other.\textsc{dat}\\}
\ex[]{\gll Šaman zakoldoval oxotnikam drug druga.\\
  shaman jinxed hunters.\textsc{dat} each other.\textsc{acc}\\}
\ex[*]{\gll Šaman zakoldoval drug drugu  oxotnikov.\\
  shaman jinxed each other.\textsc{dat} hunters.\textsc{acc}\\}
\ex[??]{\gll Šaman zakoldoval drug druga oxotnikam.\\
     shaman jinxed each other.\textsc{acc} hunters.\textsc{dat}\\}
\z
\glt (Intended:) `The shaman jinxed the hunters for each other.'\\\hfill (\citealt{Boneh-Nash2017})
\z



\noindent It can be shown that evidence from binding and from the scope of \textit{opjat’} go hand in hand: sentences with non-subcategorized datives, in which the \isi{dative} argument asymmetrically binds the \isi{direct object}, exhibit \isi{restitutive} readings when the \isi{dative} argument is scrambled outside the scope of \textit{opjat’}:


 \ea\label{ex:bondarenko:35}
\textit{Context:} Two hunters have been born jinxed and have been this way for a long time. One day a good witch relieved them from the jinx. But after some time, they had a huge fight and were very angry with each other. Each of them came to the shaman to ask him to jinx the other one.
\exi{}{\gll Šaman oxotnikam opjat’ zakoldoval drug druga\\
     shaman hunters.\textsc{dat} again jinxed each other.\textsc{acc}\\
\glt `Shaman jinxed the hunters for each other, and the hunters had been jinxed before (but the shaman had never jinxed them before).'}
\z


\noindent Thus, non-subcategorized datives are introduced higher than VPs and cannot be understood as participants of \isi{stative} subevents of predicates. But if a predicate has a \isi{stative} subevent, it can be successfully singled out by \textit{opjat’} in case the \isi{dative} argument is scrambled to the left of the \isi{repetitive} \isi{adverb}.


\section{Restitutive readings with Russian datives: Locative applicatives}\label{s5}

In the previous section I have discussed a case of the \isi{restitutive} reading in structures with a \isi{dative} argument which was not a participant in the \isi{stative} subevent singled out by \textit{opjat’}. In this section I will show that \ili{Russian} also has a construction in which a \isi{dative} argument is a participant of the \isi{stative} subevent detected by the \isi{restitutive} \textit{opjat’}.



The construction under consideration, which I will call the \textsc{\isi{locative} \isi{applicative} construction} (“N-applicatives” in the terminology of \citealt{Pshekhotskaya2012}), usually involves a motion verb that takes a \isi{direct object}, a goal PP and an optional \isi{dative} argument:\largerpage


 \ea\label{ex:bondarenko:36}
\gll Maša opjat’ položila knigu Vase na stol.\\
     Masha again put book.\textsc{acc} Vasja.\textsc{dat} on table\\
\ea \textbf{Repetitive}: Available\\
`Masha put the book on the table for Vasja, and that had   happened before.'

\newpage 
\ex \textbf{Restitutive}: Available\\
`Masha put the book on the table for Vasja, and Vasja had had   the book on the table before.'
\z\z


\noindent In \REF{ex:bondarenko:36} the \isi{dative} argument is interpreted as a possessor of the \isi{small clause} that represents the \isi{stative} subevent “the book is on the table”: Vasja’s having the book on the table is being repeated.



The \isi{locative} \isi{applicative} construction is not found exclusively with motion verbs, it is also sometimes possible with lexical causatives \REF{ex:bondarenko:37} and change-of-state predicates \REF{ex:bondarenko:38}.


 \ea\label{ex:bondarenko:37}
\gll Vasja opjat’ posadil dočku Maše na stul.\\
     Vasja again seated daughter.\textsc{acc} Masha.\textsc{dat} on chair\\
\ea \textbf{Repetitive}: Available\\
`Vasja seated the daughter on the chair for Masha, and that had   happened before.'
\ex \textbf{Restitutive}: Available\\
`Vasja seated the daughter on the chair for Masha, and Masha   had had the daughter sit on the chair before.'
\z\z

 \ea\label{ex:bondarenko:38}
\gll Maša opjat’ pobelila stenu mame v komnate.\\
     Masha again whitened wall.\textsc{acc} mother.\textsc{dat} in room\\
\ea \textbf{Repetitive}: Available\\
`Masha whitened the wall in the room for the mother, and that   had happened before.'
\ex \textbf{Restitutive}: Available\\
`Masha whitened the wall in the room for the mother, and the   mother had had the wall white in the room before.'
\z\z


\noindent The \isi{dative} argument in this structure is merged lower than the \isi{direct object}, as the evidence from binding suggests: the \isi{dative} reciprocal can be bound by the \isi{direct object}, but the \isi{accusative} reciprocal cannot be bound by the \isi{dative} argument:


 \ea\label{ex:bondarenko:39}
\ea[]{\gll Vasja posadil devoček drug drugu na stulja.\\
     Vasja seated girls.\textsc{acc} each other.\textsc{dat} on chairs\\
\glt `Vasja seated the girls -- A and B -- in such a way that A has B sitting on A's chair and B has A sitting on B's chair.'\\
(Literally: Vasja seated the girls$_i$ to each other$_i$ on the chairs.)}
 \ex[*]{\gll Vasja posadil drug druga devočkam na stulja.\\
     Vasja seated each other.\textsc{acc} girls.\textsc{dat} on chairs\\
\glt Intended: `Vasja seated the girls -- A and B -- in such a way that A has B sitting on A's chair and B has A sitting on B's chair.'\\
(Literally: Vasja seated each other$_i$ to the girls$_i$ on the chairs.)}\label{ex:bondarenko:40}
\z\z

\noindent The example in \REF{ex:bondarenko:41} shows that the \isi{dative} reciprocal that is bound by the \isi{direct object} can be a participant of the \isi{stative} subevent identified by \textit{opjat’}:


 \ea\label{ex:bondarenko:41}
\gll Vasja opjat’ posadil devoček drug drugu na stulja.\\
     Vasja again seated girls.\textsc{acc} each other.\textsc{dat} on chairs\\
\ea \textbf{Repetitive}: Available\\
‘Vasja seated the girls -- A and B -- in such a way that A has B sitting on A's chair and B has A sitting on B's chair, and that had happened before.’\\
(Literally: Vasja seated girls$_i$ to each other$_i$ on the chairs, and that had happened before.)
\ex \textbf{Restitutive}: Available\\
‘Vasja seated the girls -- A and B -- in such a way that A has B sitting on A's chair and B has A sitting on B's chair, and there was a situation before where A had B sitting on A's chair, and B had A sitting on B's chair.'\\
(Literally: Vasja seated girls$_i$ to each other$_i$ on the chairs, and the   girls$_i$ had sat by each other$_i$ on the chairs before.)\\
\z\z


\noindent It can also be demonstrated that the \isi{dative} argument forms a constituent with the \isi{locative} phrase. When a \isi{dative} argument is a wh-word, it can pied-pipe the prepositional phrase to the \isi{left periphery}:


 \ea\label{ex:bondarenko:42}
\ea\gll [\hspace{-2pt} Komu na stol] Maša položila knigu?\\
     {} who.\textsc{dat} on table Masha put book.\textsc{acc}\\
\glt `Which person $x$ is such that Masha put a book for $x$ on $x$'s table?'
\ex\label{ex:bondarenko:43}
\gll [\hspace{-2pt} Komu na stul] Vasja posadil devočku?\\
     {} who.\textsc{dat} on chair Vasja seated girl.\textsc{acc}\\
\glt `Which person $x$ is such that Vasja seated a girl for $x$ on $x$'s chair?'
\ex\label{ex:bondarenko:44}
\gll [\hspace{-2pt} Komu v školu] Maša otdala syna?\\
     {} who.\textsc{dat} in school Masha gave son.\textsc{acc}\\
\glt `Which person $x$ is such that Masha gave her son to $x$, to $x$'s school?'
\z\z

\noindent I would like to propose that in the \isi{locative} \isi{applicative} construction the \isi{dative} \isi{noun} phrase is an \isi{applicative} argument that is introduced on top of the PP that introduces a \isi{stative} subevent into the syntactic representation. Since \isi{applicative} heads introduce an abstract HAVE relation between the applied argument and the complement of Appl (\citealt{Cuervo2003,McIntyre2006}; among others), the fact that the \isi{dative} argument in \ili{Russian} \isi{locative} applicatives is interpreted as a holder of the state that the PP denotes is expected if the \isi{dative} argument is applied to an eventive PP; see \REF{ex:bondarenko:45} and \figref{fig:bondarenko:4}.\footnote{The structure in \figref{fig:bondarenko:4} feeds the relevant (\isi{restitutive}) interpretation. In order to derive the attested \isi{word order}, cf. \REF{ex:bondarenko:45}, I assume that later in the derivation the lexical verb \textit{povesil} `hung' undergoes further movement to Asp (see \citealt{Harizanov-Gribanova2018} for discussion), and the \isi{repetitive} \isi{adverb} \textit{opjat’} moves to a position before the verb (the arguments for a movement analysis of repetitives that were proposed in \citealt{Xu2016} for \ili{Chinese} hold for \ili{Russian} as well), with subsequent reconstruction into its base position at LF.}


 \ea\label{ex:bondarenko:45}
\gll Vasja opjat’ povesil kartinu Kate na stenu.\\
     Vasja again hung picture Katja.\textsc{dat} on wall\\
\ea \textbf{Repetitive}: Available\\
`Vasja hung the picture for Katja on the wall, and that had   happened before.'

\ex \textbf{Restitutive}: Available\\
`Vasja hung the picture for Katja on the wall, and Katja had the   picture on the wall before.'
\z\z

\begin{figure}
\begin{forest}for tree={l=0}%for tree= pretty nice empty nodes
[VP
  [DP
    [{\textit{kartinu}$_i$}, roof]
  ]
  [V'
    [V
      [\textit{povesil}]
    ]
    [ApplP
      [\textit{opjat'}]
      [ApplP
	[DP
	  [\textit{Kate}, roof]
	]
	[Appl'
	  [Appl]
	  [PP
	    [PRO$_i$]
	    [P'
	      [P
		[\textit{na}]	  
	      ]
	      [DP
		[\textit{stenu}, roof]
	      ]
	    ]
	  ]
	]
      ]
    ]
  ]
]
\end{forest}
\caption{The locative applicative construction \REF{ex:bondarenko:45}}
\label{fig:bondarenko:4}
\end{figure}


The \isi{restitutive} reading of \textit{opjat’} in this construction arises when \textit{opjat’} attaches to an \isi{applicative} phrase (\figref{fig:bondarenko:4}) and takes scope over the \isi{stative} subevent denoted by a goal PP. The \isi{dative} argument falls inside the scope of \textit{opjat’} since it is an applied argument of an eventive PP and not an argument of the verb.


\section{Conclusions}\label{s6}

In this paper I have argued against the \isi{small clause} analysis of \ili{Russian} ditransitives. I have observed that although \ili{Russian} \isi{repetitive} \isi{adverb} \textit{opjat’} has the same ability to look inside the decomposition structure as \ili{English} \textit{again}, it cannot have the \isi{restitutive} reading in clauses with \isi{ditransitive} verbs that take two objects, in contrast to \textit{again} in the \ili{English} \isi{double object construction}. I have shown that \ili{Russian} ditransitives have \isi{stative} subevents in their \isi{semantics} and that the unavailability of a \isi{small clause} structure for \ili{Russian} ditransitives cannot be explained by a semantic restriction, since the Principle R or its equivalent that allows to interpret a combination of a verb and a \isi{small clause} is independently required for other constructions of \ili{Russian}. I have concluded that the \isi{small clause} structure is not present in \ili{Russian} ditransitives due to syntactic reasons: the syntax cannot build such a structure. The unavailability of the \isi{restitutive} reading in \ili{Russian} ditransitives suggests that they are not equivalent to the \ili{English} \isi{double object construction} or the \textit{to-}PP construction. They also cannot be analyzed as involving a silent (incorporated) P, since the structure with a PP would make the \isi{restitutive} reading available. Although the new empirical data discussed in this paper is compatible with several analyses of ditransitives (for example, with \isi{applicative} analysis \citep{Bruening2010} or non-derivational analysis along the lines of (\citealt{Boneh-Nash2017}) and does not settle on a particular one, it clearly shows that \ili{Russian} ditransitives do not involve a \isi{small clause} structure and differ from \ili{English} ditransitives significantly.\largerpage



I have also examined two other constructions with \isi{dative} arguments in \ili{Russian}, both of which allow for the \isi{restitutive} reading of \textit{opjat’}. In sentences with “high” datives the \isi{restitutive} reading is available if the \isi{dative} argument escapes the scope of \textit{opjat’}. The \isi{dative} does not denote a participant of the \isi{stative} subevent in this case, which means that it cannot be introduced into the structure lower than the first subevent of the predicate. In the \isi{locative} \isi{applicative} construction, the \isi{dative} argument is a participant of the subevent introduced by a PP and is inside the scope of the \isi{restitutive} \textit{opjat’}. I have argued that in this construction the \isi{dative} is an applied argument to the PP, and therefore is always lower than the \isi{direct object}, forms a constituent with the PP and can be inside the scope of \textit{opjat’} under the \isi{restitutive} reading.

\section*{Abbreviations}

\begin{tabularx}{.5\textwidth}{@{}lQ@{}}
\textsc{acc}&\isi{accusative}\\
\textsc{dat}&\isi{dative}\\
\end{tabularx}%
\begin{tabularx}{.5\textwidth}{@{}lQ@{}}
\textsc{inf}&\isi{infinitive}\\
\textsc{pvb}&preverb\\
\end{tabularx}

\section*{Acknowledgments}
Many thanks to Sergei Tatevosov for his feedback and to the audience of the FDSL12 conference.

\sloppy
\printbibliography[heading=subbibliography,notkeyword=this]

\end{document}
