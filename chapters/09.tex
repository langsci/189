\documentclass[output=paper,modfonts,newtxmath,hidelinks]{langscibook}
\ChapterDOI{10.5281/zenodo.2545523}


\title{Surviving sluicing} 
\author{Franc Marušič\affiliation{University of Nova Gorica}\and Petra Mišmaš\affiliation{University of Nova Gorica}\and Vesna Plesničar\affiliation{University of Nova Gorica}\lastand Tina Šuligoj\affiliation{University of Nova Gorica}}

\abstract{In this paper, we discuss examples of sluicing in Slovenian in which, in addition to a wh-phrase (or wh-phrases in instances of multiple sluicing) discourse particles appear. This is unexpected given \citeposst{merchant2001} Sluicing-COMP generalization, as already observed in \citet{marusicetal2015}, even though there are several languages in which similar cases exist, e.g. German. In this paper we focus on discourse particles \textit{pa} and \textit{že} in (multiple) wh-questions and sluicing. These examples are not only important for our understanding of sluicing but are also crucial for analyzing discourse particles in Slovenian. Based on examples with sluicing and discourse particles in Slovenian, we argue against positioning these particles within the wh-phrase, clitic cluster or the IP. 

\keywords{Slovenian, sluicing, particles, sluicing-COMP generalization}
}

\begin{document}
\maketitle
\shorttitlerunninghead{Surviving sluicing}

\section{Introduction}\label{9:s1}
In this paper we address the phenomenon already discussed in \cite{marusicetal2015}, i.e. cases in which in addition to a wh-phrase a \isi{discourse particle} appears in sluicing in \ili{Slovenian}. These cases are unexpected given the standard understanding of sluicing, which is in \cite{ross1969} described as deletion of parts of the embedded question that are identical to some part of the antecedent \isi{clause}, leaving only the wh-phrase, as shown in (\ref{9:simplesluice}). 


\ea \label{9:simplesluice}I heard somebody, but I don't know [who \sout{I heard}].
\z
 
\noindent Despite \citeauthor{ross1969}' definition of sluicing as a phenomenon in embedded clauses, today, sluicing is taken to be a type of \isi{ellipsis} phenomenon ``in which the sentential portion of a constituent question is elided, leaving only a wh-phrase remnant'' (\citealt{merchant2006}: 271) which can occur in embedded or root causes. In what follows, all \ili{Slovenian} examples will be cases with sluicing in root clauses.\footnote{While sluicing also exist in embedded clauses in \ili{Slovenian}, as (\ref{9:embeddedpa}) shows, examples with discourse particles, which we are looking at in this paper, are limited to root clauses. This is not surprising, given that discourse particles are typically a root \isi{clause} phenomenon, which is related to their relation to both the illocutionary force and sentence type of the \isi{clause} (\citealt{bayerobenauer2011}: 452).

\ea \label{9:embeddedpa}
\gll Vid je nekoga srečal. Ne vem, koga (*\hspace{-2pt} pa).\\
Vid \textsc{aux}  someone met.\textsc{acc}  not know who.\textsc{acc} {} \textsc{ptcl}\\
\glt `Vid met someone. I don't know who.'
\zlast }

The insight that only the wh-phrase remnant appears in sluicing is formalized in Merchant's Sluicing-COMP generalization, given in (\ref{9:generalization}), in which ``operator'' stands for ``syntactic wh-XP'', ``material'' for any pronounced element, and ``COMP'' for ``material dominated by CP but external to IP'' (\citealt{merchant2001}, 62). Given a standard understanding of what CP represents, if one assumes the expanded \isi{left periphery} \`{a} la \cite{rizzi1997}, we assume this generalization was meant to be read as follows: In sluicing only wh-phrases survive \isi{ellipsis} as they are the only elements occupying the \isi{left periphery}. Apart from the wh-phrase, the \isi{left periphery} does not contain any overt elements.\footnote{This seems exactly what Merchant's informal explanation of his generalization says: ``The claim is that only segments directly associated with the syntactic operator -- the wh-XP -- will be found overtly in sluiced interrogatives.'' (\citealt[62]{merchant2001})}

\begin{exe}
\ex\label{9:generalization} \textbf{Sluicing-COMP generalization}\\
In sluicing, no non-operator material may appear in COMP.\vspace{-3pt}
\exi{}[]{\hfill(\citealt{merchant2001}: 62, (71))}
\end{exe}

\largerpage
\noindent Given this, as observed in \cite{marusicetal2015}, examples such as (\ref{9:vsi}) are unexpected. In all examples given in (\ref{9:vsi}), non-wh-material survives sluicing:\footnote{Wh-elements in \ili{Slovenian} contain the wh-morpheme \textit{k-/č}-, the particles, however, do not (cf. \citealt{marusicetal2015}).}

\begin{exe}
\ex\label{9:vsi}
\begin{xlist}
 \ex 
 \gll Vid je srečal nekoga. Koga pa? \\
 Vid \textsc{aux} met someone. Who \textsc{ptcl} \\
 \trans `Vid met someone. Who $<$did he meet$>$?'
 \ex	
 \gll Vid je srečal nekoga. Koga že?\\
	 Vid \textsc{aux} met someone. Who \textsc{ptcl} \\
 \trans	`Vid met someone. Remind me, who $<$did he meet$>$?'
  \ex	
 \gll 	Vid je srečal Janeza. Koga še?\\
		Vid \textsc{aux} met Janez. Who \textsc{ptcl} \\
 \trans	`Vid met someone. Who else $<$did he meet$>$?'
 \ex	
 \gll 	Vid je srečal nekoga. Koga to?\\
		Vid \textsc{aux} met someone. Who \textsc{ptcl} \\
 \trans	`Vid met someone. Who $<$did he meet$>$?'
 \ex	
 \gll 	Vid je srečal nekoga. Koga spet?\\
		Vid \textsc{aux} met someone. Who again\\
 \trans	`Vid met someone. Who (are you saying again) $<$did he meet$>$?'
 \ex	
 \gll Vid je srečal nekoga. Koga pa to?\\
		Vid \textsc{aux} met someone. Who  \textsc{ptcl}  \textsc{ptcl}  \\
 \trans	`Vid met someone. Who $<$did he meet$>$?'
 \ex 
 \gll	Vid je srečal Ano pa še nekoga.  Koga pa še?\\
		Vid \textsc{aux} met Ana and also someone. Who \textsc{ptcl}  \textsc{ptcl} \\
 \trans	 `Vid met Ana and someone else. Who else $<$did he meet$>$?'
 \end{xlist}
\end{exe}

\noindent As wh-phrases in sluicing can also be complex, as in (\ref{9:ktiro}), one can imagine these discourse particles that follow wh-words in (\ref{9:vsi}) could also be part of a complex wh-phrase. 

\begin{exe}
 \ex\label{9:ktiro}
 \begin{xlist}
 \exi{A:}[]{\gll Peter je videl neko punco.\\
	Peter \textsc{aux} saw some girl.\\
    \trans `Peter saw some girl.'}
  \exi{B:}[]{\gll Katero punco?\\
  which girl\\
 \trans `Which girl?'}
 \end{xlist}
\end{exe}

\noindent As shown in \cite{marusicetal2015} these particles do not form a constituent with the wh-material, but are rather a part of the extended \isi{left periphery} (in the sense of \citealt{rizzi1997}) that is not elided in sluicing in \ili{Slovenian}.\footnote{Note that examples in (\ref{9:vsi}) are also not instances of swiping \citep{merchant2002swiping}, as these particles are not prepositions, or spading \citep{van2010syntax}, as these particles are not demonstrative pronouns.}

In this paper we look at the sluicing examples with particles more closely in order to better understand where exactly particles are located and where they originate. We present new arguments against placing particles inside wh-phrases and show that particles really are part of the \isi{left periphery} and thus offer further support for the analysis according to which non-wh-material in the \isi{left periphery} does not have to be elided in sluicing in \ili{Slovenian} \citep{marusicetal2015}.

We start with an assumption that sluicing is the \isi{ellipsis} of the IP portion of a constituent question \citep{ross1969,merchant2001,merchant2006},  which means the examples in (\ref{9:vsi}) are parallel to the examples in (\ref{9:celstavek}).\footnote{We are avoiding the debate on the nature of the \isi{ellipsis site} in sluicing, particularly whether it has the same exact structure as the antecedent (which is what we adopt, following \citealt{ross1969}, \citealt{merchant2001}, and others) or whether it is structurally empty with its content being supplied by re-using syntactic structure from some accessible point elsewhere in the discourse (which is what \citealt{chungetal1995}, \citeyear{chungetal2011}, among others are arguing for). Our data do seem to favor the approach we are adopting, but we do not want to go into this discussion here.} Based on this, we discuss the role of discourse particles in both wh-questions and sluicing. From now on, we gloss the two \ili{Slovenian} particles under discussion as \textsc{pa} and \textsc{že}.

\begin{exe}
\ex\label{9:celstavek}
\begin{xlist}
 \ex 
 \gll Koga pa je Peter 	videl?\\
 who.\textsc{acc} \textsc{pa} \textsc{aux} Peter.\textsc{nom} saw\\
 \trans `Who did Peter see?'
 \ex 
 \gll Koga že je Peter videl? \\
 who.\textsc{acc} \textsc{že}  \textsc{aux} Peter.\textsc{nom} saw\\
 \trans `Who did Peter see?'
 \end{xlist}
\end{exe}

\noindent While examples in (\ref{9:vsi}) show that there are several discourse particles that can appear in sluicing in \ili{Slovenian}, we focus only on discourse particles \textit{že} and \textit{pa} here. Some initial thoughts on other particles in sluicing can be found in \cite{marusicetal2015}, but since the role of \ili{Slovenian} discourse particles in wh-questions has previously not been sufficiently described, we start off by showing some properties these elements display when they are used as discourse particles (and not topic or focus particles). Both \textit{pa} and \textit{že} have many different uses and meanings, which we will discuss in \sectref{9:s2}. In \sectref{9:s3} we take examples with sluicing and the particles \textit{pa} and \textit{že} to give new arguments both against positioning these particles within the wh-phrase and to show that in addition to (complex) wh-phrases non-wh-material can also survive sluicing in \ili{Slovenian}. In \sectref{9:s4} we discuss the position of discourse particles with respect to the  \isi{clitic cluster} and the adverbs in the IP to show that discourse particles appear in the \isi{left periphery}, higher than IP adverbs, confirming the earlier proposal by \cite{marusicetal2015}. \sectref{9:s5} concludes the paper.


\section{Discourse particles \textit{že} and \textit{pa} in Slovenian wh-questions}\label{9:s2}

In general particles \textit{že} and \textit{pa} (as do other discourse particles in \ili{Slovenian}) display some properties that are typically found with discourse particles cross-lingui\-sti\-cal\-ly. For example, as \cite{zimmermann2011} notes, discourse particles carry more than one function and can also be used as focus particles, discourse markers (i.e. markers that establish coherence in the discourse) or adverbials.\footnote{For further differences between discourse markers and discourse particles see, for example, \cite{zimmermann2011}.} This also holds for \textit{pa} and \textit{že}, e.g. \textit{že} is also an \isi{aspectual} \isi{adverb}. Furthermore, discourse particles in \ili{Slovenian} are optional (as in other languages, see \citealt{bayerobenauer2011} for \ili{German}), to a certain extent various discourse particles can appear simultaneously in the \isi{clause}, they are sensitive to \isi{clause} type, they normally do not bare stress, and are monosyllabic. And perhaps most importantly, discourse particles do not modify the proposition, but rather the utterance \citep{bayerobenauer2011} as they express speakers' attitude towards the utterance \citep{zimmermann2011}. To further show properties of particles \textit{že} and \textit{pa} in \ili{Slovenian}, we discuss them separately in this section. 

\subsection{\textit{Že} as a discourse particle}

Etymologically the origin of \ili{Slovenian} particle \textit{že} is closely related to the morpheme \textit{-r} that one finds in relative pronouns in \ili{Slovenian} (\textit{kdor} `who', \textit{kar} `what', \textit{kjer} `where'). Both are etymologically related with the Indo-European particle \textit{*g\textsuperscript{h}e/*g\textsuperscript{h}o} that has developed into particles in several \ili{Slavic} languages, for example \textit{že} in \ili{Russian} (see \citealt{hagstrommccoy2003} for its interpretation in wh-questions), and \textit{že} in \ili{Czech} \citep{skrabalova2012} (cf. \citealt{mitrovic2016benj}). But despite the common source, languages differ with respect to the actual meaning of \textit{že}.\enlargethispage{8pt}

For example, \cite{skrabalova2012} shows that the \ili{Czech} \textit{že} is a \isi{complementizer} that can be used in declarative and \isi{interrogative} clauses. In embedded contexts \textit{že} combines with the declarative \isi{clause} and it marks syntactic dependence of embedded \isi{clause}, but \textit{že} also triggers an echo-interpretation in \ili{Czech}.\footnote{\ili{Czech} particle \textit{že} is in many respects similar to \ili{Slovenian} \textit{da} `that', which can be used as a \isi{complementizer} or a \isi{discourse particle}, see \cite{marusicetal2015} for more on this topic.} That is, in wh-questions, following \cite{skrabalova2012} \textit{že} indicates that the speaker has not heard or that (s)he refuses to accept a previous utterance. For example, (\ref{9:A}) is used to check whether the part of the utterance asked by the wh-word was asserted in the previous context, see \cite{skrabalova2012} for more on \ili{Czech} \textit{že}.

\begin{exe} 
\ex\label{9:A}
\begin{xlist}
\exi{A:}[]{\gll Kam  že Petr šel?\\
 	 where that Peter went\\
\glt `Peter went where?'}
\exi{B:}[]{\gll Přece do restaurace! \\
	 indeed to restaurant \\
\glt `(I said he went) to a restaurant.'\\
\hfill (\ili{Czech}; \citealt{skrabalova2012}: 5, (10))}
\end{xlist}
\end{exe}

\noindent As \cite{mccoy2003} observes, \ili{Russian} \textit{že} can function as a modal/affective particle, focus marker, marker of \isi{contrastive focus}, emphasis marker, thematic/ organizational/ textual \textit{že}, marker of (re-)activated information, and marker of a reference point in the activated domain of reference. \citet{hagstrommccoy2003} and \cite{mccoy2003} observe the following contexts with distinctive occurrences of \textit{že} in \ili{Russian}: \textit{yes-no} questions, wh-questions, statements with phrasal scope and statements with sentential/propositional scope. Example (\ref{9:russian}) shows the use of \textit{že} in a wh-question. Crucially, it depicts a situation where the child wants to sleep in the morning, which seems unreasonable to the mother, as the child is not supposed to have a reason to feel sleepy at that time. That is, example (\ref{9:russian}) is a rhetorical question, where \textit{že} roughly corresponds to the \ili{English} \textit{in the world}. As shown in (\ref{9:slorus}), \textit{že} can be used in \ili{Slovenian} in a similar way.\footnote{As \cite{mccoy2003} observes, in these rhetorical questions the speaker does not expect a possible reasonable/true answer. Moreover, her conclusion is that \textit{že} in wh-questions applies to every member of set of contextually accessible answers to the question, generating \isi{presupposition} for each proposition in the set. Therefore, \ili{Russian} \textit{že} in wh-questions generates \isi{presupposition} that the possible answers from the set in question have already been evaluated as false and the same applies to \ili{Slovenian} \textit{že} under conditions as presented in (\ref{9:slorus}).}

\begin{exe}
\ex\gll Varen'ka, nu Varen'ka, nu začem že tebe baj-baj s utra. \\
Varen'ka \textsc{ptcl} Varen'ka \textsc{ptcl} why \textsc{ptcl} to.you night-night from morning.\\
\glt `Well, Varen'ka, why in the world do you need night-night in the morning?'\hfill(\ili{Russian}; \citealt{mccoy2003}: 125) \label{9:russian}
\ex \textit{Situation:} Ana asks Vid for help and Vid answers (in a bit irritated tone): \label{9:slorus} \\
\gll Zakaj že bi ti pomagal? \\
	why \textsc{že}  \textsc{cond} you helped \\
\trans `Why on earth should I help you?'
\end{exe}

\noindent While (\ref{9:slorus}) already shows one meaning of the particle \textit{že}, \textit{že} most commonly appears as an \isi{aspectual} \isi{adverb} meaning `already', as shown in (\ref{9:aspectualze}). Using \textit{že} as an \isi{aspectual} \isi{adverb} is very common, and while in some cases \textit{že} can receive this interpretation in addition to the \isi{discourse particle} reading, this is not directly relevant for the present discussion.\footnote{There are also other meanings, for example, \textit{že} can be used to express agreement with a statement:

\ea
\begin{xlist}
\exi{A:}[]{
\gll Miha je opral obleke.\\ 
Miha \textsc{aux} washed clothes\\
\glt `Miha washed the clothes.'}
\exi{B:}[]{\gll Že že, a ne vem, kdo jih je zlikal.\\ 
\textsc{že} \textsc{že}	but not know who it.\textsc{acc} \textsc{aux} ironed\\
\glt `True, but I don't know who ironed them.'}
\end{xlist}
\zlast
}

\begin{exe} 
\ex \label{9:aspectualze}
\gll Peter 	je že šel na počitnice. \\
	Peter   \textsc{aux} \textsc{že} went on vacation\\
\trans `Peter has already left for the vacation.' 
\end{exe}

\noindent While we can also find the \isi{aspectual} meaning in wh-questions, the use of \textit{že} in sluicing or a wh-question more importantly indicates that the speaker knows the answer to the question but does not remember it. We will refer to this reading as the `remind-me' reading, following \cite{sauerland2014wieder}, and will use \textit{že}-\textsc{r} to refer to the morpheme carrying this meaning. The morpheme carrying the \isi{aspectual} reading will be referred to as \textit{že}-\textsc{a}. The former reading is apparent in the following scenario. Imagine we visit our friend Peter in April, but his mother tells us he is not home, we remember that he is never home in the spring and we actually know where he always travels in the spring, but at the moment we cannot recall where he travels. We ask his mother the question in (\ref{9:remindmeze}) as a `remind-me' question.

\begin{exe}
\ex \label{9:remindmeze}
\gll Kam že hodi vsako leto?\\
	where \textsc{že} goes every year\\
	\trans `(Remind me) Where does he go every year?'
\end{exe}

\noindent This meaning is possible in wh-questions and in sluicing, while \textit{že} in \textit{yes/no}-questions (or in declarative sentences), such as (\ref{9:yesnoze}), can receive the \isi{aspectual} reading, but not the `remind-me' reading. 

\begin{exe}
\ex \label{9:yesnoze}
\gll A je že opral obleke?\\
	\textsc{q} \textsc{aux} \textsc{že} washed clothes\\
	\glt Available: `Did he already wash the clothes?'
\glt Unavailable: `(Remind me) Did he was the clothes?'
\end{exe}

\noindent Interestingly, as shown in (\ref{9:guernicajeze}), both the `remind-me' interpretation and the \isi{aspectual} reading of \textit{že} are available when \textit{že} and the wh-word are not adjacent. In relation to this, two things need to be noted. First, the availability of \textit{že}-\textsc{r} in (\ref{9:guernicajeze}) implies that \textit{kaj} `what' and \textit{že}-\textsc{r} do not necessarily form a constituent, as clitics do not split syntactic constituents in \ili{Slovenian}.\footnote{Consider for example the ungrammaticality of (\ref{9:ib}):
\ea
\begin{xlista}
\ex
{
\gll	Poletni dež je prekinil zabavo.\\
 summer rain \textsc{aux} stopped party\\
\trans `Summer rain stopped the party.' 
}
\ex
{
 *Poletni je dež prekinil zabavo.
}
\label{9:ib}
\end{xlista}
\zlast} 
Second, when \textit{že} precedes the auxiliary \isi{clitic}, only the `remind-me' reading is available.\footnote{A note on intonation is needed. That is, when (\ref{9:guernicajeze}) is interpreted as a wh-question with \textit{že}-\textsc{a}, it will also receive a normal wh-intonation. On the other hand, \textit{že}-\textsc{r} in (\ref{9:guernicajeze}) is emphasized and the question ends with a rising intonation (similar to the intonation in \textit{yes/no}-questions). Interestingly, (\ref{9:guernicazeje}) does not receive a true wh-reading if we change the intonation and the only interpretation it can receive is the `remind me'-reading. This implies that the intonation does not trigger the `remind-me' interpretation of the wh-question.} 

\begin{exe}
\ex 
\begin{xlist}
\ex \label{9:guernicazeje}
\gll Kdo že je naslikal Guernico? \\
		who \textsc{že} \textsc{aux} painted Guernica.\textsc{acc}\\
\trans  Available: `(I need to remember) who painted Guernica?'
\trans Unavailable: `Who already painted Guernica?'
\ex \label{9:guernicajeze}
\gll	Kdo je že naslikal Guernico?\\
		who \textsc{aux} \textsc{že} painted Guernica.\textsc{acc}\\
\trans	 Available: `(Remind me) who painted Guernica?'
\trans 	 Available: `Who already painted Guernica?'
\end{xlist}
\end{exe}

\noindent In a wh-question \textit{že}-\textsc{r} follows the wh-phrase. Examples in which \textit{že}-\textsc{r} precedes the wh-word are unacceptable, as wh-phrases need to appear in a \isi{clause} initial position in \ili{Slovenian} wh-questions, see \cite{mismas2016benj}. 

\ea[*] {
\gll Že kdo je naslikal Guernico? \\
   \textsc{že} who \textsc{aux} painted Guernica \\
\trans Intended: `(Remind me) who painted Guernica?}
\z

\noindent In sluicing, \textit{že} can only receive the `remind-me' reading, as (\ref{9:guernicasluic}) shows. That is, (\ref{9:guernicasluic}) can be used in a context where the speaker is playing a game, where (s)he needs to name the author of Guernica. The speaker knows the answer, but cannot remember it, so (s)he utters:\footnote{In sluicing, \textit{že} can be used in rhetorical questions, already discussed above. That is, \REF{9:zakajzefn} can be used in the situation described in \REF{9:slorus}.
\ea {
\gll A pomagam ti naj? Zakaj že?\\
\textsc{q} help.\textsc{1sg} you.\textsc{dat} should why \textsc{že}\\
\glt `Oh, I should help you? Why?'
}
\label{9:zakajzefn}
\zlast}

\ea \label{9:guernicasluic}
\gll  Seveda vem, kdo je naslikal Guernico. Kdo že?\\
{of course} know who \textsc{aux} painted Guernica.\textsc{acc} who \textsc{že}\\
\glt `Of course I know who painted Guernica? (I need to remember) Who?'
\z

\noindent Crucially, in (\ref{9:guernicasluic}) \textit{že} cannot be interpreted as an \isi{aspectual} \isi{adverb}. Given that \isi{aspectual} adverbs are located in the IP area and as sluicing is said to delete the entire IP area, the lack of \isi{aspectual} reading for \textit{že} is expected. And as \textit{že}-\textsc{r} is available in the structure where IP is supposedly missing, we have an argument to assume \textit{že}-\textsc{r} originates inside the \isi{left periphery}. We return to this questions below in \sectref{9:s4}. 


\subsection{\textit{Pa} as a discourse particle}

Following \cite{snoj2009}, \textit{pa} (which has counterparts in several \ili{Slavic} languages, for example in Serbo-\ili{Croatian} as \textit{pa} and \textit{pak,} meaning `again' or `then', and \ili{Czech} \textit{pak} `then, after') is related to \textit{paky} `again', `also' in Old Church \ili{Slavonic} and originates from Proto-\ili{Slavic} \textit{*pȃkъ}, which originally meant `differently', `again', `later', and probably also `wrong' and `bad'; see \cite{snoj2009} for more information on the etymology of \textit{pa}.

Today, \textit{pa} is a very common element in \ili{Slovenian}, especially in colloquial language. The particle \textit{pa} can be used in regular coordinations (similarly to standard \ili{Slovenian} `and'), (\ref{9:peterpaana}), and as a subordination \isi{complementizer} such as the standard \ili{Slovenian} \textit{ampak} `but'. In the latter use \textit{pa} typically appears in the \isi{second position} (see \citealt{marusicetal2011pa} for more data), as can be seen from the examples in (\ref{9:pojepane}).

\begin{exe}
\ex \label{9:peterpaana}
\gll Peter 	pa Ana plešeta. \\
	Peter 	and Ana dance\\
\trans  `Peter and Ana are dancing.' 
\ex \label{9:pojepane}
\begin{xlist}
\ex \gll	Peter pleše, poje pa ne. \\
 		Peter 	dances sings but not\\
\trans `Peter dances, but does not sing.'
\ex \gll 	Peter pleše, ampak ne poje. \\
 	Peter 	dances but not sings\\
 \trans `Peter dances, but does not sing.'
\end{xlist}
\end{exe}

\noindent The particle \textit{pa} can function as a topic marker or as a \isi{contrastive focus} marker in declarative sentences. \textit{Pa} used as a topic marker is given in (\ref{9:patopic}). In the context where friends are talking about various people dancing and someone asks about a certain person called `Peter', (\ref{9:patopic}) could be a natural reply. \textit{Pa} can also be a \isi{contrastive focus} marker, as in (\ref{9:contrastivepa}). 

\begin{exe}
\ex \label{9:patopic}
\gll  Petra 	pa še nisem videl plesati.\\
 	Peter.\textsc{gen} 	\textsc{pa} yet \textsc{neg.aux} see dance.\textsc{inf} \\
\trans   `As for Peter, I have not seen him dance yet.' 
\ex \label{9:contrastivepa}
\gll Jaz bom plesal tango, ti pa step.\\
 	I will.\textsc{1sg} dance tango you \textsc{pa} tap\\
\trans   `While I will be dancing the tango, you should tap dance.' 
\end{exe}

\noindent \textit{Pa} can be a topic/focus marker in wh-questions as well. In this role, \textit{pa} interacts with an emphasized constituent. Based on an emphasis (marked with smallcaps), the meaning of the question in (\ref{9:topicfocuspa}) varies slightly, however, we are here focusing on \textit{pa} as a discourse marker, so we are leaving these cases aside.

\begin{exe}
\ex  \label{9:topicfocuspa}
\begin{xlist}
\ex \gll	Kdo pa \textsc{pleše} s Petrom?\\
		who \textsc{pa} dance with Peter.\textsc{ins}\\
\trans `(We know who runs with Peter, but we want to know) who dances with Peter?'
\ex \gll	Kdo pleše pa s \textsc{Petrom}?\\
		who dances \textsc{pa} with Peter.\textsc{ins}\\
\trans `(We know about who dances with the others, but we want to know) who dances with Peter?'
\end{xlist}
\end{exe}

\noindent As a discourse marker, \textit{pa} is associated with a strongly presupposed context (see \citealt{chengrooryck2000} for this interpretation of wh-in situ questions in \ili{French}). That is, the situation is established and/or is presupposed and we are seeking details about the situation. Hence, just like what \cite{chengrooryck2000} claim for \ili{French}, a negative answer to a wh-question with the \isi{discourse particle} \textit{pa} is unexpected. For example, if we ask (\ref{9:pressuposedpa}) we already know that someone was visiting we just do not know who was visiting. Getting a negative answer ('Nobody.') is not impossible, but it would be surprising for the speaker to get this answer. Side note, (\ref{9:pressuposedjepa}) shows that \textit{pa} can appear before or after the auxiliary \isi{clitic}, just like \textit{že}, which again indicates that the particle and the wh-phrase do not form a constituent. 

\begin{exe}
\ex \begin{xlist}
\ex \label{9:pressuposedpa}
\gll Kdo pa je bil na obisku?\\
	who \textsc{pa} \textsc{aux} was on visit\\
\trans `(I know someone was visiting, tell me) Who was visiting?'
\ex \label{9:pressuposedjepa} 
\gll Kdo je pa bil to?\\
who  \textsc{aux}  \textsc{pa} was this\\
\end{xlist}
\end{exe}

\noindent This reading, related to the strongly presupposed context, is also available in sluicing.\footnote{\textit{Pa} in sluicing can also be a \isi{contrastive focus} particle -- for example (\ref{9:kajpajepojed}) can also be interpreted as a response to a context in which we already know what Ana did not eat but we want to know what she did eat (cf. \citealt{marusicetal2015}). While interesting, we are leaving this reading aside here.} So, if we hear (\ref{9:anapojedla}) and we reply with the sluices in (\ref{9:kajpajepojed}) or (\ref{9:kdajpajepojed}), this means that we potentially already knew (\ref{9:anapojedla}) or we fully accept (\ref{9:anapojedla}), but we need additional information about what and when Ana was eating. 

\begin{exe}
\ex \begin{xlist}
\ex \label{9:anapojedla}
\gll Ana je nekaj pojedla.\\
	Ana \textsc{aux} something ate\\
\trans  `Ana has eaten something.'
\ex \label{9:kajpajepojed}
\gll Kaj pa \sout{je}\sout{\ }\sout{\ }\sout{\ }\sout{\ }\sout{\ }\sout{pojedla}?\\
 		what \textsc{pa} {\textsc{aux} ate}\\
\trans  `What?'
\ex \label{9:kdajpajepojed}
\gll Kdaj pa \sout{je}\sout{\ }\sout{\ }\sout{\ }\sout{\ }\sout{\ }\sout{pojedla}?\\
 		when \textsc{pa} {\textsc{aux} ate} \\
\trans  `When?'
\end{xlist}
\end{exe}

\noindent Examples in this section show that discourse particles can appear in sluicing in \ili{Slovenian}, but more importantly, indicate that not only operator material survives sluicing, as we would expect given Sluicing-COMP generalization \citep{merchant2001}. The question is then why discourse particles in \ili{Slovenian} are able to do so.

\section{Wh-phrases, discourse particles and{\dots} what else?}\label{9:s3}

While we have only considered particles thus far, we also need to consider instances of the so called contrast sluicing, i.e. cases ``where the correlate is a focused definite expression, rather than an indefinite'' (\citealt{vicenteoupsluicing}: 12). We can find contrast sluicing in \ili{English} as well (\citealt{merchant2001}: 36): 

\begin{exe}
\ex \begin{xlist}
\ex She has five cats, but I don't know how many dogs.
\ex	We already know which streets are being repaved, but not which avenues.\vspace{-6pt}
\exi{}[]{\hfill(\citealt{merchant2001}: 36, (81a,d))}
\end{xlist}
\end{exe}

\noindent Cases just like these exist in \ili{Slovenian}, too, and in \ili{Slovenian}, just as in \ili{English}, the wh-phrase and the ``contrast'' can form a complex wh-phrase: 

\begin{exe}
\ex \begin{xlist}
\ex \gll Ima pet mačk, ne vem pa koliko psov.\\
have.\textsc{3sg} five cats not know but how.many dogs\\
\trans `(S)He has five cats, but I don't know how many dogs.'
\ex \gll	Vemo, katere ulice bodo ponovno tlakovane, a ne, katere avenije. \\
		know.\textsc{1pl} which streets \textsc{aux} 	again paved but not which avenues \\
\trans `We know which streets are being repaved, but not which avenues.'
\end{xlist}\end{exe}

\noindent However, the availability of complex wh-phrases in sluicing in \ili{Slovenian}, does not account for instances of discourse particles in sluicing, as already observed in \cite{marusicetal2015}. That is, based on the observations that discourse particles in \ili{Slovenian} (i) can be separated from the wh-word by parentheticals, shown below for \textit{pa} in a wh-question and a \isi{sluice}, (\ref{9:paseparques}) and (\ref{9:paseparsluic}), respectively, (ii) can appear after the auxiliary \isi{clitic}, cf. example (\ref{9:guernicajeze}) and (\ref{9:pressuposedjepa}), which in \ili{Slovenian} does not break syntactic constituents and (iii) that particles cannot appear with unmoved wh-phrases, \cite{marusicetal2015} conclude that in \ili{Slovenian}, discourse particles do not form a constituent with wh-phrases. 

\begin{exe}
\ex \label{9:paseparques}
\gll Kaj, po tvoje, pa kuha?\\
	what after yours \textsc{pa} cook\\
\trans  `What, in your opinion, is he cooking?'
\ex \begin{xlist}
\ex \label{9:paseparsluic}
\gll 	Ana je nekaj pojedla. \\
 		Ana \textsc{aux} something ate\\
 \trans  `Ana ate something.'
\ex \gll Kaj, po tvojem mnenju, pa \sout{je}\sout{\ }\sout{\ }\sout{\ }\sout{\ }\sout{pojedla}?\\
 		what after your opinion \textsc{pa} {\textsc{aux} ate}\\
\trans `What, in your opinion did she eat?'
\end{xlist}\end{exe}
 
\noindent In fact, the same conclusion can be made based on examples that show that the same particle cannot appear after all wh-phrases in multiple sluicing in \ili{Slovenian}. That is, while multiple sluicing by itself is acceptable in \ili{Slovenian} (a multiple wh-fronting language) and while particles can only marginally appear after each of the wh-phrases in multiple sluicing, these have to be different particles (we are not discussing the particle \textit{to} here, but see \citealt{marusicetal2015}); compare (\ref{9:kdopakomupa}) with (\ref{9:kdopakomuto}). Imagine a context in which you lend your glasses to a friend who had a party and the next day, the friend comes by to explain the situation and you demand to know:  

\begin{exe}
\ex \gll 
Na zabavi je nekdo nekomu metal kozarce in jih razbil. \\
on party \textsc{aux} somebody.\textsc{nom} somebody.\textsc{dat} throw glasses and them broke\\
\trans `At the party, somebody threw glasses at somebody and broke them.'

\ex \label{9:kdokomuvsi}\begin{xlist}
\ex[]{\gll Kdo komu?\\
who.\textsc{nom} who.\textsc{dat}\\
\trans `Who (threw the glasses) to whom?'}\label{9:kdokomu}
\ex[*]{\gll Kdo pa komu pa?\\
    	who.\textsc{nom} \textsc{pa} who.\textsc{dat} \textsc{pa}\\} \label{9:kdopakomupa}
\ex[]{\gll Kdo komu pa?\\
	who.\textsc{nom} who.\textsc{dat} \textsc{pa}\\
\trans `(I want to know) Who (threw the glasses) to whom?'}\label{9:kdokomupa}
\ex[]{ \gll Kdo pa komu?\\
	who.\textsc{nom} \textsc{pa} who.\textsc{dat} \\
\trans `(I want to know) Who (threw the glasses) to whom?'}\label{9:kdopakomu}
\ex[?]{\gll  Kdo pa komu to?   \\
 who.\textsc{nom}	\textsc{pa} who.\textsc{dat} \textsc{to}\\
 \trans `(I want to know) Who (threw the glasses) to whom?'}\label{9:kdopakomuto}
\end{xlist}\end{exe}

\noindent If particles would form a constituent with each individual wh-phrase prior to movement, we would expect (\ref{9:kdopakomupa}) to be just as acceptable as (\ref{9:katerikaterega}) in which the \isi{sluice} consists of two complex wh-phrases that only differ in their case features. But as shown, this is not the case. 

\begin{exe}
\ex \begin{xlist}
\ex \gll En slikar je drugega naslikal.\\
one painter \textsc{aux} other 	painted\\
\trans `One painter painted the other one.'	 
\ex \label{9:katerikaterega}
\gll	Kateri slikar katerega slikarja?\\
which painter.\textsc{nom} which painter.\textsc{acc}\\
\trans `Which painter which painter?'
\end{xlist}\end{exe}

\noindent This can then be taken as an additional argument against particles forming a constituent with the wh-phrase and shows that instances of sluicing with  discourse particles are not simply parallel to cases in which a complex wh-phrase survives sluicing. But, crucially, this shows that discourse particles in wh-questions in \ili{Slovenian} are not located within the wh-phrase. 

Furthermore, in \ili{Slovenian} `contrast' sluices are not necessarily complex wh-phrases, but rather consist of a wh-phrase (simplex or complex) and a non-wh-phrase. Even more, this non-wh-phrase can be discourse given, (\ref{9:skomcrt}), or new, (\ref{9:kjekekecmojco}).

\begin{exe}
\ex \begin{xlist}
\ex \gll Srečala sem Vida in Črta. Vid je bil z Ano.\\
met \textsc{aux} Vid.\textsc{acc} and Črt.\textsc{acc}  Vid.\textsc{nom} \textsc{aux} was with Ana.\textsc{ins} \\
\trans `I met Vid and Črt. Vid was accompanied by Ana.' 
\ex \gll In s kom Črt? \\
	and with who Črt.\textsc{nom}\\
\trans  `And Črt was with whom?'
\end{xlist}
\label{9:skomcrt}
\ex \begin{xlist}
\ex \gll	Ne spomni se, kje je Nik spoznal Majo? \\
not remember \textsc{refl} where \textsc{aux} Nik.\textsc{nom} met Maja.\textsc{acc}\\
\trans `(S)He doesn't remember where Nik met Maja.'
\ex \gll 	Ne, kje Kekec Mojco.\\
	no where Kekec.\textsc{nom} Mojca.\textsc{acc}\\
\trans 	 `No, (s)he can't remember where Kekec met Mojca.'
\end{xlist}
\label{9:kjekekecmojco}
\end{exe}

\noindent Based on similar examples, \cite{marusicetal2015} suggest that in sluicing in \ili{Slovenian}, the non-wh-material in the \isi{left periphery} is not elided but we can in turn take it as an indicator that the particles do not have to form a constituent with the wh-phrase in sluicing examples. In the next section, we maintain the analysis from \cite{marusicetal2015} and focus on the position of discourse particles in \ili{Slovenian} wh-questions and in doing so give new arguments for the proposed analysis.  

\section{Position of particles}\label{9:s4}

While particles are well studied in some languages, for example in \ili{German}, particles in wh-questions have not been previously studied in \ili{Slovenian} (at least not within the generative framework). In what follows we focus on the position of particles \textit{že} and \textit{pa} in \ili{Slovenian} wh-questions. Focusing on examples with sluicing we show that the particles are not a part of the \isi{clitic cluster} in \ili{Slovenian}, despite their lack of stress and what at first glance seems to be simply a \isi{clause} \isi{second position}. Furthermore, we take instances of particles in sluicing as evidence that these particles are not a part of the IP.  

\subsection{Discourse particles are not part of the clitic cluster}

Traditionally discourse particles \textit{pa} and \textit{že} are said to be part of the \isi{clitic cluster} in \ili{Slovenian}, specifically, \cite{toporisic2000} places them as the last clitics of the \isi{clitic cluster}. Similarly, \cite{oresnik1985naniz} suggests that at least one variety of the particle \textit{pa} should be seen as part of the \isi{clitic cluster}. \cite{toporisic2000} does not make any distinction between various types of particles \textit{pa} and \textit{že}, he considers all of them comparable to the \isi{negation} \isi{clitic} \textit{ne} and other particles like \textit{še} `more'/`still', \textit{da} `that'/`yes', etc. If particles are part of the \isi{clitic cluster} and if \isi{clitic cluster} is a conglomeration of syntactic heads that is adjoined to the C head (as in \citealt{goldensheppard2000}), we would expect, contrary to fact, that particles would behave like clitics and should thus, just like other clitics within the same cluster, not be possible in sluicing, as shown in  (\ref{9:sluicingclitics}). 

\begin{exe}
\ex \label{9:sluicingclitics}
\gll Ilija mu ga nekje razlaga. Kje že (*\hspace{-2pt} mu ga)?\\
 	Ilija him it somewhere explains where \textsc{že} {} him it\\
 \trans `Ilija is explaining it to him. (Remind me) Where (is Ilija  explaining it to him)?'
\end{exe}

\noindent Given the assumptions explained above and the example (\ref{9:sluicingclitics}) we cannot but conclude that the particles that we observe in sluicing in \ili{Slovenian} must be DP-internal, while the particles that we observe in wh-questions originate from a position inside the IP, as the \isi{complementizer} is the first \isi{clitic} inside the \isi{clitic cluster}. This goes against the findings of \cite{marusicetal2015} and our own conclusions about the nature of these particles in sluicing and wh-questions. Our goal now is thus to show that the ``cluster-final" particles are not truly part of the \isi{clitic cluster} and that additionally, the (mainstream) assumptions about \isi{clitic placement} explained above also need to be (at least partially) revised or discarded.

First, as claimed by \cite{marusic2008clitics}, clitics forming the \isi{clitic cluster} are not adjoined to C as they can easily appear following a word that should be located lower in the \isi{clause} (cf. \citealt{Boskovic2001NatureSyntaxPhonology} for BCS clitics). \cite{oresnik1985naniz} gives another argument against placing the \isi{clitic cluster} in the C head. As he puts it, the \isi{complementizer} should not be seen as a part of the \isi{clitic cluster} as focused phrases can split the \isi{complementizer} from the rest of the \isi{clitic cluster}, as in (\ref{9:oresnikcluster}) taken from \cite{oresnik1985naniz}.

\begin{exe}
\ex \begin{xlist}
\ex \gll {\dots} \{\hspace{-2pt} in / ker / da\} si ga Janez lahko kupi.\\
	{} {} and {} as {} that 	\textsc{refl} it Janez can buy\\
\trans   `{\dots} \{and / as / that\} Janez can buy it.'
\ex \label{9:oresnikcluster}
\gll	{\dots} \{\hspace{-2pt} in / ker / da\} \textsc{Janez} si ga lahko kupi.\\
	{} {} and {} as {} that Janez \textsc{refl} it can buy\\
\trans `{\dots} \{and / as / that\} Janez can buy it.'
\end{xlist}\end{exe}

\noindent If clitics move in overt syntax, than the \isi{clitic cluster} that is apparently not adjoined to C needs to be hosted by a lower head -- a head within IP. So for the particle at the end of the \isi{clitic cluster} that would mean its place of origin should also be somewhere inside the IP, which  further suggests our analysis is simply wrong. We can dismiss this argument saying \ili{Slovenian} clitics do not move in syntax (as suggested by \citealt{marusic2008clitics} and \citealt{marusiczaucer2017brno}) or that at least the \isi{clitic cluster} is not composed in syntax, for which there also seems to be evidence given that the order of clitics inside the cluster is not universal and does not follow any order predicted by the assumed structure (cf. \citealt{marusiclecturenotes}), but let us try and argue against the cluster-internal position of the discourse particles also within the mainstream view on clitics. 

As noted above, the two particles \textit{že} and \textit{pa} can actually appear either before or after the \isi{clitic cluster}, as shown in (\ref{9:pressuposedjepa}) for \textit{pa} and in (\ref{9:guernicajeze}) for \textit{že}, and in (\ref{9:zepaclitics}) for both. Given that all other clitics forming the \isi{clitic cluster} have a fixed word-order (with some variation in the order of \isi{dative} and \isi{accusative} clitics), we can conclude that the two clitics are not part of the \isi{clitic cluster} but appear either cluster-initially or cluster-finally by accident.

\begin{exe}
\ex \label{9:zepaclitics}
\begin{xlist}
\ex \gll 
Koga \{\hspace{-2pt} že mu je / mu je že\} Ilija predstavil?\\
who {} \textsc{že} him \textsc{aux} {} him \textsc{aux} \textsc{že} Ilija introduced\\
\trans 	 `(Remind me) who did Ilija introduce to him?'
\ex	\gll Kaj \{\hspace{-2pt} pa mu je / mu je pa\} Žodor narisal?\\
 		what {} \textsc{pa} him \textsc{aux} {} him \textsc{aux} \textsc{pa} žodor drew\\
\trans 	 `What did Žodor draw for him?'
\end{xlist}
\end{exe}

\noindent Another argument given above to show these particles do not form a constituent with the wh-word can be turned around. As shown in (\ref{9:paseparques}) repeated here as (\ref{9:kdopotvojepak}), \textit{pa} can follow the parenthetical `in your opinion', but note that \textit{pa} can also precede the parenthetical and appear on the other side of the parenthetical separated from the rest of the \isi{clitic cluster}, (\ref{9:kdopapotvoje}). This suggests \textit{pa} is an element independent from the \isi{clitic cluster} that is located structurally higher than the final position of the \isi{clitic cluster}.

\begin{exe}
\ex \label{9:kdopotvojepak} \gll 
 Kaj, po tvoje, pa kuha?\\
	What after yours	 \textsc{pa} cook\\
	\trans `What, in your opinion, is he cooking?'
\ex \label{9:kdopapotvoje}
 \gll Kdo pa, po tvojem mnenju, jih je komu metal?\\
	who \textsc{pa} after your opinion them \textsc{aux} who.\textsc{dat} threw\\
	\trans  `Who, in your opinion, threw them for whom?
\end{exe}

\noindent Further, in some cases, \textit{pa} and \textit{že} can appear also inside the complex wh-phrase as in (\ref{9:kajzedobrega}) and (\ref{9:kdopaodpetrovih}). Note that these examples do not constitute an argument for a wh-phrase-internal position of these discourse particles, as argued by \cite{marusicetal2015}, but they do suggest that these discourse particles are different syntactic elements from the clitics forming the \isi{clitic cluster}.

\begin{exe}
\ex \label{9:kajzedobrega}
\gll Kaj že dobrega je Ana 	pojedla? \\
 	what \textsc{že} good \textsc{aux} Ana	ate\\
\trans  `(Remind me) What was it that Ana ate that was good?'
\ex \label{9:kdopaodpetrovih}
\gll Kdo pa od Petrovih prijateljev je prišel?\\
 	who \textsc{pa} of Peter's friends \textsc{aux} came\\
\glt  `Who of Peter's friends was it that came?'\\
\hfill\citep[(38)]{marusicetal2015}
\end{exe}

\noindent And finally, clitics in \ili{Slovenian} typically follow the first wh-phrase of a multiple wh-question, (\ref{9:whclitics}), while discourse particles can follow the first or second wh-phrase in a multiple wh-question with two wh-phrases, as examples in (\ref{9:kdokomuvsi}) show.  

\begin{exe}
\ex  \label{9:whclitics}
 \gll Kdo \{\hspace{-2pt} jih je komu / *\hspace{-2pt} komu jih je\} metal?\\
   who.\textsc{nom} {} them \textsc{aux} who.\textsc{dat} {} {} who.\textsc{dat} them \textsc{aux} threw \\
   \glt `Who threw them to whom?'
\end{exe}

\largerpage
\noindent Given all that, regardless of our assumptions about clitics and the way \isi{clitic cluster} is formed, discourse particles are syntactic elements that behave differently from clitics, so that we have no argument to posit they originate from the same region of the \isi{clause} or that their surface position is in any way dependent on the surface position of the other clitics. Discourse particles and clitics behave differenlty in wh-questions, thus it is not unexpected that they behave differently also in sluicing.\footnote{An anonymous reviewer suggested our data are fully compatible with a view where the only relevant criterium for \isi{clitic cluster} formation is PF adjacency. If we further assume pronominal and auxiliary clitics are IP clitics whereas discourse particles are CP clitics (as they are located in the \isi{left periphery} -- in the CP area), then IP clitics and CP clitics would have been adjacent at PF in the absence of sluicing, but they would have never been syntactically adjacent or part of the same complex head. And when sluicing would elide the IP, IP clitics would get deleted whereas CP clitics would survive.} 


\subsection{Position of particles with respect to adverbs}

An argument for the analysis that places discourse particles in the \isi{left periphery} of a wh-question (and a \isi{sluice}) comes from the behavior of adverbs. Specifically, the incompatibility of high sentential adverbs and sluicing in \ili{Slovenian}. 
There are several suggestions with respect to the position of discourse particles. \cite{zimmermann2011} proposes that, perhaps universally, discourse particles tend to be realized in the periphery of the \isi{clause}, but that some languages, such as \ili{German}, should be exempt from this (i.e. in \ili{German} discourse particles do not occur in the periphery but rather in the middlefield because they do not bare stress and unstressed elements cannot appear in the prefield in \ili{German}).\footnote{\cite{ott2016deletion}, assuming that particles in \ili{German} are located outside the vP, above sentential adverbs and \isi{negation}, argue for a phonological approach to \isi{ellipsis} in which material, which is in the background, is elided. This approach does not necessarily require movement. Crucially, \cite{ott2016deletion} show that in \ili{German} sentential adverbs can appear in clausal \isi{ellipsis}, contrary to \ili{Slovenian}. This implies that while cases of sluicing with particles in \ili{Slovenian} and \ili{German} seem similar at first glance, the two are in fact different.
\ea \textit{Context:} `Peter seems to have invited some people.'\\
\gll Und 	\textsc{wen} \{\hspace{-2pt} vermutlich / 	wahrscheinlich / anscheinend\}?\\
and 	who {} presumably {}	probably {}	apparently\\
\glt `And who did he \{presumably / probably / apparently\} invite?'\\\hfill (\citealt{ott2016deletion}: (15b))
\zlast} Facts from sluicing in \ili{Slovenian} in fact suggest that discourse particles do appear higher than high sentential adverbs.


\largerpage
Specifically, high sentential adverbs in \citeposst{cinque1999} hierarchy of adverbs express speakers' attitude and are in this respect similar to discourse particles which express speakers' attitude towards the utterance \citep{zimmermann2011}. However, while particles can appear in sluicing in \ili{Slovenian}, high sentential adverbs cannot. This is shown below for the \isi{adverb} \textit{menda} `allegedly' (but the same is true for \textit{baje} in non-standard varieties of \ili{Slovenian}) -- a relatively high \isi{adverb} that is compatible with wh-questions (that is, while seemingly higher adverbs such as \textit{iskreno} `frankly' can appear in wh-questions, they only receive subject oriented reading).\largerpage

\begin{exe}
\ex \gll Kdo je menda plesal tango?\\
         who \textsc{aux} allegedly danced tango\\
 \trans `Who allegedly danced tango?'
\ex \label{9:mendaplesaltango}
\begin{xlist}
 \ex \gll	Kdo že je menda plesal tango?\\
 who \textsc{že} \textsc{aux} allegedly danced tango\\
\trans `(Remind me) Who allegedly danced tango?'
\ex \gll	Kdo je že menda plesal tango?\\
 who \textsc{aux} \textsc{že}  allegedly danced tango\\
 \glt Available: `(Remind me) Who allegedly danced tango?'
 \glt Available: `Who allegedly already danced tango?'
\ex \gll	Kdo je menda \textit{že} plesal tango?\\
 who \textsc{aux} allegedly \textsc{že} danced tango\\
\trans `Who allegedly already danced tango?'
\end{xlist}
\ex \textit{Context:} `I've heard that there are some people here who danced tango.'
\begin{xlist}
\ex[]{\gll Kdo že?\\
who \textsc{že}\\
\glt `(Remind me) Who?'}
\ex[*]{\gll Kdo menda?\\
who allegedly\\
\glt Intended: `Who, allegedly?'}
\end{xlist}
\end{exe}

\noindent First, the examples in (\ref{9:mendaplesaltango}) indicate that discourse particles precede high sentential adverbs in wh-questions in \ili{Slovenian}, since \textit{že} only gets the \isi{aspectual} reading when it follows an \isi{adverb} such as \textit{menda} `allegedly'. More importantly, high sentential adverbs cannot appear in sluices in \ili{Slovenian}, indicating that the material in the IP is elided.\footnote{The apparent exception are contrastively focused adverbs as example \REF{9:fn16exi} shows:

\ea \label{9:fn16exi}
\begin{xlist}
\exi{A:}[]{`I know Kekec danced for sure.'}
\exi{B:}[]{\gll In kdo \textsc{menda}?\\
and who allegedly\\
\glt `And who (danced) allegedly?'}
\end{xlist}
\zlast} And since particles can appear in sluicing, this suggests that discourse particles in wh-questions in \ili{Slovenian} are located above the IP.

\section{Conclusion}\label{9:s5}
Discourse particles in wh-questions in \ili{Slovenian} have not been previously studied in \ili{Slovenian} within the generative framework. In this paper we take instances of sluicing in which discourse particles \textit{pa} and \textit{že} appear as a starting point to explore discourse particles in wh-questions (and consequently sluicing) in \ili{Slovenian}. We consider cases with \textit{že} and \textit{pa} in wh-questions and sluicing to show that discourse particles in \ili{Slovenian} are not in complex wh-phrases nor are they a part of the \isi{clitic cluster} or the IP. In fact, all of the properties we explore in this paper can be captured under the analysis proposed in \cite{marusicetal2015}, i.e. an analysis according to which discourse particles are located in the \isi{left periphery}. Under this approach the projections hosting wh-phrases are not the only projections surviving sluicing in \ili{Slovenian}, but rather what survives sluicing is a larger portion of the \isi{left periphery}, hence also the grammaticality of topic and focus phrases in sluicing in \ili{Slovenian}.\largerpage[2]

A natural question that follows (also pointed out by one of the anonymous reviewers) is why particles can survive IP-deletion in the \isi{left periphery}, while auxiliaries like \textit{did} and \textit{do,} which end up in the \isi{left periphery} following T-to-C movement, do not. The elements that we observe survive sluicing in the \isi{left periphery} all originate from within the \isi{left periphery}, while \ili{English} auxiliaries do not; they are moved to the \isi{left periphery} via T-to-C movement. One option to resolve this question is to simply state that the deletion of the IP in sluicing precedes T-to-C movement, as a result of which the auxiliaries never even reach the C head, where it could survive sluicing. As T-to-C movement is an instance of head-movement and as head-movement is occasionally argued to be an instance of PF movement, it actually follows quite naturally that elements like \textit{did} cannot survive sluicing, as they do not occupy a left-peripheral position at the time when the IP is deleted.


\section*{Abbreviations}
\begin{tabularx}{.5\textwidth}{@{}lQ@{}}
\textsc{1}&1st person\\
\textsc{3}&3rd person\\
\textsc{acc}&{accusative}\\
\textsc{aux}&auxiliary verb\\
\textsc{cond}&conditional auxiliary\\
\textsc{dat}&{dative}\\
\textsc{gen}&{genitive}\\
\textsc{inf}&{infinitive}\\
\end{tabularx}%
\begin{tabularx}{.5\textwidth}{@{}lQ@{}}
\textsc{ins}&{instrumental}\\
\textsc{neg}&{negation}\\
\textsc{nom}&{nominative}\\
\textsc{pl}&{plural}\\
\textsc{ptcl}&{discourse particle}\\
\textsc{q}&question marker\\
\textsc{refl}&reflexive {pronoun}\\
\textsc{sg}&singular\\
&\\
\end{tabularx}


\section*{Acknowledgements}
We are grateful to the editors and two anonymous reviewers of this volume for comments and suggestions. We acknowledge the financial support of ARRS Program P6-0382 (PI: Marušič).

\sloppy
\printbibliography[heading=subbibliography,notkeyword=this]

\end{document}
