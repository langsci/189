\documentclass[output=paper,modfontsnewtxmath,hidelinks]{langscibook} 
\ChapterDOI{10.5281/zenodo.2545521}

\title{On the lack of φ-feature resolution in DP coordinations: Evidence from Czech}

\author{ Ivona Kučerová\affiliation{McMaster University}}

\abstract{The paper investigates a feature valuation in the context of more than one accessible goal. Concretely, the paper provides novel empirical evidence that there is no φ-feature resolution in syntactic agree. The apparent feature resolution of \textsc{gender} and \textsc{number} agreement previously reported in the Slavic literature on agreement with coordinated DPs is a side-effect of morphological realization of \textsc{person} feature that arises at the syntax--semantics interface. Furthermore, the proposal suggests that even non-default overt morphological marking of agreement might not faithfully reflect the narrow-syntax feature valuation, a result which seriously questions the validity of some core generalizations about agreement properties of natural languages. The core data comes from the agreement with coordinated noun phrases in Czech.

\keywords{agree, multiple agree, feature resolution, pronouns, copular clauses, Czech}
}

\begin{document}
\qtreecenterfalse
\shorttitlerunninghead{On the lack of φ-feature resolution in DP coordinations}
\maketitle

\section{Introduction}

The Minimalist Program \citep{Chomsky1995} shifted the focus of the syntactic investigation from lexical categories to their feature composition, which in turn yielded a growing interest in relations among syntactic features themselves, specifically, the notion of agree (\citealt{Chomsky2000}; \citealt{Chomsky2001}; among others). More recently the debate has increasingly concentrated on the status of valued and unvalued features \citep{pestorrego07} and the notion of feature valuation in and of itself. This paper addresses the question of whether syntactic agree can only copy and share existing values of features, or whether narrow syntax can derive new values of syntactic features.

The question does not directly arise in the work that investigates structures with a single accessible goal. There the focus is on the distinction of matching and valuation (\citealt{bejarrezac03}, \citealt{pestorrego07}) and the question of infallibility of these operations (e.g., the notion of failed agree in \citealt{Preminger2009}). The question becomes more intricate in the domain of investigation of syntactic structures with more than one accessible goal. While for the work on Multiple-Agree (Hiraiwa 2005), it is critical that feature values within the same agree link \emph{must match}, the literature on agreement with coordinated DPs works instead with the assumption that narrow syntax \emph{may derive} new values by combining conflicting feature values within an agree link (\citealt{Farkas1995,King2004,Heycock2005,Marusic2015}).\footnote{The existing approaches to agreement with coordinations range from strictly morpho-syntactic, as in \citet{Marusic2015}, to strictly semantic, as in \citet{Lasersohn1995}. A majority of the current approaches combines both morpho-syntactic and semantic derivation, as pioneered in \citet{Farkas1995}.}

The paper provides novel empirical evidence that there is no φ-feature resolution in syntactic agree. The apparent feature resolution of \textsc{gender} and \textsc{number} agreement previously reported in the literature is a side-effect of morphological realization of \textsc{person} feature that arises at the syntax-semantics interface. Furthermore, the proposal suggests that even non-default overt morphological marking of agreement might not faithfully reflect the narrow-syntax feature valuation, a result which seriously questions the validity of some core generalizations about agreement properties of natural languages. The core data comes from the agreement with coordinated noun phrases in Czech.


\section{Feature resolution in the Czech agreement system}
\label{sec:featureResolution}

Standard Czech\footnote{I use the label Standard Czech for a non-vernacular variety of an interdialect shared by most native speakers of Czech and based on the modern codified standard of the Czech language.} distinguishes three grammatical genders, i.e, masculine (\textsc{m}), feminine (\textsc{f}), neuter (\textsc{n}), and two grammatical numbers, i.e., singular \textsc{sg}, plural \textsc{pl}. In addition, masculine gender is marked for animacy, i.e., there is a specialized case and agreement marking for animate (\textsc{ma}) and inanimate (\textsc{mi}) masculine nouns and the elements that morphosyntactically agree with them. While the ultimately four-way distinction is fully preserved in singular agreement and case marking, there is a partial syncretism in plural. The system distinctly marks neuter plural and masculine animate plural but collapses the distinction between masculine inanimate and feminine.\footnote{The fact that feminine is collapsed with masculine inanimate in and of itself provides a strong indication that animacy plays no role in the syntactic construal of the feminine value of the gender feature.}$^,$\footnote{The syncretism pattern plays out somewhat differently in dialects, see, e.g., \citet[392--404]{KarlikEtAl:2002}, for morphological features that distingues Bohemian dialects from their Moravian counterparts (Central and Eastern Moravian). Discussing the dialectal variation goes beyond the scope of this paper but a preliminary exploration is attempted in section 5.} The richness of the morphological marking thus lends itself easily to investigating agreement with coordinated noun phrases.


According to the existing grammatical descriptions (e.g., \citealt{Panevova1997}), if nominal conjuncts differ in their φ-features, the agreement with both conjuncts is resolved along a markedness hierarchy, sensitive to animacy and gender marking.\footnote{Czech allows both first-conjunct agreement and agreement with both conjuncts. For now I leave the first-conjunct agreement pattern aside as it does not directly inform the empirical description of the feature resolution.} Thus, animate masculine is the most marked feature, with masculine inanimate and feminine ranked over neuter. This means that if one of the conjuncts is masculine animate, the plural agreement is going to be masculine animate, as shown in \REF{baseline-anim}.\footnote{Data with simple agreeement patterns are based on my native speaker intituitions and existing grammar descriptions (primarily, \citealt{Panevova1997,corbett83}). Data testing for combinations of features are based on elicitation of grammatical judgements from 4--6 native speakers.} If there is no masculine animate noun but one of the conjuncts is masculine inanimate or feminine, the plural agreement is the syncretic masculine inanimate/feminine agreement, as shown in \REF{baseline-inanim}. The order of the conjuncts does not affect the agreement pattern.\footnote{The (b) orders tend to be judged as less natural, a fact related to the asymmetric nature of coordinated noun phrases (see, e.g., \citealt{Johannessen1996}), unless the ordering becomes relevant.} For simplicity of the presentation, I refer to the former agreement pattern as \textsc{animate agreement} and the latter one as \textsc{gender agreement}. 

\ea \textbf{Feature-resolution markedness}\\
animacy (\textsc{ma}) $\succ$ gender ({\textsc{mi}/\textsc{f}}) $\succ$ neuter (\textsc{n})
\z

\ea\label{baseline-anim} \textbf{animacy (\textsc{ma}) $\succ$ gender ({\textsc{mi}/\textsc{f}})/neuter (\textsc{n})}
\ea\gll \{\hspace{-2pt} Kočka / kotě / dobytek\} a pes jedli ze stejné misky.\\
{} cat.\textsc{f.sg} {} kitten.\textsc{n.sg} {} cattle.\textsc{mi.sg} and dog.\textsc{\textbf{ma}} ate.\textsc{pp.\textbf{ma}.pl} from same bowl\\
\glt `The cat/kitten/cattle and the dog ate from the same bowl.'\\\hfill \fbox{\textsc{f/n/mi} + \textbf{\textsc{ma}} = \textbf{\textsc{ma (animate)}}}\smallskip
\ex\gll Pes a \{\hspace{-2pt} kočka / kotě / dobytek\} jedli ze stejné misky.\\
dog.\textsc{\textbf{ma}} and {} cat.\textsc{f.sg} {} kitten.\textsc{n.sg} {} cattle.\textsc{mi.sg} ate.\textsc{pp.\textbf{ma}.pl} from same bowl\\
\glt `The dog and the cat/kitten ate from the same bowl.'\\\hfill \fbox{\textsc{f/n/mi} + \textbf{\textsc{ma}} = \textbf{\textsc{ma (animate)}}}
\z\z

\ea\label{baseline-inanim} \textbf{gender ({\textsc{mi}/\textsc{f}}) $\succ$ neuter (\textsc{n})}
\ea\gll Kotě a \{\hspace{-2pt} kočka / dobytek\} jedly ze stejné misky.\\
kitten.\textsc{n.sg} and {} cat.\textsc{\textbf{f}.sg} {} cattle.\textsc{\textbf{mi}.sg} ate.\textsc{pp.\textbf{\{mi/f\}}.pl} from same bowl\\
\glt `The kitten and the cat/cattle ate from the same bowl.'\\\hfill \fbox{\textsc{n} + \textbf{\textsc{mi/f}} = \textbf{\textsc{\{mi/f\} (gender)}}}\smallskip
\ex\gll \{\hspace{-2pt} Kočka / dobytek\} a kotě jedly ze stejné misky.\\
{} cat.\textsc{\textbf{f}.sg} {} cattle.\textsc{\textbf{mi}.sg} and kitten.\textsc{n.sg} ate.\textsc{pp.\textbf{\{mi/f\}}.pl} from same bowl\\
\glt `The cat/cattle and the kitten ate from the same bowl.'\\\hfill \fbox{\textsc{n} + \textbf{\textsc{mi/f}} = \textbf{\textsc{\{mi/f\} (gender)}}}
\z\z

\noindent Upon a closer examination the markedness behaviour is rather puzzling. In other domains that involve a feature resolution along the markedness hierarchy, if there is a conflict, the system resorts to the less marked feature. This is not the case here. Not only does the masculine animate systematically emerge as the winner even though in other domains it is morphologically the most marked feature, neuter that in other environment behaves as the morphologically least marked feature, e.g., the feature used in failed-agree environments with no syntactic probe, as in \REF{def-neuter}, never survives in coordination agreement patterns.

\ea\label{def-neuter}
\ea\gll Pršelo.\\
rained.\textsc{pp.n.sg}\\
\glt `It rained.'
\ex\gll Že Petr nepřišel, nebylo dobré.\\
that Peter \textsc{neg}.came \textsc{neg}.was.\textsc{pp.n.sg} good.\textsc{n.sg}\\
\glt `That Peter didn't came wasn't good.'
\z\z

\noindent One could argue that neuter cannot participate in a syntactic resolution because it is in some sense defective. Such a conclusion goes in line with the following observation. Not only does neuter never win in a combination with other gender values, but neuter-plural agreement arises only if both conjuncts are in neuter plural, as shown in \REF{neuter}. If either or both of the conjuncts are in neuter singular, the plural agreement cannot be in neuter plural, despite the fact there is a dedicated neuter plural agreement morphology. Instead, the agreement is the syncretic gender agreement. 

\ea\label{neuter}
\ea\gll Kotě a štěně \{\hspace{-2pt} jedly / *\hspace{-2pt} jedla\} ze stejné misky.\label{n-a}\\
kitten.\textsc{n.sg} and puppy.\textsc{n.sg} {} ate.\textsc{pp.\{mi/f\}.pl} {} {} \textsc{pp.n.pl} from same bowl\\
\glt `The kitten and the puppy  ate from the same bowl.'\\\hfill \fbox{\textsc{n.sg} + \textsc{n.sg} = \textbf{\textsc{\{mi/f\} (gender)}}}\smallskip
\ex\gll Koťata a štěně \{\hspace{-2pt} jedly / *\hspace{-2pt} jedla\} ze stejné misky.\label{n-b}\\
kittens.\textsc{n.pl} and puppy.\textsc{n.sg} {} ate.\textsc{pp.\{mi/f\}.pl} {} {} \textsc{pp.n.pl} from same bowl\\
\glt `The kittens and the puppy  ate from the same bowl.'\\\hfill \fbox{\textsc{n.pl} + \textsc{n.sg} = \textbf{\textsc{\{mi/f\} (gender)}}}\smallskip
\ex\gll Koťata a štěňata jedla ze stejné misky.\label{n-c}\\
kittens.\textsc{n.pl} and puppies.\textsc{n.pl} ate.\textsc{pp.n.pl} from same bowl\label{n-pl}\\
\glt `The kittens and the puppies  ate from the same bowl.'\\\hfill \fbox{\textsc{n.pl} + \textsc{n.pl} = \textbf{\textsc{n.pl}}}
\z\z

\noindent The question of why animate masculine should behave as if it were less marked than inanimate masculine remains. The overall agreement-resolution pattern is summarized in \tabref{PPtable}.\footnote{For \citet{Panevova1997}, the plural agreement for the first conjunct being \textsc{mi} is \textsc{mi}. Since there is no empirical evidence that \textsc{mi.pl} and \textsc{f.pl} are distinct, I use the descriptive \textsc{mi/f} label instead. The same for the first conjunct being \textsc{f}.}

\begin{table}
% \captionsetup{format=hang}
\caption{Agreement-resolution patterns (adapted from \citealt{Panevova1997})}
\label{PPtable}
\begin{tabularx}{0.8\textwidth}{llX}
\lsptoprule
\textbf{1st conjunct} & \textbf{2nd conjunct} & \textbf{plural agreement} \\\midrule
\textsc{ma} & $\alpha$ & \textsc{ma}, where $\alpha \in$ \{\textsc{ma, mi, f, n}\}\\
\textsc{mi} & $\alpha$ & \textsc{mi/f}, where $\alpha \in$ \{\textsc{mi, f, n}\}\\
\textsc{f} & $\alpha$ & \textsc{mi/f}, where $\alpha \in$ \{\textsc{f, n}\}\\
\textsc{n.sg} & \textsc{n.sg} & \textsc{mi/f}\\
\textsc{n.sg} & \textsc{n.pl} & \textsc{mi/f}\\
\textsc{n.pl} & \textsc{n.pl} & \textsc{n}\\\lspbottomrule
\end{tabularx}
\end{table}

\section{The puzzle: Different probe = different feature resolution}\label{sec:puzzle}

One could dismiss the emergence of the masculine animate plural agreement as insignificant, if it was not for an additional and much more serious empirical problem. The generalization reported in the literature are strictly based on examples in the past tense. The past tense in Czech is morphologically realized by a finite auxiliary that agrees in person and number and is null for 3rd person and a past participle that agrees in number and gender with a structural subject in nominative case. Strikingly, the feature-resolution generalization reported in the previous section does not extend to other constructions in which we see plural agreement in gender, i.e., agreement with adjectival predicates and passive participles. 

Agreement with adjectival predicates and passive participles plays out rather differently. As it turns out, if the gender features on conjuncts do not match, plural agreement is fully grammatical only if one conjunct is masculine animate and the other conjunct is grammatically feminine but may be semantically construed as animate, as in \REF{anim-2}.\footnote{The consistency of masculine animate agreement in these patterns have been confirmed in \citet{Adam:2017}, a large scale ($N={}$103) elicitation study testing some of the data from an unpublished version of this paper. Adam tested only animate coordinations and confirmed that whenever one of the conjuncts is masculine animate, plural agreement is masculine animate, irrespective of the order of the conjuncts.}


\ea\label{anim-2}
\ea\gll Petr a Pavla byli unavení.\\
Petr.\textsc{ma.sg} and Pavla.\textsc{f.sg} were.\textsc{pp.ma.pl} tired.\textsc{pp.ma.pl}\\
\glt `Petr and Pavla were tired.'\smallskip
\ex\gll  Pes a kočka byli unavení.\\
dog.\textsc{ma.sg} and cat.\textsc{f.sg} were.\textsc{pp.ma.pl} tired.\textsc{pp.ma.pl}\\
\glt `The dog and the cat were tired.'\\
\z
\hfill \fbox{\textsc{ma} + \textsc{f} = \textbf{\textsc{ma (animate)}}}
\z

\noindent If there is no masculine animate gender, grammatically inanimate gender combinations are strongly degraded even if they semantically denote animate objects. When the coordination contains an inanimate masculine noun and a neuter, the expected agreement, i.e., masculine inanimate (syncretic with feminine plural), is stongly degraded, \REF{inam}. Speakers I tested strongly preferred colloquial morphology (Common Czech), which is completely syncretic in plural, i.e., no gender or animacy distinction is preserved in the system (e.g., \citealt[76]{KarlikEtAl:2002}), (\ref{colloq1}).

\ea[??]{\gll \hspace{-2pt} Dobytek a kotě byly unavené.\label{inam}\\
{} cattle.\textsc{mi.sg} and kitten.\textsc{n.sg} were.\textsc{pp.f/mi.pl}\\
\glt `The cattle and the kitten were tired.'\hfill 
\fbox{\textsc{mi} + \textsc{n} = \textbf{??\textsc{\{mi/f\} (gender)}}}}
\z

\ea[]{\gll Dobytek a kotě byly unavený.\label{colloq1}\\
cattle.\textsc{mi.sg} and kitten.\textsc{n.sg} were.\textsc{pp.pl} tired.\textsc{pp.pl}\\
\glt `The cattle and the kitten were tired.'}
\z

As for the combination of feminine and neuter, no agreement pattern is fully acceptable either. In a forced written elicitation task reported in \citet{Adam:2017}, speakers volunteered feminine plural (62\%), i.e., the prescriptively required agreement, neuter plural (about 26\%), i.e., syncretic plural in some dialects, or colloquial morphology (12\%), i.e., fully syncretic agreement, (\ref{inam-fem}). In my original data collection which was based on a spoken elicitation and a grammatical judgement task, speakers found the colloquial ending most acceptable.\footnote{\citeauthor{Adam:2017}'s study was based on data reported in the 2017 manuscript version of this paper. The judgements reported here thus reflect her finding. \citeauthor{Adam:2017} didn't test any of the other feature combinations as her focus was on animate agreement and agreement with numerals.}

\ea[??]{\gll \hspace{-2pt} Kočka a kotě byly unavené / unavená / unavený.\label{inam-fem}\\
{} cat.\textsc{f.sg}  and kitten.\textsc{n.sg} were.\textsc{pp.f.pl} tired.\textsc{pp.f.pl} {} tired.\textsc{pp.n.pl} {} tired.\textsc{pp.colloq-pl}\\
\glt `The cat and the kitten were tired.'\hfill 
\fbox{\textsc{f} + \textsc{n} = \textbf{??\textsc{f/n/colloq (gender)}}}}
\z




\noindent Strikingly, when speakers are inquired about a combination of masculine animate and neuter gender, irrespective of the number of the conjuncts, as in \REF{gaps}, they try to avoid the agreement altogether. The switch to the fully syncretic colloquial morphology improves the ratings but not as well as in \REF{colloq1}. I label this class of avoidant judgements as agreement gaps and mark them with $\circledast$.

\ea\label{gaps}
\ea[$\circledast$]{\gll Pes a kotě byli ??\{\hspace{-2pt} unavené / unavení / unavená\}.\label{gap1}\\
dog.\textsc{ma.sg} and kitten.\textsc{n.sg} were.\textsc{ma.pl} {} tired.\textsc{pp.mi/f.pl} {} \textsc{pp.ma.pl} {} \textsc{pp.n.pl}\\
\glt Intended: `The dog and the kitten were tired.' \hfill \fbox{\textsc{ma.sg} + \textsc{n} = $\circledast$}}\smallskip
\ex[$\circledast$]{\gll Psi a koťata byli ??\{\hspace{-2pt} unavené / unavení / unavená\}.\label{gap2}\\
dogs.\textsc{ma.pl} and kitten.\textsc{n.sg} were.\textsc{ma.pl} {} tired.\textsc{pp.mi/f.pl} {} \textsc{pp.ma.pl} {} \textsc{pp.n.pl}\\
\glt Intended: `The dogs and the kittens were tired.' \hfill \fbox{\textsc{ma.pl} + \textsc{n.pl} = $\circledast$}}
\z\z

\noindent Recall that the past-tense pattern was not fully syncretic, yet the feature resolution was always possible.\footnote{One could argue that the difference between the animate ending \textit{-i} and the gender-plural ending \textit{-y} is no longer preserved in modern Czech as the original phonological distinction does not exist anymore, i.e., the corresponding past tense forms are homonyms. Yet, the neuter plural ending is clearly distinct which makes a syncretic explanation untenable.}
 Furthermore, if indeed some form of morphological syncretism is in place, then the resolution pattern cannot be attributed to a narrow-syntax valuation as suggested in the existing literature. 
 
To summarize, the fact that the gender-resolution pattern does not extend to the predicative-adjective and passive agreement shows clearly that whatever the process behind the seeming feature resolution is, it cannot be a result of narrow-syntax-feature valuation as part of agree with more than one accessible goal. Next section proposes a theoretical alternative.


\section{You are what you probe}

If mismatched gender features on conjuncts were syntactically resolved within a conjunction phrase (ConjP), agreement with such a phrase should always realize the same features. As we have seen in the previous section, this prediction is not borne out. I argue that instead the resolution pattern depends on the unvalued features of the probe. In order to account for the data I propose the following generalization: If the value of the gender feature on the first conjunct and the value of the gender feature on the second conjunct do not match, feature resolution depends on whether the probe probes (a) only for gender (and number), or (b) whether it probes for person. If the probe (here, verbal predicate, including the past tense formation) probes for a valued person feature, we observe a resolution along an animacy scale. We saw this pattern in \sectref{sec:featureResolution}. If, however, the probe probes only for a valued gender feature, i.e., there is no unvalued person feature on the probe, feature resolution is severely limited and may even yield agreement gaps. This is the pattern we saw in \sectref{sec:puzzle}. The question that arises is why the apparent feature resolution plays out differently for different probes.

We already concluded that a gender-feature resolution as part of narrow-syn\-tax-agree valuation cannot be the answer. In order to understand the pattern we need to turn to the question of how the label of a conjunction phrase and its corresponding features are determined. I.e., the proposed analysis will implement two factors: unvalued features on the probe and the feature composition of the label of the conjunction phrase. 

In a nutshell, I argue that the label of the conjunction phrase is determined in the syntax-semantics interface, and the labelling process is analogical to the feature resolution attested in split-antecedent pronouns (\citealt{Heim2008}, \citealt{Sudo2012}), i.e., plural pronouns that simultaneously refer to more than one antecedent (e.g., \textit{you} and \textit{I} gives \textit{we}), provides an explicit algorithm for how the features of the referring antecedent are computed from a mixed-feature input. In the present proposal, the actual agreement is then modelled as a narrow-syntax agree that targets the conjunction-phrase label as the syntactic representation of the conjunction phrase, where label is a syntactic representation of all features present in the corresponding extended projection and relevant for next syntactic building. There are three components: First, agree is successful only if the label provides features that match the features of the probe. Second, following \citet{Sudo2012}, I assume that the syntax-semantics interface manipulates semantic indices (i.e. numerical pointers). Crucially, indices are complex structures, enriched by person, gender, and number information. Third,  this complex-index information can be mapped onto morphology. Fourth, morphology can only realize features uniquely determined and valued by the label of the probe. The consequence is that if agree probes for person, agreement reflects the complex features of the indices. If agree probes for gender, it can only use gender features available to the narrow syntax component. In other words, while semantics can build new objects (complex indices), syntax can only copy existing values of features. Consequently, if agree probes for person, it can used the complex structures built by the syntax-semantics interface. If agree probes for gender, it may only use features already present in narrow syntax.

\subsection{Features of the conjunction-phrase label}

The idea that there is a connection between agreement with coordinated noun phrases and features of split-antecedent pronouns is intellectually indebted to \citet{Farkas1995} that proposed a striking generalization, namely, that the morphological features on the predicate agreeing with a nominal conjunction are always identical to the morphological features of a pronoun anaphorically referring to the same coordination. To implement this idea I follow \citet{Heim2008} in her treatment of split-antecedent pronouns and \citet{Sudo2012} in his treatment of complex indices underlying the morphological representation of split-antecedent pronouns. 

Furthermore, I follow \citet{Narita2011} and \citet{Chomsky2013} in that labelling is a process triggered by the semantic interface (CI) and argue that the person feature is crucial in the labelling process in that it provides a formal connection between narrow syntax (person as a syntactic feature) and semantic representation (person mapped on a (referential) index; \citealt{Longobardi2008}, \citealt{Sudo2012}, \citealt{Landau2010}, among others). The connection arises via implementing the person feature as [$\pm$\textsc{participant}] (\cite{Nevins2007} and the literature cited there). Furthermore, I follow the literature on coordination that argues that the plurality of a nominal conjunction is computed as semantic plurality (\citealt{Munn1993}, \citealt{Bošković2009}, \citealt{bhattwalkow13}). Technically, I implement a semantic plurality as a conjunction of person features, more precisely, semantic plurality is a conjunction of non-matching indices based on the person feature. 

For concreteness, I assume the person-feature hierarchy and its morphological mapping as exemplified in \figref{hierarchy}. 
Note that the implementation via the [$\pm$\textsc{participant}] feature lends itself easily to accounting for the intrinsic marking of animacy that is critical for the empirical pattern at hand. Next subsection provides a detailed derivation of the attested patterns.


\begin{figure}
% \captionsetup{format=hang}
\caption{Feature hierarchy \& morphological mapping (modelled after \citealt{harleyritter02} and \citealt{Bartosova2015})\label{hierarchy}}
\begin{forest}
[±\textsc{person}
    [$+$\textsc{person} $\sim\pm$\textsc{participant}
        [$+$\textsc{participant} $\sim\pm$\textsc{speaker} 
            [$+$\textsc{speaker}\\{\Large\texttt{⇊}}\\1\textsuperscript{st}]
            [$\pm$\textsc{hearer}
                [$+$\textsc{hearer}\\{\Large\texttt{⇊}}\\2\textsuperscript{nd}] [$-$\textsc{hearer}\\{\Large\texttt{⇊}}\\\textsc{ma}]
            ]
        ] [$-$\textsc{participant} $\sim\pm$\textsc{gender} 
            [ $+$\textsc{gender}\\{\Large\texttt{⇊}}\\\textsc{f}] [$-$\textsc{gender}\\{\Large\texttt{⇊}}\\\textsc{mi}]
    ]
    ]
[$-$\textsc{person}\\{\Large\texttt{⇊}}\\\textsc{n}]
]
\end{forest}
% % \Tree[.$\pm$\textsc{person} 
% %     [.{$+$\textsc{person} $\sim\pm$\textsc{participant}} 
% %         [.{$+$\textsc{participant} $\sim\pm$\textsc{speaker}} {$+$\textsc{speaker}\\{\Large\texttt{⇊}}\\1\textsuperscript{st}} 
% %             [.$\pm$\textsc{hearer} {$+$\textsc{hearer}\\{\Large\texttt{⇊}}\\2\textsuperscript{nd}} {$-$\textsc{hearer}\\{\Large\texttt{⇊}}\\\textsc{ma}} 
% %         ] 
% %     ] [.{$-$\textsc{participant} $\sim\pm$\textsc{gender}} {$+$\textsc{gender}\\{\Large\texttt{⇊}}\\\textsc{f}} {$-$\textsc{gender}\\{\Large\texttt{⇊}}\\\textsc{mi}} 
% %     ] 
% %     ] {$-$\textsc{person}\\{\Large\texttt{⇊}}\\\textsc{n}} 
% % ] 
\end{figure}

\subsection{Accounting for the resolution pattern}

The first case to consider is the agreement patterns in which the probe probes for a person feature (the data discussed in \sectref{sec:featureResolution}). Based on the person-feature geometry in \figref{hierarchy} there are three basic cases to consider based on the label of the conjunction phrase: (a) there is a [$+$\textsc{person}] feature, valued as [$+$\textsc{participant}], (b) there is a [$+$\textsc{person}] feature, valued as [$-$\textsc{participant}], and (c) there is [$-$\textsc{person}] feature. As for number, throughout the section I assume that both conjuncts will associate with an index and that the indices will not be identical. The assumption that semantic plurality corresponds to a conjunction of non-matching indices is motivated by examples such as that in \REF{plural}. Consequently, the semantic number will be set as plurality and morpho-syntactically will correspond to a valued number feature (technically, [$-$\textsc{sg}]).

\ea\label{plural}
\ea His best friend$_i$ and editor$_j$ is by his bedside. \hfill $i=j \Rightarrow$ singular
\ex His best friend$_i$ and editor$_j$ are by his bedside. \hfill $i \neq j \Rightarrow$ plural 
\z\z

\largerpage
\noindent The first case to consider is a case in which the conjunction label will contain the features [$+$\textsc{person}] feature, valued as [$+$\textsc{participant}], and [$-$\textsc{sg}]. I argue that this labelling arises whenever one of the conjuncts is syntactically valued as masculine animate. The reason is that the masculine animate valuation corresponds to the  [$+$\textsc{person}] feature, valued as [$+$\textsc{participant}]. Since the labelling operation takes place at the syntax-semantics interfaces, the system minimally searches the embedded structure for features binding to the semantic component. Which is to say, if there is a [$+$\textsc{person}] feature and if there is a [$+$\textsc{participant}] in the searchable domain, these features must be copied (technically, identity-merged) into the label of the conjunction phrase. Consequently, irrespective of the features of the other conjunct, the labelling reflects the presence of the semantically marked features. In turn, morphology copies the feature combination onto the plural animate agreement (traditionally called masculine inanimate plural) (see \citealt{bhattwalkow13} for an argument in favour of agreement as morphological copying). This configuration is exemplified in \REF{baseline-anim-rep}, repeated from \REF{baseline-anim} above. Notice that the morphological realization does not recognize masculine animate feature as such but it solely realizes the valued [$+$\textsc{participant}] feature in the plural context.

\ea\gll \{\hspace{-2pt} Kočka / kotě / dobytek\} a pes jedli ze stejné misky.\label{baseline-anim-rep}\\
{} cat.\textsc{f.sg} {} kitten.\textsc{n.sg} {} cattle.\textsc{mi.sg} and dog.\textsc{\textbf{ma}} ate.\textsc{pp.\textbf{ma}.pl} from same bowl\\
\glt `The cat/kitten and the dog ate from the same bowl.'\\\hfill \fbox{\textsc{f/n/mi} + \textbf{\textsc{ma}} = \textbf{\textsc{ma (animate)}}}
\z

\noindent Let us now consider the next case which is the case when there is a [$+$\textsc{person}] feature but no [$+$\textsc{participant}] feature in the label of the conjunction phrase. This case arises if none of the conjuncts is syntactically valued as masculine animate but one or both conjuncts are syntactically valued as masculine inanimate or feminine. Since the label is marked as [$-$\textsc{participant}], the morphological realization resorts to the gender marking in the context of plural, i.e., the syncretic morphology for masculine-inanimate plural and feminine plural. This feature combination is exemplified by \REF{baseline-inanim}, repeated below as \REF{baseline-inanim-rep}.

\ea\gll Kotě a \{\hspace{-2pt} kočka / dobytek\} jedly ze stejné misky.\label{baseline-inanim-rep}\\
kitten.\textsc{n.sg} and {} cat.\textsc{\textbf{f}.sg} {} cattle.\textsc{\textbf{mi}.sg} ate.\textsc{pp.\textbf{\{mi/f\}.pl}} from same bowl\\
\glt `The kitten and the cat/cattle ate from the same bowl.'\\\hfill\fbox{\textsc{n} + \textbf{\textsc{mi/f}} = \textbf{\textsc{\{mi/f\} (gender)}}}
\z

\noindent Now we can finally turn to the last case which is a probe probing for person but with none of the conjuncts specified for a [$+$\textsc{person}] feature. Consequently, there is no participant-feature specification in the label of the conjunction phrase. This configuration arises when both conjuncts are in neuter. Note that according to the feature geometry in \figref{hierarchy}, neuter is syntactically not a gender feature but it arises as a realization of the [$-$\textsc{person}] feature.  In turn, plural neuter cannot be systematically computed from the label of a coordination that refers only to person. Instead, the lack of positive valuation  within the syntactic component means that the morphological realization must resort to the default gender realization (technically, failed agree, \citealt{Preminger2009}). In Czech this means that  morphology realizes the plural agreement as the syncretic plural gender form (\textsc{mi/f}). This feature combination is exemplified by \REF{n-a} and \REF{n-b}, repeated below as \REF{n-a-rep} and \REF{n-b-rep}.

\ea\gll Kotě a štěně \{\hspace{-2pt} jedly / *\hspace{-2pt} jedla\} ze stejné misky.\label{n-a-rep}\\
kitten.\textsc{n.sg} and puppy.\textsc{n.sg} {} ate.\textsc{pp.\{mi/f\}.pl} {} {} \textsc{pp.n.pl} from same bowl\\
\glt `The kitten and the puppy ate from the same bowl.'\\\hfill\fbox{\textsc{n.sg} + \textsc{n.sg} = \textbf{\textsc{\{mi/f\} (gender)}}}
\z

\ea\gll Koťata a štěně \{\hspace{-2pt} jedly / *\hspace{-2pt} jedla\} ze stejné misky.\label{n-b-rep}\\
kittens.\textsc{n.pl} and puppy.\textsc{n.sg} {} ate.\textsc{pp.\{mi/f\}.pl} {} {} \textsc{pp.n.pl} from same bowl\\
\glt `The kittens and the puppy ate from the same bowl.'\\\hfill\fbox{\textsc{n.pl} + \textsc{n.sg} = \textbf{\textsc{\{mi/f\} (gender)}}}
\z

\noindent The problem we just identified lies in the combinatorics behind the labelling operation. There is a caveat though. While the probe in these cases needs to be valued for person, it morphologically realizes gender features. Which is to say, if the label is uniquely labelled for gender from syntax, then morphology could realize the gender feature in and of itself. However, I argue that this may happen only if the syntactic features on both conjuncts are identical, i.e., only if narrow syntax provides \textsc{n.pl} as the common feature of the conjuncts. If this is the case, no feature calculation is necessary and the system solely copies the neuter plural label of its parts into the label of the conjunction phrase and this information determines the morphological mapping of the resulting agreement as neuter plural. This is the pattern we saw in \REF{n-c}, repeated below as \REF{n-c-rep}.

\ea\gll Koťata a štěňata jedla ze stejné misky.\label{n-c-rep}\\
kittens.\textsc{n.pl} and puppies.\textsc{n.pl} ate.\textsc{pp.n.pl} from same bowl\label{n-pl2}\\
\glt `The kittens and the puppies  ate from the same bowl.' \hfill\fbox{\textsc{n.pl} + \textsc{n.pl} = \textbf{\textsc{n.pl}}}
\z

\noindent The behaviour of neuter is crucial for our understanding of the overall system. Notice that there is no optionality in \REF{n-c-rep}. Which is to say, if syntax can uniquely derive the values of syntactic features of the conjunction phrase label, agree must respect these values. If, however, syntax cannot uniquely derive these values (in our cases, because there is a feature-valuation conflict), then morphology refers to the features of indices derived by the syntax-semantics interface as the only available structural information.

We have successfully derived the complete pattern of the seeming gender-feature resolution by referring only to the person feature. \tabref{label-person} summarizes the features in the label that were relevant in the process and the morphological mapping they triggered.

\begin{table}
% \captionsetup{format=hang}
\caption{Labelling of the conjunction phrase and morphological\newline realization: probe = person}
\label{label-person}
\begin{tabularx}{\textwidth}{llll}
\lsptoprule
\textbf{features of conjuncts} &  \textbf{features copied by probe}  & \textbf{morphology} & \textbf{exs}\\\midrule
\textsc{ma} \& $\alpha$,  & [\textsc{$-$sg}, $+$\textsc{person},  &   animate (\textsc{ma}) & \REF{baseline-anim},\\
where $\alpha \in$ \{\textsc{ma, mi, f, n}\} &$+$\textsc{participant}]  &  &\REF{baseline-anim-rep} \vspace{12pt}\\
\textsc{mi} or \textsc{f} \& $\alpha$, & [\textsc{$-$sg}, $+$\textsc{person}, & gender (\{\textsc{mi/f}\}) & \REF{baseline-inanim},\\
where $\alpha \in$ \{\textsc{mi, f, n}\} &$-$\textsc{participant}]  & &\REF{baseline-inanim-rep} \vspace{12pt}\\
\textsc{n.sg} \& \textsc{n.sg} & [\textsc{$-$sg}, $-$\textsc{person}] & default & \REF{n-a-rep} \\
& & ($\sim$gender=\{\textsc{mi/f}\}) & \vspace{12pt}\\
\textsc{n.pl} \& \textsc{n.sg} & [\textsc{$-$sg}, $-$\textsc{person}] & default  & \REF{n-b-rep}\\
&  &  ($\sim$gender=\{\textsc{mi/f}\}) & \vspace{12pt}\\
\textsc{n.pl} \& \textsc{n.pl} & [\textsc{$-$sg}, $-$\textsc{person}, \textsc{n.pl}] & copy (\textsc{n.pl}) & \REF{n-c-rep}\\\lspbottomrule
\end{tabularx}
\end{table}


%tady
\subsection{Accounting for the resolution failure}

Let us now turn to the data pattern discussed in \sectref{sec:puzzle}, i.e., the pattern in which the probe does not have any unvalued person feature but probes for a gender feature instead. While the derivational procedure described in the previous subsection crucially relies on the ability of the syntax--semantics interface to construct a complex semantic index from the person representation in the label of the conjunction phrase, a probe that probes only for gender cannot use this complex information but must rely on the syntactically present valuation of gender. In turn, we expect the agreement patterns to play out differently. 

Before we proceed to the individual patterns, let us consider the geometry of the gender features. According to the feature-geometry of person proposed in \figref{hierarchy}, only the masculine inanimate and feminine feature correspond to a binary gender feature. Masculine animate corresponds to a morphological realization in the context of [$+$\textsc{participant}]. The syncretic masculine inanimate and feminine plural is a default realization of the [$+$\textsc{person}] feature, i.e., without a  [$+$\textsc{participant}] feature. We have also seen that although neuter should in principle appear in the context of [$-$\textsc{person}], it does not, as it only can be copied. The core difference between the cases discussed in the previous subsection and the cases discussed in this subsection is that in the previous cases distinct values of person and participant features have been resolvable in the process of the complex index formation. The features that were used to value the unvalued person of the probe were indeed features that were mediated by the formation of the complex semantic index. The question is what happens, if there is no uniform person representation mediated by the complex-semantic-index formation?

I argue that in such a case, an unvalued gender feature on the probe can be valued only if the conjuncts share the features relevant for the morphological mapping procedure. It follows that agreement will be successful only if (a) there is no mismatch of gender features (the trivial case) or (b) both conjuncts are [$+$\textsc{participant}]. All other combinations should be degraded. This prediction is borne out.

As we already saw, if one conjunct is masculine animate, the other conjunct must also also masculine animate, or feminine  that can semantically be construed as animate. This follows from the restriction that both conjuncts must be [$+$\textsc{participant}], that is animate, as only animate entities can be modelled as participants. Consequently, in this feature combination, the plural agreement is animate, i.e., morphologically realized as \textsc{ma}. An example of this feature interaction is given in \REF{anim-2}, repeated below as \REF{anim-2-rep}.


\ea\label{anim-2-rep}
\ea\gll Petr a Pavla byli unavení.\\
Petr.\textsc{ma.sg} and Pavla.\textsc{f.sg} were.\textsc{pp.ma.pl} tired.\textsc{pp.ma.pl}\\
\glt `Petr and Pavla were tired.'\smallskip
\ex\gll  Pes a kočka byli unavení.\\
dog.\textsc{ma.sg} and cat.\textsc{f.sg} were.\textsc{pp.ma.pl} tired.\textsc{pp.ma.pl}\\
\glt `The dog and the cat were tired.'\\
\z
\hfill \fbox{\textsc{ma} + \textsc{f} = \textbf{\textsc{ma (animate)}}}
\z


\noindent Note that in this case, although there is no uniform gender feature in the label, the shared [$+$\textsc{participant}] feature is sufficient for the derivation to converge.

If gender is not specified for animacy, there is no feature information in the label of the conjunction phrase that could be used to value the gender feature on the probe. There are two cases to consider. If there is no [$+$\textsc{participant}] feature, the combination is degraded but the speakers have an intuition what the best form would be. I argue this is because there is no valuation in syntax. Yet, the speakers can use their knowledge of what the feature formation would be if there was a person feature as a formal mediator. In other words, this is a case of syntactic valuation failure, with a partial rescue by morphology. An example of this combination is in \REF{inam}--\REF{inam-fem}, repeated below as \REF{inam-rep}--\REF{inam-fem-rep}.


\ea[??]{\gll \hspace{-2pt} Dobytek a kotě byly unavené.\label{inam-rep}\\
{} cattle.\textsc{mi.sg} and kitten.\textsc{n.sg} were.\textsc{pp.f/mi.pl}\\
\glt `The cattle and the kitten were tired.'\hfill 
\fbox{\textsc{mi} + \textsc{n} = \textbf{??\textsc{\{mi/f\} (gender)}}}}
\z

\ea[]{\gll Dobytek a kotě byly unavený.\label{colloq1-rep}\\
cattle.\textsc{mi.sg} and kitten.\textsc{n.sg} were.\textsc{pp.pl} tired.\textsc{pp.pl}\\
\glt `The cattle and the kitten were tired.'}
\z


\ea[??]{\gll \hspace{-2pt} Kočka a kotě byly unavené / unavená / unavený.\label{inam-fem-rep}\\
{} cat.\textsc{f.sg}  and kitten.\textsc{n.sg} were.\textsc{pp.f.pl} tired.\textsc{pp.f.pl} {} tired.\textsc{pp.n.pl} {} tired.\textsc{pp.colloq-pl}\\
\glt `The cat and the kitten were tired.'\hfill 
\fbox{\textsc{f} + \textsc{n} = \textbf{??\textsc{f/n/colloq (gender)}}}}
\z



\noindent The more interesting case is the case when the label combines a [$+$\textsc{person}] and a [$-$\textsc{person}] feature. Without the complex semantic index being computed and used to value a person on the probe, speakers clearly lack any indication of what the morphological mapping should be. In turn, there is no morphological form that could save the failed syntactic valuation. This is what underlies the agreement gaps we saw in \REF{gaps}, repeated below as \REF{gaps-rep}.

\ea\label{gaps-rep}
\ea[$\circledast$]{\gll Pes a kotě byli ??\{\hspace{-2pt} unavené / unavení / unavená\}.\\
dog.\textsc{ma.sg} and kitten.\textsc{n.sg} were.\textsc{ma.pl} {} tired.\textsc{pp.mi/f.pl} {} \textsc{pp.ma.pl} {} \textsc{pp.n.pl}\\
\glt Intended: `The dog and the kitten were tired.' \hfill \fbox{\textsc{ma.sg} + \textsc{n} = $\circledast$}}\smallskip
\ex[$\circledast$]{\gll Psi a koťata byli ??\{\hspace{-2pt} unavené / unavení / unavená\}.\\
dogs.\textsc{ma.pl} and kitten.\textsc{n.sg} were.\textsc{ma.pl} {} tired.\textsc{pp.mi/f.pl} {} \textsc{pp.ma.pl} {} \textsc{pp.n.pl}\\
\glt Intended: `The dogs and the kittens were tired.' \hfill \fbox{\textsc{ma.pl} + \textsc{n.pl} = $\circledast$}}
\z\z

\section{Predictions}

The core property of the system proposed in the previous section is that agreement with coordinated noun phrases is always mediated by the label of the conjunction phrase. Crucially, we saw that some agreement combinations cannot be resolved because of a problem with valuation of the agree probe because the label of the conjunction phrase has not been uniquely resolved. Interestingly, in the domain of agreement gaps, we saw that even if there is a good morphological match, the lack of successful valuation yields agreement failure. Consequently, if this reasoning is correct, we expect to find problems with valuation elsewhere. This section investigates two empirical domains that confirm this prediction.

Let us start with agreement gaps. If agreement gaps result from problems of labelling, i.e., from the fact there is no unique feature in the label that could value an unvalued feature of the probe, we expect to find agreement gaps elsewhere. This prediction is born out in comitative constructions and first-conjunct agreement constructions. Although in comitative constructions only one conjunct is in nominative, agreement is with both conjuncts. Which means the agreement must be based on the features of the label of the conjunction phrase. Consequently, we expect agreement gaps to arise exactly in the same environment as with regular coordinated phrases. Which is to say, we expect agreement gaps whenever the probe does not probe for person but only for gender, and whenever the conjuncts do not share gender features or are not both marked as [$+$\textsc{participant}]. This prediction is borne out, as can be seen, for example, in \REF{comin-gap}.

\ea[??]{\gll Pes s kotětem byli unavení / unavené / unavená.\label{comin-gap}\\
dog.\textsc{nom.ma.sg} with kitten.\textsc{instr.n.sg} were.\textsc{pp.ma.pl} tired.\textsc{ma.pl} {} \textsc{\{mi/f\}.pl} {} \textsc{n.pl}\\
\glt Intended: `The dog and the kitten were tired.'}
\z

\noindent Interestingly, even if the predicate morphologically agrees only with the first conjunct, we predict that the adjectival agreement should be ungrammatical if the conjunction phrase cannot be uniquely labelled. This prediction follows if the morphological realization of agreement is post-syntactic but agree targets the label of the conjunction phrase. As the example in \REF{first} demonstrates, this prediction is indeed borne out. To my knowledge no current theory of first-conjunct agreement predicts \REF{first} to be ungrammatical.

\ea[*]{\gll Byl unaven pes a kotě.\label{first}\\
was.\textsc{pp.m.sg} tired.\textsc{m.sg} dog.\textsc{nom.ma.sg} and kitten.\textsc{n.sg}\\
\glt Intended: `The dog and the kitten were tired.'}
\z

\noindent Let us now turn to the second group of predictions. Without saying it explicitly, I assumed throughout the paper that the predicates probe only after the conjunction phrase was spelled-out. This assumption follows from the fact that the relevant notion of labelling is a process that takes at the syntax-semantics interface, which is to say, it is part of the spell-out procedure. The prediction then is clear: only elements that probe after the spell-out of the conjunction phrase can agree with both conjuncts. The reason is that without the label, there is no syntactic representation of the conjunction phrase that would combine features of both conjuncts. This prediction is borne out as well, as can be demonstrated on two agreement patterns.

First, if an adjectival adjunct modifies a conjunction, it must be syntactically adjoined before the conjoined phrase is spelled-out. Consequently, even a conjunct that semantically modifies both conjuncts must morphosyntactically agree with only one of the conjuncts. The example in \REF{adjunct} demonstrates this point. Although the adjective `young' may semantically modify only the man or it may modify both the man and the woman, it must agree only with the first conjunct. The plural agreement is ungrammatical.


\ea\gll \{*\hspace{-2pt} mladí / mladý\} muž a žena\label{adjunct}\\
{} young.\textsc{ma.pl} {} \textsc{m.sg} man.\textsc{ma.sg} and woman.\textsc{f.sg}\\
\glt `a young man and a young woman' or `a young man and a woman'
\z

\noindent This point can be further strengthened by the following fact. In Czech, determiners that semantically select for plurality cannot modify a conjunction of singular individuals. Thus, for example, \textit{oba} `both' is ungrammatical within a conjunction phrase, as shown in \REF{both}.

\ea[$\circledast$]{\gll *\{\hspace{-2pt} oba / obě\} kočka a kotě\label{both}\\
{} both.\textsc{mi} {} both.\textsc{f/n.pl} cat.\textsc{f.sg} and kitten.\textsc{n.sg}\\
\glt Intended: `both the cat and kitten'}
\z

\section{Conclusions}

This paper contributes to our understanding of syntactic agree and its morphological realizations in four important respects. First, I presented an argument that narrow syntax cannot resolve a conflicting feature valuation. Syntax can only copy and share. Second, patterns that seem to involve some form or feature resolution are mediated by feature resolution at the syntax-semantics interface. Concretely, I argued that feature resolution arises only as part of semantic index formation, dependent on person-feature representation in narrow syntax. Third, I provided an empirical argument that labelling conflicts are fatal to feature valuation as agree. There is no morphological rescue. Fourth, I demonstrated that morphological features realized on agreeing elements do not have to faithfully match the underlying bundle of syntactic features. Although the final conclusion is not surprising in the light of the work done in the Distributed morphology framework, it raises non-trivial questions about the empirical accuracy of generalizations in the domain of agreement. 

The core argument presented in the paper relies on the very existence of combinations of features that cannot be syntactically resolved. The fact that there exist combinations that cannot be syntactically resolved in and of itself provides sufficient evidence that there cannot be a default syntactic mechanism that would underlie the seeming resolution patterns. Interestingly, as pointed out by two anonymous reviewers, there are naturally attested examples with seemingly parallel combinations of features that are perceived by native speakers as more acceptable or fully acceptable. Which is to say, there appear to be agreement strategies that go beyond the mechanics proposed in this paper. Providing an exhaustive description and a theoretical account of agreement resolution patterns in Czech dialects and Slavic in general  goes beyond the present work. Yet I would like to conclude the paper with a couple of observations about the possible nature of the attested variation and its underpinning.

The data brought by the anonymous reviewers seem to fall into two groups: examples from colloquial Czech (dialects attested in the eastern part of the Czech Republic), as in \REF{colloq}, and examples with human participants, as in \REF{human}.

\ea\gll Člověk a prase jsou si navzájem souzení, stvořeni jeden pro druhého.\label{colloq}\\
man.\textsc{ma} and pig.\textsc{n} are \textsc{refl} mutually judged.\textsc{ma.pl} created.\textsc{ma.pl} one for second\\
\glt `A man and a pig are meant for each other, been created one for another.'
\z

\ea\gll Po půlhodině hraní byli tatínek i miminko úplně vyčerpaní.\label{human}\\
after half-hour of-playing were.\textsc{ma.pl} father.\textsc{ma} and baby.\textsc{n} entirely exhausted.\textsc{ma.pl}\\
\glt `After playing for half an hour, the father and the baby were entirely exhausted.'
\z

\noindent The data and judgements presented in this paper come from Standard Czech, a prescriptive variety, that  overlaps in the relevant morphological features with eastern Moravian dialects (e.g., \citealt[401--404]{KarlikEtAl:2002}). Speakers of these dialects typically have the same or similar type of morphological syncretism and range of morpho-syntactic features as preserved in Standard Czech. Speakers of western dialects or Prague-centered colloquial varieties often lack the full range of distinct morpho-syntactic patterns. One might wonder whether the distinct morphological syncretism underlies examples such as that in \REF{colloq}. If that was the case, examples of this sort would provide a challenge to the present proposal.

We know, however, that the variation in agreement goes beyond morphological syncretism. The dialects fundamentally vary in their semantic index representation, as attested by differences in binding. Consider the example in \REF{binding}.

\ea[\%]{\gll  Petr$_i$ má rád jeho$_i$ matku.\label{binding}\\
Petr has liked his mother\\
\glt `Peter likes his mother.'}
\z

\noindent While \REF{binding} yields a severe Principle B violation in Standard Czech and Moravian dialects, it is fully acceptable in some Bohemian dialects (Jakub Dotlačil, p.c.). If, indeed, there is a connection between a person-feature resolution and semantic index representation and if differences in binding follow from differences in index representations (\citealt{Heim1998}, \citealt{Roelofsen2008}), it is not altogether surprising that we might find distinct resolution patterns. The same point applies to the inter-Slavic variation as reported in \citet{corbett83} and much subsequent work. We know that agreement resolution varies in Slavic dialects. But equally there is an insufficiently studied variation in binding (e.g., \citealt{Nikolaeva2014}). 

The other point concerns an effect of humanness. It seems that at least in some cases replacing  a non-human animate DP with a human-denoting animate DP improves the resolution pattern. We know independently that humanness closely interacts with a person representation (e.g., \citealt{Ritter2014, Wiltschko2015}). It is possible that we see a related effect here as well.

A closer investigation of these intriguing patterns must, however, await future research.


\section*{Abbreviations}

\begin{tabularx}{.5\textwidth}{@{}lQ@{}}
\textsc{colloq}&colloquial\\
ConjP&conjunction phrase\\
\textsc{f}&feminine\\
\textsc{instr}&instrumental\\
\textsc{m}&masculine\\
\textsc{ma}&masculine animate\\
\textsc{mi}&masculine inanimate\\
\end{tabularx}%
\begin{tabularx}{.5\textwidth}{@{}lQ@{}}
\textsc{n}&neuter\\
\textsc{neg}&negation\\
\textsc{nom}&nominative\\
\textsc{pl}&plural\\
\textsc{pp}&participle\\
\textsc{refl}&reflexive\\
\textsc{sg}&singular\\
\end{tabularx}

% \newpage
\section*{Acknowledgements}
This research would not have been possible without funding from the Social Sciences and Humanities Research Council Insight Grants \#435-2012-1567 and \#435-2016-1034  (PI: I. Kučerová). Thanks the audiences at NELS, Amherst, MA, and FDSL, Berlin for their questions and suggestions. Special thanks go to Betsy Ritter, Adam Szczegielniak, and Nick Welch for extensive discussion of the data and the analysis with me. The remaining errors are mine.

\sloppy
\printbibliography[heading=subbibliography,notkeyword=this]
\clearpage 
\end{document}
