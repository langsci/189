\documentclass[output=paper,
modfonts, hidelinks, newtxmath
]{langscibook}  

% \papernote{\footnotesize\normalfont
% Julia Bacskai-Atkari. Doubly filled COMP in Czech and Slovenian interrogatives. To appear in: Denisa Lenertová, Roland Meyer, Radek Šimík \& Luka Szucsich (eds.),\textit{ Advances in formal Slavic linguistics 2016}. Berlin: Language Science Press. [preliminary page numbering]
% }

\title{Doubly filled COMP in Czech and Slovenian interrogatives}  

\author{%
 Julia Bacskai-Atkari\affiliation{University of Potsdam}
}

% \chapterDOI{} %will be filled in at production
% \epigram{}

\abstract{
This article investigates the syntax of doubly filled COMP patterns in Czech and Slovenian interrogatives from a cross-linguistic perspective, concentrating on the differences between Germanic and Slavic doubly Filled COMP. In Germanic, dialects that allow the doubly filled COMP pattern do so to lexicalise a C head specified as [fin] with overt material, which is regularly carried out by verb movement in main clauses (e.g. V2 in German, T-to-C in English interrogatives) and by the interrogative complementiser in embedded polar questions. The insertion of the complementiser has no interpretive effect on the clause and is restricted to embedded clauses. By contrast, in Czech and Slovenian a complementiser can be inserted even in main clauses, and while its presence is optional, its insertion triggers an interpretive difference, resulting in an echo reading. I argue that while in Germanic, the C head is specified as [wh] and is checked off by the wh-element, in Slavic the C is not specified as [wh] and the type of the clause hence matches the properties of the inserted declarative head. In turn, the wh-element moves because it is focused: echo questions are closer to focus constructions than to ordinary questions.

\keywords{complementiser, doubly filled COMP, echo questions, finiteness, interrogative clause, wh-movement}
}


\begin{document}
\maketitle
\shorttitlerunninghead{Doubly filled COMP in Czech and Slovenian interrogatives}
\section{Introduction} 
Doubly filled COMP patterns and especially their absence from the standard varieties are well known in the literature on West-Germanic languages.\footnote{The West-Germanic languages to be discussed here include English, German, and Dutch. Note that there have been claims in the literature, notably by \citet{emondsfaarlund2014} that English is not a West-Germanic but a North-Germanic language. However, as shown convincingly by \citet{bechwalkden2016}, this claim has serious problems and it cannot be maintained.} In order to illustrate the phenomenon, consider first the following interrogatives from Standard English:

%Example 1
\ea
	\ea \textbf{Which book did} she buy? \label{whichbookdid}
	\ex \textbf{Did} she buy a book? \label{did}
	\ex I don't know \textbf{which book (*that)} she bought. \label{whichbook}
	\ex I don't know \textbf{if} she bought a book. \label{if}
	\z
\z

\noindent The ban on the insertion of \textit{that} in \REF{whichbook} is traditionally referred to as the ``doubly filled COMP filter'', which is supposed to prohibit lexical material in both the specifier and the head of the same XP projection (\citealt[446]{chomskylasnik1977}, see also \citealt{koopman2000}). Hence, the wh-element \textit{which book} cannot co-occur with the complementiser \textit{that} in embedded constituent questions. The same issue does not arise in embedded polar questions containing \textit{if}, since the interrogative marker is the complementiser in these cases: the impossibility of the sequence \textit{if that} follows from the two elements being in complementary distribution and need not be accounted for by an additional filter rule.

One problem that arises with the doubly filled COMP filter as a general rule is that it is not obeyed in main clause constituent questions. As can be seen in \REF{whichbookdid} and \REF{did}, the verb moves up to C in main clause questions in English (and more generally in Germanic), and this results in the co-occurrence of an overt wh-element in SpecCP with the verb in C in main clause constituent questions, see \REF{whichbookdid}. While one could in principle argue that main clause questions with verb movement are subject to different requirements, another problem arises in connection with various non-standard dialects (as indicated by \citealt{vangelderen2009}, \citealt{bayer2004} and \citealt{bayerbrandner2008}, such dialects are found across West Germanic without a very clear geographical restriction), which show clear violations of the doubly filled COMP filter (cf. the data in \citealt{baltin2010}):

%Example 2
\ea
	\ea I don't know \textbf{which book that} she bought. \label{whichbookthat}
	\z
\z

\noindent As can be seen, the co-occurrence of the wh-phrase and \textit{that} is allowed in the non-standard pattern; this is attested across Germanic. This obviously raises the question why doubly filled COMP patterns arise in Germanic and, if applicable, cross-linguistically.

In this article, I propose the following. First, doubly filled COMP patterns in Germanic arise when a finite complementiser is inserted in addition to a wh-element in SpecCP and the complementiser serves to lexicalise [fin] in C. In principle, lexicalisation can be carried out by other elements, too (such as verbs in main clauses), and the insertion of \textit{that} causes no interpretive differences compared to \textit{that}-less interrogatives. I argue that the lexicalisation requirement on [fin] is more generally attested in the syntactic paradigm and is related to V2 and to T-to-C movement. Second, there is no such lexicalisation requirement in Slavic languages and the insertion of a complementiser causes an interpretive difference (namely, the clause is interpreted as an echo). I argue that this difference is related to syntactic features as well: while wh-movement in Germanic doubly filled COMP structures is driven by a [wh] feature on the C head, there is no such feature on C in Slavic doubly filled COMP structures.

\section{Doubly filled COMP in Germanic}
I adopt the general idea of \citet{bacskaiatkaritoappear}, according to which a C with  [fin] specification is regularly lexicalised in Germanic, with some inter-language variation. English is somewhat exceptional as it is not a V2 language: the lexicalisation rule applies to interrogatives and is manifest in the phenomenon of T-to-C movement. In German, it applies to declaratives as well and results in the matrix V2 configurations. Consider the following matrix interrogatives in English:

%Example 3
\ea
	\ea \textbf{Which book did} she buy? 
	\ex \textbf{Did} she buy a book? 
	\z
\z

\noindent The corresponding structures are shown in \REF{treematrix} below:

%Example 4
%
\begin{multicols}{2}
\ea \label{treematrix}
\ea
\begin{forest} baseline, qtree
[CP
	[\textit{which book}\textsubscript{{[}wh{]}}]
	[C$'$
		[C\textsubscript{{[}fin{]},{[}wh{]}}
			[V [\textit{did}]]
			[C]
		]
		[\ldots]
	]
]
\end{forest}
\ex \label{treedid}
\begin{forest} baseline, qtree
[CP
	[Op\textsubscript{{[}Q{]}}]
	[C$'$
		[C\textsubscript{{[}fin{]},{[}Q{]}}
			[V [\textit{did}]]
			[C]
		]
		[\ldots]
	]
]
\end{forest}
\z
\z

\end{multicols}

\noindent In either case, the C head is lexicalised by way of the verb moving up to C via head adjunction, and the SpecCP position is filled by an operator element. Note that there is a distinction between [wh] and [Q], following the idea of \citet{bayer2004}, whereby [Q] essentially stands for disjunction; wh-elements are [Q] but not all elements with a [Q] specification are [wh] (see \citealt{bacskaiatkaritoappear} for [Q] in Germanic). Further, the operator in \REF{treedid} is a covert polar operator. The polar operator can in principle be overt (e.g. English \textit{whether}) or covert, and it marks the scope of a covert \textit{or} (\citealt{larson1985}). This operator is inserted directly into SpecCP (\citealt{bianchicruschina2016}).

Consider now the following English embedded interrogatives:

%Example 5
\ea
	\ea I don't know \textbf{which book (\% that)} she bought. 
	\ex I don't know \textbf{if} she bought a book.
	\z
\z

\noindent The corresponding structures are shown in \REF{treeembedded}:\footnote{Contrary to \citet{baltin2010}, I assume that doubly filled COMP structures are literally doubly filled COMP, that is, there is only a single CP involved; see \citet{bacskaiatkari2018sardis} for arguments on this. Essentially, \citet{baltin2010} assumes that the ban on overt material in C in sluiced clauses (\citealt{merchant2001}) follows directly from the fact that the ellipsis position is located in the highest C head, eliding the complementiser in a lower C position. However, this is in fact not a sound argument since the lack of a complementiser in these cases can be due to phonological factors as well (the complementiser cliticising onto the clause in the languages he examined), which may indeed be subject to cross-linguistic variation. In Slovenian, for instance, wh-sluices can contain a complementiser (e.g. \textit{da} `that' but apparently also \textit{\v{c}e} `if'), see \citet{marusicmismasplesnicarrazborseksuligoj2015}, indicating that the generalisation does not hold. Note that the Slovenian data contradict the judgements given by \citet[76]{merchant2001}, who suggests that while doubly filled COMP patterns are possible in Slovenian in the same way they are attested in other languages (see, for instance, the Danish and Irish data given by \citealt[76--77]{merchant2001}), the sluiced version of doubly filled COMP clauses (containing an overt complementiser) is uniformly rejected.}

%Example 6
\begin{multicols}{2}
\ea \label{treeembedded}
\ea \label{treewh}
\begin{forest} baseline, qtree
[CP
	[\textit{which book}\textsubscript{{[}wh{]}}]
	[C$'$
		[C\textsubscript{{[}fin{]},{[}wh{]}}
			[(\textit{that})]
		]
		[\ldots]
	]
]
\end{forest}
\ex \label{treeif}
\begin{forest} baseline, qtree
[CP
	[Op\textsubscript{{[}Q{]}}]
	[C$'$
		[C\textsubscript{{[}fin{]},{[}Q{]}}
			[\textit{if}]
		]
		[\ldots]
	]
]
\end{forest}
\z
\z

\end{multicols}

\noindent The interrogative feature has to be marked overtly in embedded questions (there being no distinctive interrogative intonation) and it is done either by an overt complementiser or by an overt operator. Accordingly, the interrogative feature on C can be checked off by inserting an element into C (\textit{if}) or by inserting an element into the specifier (\textit{which book} in \REF{treewh} above). By contrast, [fin] can be lexicalised only by an element inserted into C (\textit{that} and \textit{if} in \REF{treeembedded} above, but not by e.g. \textit{which book} in the specifier).

Regarding the lexicalisation of [fin] in C, the following can be established. In matrix clauses, as shown in \REF{treematrix}, [fin] in C is lexicalised via verb movement, whereby the verb adjoins to C (head adjunction). In embedded clauses, a complementiser is inserted:\footnote{While [fin] is lexicalised by verb movement in main clauses, this is generally not possible in embedded clauses: certain verbs in German allow embedded V2 and there are certain dependent clauses (such as hypothetical comparatives and conditionals) that likewise allow verb fronting. As argued by \citet{bacskaiatkaritoappear}, this is due to restrictions from the matrix predicate.} there are two possible ways here. One is to insert an interrogative complementiser, see \REF{treeif}, which also checks off the [Q] feature. Further, the insertion of the regular finite subordinator is possible if [wh] is checked off by an overt operator, hence in structures like \REF{treewh}: this option can be observed in nonstandard varieties. Since, as the structures above demonstrated, lexicalisation of [fin] in C is generally attested in the syntactic paradigm, standard varieties in West Germanic have an exception in \REF{treewh} by not lexicalising the C head,\footnote{According to \citet{bacskaiatkaritoappear}, this has to do with licensing conditions on zero complementisers (i.e., they are licensed in these environments in the standard language). In addition, the ``doubly filled COMP filter'' is rather the consequence of an economy principle against multiple elements with overlapping functions, which interacts with a principle favouring overt marking, see \citet{vangelderen2009}. This question cannot be examined here in detail.} while nonstandard varieties are completely regular in this respect. Note that the insertion of an interrogative complementiser is not a viable option in cases like \REF{treewh} since the insertion of the complementiser would check off the active interrogative feature on the C head,\footnote{The C head is specified as [wh] and the complementiser has the feature [Q]. The two features are not fully incompatible, though, as [Q] is a subset of [wh] (cf. \citealt{bayer2004}). The problem with inserting the complementiser is the deactivation of the feature, as described above, not feature incompatibility.} and hence there would be no feature attracting the wh-element to move to the CP (since [Q] is a subset of [wh], an interrogative complementiser would not be incompatible with the feature specification of the head) and thus prevent the movement of the wh-element.

The insertion of the complementiser is thus in line with the general V2 property of Germanic languages and with T-to-C movement in English interrogatives. Further, the insertion of the finite complementiser causes no interpretive difference, and several dialects show optionality with respect to the insertion of the complementiser.\footnote{Optionality arises in certain dialects with head-sized wh-phrases that may be inserted into either the specifier or the head, see \citet{bacskaiatkari2018sardis}, following \citet{bayerbrandner2008}. Not all dialects have optionality, though. As there is no interpretive difference between configurations with and without the complementiser, it is actually expected that at least some dialects show optionality; note that while optionality is considered to be problematic for minimalist approaches, dialect data and diachronic data in fact support the view that at least some optionality is allowed in language, to allow gradual variation and change. These issues cannot be pursued here in detail.}

Doubling is possible in polar interrogatives as well if the operator is overt. In English, the operator \textit{whether} can appear in embedded clauses overtly and doubling with \textit{that} can be observed both historically and synchronically (see  \citealt{vangelderen2009} for modern substandard varieties); in main clauses, its appearance is restricted to historical examples.\footnote{As mentioned above, verb movement to C in embedded clauses is subject to restrictions (due to the matrix predicate).} Consider:

%Example 7
\ea 
	\ea 
    	\textbf{Whether} \textbf{did} he open the Basket?\\
		(\textit{The Tryal of Thomas Earl of Macclesfield}; source: Salmon, Thomas and Sollom Emlyn (1730) A complete collection 	 of state-trials, and proceedings for high-treason, and other crimes and misdemeanours: 1715--1725)
	\ex 
    I wot not \textbf{whether} \textbf{that} I may come with him or not. 				\label{whetherthat}\\
	`I do not know whether I may come with him or not.'\\\hfill(\textit{Paston Letters} XXXI)
	\z
\z

\noindent As can be seen, \textit{whether} is similar to ordinary wh-operators in triggering verb movement to C in main clauses and in allowing the insertion of \textit{that} in embedded clauses; hence, its behaviour contrasts with that of \textit{if}. Importantly, just like in constituent questions, there is no interpretive difference between the version with \textit{that} and the version without \textit{that} of the same sentence.

Regarding the separation of [wh] and [Q] mentioned above, it must be mentioned that the co-occurrence of two interrogative elements is possible in certain languages (\citealt{bayer2004}). This can be observed in Dutch dialects in examples like \REF{wieofdat} below:

%Example 8
\ea
	\gll Ze weet \textbf{wie} \textbf{of} \textbf{dat} hij had willen opbellen \label{wieofdat}\\
		 she knows who if that he had want call\\
\glt	`She knows who he wanted to call.'\\
{}\hfill(\citealt[66, ex. 17]{bayer2004}, citing \citealt{hoekstra1993})
\z

\noindent As can be seen, in this case three overt elements appear in the CP-domain: the wh-operator itself, the Q-element \textit{of} `if' and the finite complementiser \textit{dat} `that'. Again, no interpretive difference can be attributed to the insertion of multiple elements: clauses with the combination \textit{wie dat} `who that' and clauses with a single \textit{wie} `who' have the same interpretation, too. The structure for the CP-domain in \REF{wieofdat} is shown below:

%Example 9
\ea
\begin{forest} baseline, qtree
[CP
	[\textit{wie}\textsubscript{{[}wh{]}}]
	[C$'$
		[C\textsubscript{{[}fin{]},{[}wh{]},{[}sub{]}}
			[$\emptyset$]
		]
		[CP
			[\textit{of}\textsubscript{{[}Q{]}}]
			[C$'$
				[C\textsubscript{{[}fin{]},{[}sub{]},{[}wh{]}}
					[\textit{dat}\textsubscript{{[}fin{]},{[}sub{]}}]
				]
				[\ldots]
			]
		]
	]
]
\end{forest}
\z

\noindent The polar operator is in the scope of a wh-operator, and the clause is ultimately specified as [wh]: hence, even if the Q-element \textit{of} is inserted into the lowest SpecCP, [wh] is not checked off and the CP projects further (essentially, the [wh] feature of the lower C is inherited by the higher C).

To conclude this section, it can be established that doubly filled COMP patterns in Germanic interrogatives follow from a requirement on lexicalising [fin] on C, which ultimately follows from the V2 property of Germanic languages, whereby English is slightly exceptional in that V2 is no longer attested, but the same applies to T-to-C movement in interrogatives. The expectation is therefore that genuine doubly filled COMP patterns should be different or not available in languages where there is no lexicalisation requirement on [fin] in main clause interrogatives.\footnote{Note that while V2 (or T-to-C) is probably necessary for genuine doubly filled COMP, it is not true the other way round: it is indeed possible that the lexicalisation of [fin] does not hold in all constructions and a language may be V2 without showing doubly filled COMP effects: for instance, Standard German (and any variety of German lacking doubly filled COMP patterns) is such a language.}

\section{Czech}
In this section, I am going to overview the possible patterns in Czech main and embedded questions. I will show that doubling is possible, yet while the resulting combinations are in part surface-similar to their Germanic counterparts, they are associated with a particular (echo) interpretation.

Just like in English, constituent questions in Czech contain an overt wh-element fronted to the left edge of the clause:\footnote{Note that I am only considering questions involving a single wh-phrase in this paper and do not venture to examine multiple wh-fronting. As argued by \citet{Boskovic2012}, multiple wh-questions actually involve the movement of a single wh-phrase due to a [wh] feature, and the remaining wh-elements are either located in situ or are fronted as focused phrases: crucially, the CP does not contain multiple [wh] features attracting various wh-elements. See also \citet{gruetskrabalova2011} on Czech and \citet{mismas2016} on Slovenian. In this sense, further wh-phrases and their position in the clause are not relevant to the present discussion, which is centred on clause-typing issues.}

%Example 10
\ea
	\ea[]{
	\gll \textbf{Kdo} přijel?\\
			who arrived.\textsc{3sg}\\
	\glt 	`Who arrived?'
    }
	\ex[]{
	\gll 	Ptala se, \textbf{kdo} přijel. \label{kdoembedded}\\
			asked.\textsc{3sg.f} \textsc{refl} who arrived.\textsc{3sg}\\
	\glt	`She asked who arrived.'
    }
	\z
\z

\noindent I assume that the wh-element moves to SpecCP, following \citet{rudin1988} and \citet{kaspar2015}.

Regarding doubly filled COMP patterns, the insertion of \textit{že} `that' is possible. However, this results in an interpretive difference from ordinary questions and essentially renders echo questions where the speaker asks for the value of the wh-element\footnote{As Jiri Kaspar (p.c.) informs me, constituent questions with \textit{že} can be interpreted as canonical echo questions (where the value of the wh-element was inaudible), reminder questions (the speaker has forgotten the value), verification questions (the speaker is unsure about the value), and surprise questions (the speaker assumes a different value). Since all these types have been subsumed under the umbrella term ``echo questions'' in the literature, as opposed to ordinary questions, I will simply use the label ``echo questions'' in this paper but it should kept in mind that this term subsumes various subtypes (this applies to the Slovenian data, too).} (see \citealt{kaspar2015}, \citealt{gruetskrabalova2011}):

%Example 11
\ea
	\ea[]{
    \gll \textbf{Kdo} \textbf{že} přijel? \label{kdoze}\\
    		who that arrived.\textsc{sg.m}\\
	\glt	 `WHO has arrived?'}
	\ex[?]{
    \gll 	Ptala se, \textbf{kdo} \textbf{že} přijel. \label{kdozeembedded}\\
			asked.\textsc{sg.f} \textsc{refl} who that arrived.\textsc{sg.m}\\
	\glt 	`She asked who was said to have arrived.'}
	\z
\z

\noindent The sentence in \REF{kdoze} is an appropriate reaction to a statement such as `Peter arrived'. The sentence in \REF{kdozeembedded} is the embedded version thereof; its markedness stems from the fact that it is relatively difficult to find contexts in which an embedded echo is felicitous. As far as the status of \textit{že} is concerned, I follow \citet{kaspar2015} in assuming that this element is located in C;\footnote{As \citet{kaspar2015} shows, there is in fact more than one \textit{že} element in Czech, see also \citet{gruetskrabalova2012}; I will only concentrate on the declarative complementiser appearing in the clauses under scrutiny.} hence, its co-occurrence with the wh-element in SpecCP makes the doubly filled COMP effect possible.

Consider now the following polar questions:

%Example 12
\ea
	\ea[]{
	\gll Přijela Marie? \label{czechpolarmatrix}\\
		arrived.\textsc{sg.f} Mary\\
	\glt ‘Has Mary arrived?’
	}
	\ex[]{
	\gll Ptala se, \textbf{jestli} Marie přijela. \label{czechpolarembedded}\\
		asked.\textsc{sg.f} \textsc{refl} if Mary arrived.\textsc{sg.f}\\
	\glt ‘She asked if Mary arrived.’
	}
	\z
\z

\noindent As can be seen, the embedded polar question in \REF{czechpolarembedded} is introduced by \textit{jestli} `if', while its matrix interrogative counterpart in \REF{czechpolarmatrix} has no morphophonological marker.

The insertion of \textit{že} `that' into clauses with \textit{jestli} is impossible:


%Example 13
\ea [*]{
	\gll Ptala se, \textbf{jestli} \textbf{že} Marie přijela.\\
		 asked.\textsc{sg.f} \textsc{refl} if that Mary arrived.\textsc{sg.f}\\
	 \glt	`She asked if Mary arrived.'
    }
\z 

\noindent The elements \textit{že} and \textit{jestli} are in complementary distribution regarding their syntactic position (but not their function\footnote{This means that while they occupy the same position, C, in syntax, they do not have the same distribution and \textit{že} cannot introduce questions by itself:

%Foot note
\ea [*]{
	\gll {\normalfont Ptala} {\normalfont se,} {\normalfont \textbf{že}} {\normalfont Marie} {\normalfont přijela.}\\
		 asked.\textsc{sg.f} \textsc{refl} if Mary arrived.\textsc{sg.f}\\
	\glt `She asked if Mary arrived.'
    }
\z 

}); hence, since \textit{že} is in C, it can be concluded that \textit{jestli} is in C, too. This is in line with the etymology of \textit{jestli}, a grammaticalised form of the question particle \textit{li} and the verb `be': in Czech, if C is filled by the clitic -\textit{li}, the verb moves up to C to host the clitic (\citealt{schwabe2004}).

In addition to the constructions so far, it should be mentioned that wh-elements may appear in polar questions headed by \textit{jestli}, rendering an echo reading:

%Example 14
\ea
	\ea[]{
    \gll \textbf{Kdo} \textbf{jestli} přijel? \label{kdojestli}\\
			who if arrived.\textsc{sg.m}\\
	\glt 	`Did WHO arrive?'
    }
	\ex [*]{
    \gll Ptala se, \textbf{kdo} \textbf{jestli} přijel. \label{kdojestliembed}\\
			asked.\textsc{sg.f} \textsc{refl} who if  arrived.\textsc{sg.m}\\
	\glt 	`She asked about whom the question arose whether they arrived.'
    }
	\z 
\z 

\noindent The sentence in \REF{kdojestli} is an appropriate reaction to a question such as `Did Peter arrive?', and hence is an echo of a polar question.\footnote{The impossibility of embedding such an echo, as in \REF{kdojestliembed}, may well have pragmatic reasons, i.e. such a sentence is not felicitous in any context. Note that if the Czech pattern were an ordinary doubly filled COMP pattern, such as in (substandard) West Germanic, then \REF{kdojestliembed} should be grammatical and \REF{kdojestli} should be ruled out.} As can be expected, the insertion of \textit{že} `that' is again impossible:\footnote{Note that the impossibility of the combinations discussed in this paper is not merely due to their relative order: changing their relative order (e.g. \textit{že jestli}) results in an ungrammatical configuration, too.}

%Example 15
\ea
	\ea[*]{
    \gll \textbf{Kdo} \textbf{jestli} \textbf{že} přijel?\\
			who if that arrived.\textsc{sg.m}\\
	\glt 	`Did WHO arrive?'
    }
	\ex[*]{
    \gll Ptala se, \textbf{kdo} \textbf{jestli} \textbf{že} přijel.\\
			asked.\textsc{sg.f} \textsc{refl} who if that arrived.\textsc{sg.m}\\
	\glt 	`She asked about whom the question arose whether they arrived.'
    }
	\z 
\z 

\noindent Regarding the interrogative patterns in Czech, the following points can be established. First, doubly filled COMP effects are possible with \textit{že} `that' and with \textit{jestli} `if': both render echo questions (though these echo questions are licensed in two different kinds of context) and the elements \textit{že} and \textit{jestli} cannot occur together. Second, the insertion of the complementiser (in addition to the element in the specifier) is not attested in ordinary constituent questions. Third, the insertion of either complementiser (in addition to the wh-element) triggers an echo interpretation. Fourth, the complementiser is available in main clause echo questions, contrary to ordinary main clause questions, and in this way the echoed statement/question is surface-similar to an embedded clause, in line with the fact that it is dependent on a particular context in order to be felicitous.\footnote{Note that there are other instances of subordinating C-elements appearing in main clauses, as is the case for German \textit{ob} `if' in V-final main clause questions that are pragmatically distinct from ordinary questions, see e.g. \citet{zimmermann2013}. Naturally, the discussion of this issue would go far beyond the scope of the present paper.} This is contrary to what was seen in Germanic, where no echo interpretation is attested and where complementisers are not inserted in main clause constituent questions. Fifth, the patterns in Czech suggest that the clause type reflects the properties of the complementiser, not those of the wh-element (see the discussion in \sectref{sectionanalysis}); this is again contrary to Germanic, where the presence of a wh-element indicates that the clause is a true interrogative.

\section{Slovenian}
This section is going to overview the possible patterns in Slovenian main and embedded questions. I will show that doubling is possible in similar ways to what was attested in Czech; again, the resulting combinations are in part surface-similar to their Germanic counterparts, yet they are associated with a particular (echo) interpretation.

Just like in English and Czech, constituent questions in Slovenian contain an overt wh-element fronted to the left edge of the clause:

%Example 16
\ea
	\ea[]{
	\gll \textbf{Kdo} pride?\\
			who comes\\
	\glt 	`Who is coming?'\hfill(\citealt[13, ex. 9]{hladnik2010})
    }
	\ex[]{
	\gll Vprašal je, \textbf{kdo} pride. \label{kdoembeddedslovenian}\\
			asked.\textsc{sg.m} \textsc{aux.3sg} who comes\\
	\glt 	`He asked who was coming.'\hfill(based on \citealt[14, ex. 11]{hladnik2010}\footnote{As noted, the data are essentially taken from \citet{hladnik2010}; however, the translations have been changed in accordance with what my informants gave as more natural translations.})
    }
	\z 
\z 

\noindent I follow \citet{golden1997} and \citet{hladnik2010} in assuming that the wh-element moves to SpecCP.

Just like in Czech, the insertion of \textit{da} `that' is possible; this renders echo questions (see \citealt{hladnik2010}):\footnote{Just like in the Czech examples, the verb immediately follows the wh-element; however, this is not an effect of V2 in either language. In Slovenian, certain clitics, including auxiliaries, appear in a second position, as in \REF{meje}:

% foot note
\ea[]{
	\gll {\normalfont Deček,} {\normalfont katerega} {\normalfont sem} {\normalfont srečal} {\normalfont 			 včeraj,} {\normalfont me} {\normalfont je} {\normalfont prepoznal.} \label{meje}\\
			boy that \textsc{aux.1sg} met yesterday  me \textsc{aux.3sg} recognized\\
	\glt 	`The boy that I met yesterday, recognized me.'\hfill(\citealt[266]{marusic2008})
    }
\z

\noindent As can be seen, the clitic \textit{je} follows the element \textit{me}, and is hence the second element in the clause. However, as shown by \REF{kdodaje}, it appears that \textit{je} can follow both \textit{kdo} and \textit{da} in doubly filled COMP patterns:

% foot note
\ea
	\gll {\normalfont Kdo} {\normalfont da} {\normalfont je} {\normalfont pri\v{s}el?} \label{kdodaje}\\
			who that \textsc{aux.3sg} come.\textsc{sg.m}\\
	\glt `WHO came?'
\z

\noindent Since \textit{je} appears after the elements \textit{kdo} and \textit{da}, one might wonder whether \textit{kdo da} is a constituent or whether \textit{kdo} is in a higher clause. However, both options are unlikely: an element in the specifier cannot form a constituent with the C head, and postulating a higher clause to locate a single element would be highly problematic, too. I assume that \textit{kdo} is in SpecCP and \textit{da} in the C head of the same CP, whereby the two elements neither form a constituent nor are they located in different clauses. There is in fact no need to assume a strict surface second-position requirement on Slovenian clitics. As shown by \citet{marusic2008}, analyses assuming a fixed syntactic position such as C for clitics, as by \citet{goldenmilojevicsheppard2000}, face a number of problems and the relative position of the clitic should rather be considered phonological in nature (in line with general ``Wackernagel'' phenomena). In this case, the clitic naturally follows the element in the C head even if the specifier of the CP is filled by some additional element since there is no way of inserting the clitic in between the element in the specifier and the element in the head of a single CP projection. If the wh-element and the  complementiser were located in separate projections, one might expect the clitic to intrude, which is not the case. Note that, strictly speaking, the same holds even if one assumes a fixed syntactic position for the clitic (a projection below CP or another CP, resulting in a split CP) since the filling of the specifier in a higher projection does not influence the realisation of the clitic in some lower projection.}

%Example 17
\ea
	\ea []{
    \gll \textbf{Kdo} \textbf{da} pride? \label{kdoda}\\
			who that comes\\
	\glt	`WHO is coming?'\hfill(\citealt[13, ex. 9]{hladnik2010})
    }
	\ex[?]{
	\gll Vprašal je, \textbf{kdo} \textbf{da} pride. \label{kdodaembedded}\\
			asked.\textsc{sg.m} \textsc{aux.3sg} who that comes\\
	\glt	`He asked who was said to be coming.'\\\hfill(based on \citealt[14, ex. 11]{hladnik2010})
    }
	\z
\z

\noindent The sentence in \REF{kdoda} is an appropriate reaction to a statement such as `Peter is coming'; the sentence in \REF{kdodaembedded} shows the embedded version and is marked for pragmatic reasons, just as was the case for its Czech counterpart. Regarding the status of \textit{da}, I follow \citet{hladnik2010} in assuming that it is located in C; hence, when appearing together with a wh-element, (surface) doubly filled COMP effects are possible.\footnote{Again, one might wonder whether the wh-element is indeed in the same CP as the complementiser \textit{da}. In Slovenian, a null complementiser is licensed only if the wh-element is in the relevant specifier: it is not possible if the wh-phrase undergoes long distance movement, and in these cases \textit{da} is inserted, see \citet{golden1997}, \citet{marusic2008fdsl}. Hence, one might think that the doubly filled COMP effect in echo questions arises merely because the complementiser has to be overt if the wh-element is in a higher clause. However, as shown by \citet{mismastoappear}, echo questions in Slovenian are in fact possible even without \textit{da}, which indicates that the wh-element does not move out of the clause where it is base-generated.}

Consider now the following polar questions:\footnote{Again, I cannot examine the distribution of \textit{a} and \textit{če} beyond the constructions under scrutiny and will discuss only the differences within the given syntactic paradigm.}

%Example 18
\ea
	\ea[]{
	\gll \textbf{A} pride?\\
			\textsc{q} comes\\
	\glt 	`Is he coming?'\hfill(based on \citealt[15, ex. 12]{hladnik2010})
	}
	\ex[]{
	\gll Vprašal je, \textbf{če} pride. \label{slovenianpolarembedded}\\
			asked.\textsc{sg.m} \textsc{aux.3sg} whether comes\\
	\glt 	`He asked whether he was coming.'\\\hfill(based on \citealt[15, ex. 12]{hladnik2010})
    }
	\z
\z

\noindent As can be seen, a question particle -- \textit{a} or \textit{če} -- is licensed both in main clause and in embedded interrogatives. The insertion of \textit{da} `that' is possible in both cases and it renders an echo reading (cf. \citealt{hladnik2010}):

%Example 19
\ea
	\ea[]{
    \gll \textbf{A} \textbf{da} pride? \label{ada}\\
			\textsc{q} that comes\\
	\glt `Is it true that he is coming?'\hfill(based on \citealt[15, ex. 12]{hladnik2010})
    }
	\ex [?]{
	\gll Vprašal je, \textbf{če} \textbf{da} pride.\\
			asked.\textsc{sg.m} \textsc{aux.3sg} whether that comes\\
	\glt 	`He asked whether it was true that he was coming.'\\\hfill(based on \citealt[15, ex. 12]{hladnik2010})
    }
	\z
\z

\noindent The sentence in \REF{ada} is an appropriate reaction to a statement such as `He is coming'. Importantly, \textit{da} and \textit{a}/\textit{če} are not in complementary distribution, which suggests that \textit{a}/\textit{če} are not in C, contrary to Czech \textit{jestli}. Instead, in the given constructions they are rather operators located in SpecCP, similarly to English \textit{whether}.\footnote{Note that \textit{če} can appear in conditional clauses, too; however, the discussion of this falls outside the scope of the present paper.}

Finally, it must be mentioned that wh-elements may appear in polar questions; this renders an echo interpretation, similarly to what was observed in Czech. Note that the acceptability of these constructions in Slovenian is dependent on the dialect/idiolect, as also indicated by \citet{hladnik2010} in connection with all the doubling patterns, and this seems to be especially true in the case of the triple combination in \REF{kdoceda} below (this was not accepted as grammatical by my main informant).\footnote{Unfortunately, since the focus of \citet{hladnik2010} is relative clauses, the exact geographical distribution of the interrogative patterns cannot be recovered from his thesis, and it remains unclear whether the acceptability of \REF{polarecho} shows relatively clear regional differences or whether the differences hold rather between idiolects. As \citet[6--8]{hladnik2010} describes in the introduction, he conducted a larger pilot study of Slovenian dialects, whereby the focus was on syntactic doubling and on variation in dialects. Altogether, over 70 responses were collected from 55 test locations; further, since Slovenian speakers acquire a regional dialect as a rule, the data are quite reliable in that they reflect regional varieties rather than the standard language.} Consider the following examples:

%Example 20
\ea \label{polarecho}
	\ea[\%]{
    \gll \textbf{Kdo} \textbf{če} pride? \label{kdoce}\\
			who whether comes\\
	 \glt	`Is WHO coming?'\hfill(based on \citealt[15, ex. 13]{hladnik2010})
    }
	\ex[\%]{
    \gll \textbf{Kdo} \textbf{če} \textbf{da} pride? \label{kdoceda}\\
			who whether that comes\\
	\glt 	`Is it true that WHO is coming?'\hfill(based on \citealt[15, ex. 13]{hladnik2010})
    }
	\z
\z

\noindent The sentence in \REF{kdoce} is an appropriate reaction to a question such as `Is Peter coming?', and the sentence in \REF{kdoceda} is an appropriate reaction to a question such as `Is it true that Peter is coming?'. Crucially, in both sentences in \REF{polarecho}, the Q-element is \textit{če} and not \textit{a}, as opposed to ordinary main clause interrogatives.\footnote{As one of the reviewers informs me, this is true also if the clause is sluiced: the element \textit{kdo} can be followed by \textit{če} but not by \textit{a}. This is expected if sluiced clauses are derived from regular interrogatives. Note also that in cases like \REF{kdoce}, the wh-element may remain in situ, in line with the assumption that the movement involved here is not genuine wh-movement but rather focusing (which preferably involves fronting); see the discussion in \sectref{sectionanalysis}.} This indicates that the difference from ordinary questions is encoded morphosyntactically, too.\footnote{As was noted before, certain contexts license clauses that are surface-similar to ordinary embedded clauses, such as matrix questions with \textit{ob} `if' in German. The pattern in \REF{polarecho} again indicates that the particular echo constructions are discourse-dependent and cannot appear in the same environments as ordinary main clause questions.}

Regarding the interrogative patterns in Slovenian, the following points can be established. First, doubly filled COMP effects are possible with \textit{da} `that' and \textit{a}/\textit{če} `if'. Second, the complementiser (in addition to the element in the specifier) is not inserted in ordinary constituent questions and may be inserted in ordinary polar questions. Third, the insertion of either complementiser (in addition to the wh-element or the Q particle in the specifier) triggers an echo interpretation. Unlike Czech, the echo of a question (a ``double echo'' in \citealt{hladnik2010}) is possible in Slovenian (at least dialectally, see \REF{polarecho} above). Fourth, the complementiser is available in main clause echo questions, contrary to ordinary main clause questions, and in this way the echoed statement/question is surface-similar to an embedded clause, in line with the fact that it is dependent on a particular context in order to be felicitous. This is similar to Czech and contrary to what was seen in Germanic, where no echo interpretation is attested and where complementisers are not inserted in main clause constituent questions. Fifth, the patterns in Slovenian, just like in Czech, suggest that the clause type reflects the properties of the complementiser, not those of the wh-element (see \sectref{sectionanalysis}); this is again contrary to Germanic, where the presence of a wh-element indicates that the clause is a true interrogative.

\section{The analysis} \label{sectionanalysis}
The present paper investigates various patterns involving wh-elements, Q elements and finite subordinators in Germanic and in Slavic languages. In this section, I am going to overview the behaviour of these combinations first.

The combinations observed in Germanic are given in \tabref{tablegermanic}; these combinations are attested in embedded clauses only.

\begin{table}
\caption{Clause typing and Germanic doubly filled COMP}
\label{tablegermanic}
\begin{tabularx}{\textwidth}{llll}
\lsptoprule
\textbf{sequence} & \textbf{clause-typing feature} & \textbf{clause type} & \textbf{examples}\\
\midrule
WH Q FIN & [wh] & constituent question & \REF{wieofdat}\\
WH Q & [wh] & constituent question & --\\
WH FIN & [wh] & constituent question & \REF{whichbookthat}\\
WH & [wh] & constituent question & \REF{whichbook}\\
Q FIN & [Q] & polar question & \REF{whetherthat}\\
Q & [Q] & polar question & \REF{if}\\
\lspbottomrule
\end{tabularx}
\end{table}

\noindent As can be seen, the type of the clause always matches the leftmost element in the linear sequences. That is, once a wh-element is inserted, the clause can only be a constituent question. If there is no wh-element but a Q element is present, the clause can only be a polar interrogative. Naturally, a clause is always typed by the C head but certain features on the C head are checked off by elements moving to the specifier, as in wh-questions (yet the wh-elements do not themselves type the clause).

The combinations observed in Slavic (Czech and Slovenian) are given in \tabref{tableslavic}; these combinations are attested both in embedded and in matrix clauses.

\begin{table}
\caption{Clause typing and Slavic doubly filled COMP}
\label{tableslavic}
\begin{tabularx}{\textwidth}{llll}
\lsptoprule
\textbf{sequence} & \textbf{clause-typing feature} & \textbf{clause type} & \textbf{examples}\\
\midrule
WH Q FIN & [FIN] & declarative, double echo & \REF{kdoceda}\\
WH Q & [Q] & polar question, echo & \REF{kdojestli}, \REF{kdoce}\\
WH FIN & [FIN] & declarative, echo & \REF{kdoze}, \REF{kdoda}\\
WH & [wh] & constituent question & \REF{kdoembedded}, \REF{kdoembeddedslovenian}\\
Q FIN & [FIN] & declarative, echo & \REF{ada}\\
Q & [Q] & polar question & \REF{czechpolarembedded}, \REF{slovenianpolarembedded}\\
\lspbottomrule
\end{tabularx}
\end{table}

\noindent As indicated, the type of the clause always matches the rightmost element in the linear sequences, contrary to the Germanic pattern. That is, once the finite complementiser is inserted, the clause is typed as a declarative, but the presence of the interrogative elements leads to an echo interpretation. Consequently, there is a split between form and function that is not attested in Germanic. If there is no finite complementiser but a Q element is present, the clause is a polar interrogative, but the presence of the wh-element leads to an echo interpretation. Again, a clause is always typed by the C head but the Slavic pattern is crucial because the insertion of an operator into the specifier does not involve feature checking with the head: the C head lacks the features associated with the operator. Ordinary questions are possible only when a single interrogative element is present.

Regarding Germanic doubly filled COMP patterns, the following can be established. On the one hand, the movement of the wh-operator or the insertion of the polar operator into SpecCP take place for clause-typing reasons and can be thus drawn back to question semantics and to the requirement on feature checking with C. On the other hand, the insertion of the finite complementiser takes place in order to lexicalise [fin] in C.

By contrast, regarding Slavic doubly filled COMP patterns, the following can be established. On the one hand, the insertion of the operator (either a wh-operator or the polar operator) into SpecCP takes place due to an [\textsc{edge}] feature on the C head containing the elements introducing the echoed question, and there is no feature checking with C (given that there is no interrogative feature to be checked, as echo questions are not typed as interrogatives, see \citealt[363]{boskovic2002}).\footnote{Note that the WH Q sequence is special in this respect because the clause is typed as a polar interrogative by the Q-element, just as the declarative clause is typed as declarative by the relevant element in C. However, this configuration is also regular in the sense that the wh-element itself does not type the clause. Importantly, there is no incompatibility between an interrogative clause type and an echo reading, provided that the interrogative is typed independently of the echoing wh-phrase.} On the other hand, the insertion of the complementisers into C takes place because they type the echoed clause.

As far as echo questions are concerned, I assume that they are not true questions and are closer to focus constructions (cf. \citealt{boskovic2002}, \citealt{artsein2002}). This is in line with the analysis of \citet{boskovic2002}, who claims that the fronting of echoed wh-phrases, as well as that of non-first wh-phrases in multiple fronting constructions, are independent of a strong [wh] feature on C. Accordingly, \citet[359--364]{boskovic2002} analyses the relevant constructions as instances of focus fronting. Hence, the interrogative interpretation arises locally, similarly to English, where there is no wh-movement in echo questions, indicating that there is no [wh] feature on the C head (cf. \citealt[363]{boskovic2002}).

We saw earlier that Slavic languages may allow embedded echo questions, even though these configurations are marked compared to their matrix counterparts. That is, the clause can be taken by a predicate taking interrogative complements (e.g. \textit{ask}), which is normally possible if the clause is typed as [wh]. I assume that in echo clauses this is related to feature percolation: namely, the features of the element in the specifier can percolate up and hence the interrogative property, which is interpretable on the wh-element itself, is visible to the matrix predicate.\footnote{The idea of feature percolation is well known in the syntactic literature and is subject to debates concerning its exact application and restrictions. As described by \citet[5--7]{heck2008}, pied-piping has been treated in terms of feature percolation of the wh-feature since \citet[273]{chomsky1973}, whereby the wh-feature projects to the DP-level and then percolates up to the PP level, that is, it is allowed to cross a phrase boundary. Essentially the same is proposed here in terms of the wh-feature percolating up to the CP, without causing changes in the C head itself (just like in the case of PPs, where feature percolation does not change the properties of the P). Naturally, this again raises the question how far a feature is allowed to percolate, the discussion of which clearly cannot be carried out in the present paper.} However, there is no percolation downwards, and hence the echoed clause itself is not affected.

Consider now the structures for WH FIN sequences in Germanic (here: English) and Slavic (here: Czech), respectively:

%Example 21
% \newpage
\begin{multicols}{2}
\ea
\ea \label{treewhichbookthat}
\begin{forest} baseline, qtree
[CP
	[\textit{which book}\textsubscript{{[}wh{]}}]
	[C$'$
		[C\textsubscript{{[}fin{]},{[}wh{]}}
			[\textit{that}\textsubscript{{[}fin{]}}]
		]
		[\ldots]
	]
]
\end{forest}
\ex \label{treekdoze}
\begin{forest} baseline, qtree
[CP
	[\textit{kdo}\textsubscript{{[}wh{]}}]
	[C$'$
		[C\textsubscript{{[}fin{]}}
			[\textit{že}\textsubscript{{[}fin{]}}]
		]
		[\ldots]
	]
]
\end{forest}
\z
\z

\end{multicols}

\noindent As can be seen, both configurations result in a doubly filled COMP pattern. However, the C is specified as [wh] only in \REF{treewhichbookthat}, which is a true interrogative, while the Slavic pattern in \REF{treekdoze} is an echo question. The complementiser is inserted in certain dialects in Germanic to lexicalise [fin], while Slavic complementisers are inserted to type the clause.\footnote{Note that this does not mean that the complementiser is always overt. Declarative complementisers tend to have zero counterparts cross-linguistically and the same applies to e.g. \textit{že} and \textit{da}, too. This means that echo questions are possible without the insertion of an overt \textit{že}, too. This option has not been discussed in detail here because the present paper is devoted to doubling patterns in the CP-domain.}

Consider now the structures for Q FIN sequences in Germanic (here: English) and Slavic (here: Slovenian), respectively:

%Example 22
\begin{multicols}{2}
\ea
\ea \label{treewhetherthat}
\begin{forest} baseline, qtree
[CP
	[\textit{whether}\textsubscript{{[}Q{]}}]
	[C$'$
		[C\textsubscript{{[}fin{]},{[}Q{]}}
			[\textit{that}\textsubscript{{[}fin{]}}]
		]
		[\ldots]
	]
]
\end{forest}
\ex \label{treeceda}
\begin{forest} baseline, qtree
[CP
	[\textit{če}\textsubscript{{[}Q{]}}]
	[C$'$
		[C\textsubscript{{[}fin{]}}
			[\textit{da}\textsubscript{{[}fin{]}}]
		]
		[\ldots]
	]
]
\end{forest}
\z
\z

\end{multicols}

\noindent Again, the surface doubling configuration results in doubly filled COMP patterns in both cases. The C is specified as interrogative, this time as [Q] 
%$[Q]$
, only in Germanic, see \REF{treewhetherthat}, while in Slavic the question is merely echo, see \REF{treeceda}. Further, the complementiser is inserted in certain dialects in Germanic to lexicalise [fin], while Slavic complementisers are inserted to type the clause.

Finally, consider the structures for WH Q FIN sequences, in Germanic (here: Dutch) and Slavic (here: Slovenian), respectively:

%Example 23
%\newpage
\begin{multicols}{2}
\ea
\ea \label{treewieofdat}
\begin{forest} baseline, qtree
[CP
	[\textit{wie}\textsubscript{{[}wh{]}}]
	[C$'$
		[C\textsubscript{{[}fin{]},{[}wh{]}}
			[$\emptyset$]
		]
		[CP
			[\textit{of}\textsubscript{{[}Q{]}}]
			[C$'$
				[C\textsubscript{{[}fin{]},{[}wh{]}}
					[\textit{dat}\textsubscript{{[}fin{]}}]
				]
				[\ldots]
			]
		]
	]
]
\end{forest}
\ex \label{treekdoceda}
\begin{forest} baseline, qtree
[CP
	[\textit{kdo}\textsubscript{{[}wh{]}}]
	[C$'$
		[C\textsubscript{{[}fin{]}}
			[$\emptyset$]
		]
		[CP
			[\textit{če}\textsubscript{{[}Q{]}}]
			[C$'$
				[C\textsubscript{{[}fin{]}}
					[\textit{da}\textsubscript{{[}fin{]}}]
				]
				[\ldots]
			]
		]
	]
]
\end{forest}
\z
\z

\end{multicols}

\noindent As can be seen, the CP is split in both cases,\footnote{In the model adopted here, based on \citet{bacskaiatkari2018sardis}, the CP is split if certain features have to project further to be checked off but there is no predefined cartographic template in the sense of \citet{rizzi1997}. However, the assumption that there can be multiple CPs (similarly to VPs) is widespread in the literature.} yet the C head is specified as [wh] only in the Germanic case, see \REF{treewieofdat}, while the Slovenian configuration represents an echo, see \REF{treekdoceda}. In \REF{treewieofdat}, the [wh] feature of the lower C head is not checked off, since the polar operator in SpecCP is merely [Q], a subset of [wh]; hence, the CP projects further. In \REF{treekdoceda}, there is no feature checking associated with either of the operators; they are inserted to render the echo reading. Again, the finite complementiser is inserted in certain dialects in Germanic to lexicalise [fin], while Slavic complementisers are inserted to type the clause.

The differences between Germanic and Slavic essentially go back to differences in the requirement of lexicalising [fin]: since this requirement is present in Germanic, the finite complementiser is inserted merely due to this requirement, while its appearance in Slavic doubly filled COMP constructions contributes to the echo reading by way of typing the clause merely as [fin] but not as [wh] or [Q].

\section{Conclusion}
This paper investigated doubly filled COMP effects in Germanic and Slavic (more precisely, Czech and Slovenian). It was shown that while the two language groups represent similar surface configurations, they differ crucially in the distribution and the interpretation of these structures. In Germanic, doubly filled COMP arises due to a requirement on filling a C head specified as [fin]; this is in line with the general properties of V2 (e.g. in German) and T-to-C (English). Importantly, the insertion of the finite complementiser takes place only in embedded questions and it brings interpretive differences from complementiser-less clauses. In Slavic, doubly filled COMP arises in echo questions and the complementiser is inserted to type the clause, while the element in the specifier does not check off its features with the head. The insertion of the complementiser involves an important interpretive difference from complementiser-less clauses, since the lack of the complementiser is associated with ordinary questions, while the presence of the complementiser triggers an echo interpretation. Taking all this into account, it can be concluded that the differences between Germanic and Slavic doubly filled COMP structures can be accounted for in a principled way.




\section*{Abbreviations}

\begin{tabularx}{.5\textwidth}{@{}lQ@{}}
3&third person\\
\textsc{aux}&auxiliary\\
\textsc{f}&feminine\\
\textsc{m}&masculine\\
\end{tabularx}%
\begin{tabularx}{.5\textwidth}{@{}lQ@{}}
\textsc{ptcp}&participle\\
\textsc{q}&question particle/marker\\
\textsc{refl}&reflexive\\
\textsc{sg}&singular\\
\end{tabularx}



\section*{Acknowledgements}
This research was funded by the German Research Fund (DFG), as part of my project ``The syntax of functional left peripheries and its relation to information structure'' (BA 5201/1-1). I would like to thank Jiri Kaspar,  Mojmír Dočekal and Radek Šimík (Czech) and Moreno Mitrović (Slovenian) for their indispensable help with the data. I also owe many thanks to the audience of FDSL-12, in particular to Petra Mišmaš and Roland Meyer. Finally, I am highly grateful to the reviewers of my paper for their insightful and constructive questions and suggestions.

\printbibliography[heading=subbibliography,notkeyword=this]

\end{document}
