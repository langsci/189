\documentclass[output=paper,  modfonts,  newtxmath,  hidelinks		  ]{langscibook}

\title{How factive is the perfective?\newlineCover On the interaction between perfectivity and factivity in Polish}



\author{Karolina Zuchewicz\affiliation{Leibniz-Zentrum Allgemeine Sprachwissenschaft \&\\Humboldt-Universität zu Berlin}}


\abstract{This paper aims to provide evidence for a systematic correlation between the perfective aspect of the matrix verb and the factive interpretation of embedded object sentences in Polish. Embedding by perfective matrix verbs makes propositions systematically ‘more factive’ than embedding by their imperfective counterparts. The strength of the inference depends on the semantic class the verb belongs to. The perfective operator introduces a nearly undefined truthfulness feature, which is specified as factive, veridical or reliable depending on the relation between the truth of the proposition expressed by the embedded clause and the event described by the matrix verb.

\keywords{perfectivity, factivity, veridicality, presupposition, entailment, implicature}}

\begin{document}
\maketitle
\shorttitlerunninghead{How factive is the perfective?}

\section{Introduction}\label{21:intro}

There is no one simple way to define factivity, and especially the truth-related inferences in general. In this paper, I will adopt assumptions which can be used to describe perfectivity-dependent truthfulness in Polish. It should be pointed out that the whole spectrum of the literature available is much broader.

According to \cite{KiparskyKiparsky1970}, a verb $V$ that takes a that-clause $p$ is called \textsc{factive} if asserting $\mathit{V\!p}$ presupposes the truth of the complement $p$ (but see also \citealt{Karttunen1971} for a discussion of the presuppositional account). Following \citet[101]{Egré2008}, a verb $V$ is called \textsc{veridical} if it entails the truth of its complement when used in the positive declarative form, more precisely if it satisfies the scheme $\mathit{V\!p}{}\rightarrow{}p$ for all $p$, where $p$ is a that-clause.

I will refer to the first option as \textsc{truth presupposition} (factivity in a common sense). It holds when the inference remains under negation or after the insertion of a modal adverbial (for a semantic definition of presupposition, see \citealt{Strawson1950}). Truth presupposition concerns for instance perfective \textit{przewidzieć}, as can be seen in the following examples. ‘$\gg$’ marks presupposition.\footnote{All embedded verbs are marked for the imperfective aspect and used in the present tense in order to exclude the influence of perfectivity and past tense morphology within the subordinate clause on the truth inferences observed.}

%Example 1
\ea\label{21:1} 
\gll Ola nie / prawdopodobnie przewidziała, że Marek boi się duchów.\\
    Ola \textsc{neg} {} probably predicted.\textsc{pfv} that Marek fears.\textsc{ipfv} \textsc{refl} ghost.\textsc{pl}\\
\glt	 `Ola did not predict / probably predicted that Marek fears ghosts.'\newline$\gg$ Marek fears ghosts.
\z

%Example 2
\ea\label{21:2}
\gll Ola nie / prawdopodobnie przewidywała, że Marek boi się duchów.\\  
    Ola \textsc{neg} {} probably predicted.\textsc{ipfv} that Marek fears.\textsc{ipfv} \textsc{refl} ghost.\textsc{pl}\\
\glt	 `Ola was not predicting / was probably predicting that Marek fears ghosts.'\newline${}\ngg{}$Marek fears ghosts.
\z

\noindent Examples \REF{21:1} and \REF{21:2} consist of an aspectual minimal pair exhibiting complementary behavior of the feature [$\pm$perfective] with respect to the enforcing of a factive interpretation of their complement sentences. Whereas the perfective variant presupposes the truth of its sentential argument, the imperfective one does not. After the insertion of a sentence negation or a modal adverbial, \REF{21:1} implicates that Marek fears ghosts. Sentence \REF{21:2} only says that Ola was guessing / tried to predict that Marek fears ghosts, but it leaves it open whether she was correct or not.

The second option will be called \textsc{truth entailment}. Truth entailment results in an occurrence of a veridical meaning of the proposition expressed by the subordinate clause. Here, the inference is present in affirmative sentences, but it does not project. We can find it for example in the perfective \textit{potwierdzić}. I will use ‘$\rightarrow$’ to mark entailment.

%Example 3
\ea\label{21:3}
\gll Komisarz potwierdził, że Marek boi się duchów.\\
    commissioner confirmed.\textsc{pfv} that Marek fears.\textsc{ipfv} \textsc{refl} ghost.\textsc{pl}\\
\glt	`The commissioner confirmed that Marek fears ghosts.'\newline$\rightarrow$ Marek fears ghosts.
\z

%Example 4
\ea\label{21:4}
\gll Komisarz potwierdzał, że Marek boi się duchów.\\  
   	commissioner confirmed.\textsc{ipfv} that Marek fears.\textsc{ipfv} \textsc{refl} ghost.\textsc{pl}\\
\glt	`The commissioner was confirming that Marek fears ghosts.'\newline$\nrightarrow$  Marek fears ghosts.
\z

\noindent Truth entailment can be found in \textit{potwierdzić} and is absent in \textit{potwierdzać}; where\-as it seems to follow from \REF{21:3} that Marek fears ghosts, \REF{21:4} states that this is a possible, but not an obligatory interpretation. The inference presented in \REF{21:3} does not project, which excludes it from being a presupposition. Consider \REF{21:5}.

%Example 5
\ea\label{21:5}
\gll Komisarz nie / prawdopodobnie potwierdził, że Marek boi się duchów.\\
    commissioner \textsc{neg} {} probably confirmed.\textsc{pfv} that Marek fears.\textsc{ipfv} \textsc{refl} ghost.\textsc{pl}\\
\glt	 `The commissioner did not confirm / probably confirmed that Marek fears ghosts.'\newline$\nrightarrow$ Marek fears ghosts.
\z

\noindent The third and weakest option is the \textsc{truth implicature} (see also \citealt{Hacquard2006} for the so-called actuality implicature, as illustrated in \REF{21:11}). I will use this term to refer to an inference which cannot be captured by factivity or veridicality (it is clearly pragmatic, since it can be canceled). Here, the proposition embedded under a perfective communication verb is taken for granted due to the reliability of the sentence subject (cf. \citealt{Schlenker2010} for the factivity of announcements). The same proposition embedded under a particular imperfective counterpart is neutral with respect to the reliability condition. Consider examples \REF{21:6} and \REF{21:6b}. I will use ‘$\rightlsquigarrow$’ to mark implicature.
 
\ea
	\ea[]{
	\gll Ela powiedziała, że jest w pracy.\\
   		Ela said.\textsc{pfv} that is.\textsc{ipfv} at work\\
\glt	 		`Ela said that she is at work.'\newline$\rightlsquigarrow$ Ela is at work.\label{21:6}
   }
	\ex[]{
	\gll Ela mówiła, że jest w pracy.\\
   		Ela said.\textsc{ipfv} that is.\textsc{ipfv} at work\\
\glt			`Ela was saying that she is at work.'\newline$\nrightlsquigarrow$ Ela is at work.\label{21:6b}
   }
	\z
\z


\noindent In the perfective variant \REF{21:6}, the speaker takes the truth of what the sentence subject said for granted; Ela is considered reliable and she is expected to tell the truth. In the imperfective variant \REF{21:6b}, the speaker does not want to commit herself to the truth of the proposition. It is left open whether the speaker considers the sentence subject reliable. As a result, there is no implicature that Ela is at work. The reliability effect seems to correlate with fulfilling all the parts of the speech act (see \sectref{21:trimpl}), which is necessarily the case when using the perfective and which does not have to be the case when using the imperfective (see \citealt{CohenKrifka2014,Krifka2015} for commitment space semantics).

We could refer to the abovementioned inferences as a \textsc{truthfulness scale}. The strongest inference -- truth presupposition -- represents the highest value on that scale, while truth implicature stands for the lowest one. Between them there is the medium strength inference -- truth entailment. A more detailed classification may be developed after more conclusive research has been done.

Perfectivity-dependent truthfulness needs to be distinguished from the truthfulness of inherently factive imperfectives, where the perfectivizing operation only results in the specification of a temporal boundary of an event, for instance: \textit{żałować} `regret.\textsc{ipfv}’ vs. \textit{pożałować} `start regretting.\textsc{pfv}’ (cf. \citealt{Egré2008} for an interesting discussion about \textit{regret}, though), unless the meaning of the derivate becomes non-compositional (\textit{wiedzieć}
`know.\textsc{ipfv}’ + factive vs. \textit{powiedzieć} `say.\textsc{pfv}’ + non-factive). In contrast, all truth inferences which originate from the perfective do not occur in the case of the respective imperfective counterparts.\par At this point, I would like to make an important remark concerning the tense of the matrix verb. I use the past tense in all examples, because it is available for any verbal stem regardless of the aspectual marking. The present tense morphology results in future reference in the case of the perfective, whereas both present tense and the periphrastic future construction are available for the imperfective. Because the analyzed sentences are supposed to be minimal pairs (differing only in the aspectual marking on the matrix verb), using the past tense was the only option.\par In this paper, I will examine different verbs falling into class 1 (truth presupposition), class 2 (truth entailment) and class 3 (truth implicature). I provide an account of perfectivity-dependent truth inferences in Polish, which will be presented in \sectref{21:perfdep}. Before coming to that, I will briefly discuss the influence of aspect on the interpretation of nominal arguments, which serves as a starting point for an investigation of the correlation between the perfectivity of the matrix verb and the interpretation of complement sentences.

\section{Aspect and the interpretation of nominal arguments}
It has been pointed out by \cite{Wierzbicka1967} that in perfective sentences in Polish the direct object is interpreted as definite, while in imperfective ones it is understood as indefinite. Consider example \REF{21:8}.

%Example
\ea\label{21:8}
\gll On zjadł / jadł orzechy.\\
   he ate.\textsc{pfv} {} ate.\textsc{ipfv} nut.\textsc{pl}\\
\glt	`He ate all of the nuts / was eating (some) nuts.'
\z

\noindent In the case of \textit{zjadł}, the reference is to a definite group of entities -- the nuts. The object is completely affected by the verbal process (as a result, there are no more nuts left). In contrast, neither the definite nor the totality reading is enforced when using \textit{jadł}. Here, the partitive interpretation is available amongst others, corresponding to `some of the nuts'.\par However, \cite[128]{Filip2005} shows that perfective aspect does not always require that bare nominal arguments in its scope refer to one whole and specific individual (consider for instance the perfective Czech and Polish equivalents of the English verb \textit{bring}). That means that not only aspect, but also verb semantics and especially the thematic relation between the nominal object and the verb determine the referential properties of the entire predicate.\par 
The crucial point is that the perfective operator can take scope over both the matrix verb and its nominal complement. A formal analysis of this correlation has been developed by \cite{Krifka1989a,Krifka1989b,Krifka1992,Krifka1998}. Different theoretical implementations are possible; because it is not the main focus of this paper, I will not discuss them in greater detail.\par
According to Krifka, complex verbal expressions (verb plus direct object) have either a cumulative or a quantized reference. We can define them in terms of the sum operation: $x$ $\sqcup$ $y$ ‘the sum of $x$ and $y$’. For example, the sum of two events of ‘eating grapes’ still yields an event of ‘eating grapes’. The predicate ‘eating grapes’ has a cumulative reference – we can apply it not only to the single events, but also to the sum of them. In contrast, the joining of two events of ‘eating two grapes’ can no longer be described with ‘eating two grapes’, because ‘eating two grapes’ plus ‘eating two grapes’ does not equal ‘eating two grapes’. The predicate ‘eating two grapes’ has a quantized reference – we can apply it to the single events, but not to the sum of them. Apart from the sum operation, the proper part relation can be defined: $x$ $\sqsubset$ $y{}\leftrightarrow{}x$ $\sqsubseteq$ $y$ and $x\neq y$. For example, there is no proper part of an event ‘eating two grapes’ which is an event of ‘eating two grapes’. This illustrates another property of the quantized reference.

Krifka assumes that the perfective operator presupposes the quantization of the entire predicate, whereas the imperfective operator requires its cumulativity. This correlation primarily (but not exclusively, cf. \citealt{Krifka1998}) holds for predicates which allow mapping of objects to events and vice versa (the so-called homomorphism of objects to events). Certainly, it cannot be considered a 1:1 relationship (cf. \citealt{Filip1996,Filip2005} or \citealt{Borik2006}). In Polish, the verbal predicate marked with the perfective aspect is quantized iff the whole verbal complex receives a telic interpretation, particularly in the case of predicates with nominal objects that are incremental themes. On the other hand, we get the combination of features [$+$perfective] and [$–$telic] after adding the delimitative prefix \textit{po}- to the imperfective stem; in these cases the predicate is to be interpreted as atelic despite the perfective marking on the verb. Thus, the following generalization holds for Polish: telicity implies quantization, but perfectivity does not imply telicity (see also \citealt{Gehrke2008}). As will become clear later, the truth inference of a sentential complement is triggered by perfectivity.

\section{Cross-linguistic evidence for the interaction between perfectivity and factivity}

The influence of the perfective aspect of a matrix verb on the factive interpretation of complement clauses has already been observed. \cite{Hacquard2006} shows that both actuality entailment and actuality implicature can be found in some modal constructions in French, when a modal is marked with the perfective.\par Actuality entailment refers to the uncancelable inference stating that the proposition expressed by the complement clause holds in the actual world.\footnote{\cite{Bhatt1999} observed the correlation between perfectivity marked on ability modals and the presence of the actuality entailment in Greek and Hindi.} Consider \REF{21:10}, adapted from \cite[21]{Hacquard2006}.
% Example

\ea\label{21:10}
\gll Jane {\#}\hspace{-2pt} a pu / pouvait soulever cette table, mais elle ne l’a pas soulevée. \\
    Jane {} \textsc{aux} could.\textsc{pfv} {} could.\textsc{ipfv} lift this table but she \textsc{neg} it.\textsc{aux} \textsc{neg} lift\\
\glt	`Jane could lift this table, but she did not lift it.'
\z

\noindent Example \REF{21:11} demonstrates an actuality implicature (adapted from ibid. 16).

% Example

\ea\label{21:11}
\gll Darcy a eu / avait la possibilité de rencontrer Lizzie. \\
	Darcy \textsc{aux} had.\textsc{pfv} {} had.\textsc{ipfv} the possibility to meet Lizzie\\
\glt `Darcy had the possibility to meet Lizzie.'
\z

\noindent When used with the perfective, \REF{21:11} strongly suggests (but does not entail) that Darcy did meet Lizzie.

The correlation between perfectivity and factivity can also be seen in Hungarian; it concerns the influence of embedding verbs of saying on the interpretation of their sentential complements. Whereas \textit{megmond} ‘say.\textsc{pfv}’ requires the argument to be true, \textit{mond} ‘say.\textsc{ipfv}’ does not (see \citealt{Kiefer1986}). Even though aspect is not grammaticalized in Hungarian (it is not obligatory for every verb to have its (im)perfective twin), informal investigations among speakers show that we can observe clear aspect-dependent differences with respect to the truthfulness of propositions embedded under verbs marked as perfective.

In the next section I am going to present Polish data showing a systematic interaction between perfectivity and truthfulness. In Polish, the category of aspect is fully grammaticalized, which allows us to take a closer look at the abovementioned dependency. 

\section{Aspect-dependent truth inferences in Polish}\label{21:aspinf}
\subsection{Case 1: {T}ruth presupposition}

One group of verbs where the truth presupposition of the perfective can be found is verbs of guessing.\footnote{The strength of the inference may also depend on aktionsart. For example, a resultative verb \textit{wyczuć} `sense.\textsc{pfv}’ is factive, whereas the inchoative \textit{poczuć} `start feeling.\textsc{pfv}’ is not (a similar observation holds for Czech, Radek \v{S}imík, p.c.). It seems that inchoativity does not give rise to factivity, but to a weak truth implicature.} From \REF{21:13} it follows that the proposition from the embedded clause – Marek fears ghosts – is true. Example \REF{21:14} demonstrates that this inference projects, i.e. it remains under negation and after the insertion of a modal adverbial.

% Example
\ea\label{21:13}
\gll Jan zgadł / wyczuł, że Marek boi się duchów.\\
Jan guessed.\textsc{pfv} {} sensed.\textsc{pfv} that Marek fears.\textsc{ipfv} \textsc{refl} ghost.\textsc{pl}\\
\glt	`Jan guessed that Marek fears ghosts.'\newline$\gg$ Marek fears ghosts.
\z

% Example
\ea\label{21:14}
\gll Jan nie / prawdopodobnie zgadł / wyczuł, że Marek boi się duchów.\\
Jan \textsc{neg} {} probably guessed.\textsc{pfv} {} sensed.\textsc{pfv} that Marek fears.\textsc{ipfv} \textsc{refl} ghost.\textsc{pl}\\
\glt	`Jan did not guess / probably guessed that Marek fears ghosts.'\newline$\gg$ Marek fears ghosts.
\z

\noindent Contrary to this, no such inference appears with the particular imperfective counterparts. Example \REF{21:15} shows that there is no entailment, let alone presupposition, that Marek fears ghosts when the subordinate clause is embedded under the imperfective variants of `guess’ / `sense’.

% Example
\ea\label{21:15}
\gll Jan zgadywał / wyczuwał, że Marek boi się duchów.\\
Jan guessed.\textsc{ipfv} {} sensed.\textsc{ipfv} that Marek fears.\textsc{ipfv} \textsc{refl} ghost.\textsc{pl}\\
\glt	`Jan supposed that Marek fears ghosts.'\newline$\nrightarrow$  Marek fears ghosts.
\z

\noindent As expected, the truth inference is also absent under negation and after the insertion of a modal adverbial. Consider \REF{21:16}.

% Example
\ea\label{21:16}
\gll Jan nie / prawdopodobnie zgadywał / wyczuwał, że Marek boi się duchów.\\
Jan \textsc{neg} {} probably guessed.\textsc{ipfv} {} sensed.\textsc{ipfv} that Marek fears.\textsc{ipfv} \textsc{refl} ghost.\textsc{pl}\\
\glt	`Jan did not suppose / probably supposed that Marek fears ghosts.'\newline$\nrightarrow$  Marek fears ghosts.
\z

\noindent Examples \REF{21:15} and \REF{21:16} leave it open whether it is true that Marek fears ghosts. Other members of this class are: \textit{odkryć, odkrywać} `discover’, \textit{rozgryźć, rozgryzać} `figure out’, and \textit{rozpoznać, rozpoznawać} `identify’.

\subsection{Case 2: {T}ruth entailment}

Many perfective matrix verbs show an implicative behavior with respect to the truth inference of the proposition from the subordinate clause. For instance, verbs of proving seem to entail that their sentential argument is true, which can be seen in \REF{21:17}. \textit{Udowodnić} and \textit{wykazać} are much stronger in their veridicality than \textit{pokazać} however.

% Example
\ea\label{21:17}
\gll Jan udowodnił / wykazał / pokazał, że Marek boi się duchów.\\
Jan proved.\textsc{pfv} {} revealed.\textsc{pfv} {} showed.\textsc{pfv} that Marek fears.\textsc{ipfv} \textsc{refl} ghost.\textsc{pl}\\
\glt	`Jan proved / revealed / showed that Marek fears ghosts.'\newline$\rightarrow$ Marek fears ghosts.
\z

\noindent Interestingly, this inference is apparently cancelable in particular contexts. Consider \REF{21:18} (cf. \citealt[74]{AnandHacquard2014}).

% Example
\ea\label{21:18}
\gll Jan udowodnił Basi, że Marek boi się duchów, jednak Krzysiek w to wątpi.\\
Jan proved.\textsc{pfv} Basia.\textsc{dat} that Marek fears.\textsc{ipfv} \textsc{refl} ghost.\textsc{pl} but Krzysiek in this doubts\\
\glt	`Jan proved to Basia that Marek fears ghosts, but Krzysiek doubts that.'\newline$\nrightarrow$  Marek fears ghosts.
\z

\noindent All the predicates in \REF{21:17} allow an overt experiencer, which makes veridicality questionable. \REF{21:18} says that Jan succeeded in convincing Basia that Marek fears ghosts, but he did not manage to convince Krzysiek. As a result, the lexical entry of the matrix predicate corresponds more to \textit{convince} than to \textit{prove}.

The ‘weak entailment’ from \REF{21:17} does not project under negation or after the insertion of a modal adverbial, which can be seen in \REF{21:19}.

% Example
\ea\label{21:19}
\gll Jan nie / prawdopodobnie udowodnił / wykazał / pokazał, że Marek boi się duchów.\\
Jan \textsc{neg} {} probably proved.\textsc{pfv} {} revealed.\textsc{pfv} {} showed.\textsc{pfv} that Marek fears.\textsc{ipfv} \textsc{refl} ghost.\textsc{pl}\\
\glt	`Jan did not prove / reveal / show / probably proved / revealed / showed that Marek fears ghosts.'\newline$\nrightarrow$  Marek fears ghosts.
\z

\noindent Example \REF{21:19} only says that Jan did not succeed / that Jan probably succeeded in providing arguments for Marek’s fear of ghosts, but it leaves it open whether the complement sentence is true or not.

We have just seen that the weak truth entailment in the case of perfective verbs of proving can disappear in particular contexts, especially after an overt realization of an experiencer. Furthermore, the significance or trustworthiness of the authority also plays a role in acknowledging a complement proposition as veridical. No projection pattern can be observed, which means that we are not dealing with a presupposition here.

Particular imperfective forms lack any kind of truth-contributing potential. Consider example \REF{21:20} for affirmative sentences.

% Example
\ea\label{21:20}
\gll Jan udowadniał / wykazywał / pokazywał, że Marek boi się duchów.\\
Jan proved.\textsc{ipfv} {} revealed.\textsc{ipfv} {} showed.\textsc{ipfv} that Marek fears.\textsc{ipfv} \textsc{refl} ghost.\textsc{pl}\\
\glt	`Jan was proving / revealing / showing that Marek fears ghosts.'\newline$\nrightarrow$  Marek fears ghosts.
\z

\noindent Example \REF{21:20} asserts that Jan was trying to prove / reveal / show that Marek fears ghosts, but it does not make any statement about the final results of Jan’s investigations. As expected, no truth inference can be found under negation or after the addition of a modal adverbial, which can be seen in \REF{21:21}.

% Example
\ea\label{21:21}
\gll Jan nie / prawdopodobnie udowadniał / wykazywał / pokazywał, że Marek boi się duchów.\\
Jan \textsc{neg} {} probably proved.\textsc{ipfv} {} revealed.\textsc{ipfv} {} showed.\textsc{ipfv} that Marek fears.\textsc{ipfv} \textsc{refl} ghost.\textsc{pl}\\
\glt	`Jan was not / probably proving / revealing / showing that Marek fears ghosts.'\newline$\nrightarrow$  Marek fears ghosts.
\z

\noindent Example \REF{21:21} demonstrates possible modifications of the likelihood of Jan having tried to prove / reveal / show that Marek fears ghosts. No contribution to the truth-related meaning of the complement sentence can be observed. Another member of this group is for instance \textit{przekonać}, \textit{przekonywać} `convince’.

\largerpage[-1]
\subsection{Case 3: {T}ruth implicature}\label{21:trimpl}
Truth implicature refers especially to the perfective communication verbs, which differ from their imperfective counterparts in that the former, but not the latter, entail the complete realization of all parts of the speech act. \cite{Austin1962} defines a speech act as consisting of three partial acts. The first one, a locutionary act, is the act of uttering itself. The second one, an illocutionary act, affects the area of the speaker’s intention. Finally, a perlocutionary act describes an actual effect the particular speech act had on the hearer. A speech act is presumed to be completely realized only if all three parts have been fulfilled. In Polish, perfective communication verbs, in contrast to imperfective ones, enforce complete fulfillment of all parts of the speech act, as example \REF{21:sp} illustrates.\footnote{I would like to thank Manfred Krifka for inspiring this idea.}

% Example
\ea\label{21:sp}
\gll Iza właśnie go o tym {\#}\hspace{-2pt} poinformowała / informowała, ale przerwał jej w pół słowa.\\
Iza just him about that {} informed.\textsc{pfv} {} informed.\textsc{ipfv} but interrupted her in middle word\\
\glt	`Iza has just informed / was just informing him about that, but he interrupted her in the middle of the sentence.'
\z

\noindent Only \textit{poinformowała} entails that the hearer received the information.\par


\section{Perfectivity-dependent truthfulness}\label{21:perfdep}

First of all, a short note on telicity should be made. My object of investigation is embedding predicates, which are transitive verbs. They all require a direct object, realized as a sentential complement; for the purpose of my analysis, I consider that-clause a definite argument. For this reason, the whole complex predicate receives a telic interpretation, independently of the (im)perfective marking on the verb. The truth inference is present when the matrix predicate has the features [$+$telic$, +{}$perfective], and it is absent when the matrix predicate has the features [$+$telic$,–{}$perfective].

Based on the influence of aspect on the interpretation of nominal arguments, I also assume a dependency between aspect and a propositional argument. The aspectual operator \cnst{pfv} introduces a further undefined truthfulness feature, which is specified as factive, veridical or reliable via the dependency between the truth of $p$ (where $p$ stands for the proposition expressed by the that-clause) and an event $e$ described by the matrix verb. For now, the three truthfulness-realizations can be formalized as follows:\footnote{The operations are based on the semantics of the perfective and not on the formation patterns. In future work, the morphology will be integrated into the semantic account (cf. \citealt{Młynarczyk2004}).}
% \newpage

\newpage 
\ea For a VP with a propositional complement $p$
\ea \cnst{pfv}\hspace{1pt}$(\lambda e.$\sx{VP}$(e)$ such that the truth of $p$ is independent of $e)$\newline$\rightarrow p$ is factive
\ex \cnst{pfv}\hspace{1pt}($\lambda e.$\sx{VP}$(e)$ such that the truth of $p$ is dependent on $e)$\newline$\rightarrow p$ is veridical
\ex \cnst{pfv}\hspace{1pt}($\lambda e.$\sx{VP}$(e)$ such that the truth of $p$ is communicated by $e)$\newline$\rightarrow p$ is reliable
\z
\z



\noindent Truth presupposition comes about when the truth of $p$ is independent of the truth of $e$. Here, no incremental creation of belief can be observed. For example, the truth of propositions embedded under \textit{zgadnąć} or \textit{przewidzieć} holds independently of the process of guessing or predicting. In contrast, the truth of propositions embedded under \textit{udowodnić}, \textit{wykazać} or \textit{pokazać} does depend on the result of the proving-process; we have an incremental creation of belief. This explains why the authority of an experiencer or its overt realization are crucial for judging complement sentences as veridical. In the case of truth implicature, the truth of $p$ is ‘only’ communicated by $e$.

The question remains whether ‘being reliable’ should be considered a feature at all, or if it should be labeled as ‘no feature present’. In the latter case, truthfulness set up by the perfective operator would remain unrealized if the inference was an implicature. Another open question concerns the role of morphology in determining the strength of the inference. It seems that perfective underlying forms tend to enforce factive meaning of the proposition expressed by the subordinate clause. Additionally, verb semantics and argument structure may also be taken into consideration, since specifying an experiencer can influence the entailment pattern. In general, the semantic type of the matrix verb could be used to distinguish between different verb classes and to establish a more fine-grained truthfulness scale. All this will be the subject of further investigations.

In the last section of this paper I will briefly discuss the inherently factive imperfectives and their perfective counterparts. It will be shown that they constitute a unique group with factivity being an aspect-independent, lexical property of the root form, which automatically projects to the perfective derivate.

\section{Remark on inherently factive imperfectives}

As has been mentioned before, inherently factive imperfectives (for example emotive factives) require their complements to be true. Consider example \REF{21:inh}.

% Example
\ea\label{21:inh}
\gll Ania cieszyła / ucieszyła się, że idzie lato.\\
Ania was.happy.\textsc{ipfv} {} was.happy.\textsc{pfv} \textsc{refl} that comes.\textsc{ipfv} summer\\
\glt	`Ania was happy / started being happy about the fact that the summer was coming.'\newline$\gg$ The summer was coming.
\z

\noindent The only difference between \textit{cieszyła} and \textit{ucieszyła} lies in the marking of the beginning of a state in the case of the latter. The underlying imperfective form is inherently factive (lexical factivity), so it remains factive when perfectivized. In the case of inherently factive imperfectives the perfectivizing operation leads to the marking of a temporal boundary of an event, but it does not enforce or change the truth inference of the proposition from the embedded clause (see also \sectref{21:intro}). This pattern needs to be distinguished from the ones discussed in \sectref{21:aspinf} and \sectref{21:perfdep}, where the truth inference ascribed to the perfective was absent in the particular imperfective forms. Other inherently factive imperfectives are \textit{rozumieć} `understand’ and \textit{kapować} `get’.

\section{Conclusion}
In this paper, I demonstrated three kinds of perfectivity-dependent truth inferences in Polish: truth presupposition, truth entailment and truth implicature. In the case of truth presupposition, the proposition from the embedded clause receives a factive interpretation. The inference remains under negation or after the insertion of a modal adverbial. In the case of truth entailment, a veridical interpretation of the complement sentence can be observed; only the positive sentence is interpreted as true. In the case of truth implicature, the inference in question is neither factivity nor veridicality. It is due to a pragmatic principle giving preference to the perfective verb if the speaker assumes that the sentence subject is reliable (speaker commitment to the truth of $p$).

Despite the differences in the strength of particular inferences, the truthfulness of the proposition from the embedded clause is only due to perfectivity – it is absent with imperfective forms. Embedding by imperfective matrix verbs results in the occurrence of a neutral interpretation of a that-clause with respect to its truthfulness, provided that the embedding imperfective verb is not inherently factive. The aspectual operator PFV introduces a truthfulness feature, which is realized as factive, veridical or reliable depending on the relation between the truth of the proposition expressed by the embedded clause and an event described by the matrix verb.

The question remains as to how truthfulness interacts with perfectivity itself. In the case of communication verbs, the completedness condition of the perfective enforces the complete performance of the speech act denoted by the matrix verb. The speaker of the sentence chooses the perfective if she considers the speaker of the speech act reliable. As a result, the proposition expressed by the that-clause is understood to be true. In the case of verbs of proving, the completedness effect of the perfective interacts with the incrementality, which is a part of lexical verb semantics. A proof is a proof after its final step is completed. For verbs of guessing, the truth presupposition is triggered in combination with the integration of the proposition `someone guessed something' into the common ground. The speaker uses the perfective in order to demonstrate that the guessing event has been completely realized. The cooperative hearer accepts the proposition as true, which triggers the presupposition rooted in the lexical verb semantics.\footnote{I would like to thank one of the anonymous reviewers for inspiring this idea.}

In future work, a detailed study with different semantic groups of verbs will be conducted. In addition the type of embedding is to be controlled for, since it may be involved in determining the strength of the inference available. An interesting observation concerns perfective verbs of saying which embed wh-phrases; they seem to function as exhaustivity triggers. Thus, exhaustivity could also be used to make the truthfulness scale more fine-grained.
 




\section*{Abbreviations}
\begin{tabularx}{.333\textwidth}{@{}lQ@{}}
\textsc{dat} & dative \\
\textsc{(i)pfv} & (im)perfective\\
\end{tabularx}%
\begin{tabularx}{.333\textwidth}{@{}lQ@{}}
\textsc{neg} & negation\\
\textsc{pl}&plural\\
\end{tabularx}%
\begin{tabularx}{.333\textwidth}{@{}lQ@{}}
\textsc{refl}&reflexive\\
{}&{}\\
\end{tabularx}

 \section*{Acknowledgements}
I would like to thank Luka Szucsich, Manfred Krifka and Radek \v{S}imík. I thank Berit Gehrke, Kyle Johnson, Dara Jokilehto, Denisa Lenertová, Clemens Mayr, Roland Meyer, Brandon Waldon, Ilse Zimmermann and audiences from ZAS, HU Berlin, ELTE and MTA. I am also grateful to the anonymous reviewers for their helpful comments. Finally, I thank Kirsten Brock and Jake Walsh for correcting my English. All remaining errors are my own.

This work was supported by the German Bundesministerium für Bildung und Forschung (BMBF) (Grant No. 01UG1411).

\sloppy
\printbibliography[heading=subbibliography,notkeyword=this]


\end{document}


%\section*{Abbreviations}
%\section*{Acknowledgements}

%\printbibliography[heading=subbibliography,notkeyword=this]


%\end{document}
