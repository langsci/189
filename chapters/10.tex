\documentclass[output=paper,modfonts,newtxmath,hidelinks,]{langscibook} 
\ChapterDOI{10.5281/zenodo.2545525}


\title{The markedness of coincidence in Russian}


\author{Emilia Melara\affiliation{University of Toronto}}

\abstract{This paper presents a novel analysis of the Russian Infl domain. Specifically, it is argued in this paper that in Russian, the past tense, as opposed to the non-past, is the default, unmarked tense. Consequently, non-past in Russian is marked by the specification of a privative feature on T$^0$, which associates the event/state expressed by vP to some anchoring time. This analysis stems from observations of how subjunctive matrix and complement clauses are interpreted. The analysis captures how, unlike other languages with the subjunctive mood, Russian allows main independent clauses to appear in the subjunctive. It additionally furthers work on features and properties of the Infl domain, showing how languages use different features, from what appears to be a limited set, to express time and realis contrasts.

\keywords{Russian, tense, subjunctive, Infl, realis and irrealis moods}
}

\begin{document}
\maketitle
\shorttitlerunninghead{The markedness of coincidence in Russian}

%%%%%%%%%%%%%%%%%%%%%%% (BEGIN) INSERTION LATEX ANDREI %%%%%%%%%%%%%%%%%%%%%%%%%%%%%%%%%

% DONT FORGET TO CHANGE ALL FOOTNOTE REFERENCES because of first 2 footnotes are Abbreviations and Acknowledges

% SECTION 1
\section{Introduction}\label{10:s1}


% \section{\sffamily\bfseries Introduction}
This study examines the morphosyntactic features of the \ili{Russian} inflectional domain by focusing on the selectional properties of the \ili{Russian} \isi{subjunctive}. Traditionally, the \isi{subjunctive} is held to be a mood (whether or not there is overt morphology) that expresses an eventuality as hypothetical, advisable, desirable, or obligatory with respect to the sentential subject  \citep[142]{HarrisonLeFleming2000}. In \ili{Russian}, the \isi{subjunctive mood} is expressed with the particle \textit{by} and typically with the past-tense form of the predicate.

% example 1
\ea \label{10:ex1}
\gll Ty uš-\textbf{l}{}-a \textbf{by} domoj.\\
     you leave\textsc{-pst-sg.f} \textsc{by} home.\\
\glt `You would \{go / have gone\} home.' \hfill \citep[152]{Mezhevich2006}
\z

\noindent Despite co-occurring almost exclusively with the past-tense verb form, however, constructions containing \textit{by} show no semantic tense contrasts whatsoever \citep[298]{Spencer2001}. This is illustrated in \REF{10:ex2}, where \isi{past}, \isi{present}, and future-oriented temporal adverbs are shown to licitly co-occur with the \isi{past tense} verb form when \textit{by} is present. 

% example 2
\ea \label{10:ex2}
%\langinfo{}{}{\citet[136]{Mezhevich2006}}\\
\gll Ja \textbf{by} uexa-\textbf{l}{}-a \{\hspace{-2pt} včera / sejčas / zavtra\}.\\
     I \textsc{by} leave\textsc{-pst-sg.f} {} yesterday {} now {} tomorrow\\
\glt `I would \{have left yesterday / leave now / leave tomorrow\}.'\\
\hfill \citep[136]{Mezhevich2006}
\z

\noindent \textit{By} can also co-occur with the \isi{infinitive} form of the verb in an independent \isi{matrix clause}.

% example 3
\ea \label{10:ex3}
%\langinfo{}{}{\citet[10]{Asarina2006}}\\
\gll Oj s”es\textbf{t’} \textbf{by} Pete \{\hspace{-2pt} včera / zavtra\} jabloko!\\
     oh eat.\textsc{inf} \textsc{by} Peter.\textsc{dat} {} yesterday {} tomorrow apple\\
\glt `If only Peter would eat an/the apple tomorrow!' or
\glt `If only Peter would have eaten an/the apple yesterday!'\\\hfill \citep[10]{Asarina2006}
\z

\noindent Non-\isi{past} finite forms of the predicate, on the other hand, are completely illicit with \textit{by}.

% example 4
\ea \label{10:ex4}
	\ea[*]{ \label{10:ex4a}
    \gll Ja propuskaj-\textbf{u} \textbf{by} \.etot doklad.\\          
         I miss.\textsc{ipfv/prs-1sg} \textsc{by} this talk \\
         \glt Intended: `I would skip this talk.'
    }
	\ex[*]{ \label{10:ex4b}
    \gll Ja ujd-\textbf{u} \textbf{by} domoj.\\
         I leave.\textsc{pfv/fut-1sg} \textsc{by} home\\ 
         \glt Intended: `I would go home.' \hfill (adapted from \citealt[133]{Mezhevich2006})}
	\z
\z

\noindent This study stems from these observations. It asks: What can these co-occurrence patterns tell us about the interpretable features of the \ili{Russian} inflectional system? I argue that \textit{by} is the phonological spell-out of an irrealis head in the \ili{Russian} inflectional domain, whose projection is semantically incompatible with the specification of any feature that situates a \isi{clause} at the utterance context. Specifically, I will claim that this feature is [Coin(cidence)] (cf. \citealt{RitterWiltschko2005}, \citealt{RitterWiltschko2009}), which is hosted in T. A consequence, and perhaps the main take-away of this proposal is that the contrast between \isi{past} and non-\isi{past} in \ili{Russian} is distinguished by the specification of [Coin], \isi{past tense} being the unmarked tense. This proposal is rooted in Distributed Morphology (\citealt{HalleMarantz1993}; \citealt{EmbickNoyer2007}) and builds on the \isi{feature geometry} work of \citet{Cowper2002,Cowper2005} and others.

The outline of this paper is as follows. In \sectref{10:s2}, I describe the data considered for the analysis to be presented. It describes the tense system in \ili{Russian} along with how the \isi{subjunctive} is expressed in the language. \sectref{10:s3} provides a background sketch of the \isi{subjunctive mood} cross-linguistically and in the literature. In \sectref{10:s4}, I present an analysis of the data presented in \sectref{10:s2}. \sectref{10:s5} expands the analysis presented to account for \ili{Russian} \isi{subjunctive} constructions as complement clauses. Finally, I conclude in \sectref{10:s6}. 

% SECTION 2
\section{The Russian system}\label{10:s2}

In \ili{Russian}, most verbs come in \isi{aspectual} pairs \citep[371]{Mezhevich2008} – an \isi{imperfective} form and corresponding \isi{perfective} form – and tense is often defined with respect to aspect \citep[373]{Mezhevich2008}. In the indicative mood (that of “independent main assertive \isi{clause} type[s]” \citep[1]{Wiltschko}), \isi{imperfective aspect} allows for temporal distinctions among \isi{past}, \isi{present}, and a periphrastic \isi{future}; \isi{perfective} only allows for \isi{past} and \isi{future} readings \citep[371]{Mezhevich2008}. Among non-\isi{past} forms, aspect plays a role in distinguishing \isi{present} from \isi{future}. The examples in \REF{10:ex:ipfv} and \REF{10:ex:pfv} shows the temporal-\isi{aspectual} realizations for the verb ‘fall’, illustrating the \ili{Russian} tense system.

\begin{multicols}{2}
\ea \textbf{Imperfective}\label{10:ex:ipfv}
\ea\gll On padal.\\
he fall.\textsc{ipfv.pst}\\\hfill\textsc{\textbf{pst}}
\glt `He was falling.'
\ex\gll On padaet.\\
he fall.\textsc{ipfv.prs}\\\hfill\textsc{\textbf{prs}}
\glt `He is falling.'
\ex\gll On budet padat'.\\
he will fall.\textsc{ipfv.inf}\\\hfill\textsc{\textbf{fut}}
\glt `He will be falling.'
\z\z\columnbreak
\ea \textbf{Perfective}\label{10:ex:pfv}
\ea\gll On upal.\\
he fall.\textsc{pfv.pst}\\\hfill\textsc{\textbf{pst}}
\glt `He fell.'
\ex\gll N/A\\
{}\\\hfill\textsc{\textbf{prs}}
\glt {}\vspace{14pt}
\ex\gll On upadet.\\
he fall.\textsc{pfv.prs}\\\hfill\textsc{\textbf{fut}}
\glt `He will fall.'
\z\z

\end{multicols}

\noindent Unlike Modern \ili{Russian}, Old \ili{Russian} made a distinction among four \isi{past} tenses, namely, the aorist, the perfect, the pluperfect, and the imperfect \citep[38]{Mezhevich2006}. Perfect and pluperfect constructions contained an inflected form of \textit{byti} ‘be’ and a form commonly referred to as the \textsc{l}-\isi{participle}: a verb containing the \textit{-l} suffix. The distinction among the four \isi{past} tenses was lost over time. What has remained is the \textit{-l} suffix as the sole marker of \isi{past tense} (ibid.).

Although historically it was the case that the \textit{-l} suffix of the \textsc{l}-\isi{participle} did not mark \isi{past tense} itself, it has been argued that the suffix has been reanalyzed as the \isi{past tense} morpheme in Modern \ili{Russian} (see \citealt{Mezhevich2006} for a discussion and references). The form’s distribution and interpretation in Modern \ili{Russian} contrast with what are considered to be non-\isi{past} predicate forms. I therefore treat the \textit{-l} suffix that attaches to verbs as the \isi{past tense} form here. In no way, however, do I assume that it exclusively expresses \isi{past tense}. As shown in \REF{10:ex2} and to be seen in later examples, when \textit{{}-}\textit{l\-} co-occurs with \textit{by}, one interpretation the \isi{clause} may receive is a \isi{past} interpretation but in no way is such a construction restricted to that interpretation. A \isi{clause} containing both these morphemes may also receive non-\isi{past} readings.

Apart from the indicative, Modern \ili{Russian} has only two formal moods: the imperative and the \isi{subjunctive}/conditional \citep[157]{Cubberley2002}. \ili{Russian} does not have specific \isi{subjunctive} verb forms \citep[118]{Mezhevich2006}. Rather, \isi{subjunctive} clauses are generally formed with the particle \textit{by} and the \textsc{l}-\isi{participle}, as in \REF{10:ex5}, repeated from \REF{10:ex1}, and \REF{10:ex6}.

% example 5
\ea \label{10:ex5}
\gll Ty uš-l-a by domoj.\\
     you leave\textsc{-pst-sg.f} \textsc{by} home\\
\glt `You would go / have gone home.' \hfill \citep[152]{Mezhevich2006}
\z

% example 6
\ea \label{10:ex6}
\gll Liza xote-l-a, [čtoby Philemon uše-l].\\
     Liza want\textsc{-pst-sg.f} { }\textsc{čtoby} Philemon leave-\textsc{pst.sg.m}\\
\glt `Liza wanted Philemon to leave.' \hfill \citep[148]{Mezhevich2006}
\z

\noindent Traditionally, the \isi{subjunctive} is held to be a mood (whether or not there is overt morphology) that expresses an eventuality as hypothetical, advisable, desirable, or obligatory \citep[142]{HarrisonLeFleming2000}, as in \REF{10:ex7}, with respect to the sentential subject.\largerpage[-1]

% example 7
\ea \label{10:ex7}
%\langinfo{}{}{\citet[142]{Harrison \& le Fleming2000}}\\
	\ea[]{ \label{10:ex7a}
    	They would like [to go].            \hfill\textbf{desirability}\\
    }
	\ex[]{ \label{10:ex7b}
		I should [write to my mother].      \hfill\textbf{obligation}
    }
    \hfill \citep[142]{HarrisonLeFleming2000}
	\z
\z

\noindent In \ili{Russian}, the \isi{subjunctive} pattern described above is used to express these semantic notions, for example, in \REF{10:ex8} and \REF{10:ex9}.

% example 8
\ea \label{10:ex8}
%\langinfo{}{}{\citet[142]{Harrison \& le Fleming2000}}\\
	\ea[]{ \label{10:ex8a}
    \gll Vy čita-l-i by gazetu.\\
         you read\textsc{-pst-pl} \textsc{by} paper\\\hfill\textbf{advisability}
    }
	\ex[]{ \label{10:ex8b}
    \gll Vy pro-čita-l-i by gazetu.\\
         you \textsc{pfv-}read\textsc{-pst-pl} \textsc{by} paper\\
    }
	\z
        \glt `You should (have) read the paper.'\hfill\citep[142]{HarrisonLeFleming2000}
\z

% example 9
\ea \label{10:ex9}
\gll Zavtra ja s udovol’stviem poše-l by v teatr\\
     tomorrow I from pleasure go\textsc{-pst} \textsc{by} at theatre\\\hfill\textbf{desirability}
\glt `I would very much like to go to the theatre tomorrow.'
\z

\noindent That is, in \REF{10:ex8}, the \isi{subjunctive} is used to express advisability with respect to the subject and in \REF{10:ex9}, desirability. \REF{10:ex8a} and \REF{10:ex8b} illustrate that the imperfective-\isi{perfective} distinction is maintained in the \isi{subjunctive mood}.

Although \textit{by} derives from the aorist of the Old \ili{Russian} auxiliary \textit{byti} ‘be’, it has been reanalyzed as a marker of the \isi{subjunctive}/conditional separate from the Modern \ili{Russian} form \textit{byt’} ‘be’. The main distinguishing property between \textit{by} and \textit{byt’} is that the latter has a paradigm of inflected forms while the former does not; rather, it is a frozen morpheme (see \citealt{Spencer2001}; \citealt{Mezhevich2006}).

In matrix clauses, \textit{by} most naturally appears following the main verb \citep[200]{Cubberley2002}. However, it can also follow a focused element, appearing in the second sentential position \citep[298]{Spencer2001}, as in \REF{10:ex10}. In theory, though, \textit{by} can occur in any position except clause-initially (\citealt{Hacking1998}, cited in \citealt[152]{Mezhevich2006}; \citealt[298]{Spencer2001}); see \REF{10:ex11}.

% example 10
\ea \label{10:ex10}
	\ea[]{ \label{10:ex10a}
    \gll Ja uš-l-a by.\\       
         I leave\textsc{-pst-sg.f} \textsc{by}\\
    }
	\ex[]{ \label{10:ex10b}
    \gll Ja by uš-l-a.\\
         I \textsc{by} leave\textsc{-pst-sg.f} \\
    }
	\z
        \glt `I would \{leave / have left\}.' \hfill \citep[298]{Spencer2001}
\z

% example 11
\ea \label{10:ex11}
	\ea[]{ \label{10:ex11a}
    \gll Ty \textbf{by} uš-l-a domoj.\\      
         you \textsc{by} leave\textsc{-pst-sg.f} home\\
    }
	\ex[]{ \label{10:ex11b}
    \gll Ty uš-l-a \textbf{by} domoj.\\
         you leave\textsc{-pst-sg.f} \textsc{by} home\\
    }
    \ex[]{ \label{10:ex11c}
    \gll Ty uš-l-a domoj \textbf{by}.\\
         you leave\textsc{-pst-sg.f} home \textsc{by}\\
    }
    \ex[*]{ \label{10:ex11d}
    \gll \textbf{By} ty uš-l-a domoj.\\
         \textsc{by} you leave\textsc{-pst-sg.f} home\\
    }
	\z
    \glt (Intended:) `You would go / have gone home.' \hfill \citep[152]{Mezhevich2006}
\z

\noindent It was noted in \sectref{10:s1} that \textit{by} cannot co-occur with a non-past-tense predicate. This is shown again in \REF{10:ex12}, repeated from \REF{10:ex4}.

% example 12
\ea \label{10:ex12}
	\ea[*]{ \label{10:ex12a}
    \gll Ja propuskaj-u by \.etot doklad.\\          
         I miss.\textsc{ipfv/prs-1sg} \textsc{by} this talk \\
         \glt Intended: `I would skip this talk.'
    }
	\ex[*]{ \label{10:ex12b}
    \gll Ja ujd-u by domoj.\\
         I leave.\textsc{pfv/fut-1sg} \textsc{by} home\\ 
         \glt Intended: `I would go home.' \hfill (adapted from \citealt[133]{Mezhevich2006})}
	\z
\z

\noindent Embedded under predicates that license \isi{subjunctive} clauses, \textit{by} surfaces clause-initially with the indicative \isi{complementizer} \textit{čto} as a fused form \citep{Brecht1977}.

% example 13
\ea \label{10:ex13}
\gll Liza xote-l-a, [čtoby Philemon uše-l].\\
     Liza want\textsc{-pst-sg.f} { }\textsc{čtoby} Philemon leave\textsc{-pst}\\
\glt `Liza wanted Philemon to leave.' \hfill \citep[148]{Mezhevich2006}
\z

\noindent Like \textit{by} in matrix clauses, \textit{čtoby} never appears with \isi{present} or \isi{future} morphology on the predicate.

% example 14
\ea \label{10:ex14}
	\ea[]{\label{10:ex14a}
    \gll Maša xočet čtoby Petja s”e-l jabloko.\\          
         Maša wants \textsc{čtoby} Peter eat.\textsc{-pfv.pst} apple\\
    \glt `Mary wants for Peter to eat an apple.'
    }
	\ex[*]{ \label{10:ex14b}
    \gll Maša xočet čtoby Petja \{\hspace{-2pt} est / s”est\} jabloko.\\
         Maša wants \textsc{čtoby} Peter {} eat.\textsc{ipfv.prs} {} eat.\textsc{pfv.prs(=fut)} apple\\
         \glt Intended: `Mary wants for Peter to eat an apple.' \hfill \citep[7]{Asarina2006}}
    
	\z
\z

\noindent Unlike matrix \isi{subjunctive} clauses, a past-tense reading is unavailable for a \isi{subjunctive} \isi{complement clause}, as shown in \REF{10:ex15c}; while \isi{present} and \isi{future} interpretations are possible, as shown in \REF{10:ex15a} and \REF{10:ex15b}. 

% example 15
\ea \label{10:ex15}
	\ea[]{ \label{10:ex15a}
    \gll Ja xoču, čtoby Maša zavtra s”e-l-a jabloko.\\      
         I want \textsc{čtoby} Mary tomorrow eat\textsc{-pst-sg.f} apple\\
    \glt `I want for Mary to eat an apple tomorrow.'
    }
	\ex[]{ \label{10:ex15b}
    \gll Ja xoču, čtoby Maša sejčas e-l-a jabloko.\\
         I want \textsc{čtoby} Mary now ate\textsc{-pst-sg.f} apple\\
    \glt `I want for Mary to be eating an apple right now.'
    }
    \ex[*]{ \label{10:ex15c}
    \gll Ja xoču, čtoby Maša včera s”e-l-a jabloko.\\
         I want \textsc{čtoby} Mary yesterday eat\textsc{-pst-sg.f} apple\\
    \glt Intended: `I want for Mary to have been eating an apple yesterday.'
    }
    \hfill \citep[8]{Asarina2006}
	\z
\z

\noindent In the case that the subjects of the complement and matrix clauses are coreferential, however, the subordinate predicate appears in its infinitival form \citep[160, 236]{Cubberley2002}, as shown in \REF{10:ex16}. When the subjects of the complement and matrix clauses have disjoint reference, the \isi{subordinate clause} appears with the \isi{complementizer} \textit{čtoby} and the \isi{past tense} form of the embedded verb, as in \REF{10:ex17}. The disjoint reference requirement for the subject of the embedded \isi{subjunctive} \isi{clause} with respect to the subject of the \isi{matrix clause} is called “subject obviation” (cf. \citealt[1]{Antonenko2010}).

% example 16
\ea \label{10:ex16}
	\ea[]{ \label{10:ex16a}
    \gll Ja xoču poj-ti domoj.\\       
         I want.\textsc{prs.1sg} go\textsc{-inf} home\\
    \glt `I want to go home.'
    }
	\ex[]{ \label{10:ex16b}
    \gll My xote-l-i \.eto sdelat' zavtra.\\
         we want\textsc{-pst-pl} this do.\textsc{inf} tomorrow\\
    \glt `We wanted to do that tomorrow.' \hfill \citep[143]{HarrisonLeFleming2000}
    }
	\z
\z

% example 17
\ea \label{10:ex17}
	\ea[]{ \label{10:ex17a}
    \gll Ja xoču, čtoby on poše-l domoj.\\       
         I want\textsc{.prs.1sg} \textsc{čtoby} he go\textsc{-pst} home\\
    \glt `I want him to go home.'
    }
	\ex[]{ \label{10:ex17b}
    \gll My xote-l-i čtoby vy \.eto~ sdela-l-i zavtra.\\
         we want\textsc{-pst-pl} \textsc{čtoby} you this do\textsc{-pst-pl} tomorrow\\
    \glt `We wanted you to do this tomorrow.' 
    }
	\z
\z

\noindent Matrix subjunctives, though, do not differ semantically regardless of whether the predicate appears with \isi{past} morphology or in the \isi{infinitive} \citep[10]{Asarina2006}. Note, however, that the subject of the \isi{clause} appears in its \isi{nominative} form when the verb appears with \textit{{}-l} but in its \isi{dative} form when the verb is infinitival.

% example 18
\ea \label{10:ex18}
%\langinfo{}{}{\citet[10]{Asarina2006}}\\
	\ea[]{ \label{10:ex18a}
    \gll Oj s”e-l by Petja \{\hspace{-2pt} včera / zavtra\} jabloko!\\       
         oh ate\textsc{-pst} \textsc{by} Peter.\textsc{nom} {} yesterday {} tomorrow apple\\
\glt `If only Peter would eat an/the apple tomorrow!' or
\glt `If only Peter would have eaten an/the apple yesterday!'}
	\ex[]{ \label{10:ex18b}
    \gll Oj s”est’ by Pete \{\hspace{-2pt} včera / zavtra\} jabloko!\\
         oh eat.\textsc{inf} \textsc{by} Peter.\textsc{dat} {} yesterday {} tomorrow apple\\
\glt `If only Peter would eat an/the apple tomorrow!' or
\glt `If only Peter would have eaten an/the apple yesterday!'\\\hfill \citep[10]{Asarina2006}}
\z\z

\noindent The following section outlines properties of the \isi{subjunctive mood} from a cross-linguistic perspective. 

% SECTION 3
\section{The subjunctive mood}\label{10:s3}

The \isi{subjunctive mood} contrasts minimally with the indicative (\citealt[660]{Quer2006}; \citealt[218]{Wiltschko}). However, neither cross- nor intra-linguistically does the \isi{subjunctive mood} constitute a uniform category \citep[661]{Quer2006}. Some subjunc\-tive-related phenomena are present in some languages but absent in others that have the mood (ibid.). For example, \ili{Icelandic} \isi{subjunctive} clauses allow long-distance anaphors while Upper Austrian \ili{German} \isi{subjunctive} clauses do not (ibid.). Further, within a single language that has the \isi{subjunctive mood}, there are sub\-junc\-tive-related phenomena that are evident in some \isi{subjunctive} clauses but not all (ibid.).

The \isi{subjunctive} has frequently been considered a defective tense (e.g. \citealt{Picallo1984} and \citealt{Giannakidou2009}) or at least impoverished semantically with respect to the indicative (see \citealt{Cowper2002}; \citealt{Cowper2005}; \citealt{Schlenker2005}). As a completely defective tense, the \isi{subjunctive} is claimed to be dependent on some higher structure for its temporal interpretation \citep[2]{Wiltschko}. Proposals of this sort stem from the fact that in some languages (e.g. \ili{Spanish} and \ili{Catalan}), subjunctives cannot be used in matrix clauses; in these same languages, where the \isi{subjunctive} appears in a \isi{complement clause}, the time of the embedded \isi{clause} is interpreted relative to that of the \isi{matrix clause} \citep{Wiltschko}.\largerpage[-1]

A problem that has been noted concerning the idea that the \isi{subjunctive} is a defective tense/impoverished morphosyntactically is that there are languages that have been argued to lack tense but have an active indicative-\isi{subjunctive} distinction \citep{Wiltschko}. For example, \citet{Wiltschko} demonstrates that in Upper Austrian \ili{German}, there is no dedicated form for the simple \isi{past tense} and the bare verb in the indicative is compatible with a \isi{past}, \isi{present}, or \isi{future} interpretation.

% example 19
\ea \label{10:ex19}
%\langinfo{}{}{\citet[13-14]{Wiltschko n.d.}}\\
	\ea[]{ \label{10:ex19a}
    \gll I \textbf{koch} grod.\\      
         I cook now\\\hfill\textbf{present}
    \glt `I am cooking right now'
    }
	\ex[]{ \label{10:ex19b}
    \gll I \textbf{koch} gestan.\\
         I cook yesterday\\\hfill\textbf{past}
    \glt `I was cooking yesterday.'
    }
    \ex[]{ \label{10:ex19c}
    \gll I \textbf{koch} moagn.\\      
         I cook tomorrow\\\hfill\textbf{future}
    \glt `I will cook tomorrow.' \hfill \citep[13--14]{Wiltschko}
    }
	\z
\z

\noindent \citet{Wiltschko} argues that in Upper Austrian \ili{German} there is a subjunctive--indicative contrast active where a tensed language, for example Standard \ili{German}, would employ the past-non-\isi{past} distinction. For example, as shown in \REF{10:ex20}, \isi{subjunctive} morphology appears on the verb, closer than \isi{agreement} marking.

% example 20
\ea \label{10:ex20}
	\ea[]{ \label{10:ex20a}
    \gll Nua du kumm-\textbf{at}{}-st.\\          
         only you come\textsc{-sbj-2sg}\\
    \glt `Only you would come.'
    }
	\ex[]{ \label{10:ex20b}
    \gll Nua es kumm-\textbf{at-}ts.\\
        only you.\textsc{pl} come\textsc{-sbj-2pl}\\
    \glt `Only you guys would come.' \hfill \citep[17]{Wiltschko}
    }
	\z
\z

\noindent Wiltschko claims that the subjunctive-indicative contrast is how the language anchors its clauses. This is evident from the fact that the \isi{subjunctive} may be used in main independent clauses in Upper Austrian \ili{German}, and therefore: a) \isi{subjunctive} clauses are temporally independent, and b) the \isi{subjunctive} does not create a transparent \isi{clause}. The proposal, following \citet{RitterWiltschko2005,RitterWiltschko2009}, is that Infl, the locus of clausal anchoring, contains a [Coin(cidence)] feature which establishes a relation of either overlap or coincidence between Infl’s two arguments (in the case of [$+$Coin]) or disjointness (as in the case of [$-$Coin]). It is the substantive (a.k.a. semantic) content of the morphology that determines the relation between Infl arguments, for example, time. In the case of Upper Austrian \ili{German}, \isi{subjunctive} marking values the [\textit{u}Coin] feature in Infl as [$-$Coin], while indicative marking values it as [$+$Coin].\largerpage

The negatively valued [Coin] feature of \citet{RitterWiltschko2005,RitterWiltschko2014} roughly corresponds to  \citeposst{Iatridou2000} exclusion feature: ExclF. ExclF can range over times or worlds and has the basic meaning presented in \REF{10:ex21}.

\newpage 
% example 21
\ea \label{10:ex21}
ExclF: \cnst{t}$(x)$ excludes \cnst{c}$(x)$,\\where \cnst{t}$(x)$ means \cnst{topic}$(x)$ (“the $x$ that we are talking about”) and \cnst{c}$(x)$ means \cnst{context}$(x)$ (“that $x$ that for all we know is the $x$ of the speaker“)
	\ea[]{ \label{10:ex21a}
    Ranging over times, \cnst{t}$(t)$ is the set of times under discussion and \cnst{c}$(t)$ is the set of times that for all we know are the times of the speaker (i.e. the utterance time). What this yields is the interpretation: The topic time excludes the utterance time.
    }
	\ex[]{ \label{10:ex21b}
    Ranging over worlds, the interpretation the ExclF yields is: The topic worlds exclude the actual world.
    }
    \hfill \citep[246]{Iatridou2000}
	\z
\z

\noindent Essentially, ExclF and the negatively valued [Coin] feature share the property of establishing that two elements are disjoint.

The analysis to be presented in this paper adopts the feature proposed by \citet{RitterWiltschko2005,RitterWiltschko2009}, however as a privative interpretable feature of Infl. It also employs \citeauthor{Cowper2002}'s (\citeyear{Cowper2002,Cowper2005}) \isi{feature geometry} of interpretable Infl features. It will also be explained how ExclF, bearing basically the opposite \isi{semantics} of [Coin], would be less parsimonious in accounting for the behaviour exhibited by the \ili{Russian} \isi{subjunctive}. To give away the punch-line, what surfaces is the claim that in \ili{Russian}, the \isi{past tense} is morphosyntactically unmarked (non-\isi{past} being the marked tense) and the \ili{Russian} \isi{subjunctive} involves the spell-out of an irrealis head in Infl that is incompatible with the morphosyntactic specification of [Coin]. 

% SECTION 4
\section{The proposal}\label{10:s4}

I argue in this section that \textit{by} is an irrealis particle that spells out the head of a functional projection IrrP, which merges with TP in a fully articulated Infl structure. Despite proposing IrrP as a modified version of \citeposst{Cowper2010} MP, I make no claims here about modal operators in \ili{Russian} \isi{subjunctive} clauses or \isi{subjunctive} clauses in general. 

% SUB-SECTION 4.1
\subsection{Theoretical framework}\label{10:s4.1}

The analysis to be presented adopts the inflectional system proposed by \citet{Cowper2010}, based on the \isi{feature geometry} of the inflectional domain proposed in \citet{Cowper2005}. Her framework and the one presented here are rooted in Distributed Morphology (DM) (\citealt{HalleMarantz1993}; \citealt{EmbickNoyer2007}; \citealt{Bobaljik}), a theoretical approach according to which the syntax operates on feature bundles (i.e. lexical items or LIs) taken from the lexicon, combined in terminal nodes. Vocabulary items (or VIs) spell these features out at the phonological interface. 

The interpretable, privative features of the Infl domain proposed by \citet{Cowper2005} are divided according to mood, narrow tense, and viewpoint aspect, as shown in \REF{10:ex22}, where $\alpha$ and $\beta$ are features in a dependency structure, in $\alpha >\beta$, $\beta$ is a dependent of $\alpha$.\largerpage[2]

% example 22
\ea \label{10:ex22}
Mood: [Proposition] $>$ [Finite/Deixis] $>$ [Modality]
\newline Narrow tense: [Precedence]
\newline Viewpoint aspect: [Event] $>$ [Interval] \hfill \citep[1]{Cowper2010}
\z


\noindent The proposed dependency structure from \citet{Cowper2010} for the \ili{English} Infl domain is provided in \figref{10:fig:tree_1}.\footnote{\label{10:fn1}While \citet{Cowper2010} proposes heads higher than TP, only the projections relevant to the present proposal are provided here.}

The specification of [Proposition] contrasts propositions from bare events or states. [Finite] is a syntactic feature that licenses \isi{nominative} case and \isi{verbal} \isi{agreement}. [Deixis] anchors a \isi{clause} to the moment of speech. [Modality] carries the \isi{semantics} of necessity or possibility. [Precedence] encodes the meaning of \isi{past} versus non-\isi{past}. [Event] encodes the eventive (as opposed to \isi{stative}) property of a predicate. Finally, the specification of [Interval] derives imperfectivity versus perfectivity. These features are realized on multiple functional heads which together constitute the inflectional domain of the \isi{clause}.

%%%%%%%%%%%%%%%%%%%%%%%%%%%%%%%%%%%%%%%%%%%%%%%%%%%%%%
% FIGURE 1
%\ea \label{10:tree_1}
\begin{figure}[t]
%\centering
%\label{10:tree_1}
\caption{English Infl domain \citep[2]{Cowper2010}}
\begin{forest} baseline, qtree
%\caption{English Infl domain}
  [TP, s sep=1cm
  	[{T = Proposition}, s sep=0.5cm
    	[Finite/Deixis
        	[Modality]
        ]
        [Precedence]
    ]
    [MP
    	[Modality]
        [EP
        	[Event
            	[Interval]
            ]
            [vP]
        ]
    ]
  ]
\end{forest}
\label{10:fig:tree_1}
\end{figure}
%\z

Under \citeauthor{Cowper2010}’s proposal, \ili{English} modals merge in M(od) and subsequently move to T. TP, accordingly, is the projection of the feature [Proposition] given that only in propositions may the \isi{past}/non-\isi{past} distinction be realized. The viewpoint aspect features are realized in EP, which is not projected in \isi{stative} clauses \citep[2]{Cowper2010}. Moreover, the EPP is a property of the domain as a whole and is instantiated by the highest Infl head projected.

I assume here the TP, MP, and EP projections from \citet{Cowper2010} along with the features [Finite], [Modality], and [Event]. [Modality] in my proposal is semantically impoverished in relation to its original proposal: (i) to avoid making any claims about subjunctivity and some relation with modality and (ii) because the \isi{semantics} of \textit{by} allows for modal interpretations within a superset of additional irrealis readings. I therefore refer to it simply as IrrP, projected by the instantiation of [Irrealis]. Another difference between the \isi{feature geometry} proposed here and that of \citeauthor{Cowper2010}’s is that I follow \citet{RamchandSvenonius2014}, assuming that propositional content is encoded higher in the \isi{clause}, namely in the CP domain, rather than within Infl. For \citeauthor{RamchandSvenonius2014}, clauses are comprised of event (VP), situation (TP), and proposition (CP) domains, with transitional projections establishing relations among the domains. Specifically, AspP – essentially \citeposst{Cowper2010} EP – establishes a relation between the v/VP and TP, where an event is converted to a situation, while FinP (the lowest projection in \citeposst{rizzi1997} split CP) establishes a relation between TP and CP, where a situation is converted to a proposition. It is in the CP that the propositional content of the \isi{clause} becomes anchored to the utterance context, since that is the domain where speaker-oriented parameters reside. The diagram in \figref{10:fig:tree_2} shows these domain associations.
%\textbf{\figref{10:fig:key:2}}

%%%%%%%%%%%%%%%%%%%%%%%%%%%%%%%%%%%%%%%%%%%%%%%%%%%%%%%%%%%
% FIGURE 2
%\ea \label{10:tree_2}
\begin{figure}[h]
\captionsetup{format=hang,justification=centering}
\caption{Domains \& transitional projections\\\citep[164]{RamchandSvenonius2014}}
\begin{forest}   for tree={calign=fixed edge angles,calign secondary angle=35}%baseline%, qtree
  [C,tikz={\node [draw,black,fit=()()(!l)] {};}
  	[,phantom]
  	[Fin*,tikz={\node [draw,black,fit=()()(!ll)] {};}%, s sep=1.2cm
    	[,phantom]
    	[T%, s sep=1.2cm
        	[,phantom]
        	[Asp*,tikz={\node [draw,black,fit=()()(!l)] {};}%, s sep=1.1cm
            	[,phantom]
                [V]
            ]
        ]
    ]    	
  ]
\end{forest}
\label{10:fig:tree_2}
\end{figure}
%\z


\noindent I will claim that whereas \isi{past} is marked relative to non-\isi{past} in \ili{English}, the opposite holds in \ili{Russian}. That is, whereas \isi{past} in \ili{English} is the spellout of (minimally) [Precedence], \ili{Russian} does not have [Precedence] in its Infl feature inventory. Rather, \ili{Russian} has the feature [Coin] (\citealt{RitterWiltschko2005}; \citealt{Wiltschko,Wiltschko2014}) as a dependent of [Finite], and does not have [Deixis].\footnote{\label{10:fn2}The difference between [Deixis] and [Coin] lies in [Deixis] having been proposed as a feature that in \ili{English} links temporal and speaker properties to the utterance context, whereas what [Coin] associates to the utterance context depends on where in the syntactic spine it is specified à la \citet{RamchandSvenonius2014}.}$^,$\footnote{\label{10:fn3}[Interval], I claim, is also absent in \ili{Russian}. Instead, the feature [Atomic] is a dependent of [Event], as I have argued based on the fact that \isi{stative} predicates in \ili{Russian} cannot bear non-derivational \isi{perfective} morphology. See \citet{Melara2014} for further discussion.} Unlike in \citet{Wiltschko}, as was previously described, however, [Coin] is a privative feature. Moreover, while [Deixis] establishes an anchor to the utterance time relative to which [Precedence] situates the event, I claim that [Coin] anchors a proposition to the utterance context temporally within the Infl domain and personally (to the speaker) within the C domain. As a feature in Force, the head that hosts complementizers like \ili{English} \textit{that} and provides information about \isi{clause} type, [Coin] associates the clausal content to the speaker’s perspective.

% SUB-SECTION 4.2
\subsection{The Infl system in Russian}\label{10:s4.2}

Adopting the tools from \citet{Cowper2005,Cowper2010}, I propose the fully articulated dependency structure in \figref{10:fig:tree_3} for the \ili{Russian} Infl system. Note that here, as mentioned above, [Irr] heads its own projection rather than being part of T, unlike in \citet{Cowper2010} (note that my Irr corresponds to \citeauthor{Cowper2010}'s Mod). I assume that a functional head cannot be projected in the absence of any specified features. Thus, while for \citeauthor{Cowper2010}, the lexical properties of modals also reside in Mod, I take Irr to be a purely functional head, merged only when [Irr] is specified. This is where a modal particle such as \textit{by} in \ili{Russian} is merged. Similarly, T is the projection of the feature [Fin(ite)].

% FIGURE 3
\begin{figure}
\caption{Russian Infl dependency structure}
\begin{forest} baseline, qtree
  [IrrP, s sep=1.8cm
    [{[}Irr{]}]
    [TP, s sep=0.7cm
      [T/{[}Fin{]}
      	[{[}Coin{]}]
      ]
      [E(event)P, s sep=0.7cm
      	[{[}Event{]}
        	[{[}Atomic{]}]
        ]
      	[vP]
      ]
    ]
  ]
\end{forest}
\label{10:fig:tree_3}
\end{figure}


The fact that the \ili{Russian} \isi{subjunctive} is compatible only with the \isi{past} marker \textit{-l} or the \isi{infinitive} results from the selectional requirements of the functional heads in the Infl system. As stated earlier, I assume, based on \citet{RamchandSvenonius2014}, that the Infl domain temporally situates an event, while the C domain anchors the situation personally, both with respect to the utterance context.

As in \citet{Cowper2010}, EP is projected in non-\isi{stative} clauses, selecting the vP. It is in E that non-derivational \isi{aspectual} affixes reside. TP hosts the features [Fin] and [Coin]. [Fin] is the locus of \isi{nominative} case and \isi{agreement}. [Coin] establishes coincidence between the event described by the vP and the temporal properties of the utterance context. \ili{Russian}, I claim, lacks any tense features. Instead, the \isi{past}/non-\isi{past} distinction is attributable to the presence or absence of [Coin]. Specified in Infl -- the temporal domain -- [Coin] semantically situates the event described by the \isi{clause} to a non-\isi{past} time and is spelled out by non-\isi{past} morphology. Both T and E bear a strong uninterpretable V feature [\textit{u}V], requiring that v, containing V, move up at least to T to satisfy and check the [\textit{u}V] of each head locally. I propose that in \ili{Russian}, when [Coin] is absent, the \isi{past} suffix \textit{{}-l} is spelled out on the verb. That is, \textit{{}-l} spells out a T specified only for [Fin], hence the \isi{past tense} morpheme being unmarked relative to the non-\isi{past}.

The [Irr] feature that \textit{by} spells out encodes irrealisness. The irrealis meaning of [Irr] is semantically at odds with the binding established by [Coin]. When IrrP is projected, [Irr] scopes over the entire Infl domain (but cf. \citealt{Cowper2010} for discussion on how NegP is the highest projection in Infl) and essentially has the \isi{semantics} of ExclF scoping over times, proposed by \citet{Iatridou2000}. As described in \sectref{10:s3}, ExclF is equivalent to [$-$Coin] from \citeauthor{RitterWiltschko2005}'s (\citeyear{RitterWiltschko2005,RitterWiltschko2014}) proposals. Thus, under an analysis according to which [Coin] is a privative feature, its specification coincides with the [$+$Coin] valuation and the anchoring of the proposition described by the \isi{clause} to the utterance context. In case [Irr] and [Coin] were to be specified together, the Infl domain would be specified, in essence, for both [$-$Coin] and [$+$Coin]. If the Infl domain is what indicates whether an eventuality is anchored to the utterance context (temporally) as a whole, it cannot be both necessarily associated with and not associated with the utterance context, which is what specifying both $+$ and $-$ values for [Coin] would entail. Overall, there must be \isi{agreement} within the domain with respect to the \isi{clause}’s association to the utterance context. Therefore, while Irr must check its [\textit{u}V] feature, it cannot do so if [Coin] is specified on T. On the other hand, Irr may freely merge with a TP lacking [Coin]. In this case, \textit{by} is spelled out with \isi{past} morphology on the verb.\largerpage

The well-formedness of \textit{by} with the \isi{infinitive} form of the verb is predicted in a similar fashion. In the absence of TP, Irr may merge directly with EP, satisfying its requirements for [\textit{u}V]-checking in the same way as it would have in being merged with TP. As long as [Coin] is absent, Irr can freely merge with EP (or vP for that matter). Observe that the absence of [Fin] -- whose specification licenses \isi{nominative} \isi{case assignment} and \isi{agreement} -- would predict that the subject not appear in its \isi{nominative} form and the \isi{infinitive} form of the verb would arise without subject \isi{agreement} marking. This prediction is borne out, at least with respect to \isi{case assignment}. Note again in the following example, repeated from \REF{10:ex18}, that the \isi{matrix clause} containing \textit{by} and the \textit{-l} suffix contains a subject in the \isi{nominative} case. Conversely, the construction with \textit{by} and the \isi{infinitive} form of the verb contains a \isi{dative} subject. 

% example 23
\ea \label{10:ex23}
%\langinfo{}{}{\citet[10]{Asarina2006}}\\
	\ea[]{ \label{10:ex23a}
    \gll Oj s”e-l by Petja \{\hspace{-2pt} včera / zavtra\} jabloko!\\       
         oh ate\textsc{-pst} \textsc{by} Peter.\textsc{nom} {} yesterday {} tomorrow apple\\
\glt `If only Peter would eat an/the apple tomorrow!' or
\glt `If only Peter would have eaten an/the apple yesterday!'}
	\ex[]{ \label{10:ex23b}
    \gll Oj s”est’ by Pete \{\hspace{-2pt} včera / zavtra\} jabloko!\\
         oh eat.\textsc{inf} \textsc{by} Peter.\textsc{dat} {} yesterday {} tomorrow apple\\
\glt `If only Peter would eat an/the apple tomorrow!' or
\glt `If only Peter would have eaten an/the apple yesterday!'\\\hfill \citep[10]{Asarina2006}}
\z\z

\noindent Recall that the subject surfaces in a position higher than \textit{by}. I assume that the EPP property holds of the highest head in the Infl domain. I make no commitment to any particular version of the EPP; for our purposes, it simply requires that the \isi{external argument} appear in the \isi{specifier} of the highest Infl head. I conjecture that the \isi{external argument} may move to the \isi{specifier} of T, where it receives case and values the uninterpretable phi-features of T. It may then move on further to the \isi{specifier} of Irr, where it satisfies the EPP. In \textsc{by}+\isi{infinitive} constructions, TP is absent, hence the lack of \isi{agreement} on the verb.

I speculate that Irr, when [Irr] is specified, bears some sort of feature that is optionally strong, allowing for the various available positions of \textit{by} within the \isi{clause}. It is unclear what exactly this feature is and why it optionally takes the verb or the VP more locally. An alternative explanation would be that \textit{by} is phonologically a \isi{clitic}, which would capture why the form cannot appear clause-initially. In fact, there is no generally accepted theory of \ili{Russian} \isi{word order} as of yet (see \citealt{KallestinovaSlabakova2008} and \citealt{Bailyn2011} for discussion), with \isi{subjunctive} data muddying the waters even more. What the reader, I hope, has been convinced of is that \textit{by} spells out a head in the Infl domain. The form interacts directly with Infl categories/properties, namely tense and finiteness, both in terms of distribution and interpretation. If \textit{by} were to spell-out a feature in the CP domain, one would expect it to licitly appear clause-initially, which it can’t. While I have discussed only SVO-ordered clauses, work on \textit{by} in other word orders would shed light on \textit{by}’s position variability.

In summary, \textit{by} is incompatible with the non-\isi{past tense} because the non-\isi{past} morphology spells out the feature [Coin], which itself is semantically at odds with the lack of connection to the utterance context encoded by [Irr], which \textit{by} spells out. It is the lack of [Coin] in infinitival constructions that allows them to appear with \textit{by}. \tabref{10:table:table_2} lists the featural specifications of the indicative and \isi{subjunctive} possibilities that have been discussed.\footnote{\label{10:fn4}Concerning line 3 in \tabref{10:table:table_2}, one could think of \textit{by} as requiring that the \isi{clause} within which it appears is specified for [$-$Coin] (in both the Infl and C domains). The postulation of binary features in this analysis, however, would lead to overgeneration.}

\begin{table}[h]
\caption{Indicative and subjunctive morphology in Russian}
\label{10:table2}
%\begin{tabularx}{\textwidth}{ll}
\begin{tabularx}{.80\textwidth}{rlX}
\lsptoprule
&\textbf{Infl heads} & \textbf{Morphological spell-out}\\
\midrule
1&T: [Fin], (E) & Past tense\\
2&T: [Fin]>[Coin], (E) & Present tense\\
3&Irr: [Irr], T: [Fin], (E) & \textit{By} + Past tense\\
4&Irr: [Irr], (E) & \textit{By} + Infinitive\\
\lspbottomrule
\end{tabularx}
\label{10:table:table_2}
\end{table}

Overall, \textit{by} requires that the event not be bound by the utterance situation, therefore it cannot be anchored with respect to person or time. This conforms to \citeposstpg{Jespersen1924}{319}, cited in \citet[10]{Cowper2002} claim that the \isi{subjunctive} expresses a perspective other than the speaker’s. Moreover, the \isi{semantics} expressed by \textit{by}, such as obligation, desirability, advisability, hypothesis, are captured by this analysis in treating \textit{by} as an irrealis particle.



% SECTION 5
\section{\textit{By} in complement clauses}\label{10:s5}

Work on \textit{by} typically makes note of the particle’s tendency to move to \isi{second position} in a \isi{clause} when some sort of \isi{complementizer} appears in C \citep[29]{Hacking1998}. For instance, there is a strong tendency for \textit{esli} ‘if’ and \textit{by} to appear adjacent to one another in the antecedent of a conditional, as in \REF{10:ex24}. An antecedent with \textit{esli} in which \textit{by} appears farther from the \isi{complementizer}, as in \REF{10:ex25}, is degraded for many speakers. 

% example 24
\ea 
\ea[]{\label{10:ex24}\gll \textbf{Esli} \textbf{by} my zna-l-i ob \.etom, my by vam skaza-l-i.\\
     if \textsc{by} we know\textsc{-pst-pl} about this we \textsc{by} you tell\textsc{-pst-pl}\\}
    \ex[?]{\label{10:ex25}
\gll \textbf{Esli} my zna-l-i \textbf{by} ob \.etom, my by vam skaza-l-i.\\
     if we know\textsc{-pst-pl} \textsc{by} about this we \textsc{by} you tell\textsc{-pst-pl}\\ }
\z 
\glt `If we had known about this, we would have told you.'\hfill \citep[29]{Hacking1998}
\z

\noindent \textit{Čto} ‘that’ also bears a tight relation to \textit{by}. It has been noted, however, that there are speakers for which \REF{10:ex26} is interpreted as equivalent to \REF{10:ex27}. For those who do not get the same interpretation, \REF{10:ex26} merely sounds like an incomplete embedded conditional (\citealt[40]{Brecht1977}).

% example 26
\ea \ea\label{10:ex26}
\gll Ja nikogda ne duma-l, \textbf{čto} Jura \textbf{by} \.eto sdela-l.\\
     I never \textsc{neg} think\textsc{-pst} that Jura \textsc{by} this do\textsc{-pst}\\
     \ex \label{10:ex27}
\gll Ja nikogda ne duma-l, \textbf{čtoby} Jura \.eto sdela-l.\\
     I never \textsc{neg} think\textsc{-pst} \textsc{čtoby} Jura this do\textsc{-pst}\\
\z
\glt `I never thought that Jura would do that.' \hfill (\citealt[40, fn. 10]{Brecht1977})
\z


\noindent Given the high markedness for speakers, it might be that \textit{esli by} and \textit{čtoby} are separate lexical items from the independent \textit{esli}, \textit{čto}, and \textit{by}. \citet{Brecht1977} shows, though, that when the embedded \isi{clause} is comprised of two (and presumably more) conjuncts, \textit{čtoby} appears in the first \isi{clause} and the second \isi{conjunct} contains only an instance of \textit{by}, as in \REF{10:ex28} (see similar discussion on \textit{esli by} in \citealt[29-32]{Hacking1998}).

% example 28
\ea \label{10:ex28}
	 \gll Ty vele-l, \textbf{čtoby} ja uexa-l v Minsk odin, a Vasja \textbf{by} ostalsja s toboj?\\
     you order\textsc{-pst} \textsc{čtoby} I leave\textsc{-pst} at Minsk alone and Vasja \textsc{by} remain with you\\
\glt `Did you order that I leave for Minsk alone and Vasja remain with you?'\\
\hfill \citep[36]{Brecht1977}
\z

\noindent Furthermore, \citet{BarnetovaEtAl1979}, cited in \citet{Hacking1998}, suggest that an element that appears between \textit{esli} and \textit{by} receives a focused reading. In fact, according to a consultant of my own, the following receives a reading according to which \textit{Nikol’} has \isi{contrastive focus}.

% example 29
\ea \label{10:ex29}
\gll Esli Nikol’ by mne skaza-l-a ja by vstreti-l ee v škole.\\
     if Nicole \textsc{by} me.\textsc{dat} tell\textsc{-pst-sg.f} I.\textsc{nom} \textsc{by} meet\textsc{-pst} her.\textsc{acc} at school\\
\glt `If Nicole had told me, I would have met her at school.'
\z

\noindent Suppose \textit{čto} and \textit{esli} and other related complementizers appear in Force, assuming \citeposst{rizzi1997} split CP analysis. The structure of the C domain is shown in \REF{10:ex30}, where “>” simply expresses dominance. Suppose that this full-fledged structure may also be projected in \ili{Russian}.

% example 30
\ea \label{10:ex30}
%\langinfo{}{}{\citet[297]{rizzi1997}}\\
	ForceP > TopP > FocP > TopP > FinP \hfill \citep[297]{rizzi1997}
\z

\noindent As previously mentioned, Force encodes information about \isi{clause} type and FinP works in tandem with ForceP to select either finite or non-finite IPs \citep{rizzi1997}. I have argued in \citet{Melara2014} that complement clauses selected by propositional attitude verbs lack a feature that links a \isi{clause} to the perspective of the speaker, accounting for cross-linguistic differences in what has traditionally been referred to as sequence of tense phenomena. For example, in \ili{English}, a \isi{past tense} in a \isi{complement clause} embedded under a matrix \isi{past tense} will be interpreted either at or before the time of the \isi{matrix clause} event (thus, exhibiting sequence of tense). This is shown in \REF{10:ex31}. In \ili{Russian}, the embedded \isi{clause} in the same tense configuration can instead only be interpreted as prior to the time of the matrix event, not coinciding with it (i.e. it does not exhibit sequence of tense with complement clauses). This is shown in \REF{10:ex32}. Crucially for both languages, the forward-shifted reading in complement clauses is impossible.

% example 31
\ea \label{10:ex31}
	John \textbf{said} that Mary \textbf{was} pregnant.\\
	\ea \label{10:ex31a}
    \textit{Embedded situation coincides with matrix situation}
    \newline John said: “Mary is pregnant.”\hfill\textbf{available}
    
	\ex \label{10:ex31b}
    \textit{Embedded situation precedes matrix situation}
    \newline John said: “Mary was pregnant.”\hfill\textbf{available}
    
	\z
\z

% example 32
\ea \label{10:ex32}
\gll Maša \textbf{skazala}, [\hspace{-2pt} čto Petja \textbf{byl} bolen].\\
    Masha said.\textsc{prf.pst} {} that Petya was sick\\
\glt `Masha said that Petya was sick (i.e., Petya had been sick).'
\ea \textit{Embedded situation does not coincide with matrix situation}\\
      Masha said: “Petya is sick.”\hfill\textbf{unavailable}
\ex      \textit{Embedded situation precedes matrix situation}\\
	Masha said: “Petya was sick.”\hfill\textbf{available}\\\hfill (\citealt[183]{Kondrashova1999}, as cited in \citealt[174]{Mezhevich2006})
\z\z

\noindent In line with what I am arguing for here, I proposed that indicative clauses must be both personally and temporally anchored. In matrix clauses, this is accomplished by a temporal deixis feature in Infl, a personal deixis feature in C/Force, both, or by default when there is no feature specified to express otherwise. In the absence of these anchoring features in T or C, perhaps because a language lacks them altogether, the \isi{clause} is anchored by default to the utterance time and speaker in matrix clauses. Embedded clauses lacking these features are temporally and personally anchored to the time and viewpoint of the (Agent/Experiencer) subject of the embedding \isi{clause}. Accordingly, in both of the \ili{English} and \ili{Russian} sentences above, the embedded \isi{clause} lacks the personal anchoring feature in Force and the embedded clauses are interpreted relative to the perspective of the matrix subject. What makes the temporal interpretations different between the two languages, though, is that \ili{English} has an anchoring feature in Infl (\citepossalt{Cowper2005} [T-deixis]), while \ili{Russian} does not, hence the \ili{English} \isi{complement clause} is thus temporally independent while the \ili{Russian} one depends on the temporal interpretation of the higher \isi{clause}.

I claim that in \ili{Russian}, the same personal anchoring feature is in \isi{complementary distribution} with \textit{čto} `that'. Let’s also call this feature [Coin], manifested in the propositional domain, where anchoring to the utterance context via point-of-view is established. As I have claimed, \textit{by} cannot be bound by the utterance context, due to the irrealis \isi{semantics} of [Irr]. If Fin is the head that establishes a transition from situation to proposition \citep{RamchandSvenonius2014}, then it is possible that [Irr] moves into Fin when the CP domain is projected in order to scope upward within the C domain to ensure that it is not being bound to the utterance context, in violation of [Irr]. This correctly predicts that it is possible, though marked for many speakers, to have a focused element between the \isi{complementizer} and \textit{by}. Furthermore, it captures \textit{by}’s preference for the \isi{second position} in the \isi{clause} when the C domain is overtly projected. 

If indeed [Coin] in Force creates a barrier for inter-clausal operations like temporal anchoring, then we can explain why \isi{subjunctive} complement clauses embedded under a non-\isi{past} \isi{matrix clause} cannot receive a \isi{past tense} interpretation. \REF{10:ex33}, repeated from \REF{10:ex15}, shows that a \isi{past tense} \isi{subjunctive} \isi{clause} under a non-\isi{past} \isi{matrix verb} can receive a \isi{present} or \isi{future} reading but not a \isi{past} one.

% example 33
\ea \label{10:ex33}
	\ea[]{ \label{10:ex33a}
    \gll Ja xoču, čtoby Maša \textbf{zavtra} s”e-l-a jabloko.\\      
         I want \textsc{čtoby} Mary tomorrow eat\textsc{-pst-sg.f} apple\\
    \glt `I want for Mary to eat an apple tomorrow.'
    }
	\ex[]{ \label{10:ex33b}
    \gll Ja xoču, čtoby Maša \textbf{sejčas} e-l-a jabloko.\\
         I want \textsc{čtoby} Mary now ate\textsc{-pst-sg.f} apple\\
    \glt `I want for Mary to be eating an apple right now.'
    }
    \ex[*]{ \label{10:ex33c}
    \gll Ja xoču, čtoby Maša \textbf{včera} s”e-l-a jabloko.\\
         I want \textsc{čtoby} Mary yesterday eat\textsc{-pst-sg.f} apple\\
    \glt Intended: `I want for Mary to have been eating an apple yesterday.'
    }
    \hfill \citep[8]{Asarina2006}
	\z
\z

\noindent The presence of \textit{čto} in Force tells us that Force is not specified for [Coin]. This means the lower \isi{clause} is temporally anchored to the time of the matrix situation. Given that in a matrix non-\isi{past} context, the higher \isi{clause} is specified for the temporal [Coin], the lower \isi{clause} may only be compatible with readings that arise from the specification of temporal [Coin]. In order to get a \isi{past} interpretation of the \isi{subjunctive} \isi{complement clause}, the \isi{matrix verb} must appear in its \isi{past tense} form, as in \REF{10:ex34}.

% example 34
\ea \label{10:ex34}
\gll Ja xote-l, čtoby Maša včera s”e-l-a jabloko.\\
     I want\textsc{-pst} \textsc{čtoby} Mary yesterday ate\textsc{-pst-sg.f} apple \\
\glt `I wanted for Mary to have eaten the apple yesterday.’
\z

\noindent I claim that \ili{Russian} \isi{present} and \isi{future} tense forms both spellout [Coin], hence their similar morphological forms. Their interpretation as \isi{present} or \isi{future} arises from their \isi{aspectual} properties. The \isi{future} reading in \REF{10:ex33a} is therefore licit, since nothing featurally blocks the reading.

Finally and speculatively, it is possible that \textit{čto} and \textit{by} are over time lexicalizing as a single item, with \textit{esli} + \textit{by} lagging slightly in the same process. I leave this question for \isi{future} research.

% SECTION 6
\section{Conclusion}\label{10:s6}

This paper has investigated the morphosyntactic properties of what the literature refers to as the \ili{Russian} \isi{subjunctive}. The particle \textit{by}, which is used to form this type of construction in \ili{Russian}, has been argued to be the spellout of an irrealis head Irr. This functional head was proposed to be the highest head of the \ili{Russian} Infl system, taking a TP, EP, or vP as its complement. I have claimed that Irr encodes irrealis \isi{semantics}. That is, the projection of this head -- the specification of the feature [Irr] -- establishes that the proposition denoted by the \isi{clause} is not bound to the utterance context. Its projection is therefore incompatible with the feature [Coin] in either the Infl or C domains as [Coin]’s specification binds a \isi{clause} to the utterance context temporally or personally, depending on where it is specified. This captures the lack of temporal dependency matrix \isi{subjunctive} clauses exhibit and the lack of commitment on the speaker’s part towards the proposition expressed by the \isi{subjunctive} \isi{clause}. Moreover, the fact that \textit{by} cannot appear with non-\isi{past} morphology stems from the proposal that non-past-tense morphology is the spellout of [Coin]. In essence, then, the subjunctive--indicative mood (or better yet, the irrealis--realis) distinction in \ili{Russian} is one that lies in the projection or non-projection of [Irr].

The analysis presented in this paper ultimately results in the proposal that the non-\isi{past tense} is marked relative to the \isi{past} in \ili{Russian}. Additionally, \textit{by} spelling out a head whose \isi{semantics} are inherently irrealis, the analysis presented also captures the modal-like interpretations of the \ili{Russian} clauses that contain \textit{by}, which namely express obligation, desire, advisability, hypothesis, and so forth on the part of the subject. Also shown was the fact that \textit{by} cannot appear in clause-initial position. This restriction was argued to be due to the fact that \textit{by} moves to the head of FinP in the C domain, which itself is selected by one of the higher heads of an expanded CP layer. 

As noted by a reviewer, clearly the analysis presented here runs \textit{contra} the literature on the \isi{subjunctive}. The \isi{subjunctive} has typically been considered syntactically/semantically impoverished relative to the indicative mood. Under the analysis presented in this paper, the structure of the \ili{Russian} \isi{subjunctive} is structurally more marked compared to the indicative. Ultimately, this analysis supports Wiltschko’s conclusion that while categories like indicative and \isi{subjunctive} may be universal, the way in which they are constructed is language specific. While further work on the morphological closeness of \textit{čto} and \textit{by} ought to be conducted, the analysis presented in this paper has nonetheless proposed a framework of the language’s Infl properties from which further work can springboard.




\section*{Abbreviations}


\begin{tabularx}{.5\textwidth}{@{}lQ@{}}
\textsc{1}&first person\\
\textsc{2}&second person\\
\textsc{3}&third person\\
\textsc{acc}&{accusative}\\
\textsc{dat}&{dative}\\
\end{tabularx}%
\begin{tabularx}{.5\textwidth}{@{}lQ@{}}
\textsc{fut}&\isi{future}\\
\textsc{f}&{feminine}\\
\textsc{gen}&{genitive}\\
\textsc{ipfv}&{imperfective}\\
\textsc{ind}&indicative\\
\end{tabularx}%


\begin{tabularx}{.5\textwidth}{@{}lQ@{}}
\textsc{inf}&{infinitive}\\
\textsc{neg}&negative\\
\textsc{neut}&{neuter}\\
\textsc{nom}&{nominative}\\
\textsc{pl}&{plural}\\
\textsc{pfv}&{perfective}\\
\end{tabularx}%
\begin{tabularx}{.5\textwidth}{@{}lQ@{}}
\textsc{prog}&progressive\\
\textsc{prs}&\isi{present}\\
\textsc{pst}&\isi{past}\\
\textsc{sbj}&{subjunctive}\\
\textsc{sg}&singular\\
\textsc{top}&topic\\ 
\end{tabularx}

\section*{Acknowledgements}
I am greatly indebted to my committee members: Elizabeth Cowper -- my advisor, Michela Ippolito, and Alana Johns, for their guidance, comments, and support on this project. Without any of it, this project would not have been possible. I also owe many thanks to Alëna Aksënova, Oleg Chausovsky, Julie Goncharov, Iryna Osadcha, and Magda Makharashvili for dedicating time to share their knowledge of \ili{Russian} with me -- thank you. Finally, I sincerely appreciate the time the reviewers of this paper contributed to help improve it -- your feedback has been invaluable. Of course, any remaining errors are my own.

%%%%%%%%%%%%%%%%%%%%%%% (END) INSERTION LATEX ANDREI %%%%%%%%%%%%%%%%%%%%%%%%%%%%%%%%%

\sloppy
\printbibliography[heading=subbibliography,notkeyword=this]

\end{document}
