\documentclass[output=paper,modfonts,newtxmath,hidelinks
\ChapterDOI{10.5281/zenodo.2545513}
]{langscibook}

\title{Imperfective past passive participles in Russian}  

\author{Olga Borik\affiliation{The National Distance Education University (UNED)}\lastand  Berit Gehrke\affiliation{Humboldt-Universität zu Berlin}}

\abstract{Contra the received view that Russian past passive participles (PPPs) can only be derived from perfective verb forms, we show that imperfective (IPF) PPPs can be found in corpora as well. A substantial subset of these should receive a compositional analysis, given that they can be used in periphrastic passive constructions with predictable meaning contribution. However, these IPF PPPs commonly require a modifier and occur with a particular information structure, often accompanied by a marked word order, where the event described by the PPP is backgrounded (occurs first) and focus is on the modifier (appearing somewhere after the PPP). We propose an analysis, under which such uses of the IPF are parallel to definite descriptions, in the sense that the IPF signals an anaphoric link to a previously introduced or inferable eventive discourse referent, and the modifier provides new information about this event.

\keywords{presuppositional imperfective, passive, past passive participle, Russian}
}

\begin{document}

\maketitle
\shorttitlerunninghead{Imperfective past passive participles in Russian}

\section{Introduction}
\label{intro}

In \ili{Russian}, as in other \ili{Slavic} languages, there are two types of passives. The \textsc{reflexive passive} is formed by the reflexive marker/postfix \textit{-sja}, whereas the \textsc{periphrastic passive} combines a past passive \isi{participle} (PPP) with a form of \textit{byt'} `be'. It is generally assumed for \ili{Russian} (but not necessarily for other \ili{Slavic} languages; see \sectref{concl}) that the two types of passives are aspectually restricted \citep[e.g.,][]{babbybrecht75}, in the sense that imperfectives only appear in reflexive \REF{storoz}, perfectives only in periphrastic passives \REF{storoz2}.

\ea \label{storoz}
\ea[]{\gll 	Storož otkryval vorota.   \\
	watchman.\textsc{nom} opened.\textsc{ipf} gates.\textsc{acc}\\
\glt `The watchman opened/was opening a/the gate.'}
\ex[]{\gll 	Vorota otkryvalis' storožem.\\
	gates.\textsc{nom} opened.\textsc{ipf}.\textsc{rfl} watchman.\textsc{instr}\\
\glt	`The gate was (being) opened by a/the watchman.'\label{Olegom}}
\ex[*]{\gll	Vorota byli otkryvany  storožem.\\
 	  gates.\textsc{nom} were opened.\textsc{ipf}.\textsc{ppp} watchman.\textsc{instr}\\ }
\z\z

\ea\label{storoz2} 
\ea[]{\gll 	Storož otkryl vorota.	\\				     
	watchman.\textsc{nom} opened.\textsc{pf} gates.\textsc{acc}\\
\glt	`The watchman opened a/the gate.'}
\ex[]{\gll 	Vorota byli otkryty storožem. \\
	gates.\textsc{nom} were opened.\textsc{pf}.\textsc{ppp} watchman.\textsc{instr} \\
\glt	`The gate was opened by a/the watchman.'}
\ex[*]{\gll	Vorota otkrylis' storožem.\\
	  gates.\textsc{nom} opened.\textsc{pf}.\textsc{rfl} watchman.\textsc{instr}\\ }
\z\z

\noindent In this paper, we show that this is an oversimplified view. In particular, we address the occurrence of \isi{imperfective} PPPs in \ili{Russian} periphrastic passives, such as \REF{sity}, which, according to the generalization exemplified above should either not exist at all or be at most exceptional.\footnote{There are also possibly exceptional examples for reflexive passives of \isi{perfective} verbs; see, e.g., \citet{schoorlemmer95} and \citet{fehrmann+10} for relevant examples.} 

\ea\gll Oni byli šity kornjami berezy ili vereska i byli očen' krepki.\\
	they were sewn.\textsc{ipf} roots.\textsc{instr} birch.\textsc{gen} or heather.\textsc{gen} and were very tough\\
\glt	'They were sewn with birch or heather roots and were very tough.' \label{sity}
\z

\noindent From a purely morphological perspective, and also from a cross-\ili{Slavic} perspective, nothing is wrong with \isi{imperfective} PPPs per se. While \REF{pfppp} shows that PPPs are regularly derived from \isi{perfective} verbs, we can see in \REF{ipfppp} that \isi{imperfective} ones exist as well.\footnote{In this paper we set aside long form PPPs and focus on short form PPPs only, such as those in \REF{pfppp} and \REF{ipfppp}, since these are the ones used in passives (see \citealt{borik14} for further discussion).}

% \newpage
\ea\label{pfppp}
\ea \textit{sdelat'} `make.\textsc{pf}' $>$ \textit{sdelan} `made.\textsc{pf}'
\ex \textit{rasserdit'} `make.angry.\textsc{pf}'  $>$ \textit{rasseržen}  `made.angry.\textsc{pf}
\ex \textit{zakryt'} `close.\textsc{pf}' $>$ \textit{zakryt} `closed.\textsc{pf}'
\z\z

\ea\label{ipfppp}
\ea \textit{delat'} `make.\textsc{ipf}' $>$ \textit{delan} `made.\textsc{ipf}'
\ex \textit{slyšat'} `hear.\textsc{ipf}' $>$ \textit{slyšan} `heard.\textsc{ipf}'
\ex \textit{krasit'} `paint.\textsc{ipf}' $>$ \textit{krašen} `painted.\textsc{ipf}'
\z\z

\noindent Nevertheless, the received view is that \isi{imperfective} PPPs like those in \REF{sity} and in \REF{ipfppp} are rare, idiomatic or frozen forms that function like adjectives \citep[e.g.][]{svedova80, schoorlemmer95}.  A common strategy in the discussion of periphrastic passives in \ili{Russian} is therefore to completely ignore such participles \citep{babbybrecht75, paslawskastechow}. A non-standard and somewhat more refined view, and one that we share, is found in \citet{knjazev07}, who notes that \isi{imperfective} PPPs are somehow restricted in use, in comparison to more ``regular'' \isi{perfective} ones. However, he does not give a formal account of their \isi{semantics}, nor a detailed description of when and why such participles appear.

Our goal in this paper is to show, based on naturally occurring data in a corpus, that \isi{imperfective} past passive participles are indeed participles, not only by name and by their morphology, but also by their distribution. We show that they can be participles, not adjectives, based on their predictable compositional \isi{semantics}, as well as their occurrence in regular periphrastic passive constructions, both verbal and \isi{adjectival}. We argue that a subgroup of such participles constitutes a case of the presuppositional \isi{imperfective} \citep[in the sense of][]{gronndiss}, a subtype of the so-called general-factual \isi{imperfective}, which expresses the sheer fact that an/the event took place. 

Among the readings generally associated with the \isi{imperfective aspect} in \ili{Russian}, the general-factual reading, which we will have more to say about in \sectref{whichIPF}, is the most well-studied one. It is usually characterized as a non-canonical reading, in which the \isi{imperfective aspect} is in ``\isi{aspectual} competition'' with the \isi{perfective aspect} \citep[a term that goes back to at least][]{mathesius38}. Canonical \isi{imperfective} meanings that in \ili{Russian} are expressed almost exclusively by \isi{imperfective} forms are process and habitual readings. 

\largerpage[2]
As a side note we want to emphasize that we reserve the terms (im)\isi{perfective} for morphological forms of a given verb, regardless of the \isi{semantics} associated with such forms in a given context. In particular, we study \isi{imperfective} forms used in contexts that might semantically be called \isi{perfective}, namely completed bounded events in the past.

The paper is structured as follows. \sectref{data} outlines the empirical generalization from our corpus study and establishes that \isi{imperfective} PPPs appear in regular periphrastic passives. We also show that the \isi{imperfective} contexts that such participles are found in express non-canonical \isi{imperfective} meanings, and we hypothesize that they always involve either the existential or the presuppositional subtype of the general-factual \isi{imperfective}. \sectref{analysis} provides an analysis of presuppositional \isi{imperfective} PPPs and provides further arguments in favour of such an analysis. Finally, \sectref{concl} concludes and gives an outlook on further research questions and open issues. 

\section{The data}
\label{data}

We extracted data from the \ili{Russian} National Corpus (RNC)\footnote{\url{http://ruscorpora.ru/}} of 109,028 documents, which contained 22,209,999 sentences and 265,401,717 words.  Based on the grammatical features partcp,praet,pass,ipf, we focused on \isi{imperfective} past passive participles directly preceding or following a finite form of \textit{byt'} `be' (BE). Respectively, we found 2,632 and 17,015 contexts, and this reflects the unmarked \isi{word order} status of BE preceding the \isi{participle}. Our search thus excludes participles with non-finite or a null form of BE (i.e. \isi{present tense}), participles as second conjuncts in coordination with, e.g., other participles, etc. Since we used the non-disambiguated corpus version, we manually excluded biaspectual forms, which are marked as \isi{imperfective} in the RNC, such as \textit{obeščan} `promised', \textit{velen} `ordered', and verbs in \textit{-ovat'} (e.g. \textit{ispol'zovan} `used', \textit{realizovan} `realized'). We furthermore excluded all long form participles, given that only short form participles canonically appear in \ili{Russian} periphrastic passive constructions. Finally, we excluded errors in tagging, such as \textit{Sezan} (the \ili{French} painter Cézanne), \textit{strašen} `terrible/scary.\textsc{adj}' (tagged as a \isi{participle}), or \isi{perfective} participles erroneously tagged as \isi{imperfective} (e.g. \textit{otvečen} `answered.\textsc{pf}'). Given these limitations, we will not provide a quantitative analysis. 
 
\largerpage
In the following, we will show that \isi{imperfective} PPPs are not limited to idiomatic expressions, but that we find regular, repeated forms with predictable compositional meaning (\sectref{nonidiom}) that occur in both \isi{adjectival} and verbal passives (\sectref{passive}). We will therefore conclude that such participles (both \isi{adjectival} and verbal ones) need to be accounted for, uniformly, and not just discarded as exceptions.\footnote{A reviewer points out that our data sound archaic. However, we carefully separated all the truly archaic examples (e.g., 17th--18th century and before); only one of those appears in the paper, in \REF{7010}, and we state explicitly that this is an archaic example. All the other examples here are mostly from literary sources from the 1950s--60s, so they cannot be classified as `archaic'. We think that the reviewer might not be used to these kinds of examples because they are not part of the literary norm.} In \sectref{whichIPF} we will conjecture that \isi{imperfective} PPPs always involve the general-factual meaning of the \isi{imperfective aspect}.

\subsection{Non-idiomatic, regular imperfective past passive participles} 
\label{nonidiom}

A first research question was to see whether the wideheld assumption, briefly outlined in \sectref{intro}, according to which all \isi{imperfective} PPPs are idiomatic or frozen forms that should be analyzed as adjectives, withstands closer data scrutiny. Of course we found idiomatic participles, such as the idiom \textit{ne lykom šit}, which is literally `not sewn with bast fiber' but means `not simple(-minded)'. There are also fixed expressions, such as \textit{rožden/kreščen} `born/baptized', and genuine adjectives, such as \textit{viden}, literally `seen' but actually meaning `visible'.

However, we found a number of regular, repeated forms with predictable\linebreak meaning. A non-exhaustive list of such participles is given in \REF{list}.

\ea	\textit{pisan} `written.\textsc{ipf}', \textit{čitan} `read.\textsc{ipf}', \textit{pit} `drunk.\textsc{ipf}', \textit{eden} `eaten.\textsc{ipf}', \textit{delan} `made.\textsc{ipf}', \textit{šit} `sewn.\textsc{ipf}', \textit{čekanen} `minted.\textsc{ipf}', \textit{bit} `beaten.\textsc{ipf}', \textit{strižen} `haircut.\textsc{ipf}', \textit{myt} `washed.\textsc{ipf}', \textit{brit} `shaved.\textsc{ipf}', \textit{kormlen} `fed.\textsc{ipf}', \textit{nesen} `carried.\textsc{ipf}', \textit{govoren} `said.\textsc{ipf}', \textit{prošen} `asked.\textsc{ipf}', \textit{zvan} `called.\textsc{ipf}', \textit{kusan} `bitten.\textsc{ipf}', \textit{kryt} `covered.\textsc{ipf}', \textit{njuxan} `smelled.\textsc{ipf}' \label{list}
\z

\noindent We take these forms to be regular because we found various occurrences (tokens) of a given \isi{participle} (type), in combination with different types of arguments. We furthermore take them to be compositional because we could not detect any idiomatic or idiosyncratic meaning in the contexts we found them in, when compared to the base verbs they are derived from. In particular, their meaning is composed of the meaning of the underlying verb and the meaning of the past passive \isi{participle} (under any account of such participles; see \sectref{passive} for further discussion). 

To get a first impression of the data, some relevant examples in context are given in (\ref{zvany}--\ref{pito}), which we leave uncommented at this moment but will come back to in later discussion. 

\ea\gll	V silu delikatnosti situacii gosti zvany byli s osobym razborom. \label{zvany}\\
	in power delicacy.\textsc{gen} situation.\textsc{gen} guests called.\textsc{ipf} were with particular selection\\
\glt	`Due to a delicate situation the guests were invited upon careful selection.'
\z

\ea\gll	Ništo vam, prinjuxaetes', i ne takoe njuxano bylo. \label{njuxano}\\ 
	nothing you.\textsc{dat}.\textsc{pl} sniff.\textsc{pf} and not such smelled.\textsc{ipf} was\\
\glt	`It does not matter, you will get used to the smell, there are worse smells.'
\z

\ea\gll	Bylo pito, bylo edeno, byli slezy prolity. \label{pito}\\
	was drunk.\textsc{ipf} was eaten.\textsc{ipf} were tears poured.\textsc{pf}\\
\glt	`(Things) were drunk, (things) were eaten, tears were shed.'
\z

\noindent As (\ref{list}--\ref{pito}) show, compositional \isi{imperfective} past passive participles are not limited to one particular verb class. Nevertheless, our manual check reveals that they are often formed from verbs of saying (`say', `ask', etc.) and incremental verbs (`write', `sew', etc.), though not exclusively. This suggests that there might still be lexical restrictions, but this could also be due to limitations of the corpus. In \sectref{concl} we speculate why this might be the case. 

We furthermore found no contemporary participles derived from secondary imperfectives. The ones we did find are all archaic, i.e. at least from before the 19th century, such as the biblical \REF{7010}. 

\ea\gll	V leto 7010 mesjaca avgusta v šestoe na Preobraženie Gospoda našego Iisusa Xrista načata byst' podpisyvana cerkov' [...] \label{7010} \\
	in summer 7010 month.\textsc{gen} august.\textsc{gen} in sixth on transfiguration lord.\textsc{gen} our.\textsc{gen} Jesus.\textsc{gen} Christ.\textsc{gen} begun.\textsc{pf} be.\textsc{aor} signed.\textsc{si} church \\ 
\glt    `In the summer of 7010 on August 6th, on the day of the transfiguration of our Lord Jesus Christ they begun to decorate the walls of the church (lit.: the church was begun to be painted).'
\z

\noindent We therefore conclude for now that PPPs formed from secondary imperfectives are at most extremely rare, and in \sectref{concl} we will provide some informal discussion as to why this may be. 

To sum up, there are clearly compositional \isi{imperfective} PPPs, which cannot simply be discarded as exceptional but need to be accounted for. Let us then turn to the kinds of passives that \isi{imperfective} PPPs occur in.

\subsection{Imperfective past passive participles in periphrastic passives} 
\label{passive}

In this section we address the question whether \isi{imperfective} PPPs can be found in all kinds of passives. For example, if there were only \isi{adjectival} participles, proponents of a lexical approach to such participles could still maintain that they are adjectives, not related to \isi{imperfective} verbs. This would then still be in line with the widespread assumption that there are no \isi{imperfective} PPPs in periphrastic passives, which are then always verbal. It should be noted, however, that we do not take \isi{adjectival} participles to be non-decomposable adjectives, so ultimately we would want to provide a compositional account that also covers \isi{adjectival} participles.

Let us give some general background on verbal vs. \isi{adjectival} passives. We follow the, by now, standard assumption that \isi{adjectival} participles involve adjectivization and combine with a copula, whereas verbal participles `stay' verbal and combine with an auxiliary. For languages like \ili{English}, \ili{German}, and \ili{Spanish}, it has been argued \citep[see][and literature cited therein]{gehrkesub15, gehrkenllt, gehrkemarcolingua, alexiadou+lingua} that unlike with verbal passives, the underlying event in \isi{adjectival} passives lacks spatiotemporal location or referential event participants, and only the state associated with the \isi{adjectival} \isi{participle} can be located temporally. Therefore, spatiotemporal event modifiers, referential by-/with-phrases, and similar such expressions that need to access an actual event, can only appear with verbal participles. In \REF{reifen}, this contrast is illustrated with examples from \ili{German}, which makes a formal distinction between verbal and \isi{adjectival} passives: the former appear with the auxiliary \textit{werden} `become' and the latter with the copula \textit{sein} `be'.\footnote{These and the following \ili{German} examples are based on examples discussed in \citet{gehrkenllt} and literature cited therein.}

\ea\label{reifen}
\ea\gll	Der M\"{u}lleimer \{*\hspace{-2pt} ist / wird\} \{\hspace{-2pt} von meiner Nichte / mit der Heugabel\} geleert.\\
	the {rubbish bin} {} is {} becomes {} by my niece {} with the pitchfork emptied\\
\glt	`The rubbish bin is *(being) emptied \{by my niece / with the pitchfork\}.'
\ex\gll	Der Computer ist vor drei Tagen repariert \#(\hspace{-2pt} worden).\\
	the computer is before three days repaired {} become.\textsc{ppp} \\
\glt	`The computer \{\#is / has been $\sim$ was (being)\} repaired three days ago.'
\z\z

\noindent The modifiers in \REF{reifen} relate to a spatiotemporally located event token with referential event participants, and we assume, following the above-mentioned literature, that only verbal participles make available such an event token. In contrast, non-referential by-phrases, \REF{zeichnung}, and manner modifiers, \REF{schlampig}, which, we assume,  derive an event subkind, are acceptable with \isi{adjectival} participles. 

\ea\label{acc}
\ea\gll	Die Zeichnung ist / wird von einem Kind angefertigt.\label{zeichnung}\\
	the drawing is {} becomes by a child produced \\
\glt	`The drawing is (being) produced by a child.'
\ex\gll	Das Haar war / wurde ziemlich schlampig gek\"{a}mmt.\label{schlampig}\\
	the hair was {} became rather slopp(il)y combed\\
\glt	`The hair was (being) combed in a rather sloppy way.'
\z\z

\noindent Finally, since \isi{adjectival} passives always make available a state, any state-related modification is acceptable as well (see op.cit. for examples).

For \ili{Russian}, we follow \citet{schoorlemmer95} and \citet{borik13, borik14} in  taking short form \isi{perfective} PPPs to be either verbal or \isi{adjectival}; in principle, this should also hold for \isi{imperfective} ones. We take the same modifier restrictions illustrated for \ili{German} in (\ref{reifen}--\ref{acc}) to hold for \ili{Russian} \isi{adjectival} participles, even if we cannot see from the form of BE alone whether we are dealing with an \isi{adjectival} or a verbal \isi{participle}. For example, the temporal modifier in \REF{dom} (discussed in \citealt{borik14}, after an example from \citealt{paslawskastechow}) does not locate the state associated with the \isi{participle} but the underlying event, and therefore, irrespective of the presence/absence of BE, we have to be dealing with a verbal \isi{participle} that makes available an event token for modification.

\ea\gll	Dom (\hspace{-2pt} byl) postroen v prošlom godu.\\
	house.\textsc{nom} {} was built.\textsc{pf} in last year\\
\glt `The house was built last year.'\label{dom}
\z
	
\noindent Thus, if we find such event-related modifiers in our data with \isi{imperfective} PPPs, we can take these to be verbal. This would then refute (or at least seriously jeopardize) the claim that they can appear only in \isi{adjectival} passives. 

As the examples in \REF{pisano} show, we indeed found \isi{imperfective} PPPs co-occurring with such event-related modifiers, highlighted in boldface. In \REF{pisanodost} we find a temporal modifier that locates the underlying event. (\ref{pisanodost}--\ref{vedeno}) contain by-phrases (in \ili{Russian}: instrumental-marked nominals), which are referential, since they contain a proper name, a personal \isi{pronoun}, and an (inherently definite) possessive \isi{pronoun}, respectively. In \REF{Sion} we have a definite spatial expression locating the underlying event.

\ea\label{pisano}
\ea\gll	Pisano \.{e}to bylo \textbf{Dostoevskim} \textbf{v 1871 godu} [\dots]\\ 
	written.\textsc{ipf} that was Dostoevskij.\textsc{instr} {in 1871 year} \\
\glt	`That was written by Dostoevskij in 1871.'\label{pisanodost}
\ex\gll	Recepty \textbf{im} pisany byli i na drugoe imja [\dots]\\ 
	prescriptions he.\textsc{instr} written.\textsc{ipf} were and on other name \\ 
\glt	`The prescriptions were written by him for different names as well.'\label{pisanoim}
\ex\gll 	\.{E}to [\dots] vedeno bylo \textbf{moeju} \textbf{rukoj}!\\
	this {} led.\textsc{ipf} was my.\textsc{instr} hand.\textsc{instr}\\
\glt	`This was orchestrated by me (lit. led by my hand)!'\label{vedeno} 
\ex\gll	[\dots] sleduja \textbf{tem} \textbf{putem}, \textbf{kotorym} neseno bylo v Gefsimaniju dlja pogrebenija telo Bogomateri\\
	{} following that.\textsc{instr} path.\textsc{instr} which.\textsc{instr} carried.\textsc{ipf} was in Gethsemane for burial body {Mother of God} \\
\glt	`\dots on the same path on which the body of the Mother of God was brought to Gethsemane for the burial'\label{Sion}
\z\z

\noindent We thus conclude that \isi{imperfective} PPPs can appear in unambiguously verbal passives and can therefore not be reduced to adjectives. 

On the other hand, it is also not the case that all \isi{imperfective} PPPs are verbal. The following two examples illustrate \isi{adjectival} PPPs: \REF{kryta} involves a non-referential \isi{instrumental} case-marked NP that characterizes the state that the house is in,\footnote{We take `cover' here to be used as a \isi{stative} extent predicate, rather than an eventive change-of-state predicate; see \citet{gawron09}.} and the \isi{adverbial} manner modifier in \REF{nagolo} can only describe a resulting haircut `style', but not the process of cutting hair.

\ea\label{kryt}
\ea\gll 	Kryt byl dom \textbf{solomoj} [\dots]\\ 		
	covered.\textsc{ipf} was house hay.\textsc{instr} 	\\
\glt	`The house was covered with hay.'\label{kryta}
\ex\gll	My oba byli striženy \textbf{nagolo} [\dots]\\
	we both were haircut.\textsc{ipf} bald\\
\glt	`We were both shorn / we both had shaven heads.'\label{nagolo} 
\z\z

\noindent We therefore conclude this section by stating that \isi{imperfective} PPPs appear in both verbal and \isi{adjectival} passives in \ili{Russian}, and that their distribution is not limited to a specific passive construction. In the next section, we turn to the meaning expressed in such passives, namely the general-factual meaning of the \isi{imperfective aspect}.

\subsection{General-factual imperfective past passive participles} 
\label{whichIPF}

In this section, we discuss the \isi{imperfective} contexts that the participles in question appear in. We could corroborate \possessivecite{knjazev07} generalization that they are found in non-progressive \isi{imperfective} contexts only. In particular, we hypothesize that all the examples with \isi{imperfective} PPPs that we found can be analyzed as one or the other type of the general-factual meaning of the \isi{imperfective}. In the following, we give a brief introduction to this kind of reading.

\subsubsection{The general-factual meaning of the Russian imperfective}
\label{OF}

The term \textsc{general-factual} (\textit{obščefaktičeskoe}) goes back to \citet{maslov59} \citep[for recent discussion see][]{mehlig16}. While this is a well-discussed \isi{imperfective} meaning, there is no real consensus in the literature \citep[see][chapter 4 for an overview and references]{gronndiss} as to the precise empirical delineation of this meaning, the question whether or not there are subtypes and if there are, how many, or the theoretical account: Is this an \isi{imperfective} meaning in its own right, or is it a subtype of core \isi{imperfective} meanings (i.e. process or iterative/habitual)? What most authors agree on, however, is that factual imperfectives are in \isi{aspectual} competition with their \isi{perfective} counterparts, in the sense that in many such contexts the \isi{imperfective} can be replaced by the \isi{perfective}, with only subtle meaning differences. In particular, if we are to find a meaning difference at all, it has nothing to do with, e.g., a completed event for the PF and an incompleted one for the IPF. We illustrate this with some of \possessivecite{paduceva96} classical general-factual examples in \REF{oranges}, and their \isi{perfective} counterparts in \REF{orangesPF}. 

\ea\label{oranges}
\ea\gll Ja ubiral komnatu včera.\\
       I cleaned.\textsc{ipf} room.\textsc{acc} yesterday\\
\glt       `I cleaned the room yesterday.'
\ex\gll Gde apel'siny pokupali?\\
               where oranges.\textsc{acc} bought.\textsc{ipf}.\textsc{pl} \\
\glt               `Where did they/you buy the(se) oranges?'
\z\z

\ea\label{orangesPF}
\ea\gll Ja ubral komnatu včera.\\
       I cleaned.\textsc{pf} room.\textsc{acc} yesterday\\
\glt       `I cleaned the room yesterday.'
\ex\gll Gde apel'siny kupili?\\
     where oranges.\textsc{acc} bought.\textsc{pf}.\textsc{pl}\\
\glt               `Where did they/you buy the(se) oranges?'
\z\z

\noindent In both these examples, we are dealing with one-time completed events in the past (cleaning the room and buying oranges), no matter whether the IPF or the PF is used.

\citet{gronndiss} discerns two subtypes of the general-factual meaning: \textsc{existential} and \textsc{presuppositional}.\footnote{These roughly correspond to \possessivecite{paduceva96} existential/concrete general-factual vs. actional distinction.} Existential imperfectives often (but not always) have intonational focus on the verb and are incompatible with precise temporal expressions locating an event. Thus, if we find temporal modifiers at all, these have to be rather vague, or they are temporal frame adverbials specifying a larger interval within which a (series of) event(s) happened (at some point in time or other). There are also contexts which actually require existential imperfectives, such as the epistemically indefinite \textit{kogda-nibud'} `ever' in \REF{Proust}.

\ea\gll Ty kogda-nibud' \{\hspace{-2pt} pročityval /\#\hspace{-2pt} pročital / čital\} roman Prusta do konca? \\
you ever {} read.\textsc{si} {} read.\textsc{pf} {} read.\textsc{ipf} novel Proust.\textsc{gen} until end\\
\glt `Have you ever read a novel by Proust to the end?' \hfill \citep[][73]{gronndiss}\label{Proust}
\z

\noindent Since we will mostly focus on the other type of factual meaning, the presuppositional one, we will not discuss theoretical accounts of existential imperfectives here. Informally this reading can be characterized as `there was (at least) one event of that type', or, under \isi{negation}, `there was no ($\sim$ never any) event of that type' \citep[see][]{mehlig01, mehlig13, muellerkrat, muellerPI, gehrkemueller}. We follow a more general assumption in the literature that the use of existential imperfectives is due to the non-uniqueness, or temporal indefiniteness / non-specificity of the event; when this is marked explicitly, e.g. by \textit{kogda-nibud'} in \REF{Proust}, the use of the \isi{perfective} becomes impossible (see op.cit. for further discussion). 

Presuppositional imperfectives, in turn, come with a different \isi{information structure}: The verb is never accentuated, and  focus is on some other constituent in the sentence. This \isi{imperfective} use is found in the examples in \REF{oranges} and is furthermore illustrated by the boldfaced verb form in \REF{Anna}, where focus is on the clefted \isi{pronoun} \textit{ty} `you' (focus is marked by subscript F).

\ea\gll	Anna otkrovenno brosila emu v lico obvinenie: \.{e}to ty\un{\hspace{1pt}F} \textbf{ubival} ix, a ispol'zoval dlja \.{e}togo menja!\\ 						
	Anna openly threw.\textsc{pf} him in face accusation that you killed.\textsc{ipf} them and used.\textsc{(i)pf} for that me\\
\glt	`Anna openly accused him: It was you who killed them, and you used me to achieve your goal!' \hfill \citep[after][131]{gronndiss}\label{Anna}
\z

\noindent The second sentence in \REF{love letter} \citep[attributed to][]{forsyth70} is another case of the presuppositional \isi{imperfective}, as discussed in \citet[][192f.]{gronndiss}. The first sentence introduces the completed past event `write my first love letter' with a \isi{perfective verb} form (\textit{napisal}). The second sentence is still about this very same event, picked up by the \isi{imperfective} `write'; the event, however, is backgrounded and the intonational focus is on the modifier \textit{karandašom} `with pencil'.

\ea\label{love letter} \gll V \.{e}toj porternoj ja [\dots] napisal pervoe ljubovnoe pis'mo. \textbf{Pisal} karandašom\un{\hspace{1pt}F}.\\
in this tavern I {} wrote.\textsc{pf} first love letter 			 wrote.\textsc{ipf} pencil.\textsc{instr}\\
\glt `In this tavern, I wrote my first love letter. I wrote it with a pencil.'
\z

\sloppy \noindent \citeauthor{gronndiss} assumes that at the VP level this \isi{information structure} leads to a~background--focus division \citep[in the sense of][]{krifka01}. Backgrounded material is argued to be transformed into a \isi{presupposition}, following The Background/Presupposition Rule in \citet{geurtssandt97}. \citeauthor{gronndiss}'s DRT formalization of the \isi{semantics} of the VP in this second sentence in \REF{love letter}, after application of the Background/Presupposition Rule, is given in \REF{gronnanalysis} \citep[][193]{gronndiss}.\footnote{Instead of the probably more familiar box notation for DRSs, \citeauthor{gronndiss} employs a linear simplified notation: To the left of | are the discourse referents one normally finds at the top of a DRS box ($x$ in \REF{gronnanalysis}) and to the right of it are the conditions on such discourse referents, separated by commata \citep[for further discussion see][43]{gronndiss}. 

The VP in \REF{gronnanalysis} is further embedded under AspP. \citet{gronndiss} argues for an underspecified meaning of the \isi{imperfective}, with the event time overlapping the reference time  \citep[building on][]{klein95}. He assumes that this meaning can be strengthened, in the right context, to the kind of \isi{perfective} meaning we get with factual IPFs. In a more recent paper, \citet{gronn15} refrains from giving the \ili{Russian} IPF a uniform denotation, and factual IPFs are argued to have the same denotation as PFs (the event time is included in the reference time). For the full formalization of this example, which also takes into account the contribution of Aspect, Tense and the overall discourse, see op.cit.} 

\ea\label{gronnanalysis}
\sx{VP}${}=\lambda e[x\hspace{0.1em} |\hspace{0.1em} \cnst{instrument}(e, x), \textsc{pencil}(x)]_{\hspace{2pt}[ \hspace{0.3em} |\hspace{0.1em}\textsc{write}(e)]}$
\z
	
\noindent The subscripted part of \REF{gronnanalysis} is argued to introduce presupposed content into the DRS: the writing event is in the background and thus presupposed, whereas `with pencil' is in focus and part of the assertoric content. According to \citet[][192]{gronndiss}, ``the verbal predicate has an eventive argument, an instantiation of which is presupposed, i.e. given (more or less entailed) in the input context". Presuppositions are treated as \isi{anaphora}, which can be bound to an antecedent, e.g. the \isi{perfective} \textit{napisal} in the first sentence in \REF{love letter}, or justified by the input context, as in \REF{departure}.

\ea Dlja bol'šinstva znakomyx vaš [\textbf{ot"ezd}]\un{\hspace{1pt}(pseudo-)antecedent} stal\un{\hspace{1pt}PF} polnoj  neožidannost'ju\dots Vy [\textbf{uezžali}\un{\hspace{1pt}IPF}]\un{\hspace{1pt}anaphora} v Ameriku [ot čego-to, k čemu-to ili že
	prosto voznamerilis'\un{PF} spokojno provesti\un{\hspace{1pt}PF} tam buduščuju starost']\un{\hspace{1pt}F}?\\
`For most of your friends your departure to America came as a total surprise ... Did you leave for America for a particular reason or with a certain goal, or did you simply decide to spend your retirement calmly over there?'\label{departure}\hfill \citep[][207f.]{gronndiss}
\z

\noindent The nominalization \textit{vaš ot"ezd} `your departure' (lit. `off-drival') in the first sentence of \REF{departure} introduces a (one-time, completed) departure event by the addressee. This event is picked up again by the \isi{imperfective} verb form \textit{uezžali} `away-drove' (lit.), which contains a semantically related prefix and the same verbal root (`drive'). In this second sentence, the departure event is backgrounded with respect to the focused elements that inquire about the reason or purpose of the departure. 

Returning to \isi{imperfective} PPPs, a crucial indication that they express a (subtype of the) general-factual \isi{imperfective} meaning is the following. Recall from the beginning of \sectref{OF} that it holds for the general-factual meaning more generally that (in most cases) both \isi{imperfective} and \isi{perfective} word forms can be used, with only subtle meaning differences. When we compare our \isi{imperfective} participles with their \isi{perfective} variants (in those cases where a \isi{perfective} option exists), we get the same effect. This is true of both verbal and \isi{adjectival} participles, hence we classify them as factual imperfectives. \REF{pisanoPF} illustrates this for some of the examples in \REF{pisano} and \REF{kryt} (other examples that we identified as presuppositional imperfectives behave similarly). 

\ea\label{pisanoPF}
\ea\gll	(Na)pisano \.{e}to bylo Dostoevskim v 1871 godu [\dots]\\ 
	(\textsc{pf})written.\textsc{ipf} that was Dostoevskij.\textsc{instr} in 1871 year \\
\glt	`That was written by Dostoevskij in 1871.'
\ex\gll	(Po)kryt byl dom solomoj [\dots]\\ 		
	\textsc{(pf)}covered was house hay.\textsc{instr} 	\\
\glt	`The house was covered with hay.'
\ex\gll	My oba byli (po)striženy nagolo [\dots]\\
	we both were (\textsc{pf})haircut.\textsc{ipf} bald\\
\glt	`We were both shorn / we both had shaven heads.'
\z\z

\noindent The meaning differences between \isi{imperfective} and \isi{perfective} participles are, as expected, very fuzzy and difficult to describe, since in all these cases we have one-time, completed events or states located in the past. 

In the following,  we will first briefly describe existential \isi{imperfective} PPPs, although an account of this class is left for future research. Then we  zoom in on the presuppositional ones and their analysis.

\subsubsection{Existential imperfective past passive participles}

Typical imperfectivity-inducing contexts discussed in the literature include \isi{negation}, repetition, and habituality. Some of the contexts in which we found \isi{imperfective} participles could, in principle, be described as such. For example, \REF{negOF} illustrates negated or negative events. 

\ea\label{negOF}
\ea\gll	[\dots] i ja uže ne byl zvan v gosti [\dots]\\
	{} and I already not was called.\textsc{ipf} in guests \\
\glt `And I was not invited anymore.' 
\ex\gll	Mojka byla perepolnena nemytoj posudoj. Ne myto bylo davno.\\
   	sink was overflown.\textsc{pf} unwashed.\textsc{instr} dishes.\textsc{instr} not washed.\textsc{ipf} was long-time\\
\glt	`The sink was overflowing with unwashed dishes. The dishes had not been done in a long time.'\label{17c}
\z\z

\noindent The following examples involve event repetition (in the broadest sense), evidenced by pluractional markers \REF{golodal} or markers of repeatability/iterativity \REF{neraz} (in boldface).

\ea\gll	Vsego nagljadelsja -- i golodal, i syt \textbf{byval} po gorlo, i bit byl, i sam bil [\dots] \label{golodal}\\
	all.\textsc{gen} saw.\textsc{ipf} {} and starved.\textsc{ipf} and full was.\textsc{freq} until throat and beaten.\textsc{ipf} was and self beat.\textsc{pst}.\textsc{ipf}\\
\glt	`[I] experienced it all -- I starved, and I was full to the top, I was beaten, and I did the beating myself.'\label{bit} 
\z

\ea\label{neraz}
\ea\gll	\textbf{Ne} \textbf{raz} ja byl učen, molču i znaju [\dots]\\ 
	not once I was educated.\textsc{ipf} silent.\textsc{1sg} and know.\textsc{1sg}\\
\glt	`Not just once was I lectured, I remain silent and know ...'
\ex\gll	Za čto \textbf{neodnokratno} byla bita [\dots]\\
	for what not-once was beaten.\textsc{ipf}\\
\glt	`For what she was beaten more than once.'
\z\z
	
\noindent We propose that all these contexts have the informal characteristics of existential imperfectives, outlined in the previous section. In particular, they state that `there were no events of that type (at some point in time or other)' (for the negated examples) and `there were events of that type (at some point in time or other)' (for the other examples). We conjecture that among our previous examples, also \REF{njuxano} (\isi{negation}) and \REF{pito} (event repetition) contain existential imperfectives, but we will leave this for further research. The main focus of this paper are presuppositional \isi{imperfective} PPPs, to which we turn now.

\subsubsection{Presuppositional imperfective past passive participles}

We argue that a prominent subset of the \isi{imperfective} PPPs we found should be analyzed as presuppositional imperfectives, because they display hallmark properties of presuppositional imperfectives: Intonational focus is never on the verb but on some other element in the sentence, and a completed event is backgrounded and presupposed. In focus we find modifiers specifying the manner, quality, purpose or other aspect of the event itself (and not its culmination).\footnote{An anonymous reviewer pointed out that our corpus only contains written texts so that we cannot know where focus is in these sentences. We are reporting here the native \ili{Russian} intuitions of the first author of this paper.} In fact, removing the modifiers sufficiently decreases the acceptance of these examples, though it might be possible to leave them out in the right context. Relevant examples are given in \REF{stroeno}.  

\ea\label{stroeno}
\ea\gll 	Stroeno bylo \.{e}to [\hspace{-2pt} ploxo, xromo, ščeljasto]\un{\hspace{1pt}F}.\\
	built.\textsc{ipf} was that {} badly lamely with.holes\\
\glt	`It was built badly, lamely, with holes.' \label{stroenoa}
\ex\gll 	Zapiski byli pisany ne dlja pečati\un{\hspace{1pt}F} [{\dots} no\dots]\\	
	notes were written.\textsc{ipf} not for print {} but {}\\
\glt	`The notes were written not for print, but ...' \label{zapiski}
\z\z

\noindent The kind of background--focus division typical for presuppositional imperfectives, as described in the previous subsection, is thus also found in our examples. This \isi{information structure} is frequently accompanied by a marked \isi{word order} that has the \isi{participle} (i.e. the backgrounded material) in sentence-initial topic position and the modifier (i.e. the focused material) at the end, after BE, or in some other prominent position, see \REF{textbooks}. This \isi{word order} is marked with respect to the unmarked order of the \isi{participle} following BE, which is otherwise much more frequent (recall our context count in the beginning of \sectref{data}). More such examples are given in \REF{presOF-WO}.

\ea\label{presOF-WO}
\ea\gll	[...] ne skazal, čto vagon-to naš učebnikami\un{\hspace{1pt}F} gružen byl?\\
	{} not said.\textsc{pf} that waggon-\textsc{ptl} our textbooks.\textsc{instr} loaded.\textsc{ipf} was\\
\glt	`He did not tell us that our waggon was loaded with textbooks?'\label{textbooks}
\ex\gll Znamenityj pokojnik nesen byl do mogily {\hspace{60pt}} [\hspace{-2pt} na rukax]\un{\hspace{1pt}F} [\dots] \label{nesen} \\
	Famous deceased.\textsc{nom} carried.\textsc{ipf} was until grave {} {} on arms\\
\glt	`The famous deceased was carried in arms until the grave.'
\z\z

\noindent We also find this \isi{word order} in examples already discussed, namely \REF{zvany}, (\ref{pisanodost}--\ref{vedeno}), \REF{kryta}, and \REF{stroenoa}, which, we argue, also involve presuppositional imperfectives, evidenced by the focussed additional modifiers. However, this marked \isi{word order} is not obligatory for presuppositional \isi{imperfective} participles, as we see in \REF{zapiski}; what is relevant is the background--focus division described above. Finally, this marked \isi{word order} is also found not only with presuppositional imperfectives. For example, in \REF{bit}, which was argued to involve an existential \isi{imperfective}, we find the same marked \isi{word order}. This example is crucially different from the presuppositional imperfectives discussed here, though, in that there is no modifier in focus and instead the intonational focus is on the predicate.

\section{The semantics of presuppositional imperfective past passive participles}
\label{analysis}

We propose to extend \possessivecite{gronndiss} account of presuppositional imperfectives, which originally only covered active cases and which was illustrated in \REF{gronnanalysis}, to passives.\footnote{Note that \citet{gronndiss} acknowledges that factual IPFs are not restricted to \isi{past tense} contexts but that he only concentrated on such contexts for convenience. In \citet{gronn15} he briefly mentions other IPF forms that could be analyzed along the same line, including, e.g., past active participles like \textit{čitavšij} `having read'. Our contribution in this respect is that we broaden the empirical coverage to include the passive data that has previously gone unnoticed, due to the (we hope to have shown) erroneous assumption that IPF PPPs do not deserve a proper compositional analysis.} For example, the analysis of the VP in \REF{stroenoa}, repeated as \REF{ploxo}, is given in \REF{ploxoanalysis}.

\ea\gll	Stroeno bylo \.{e}to ploxo, xromo, ščeljasto.\\
	built.\textsc{ipf} was that badly lamely with.holes\\
    \glt	`It was built badly, lamely, with holes.'\label{ploxo}
    \z

\ea \sx{VP}${}=\lambda e[ \hspace{0.3em}|\hspace{0.1em} \textsc{bad}(e), \textsc{lame}(e), \textsc{with holes}(e)]_{\hspace{2pt}[ \hspace{0.3em} |\hspace{0.1em}\textsc{build}(e)]}$\label{ploxoanalysis}
\z

\noindent Under this analysis, the completion/culmination of the event is not part of the asserted meaning, and the \isi{imperfective} shifts the focus to another aspect of the event, expressed by the  modifier, instead of the culmination of the event itself. 

The presuppositional account makes a number of predictions. One is that presuppositions project, in the sense that, e.g., \isi{negation} affects only the asserted but not the presuppositional content. Thus, if the existence of a completed event is presupposed in the positive counterpart, as illustrated in \REF{stroeno}, the same holds in a corresponding negated sentence in \REF{stroenoneg}.

\ea\label{stroenoneg}
\ea\gll 	Stroeno \.{e}to ne bylo ploxo, xromo, ščeljasto. \\
	built.\textsc{ipf} that not was badly lamely with.holes\\
 \glt   `It was not built badly, lamely, with holes.'
\ex\gll 	Zapiski ne byli pisany ne dlja pečati [... no ...]	\\
	notes not were written.\textsc{ipf} not for print          {}       but {}\\
\glt    `It is not the case that the notes were written not for print, but ...'\label{stroenonegb}
\z\z

\noindent From both the original and the negated examples we infer the existence of a (completed) event, and what is negated in \REF{stroenoneg} is only the contribution of the modifier.\footnote{The negated examples in \REF{stroenoneg} (in particular \REF{stroenonegb} with the double \isi{negation}) sound somewhat unnatural, due to the fact that sentential \isi{negation} usually negates the whole predicate, including the event. Nevertheless, to the extent that they are ok, they still imply event completion.}

Furthermore, if our \isi{imperfective} PPPs are indeed presuppositional, the presupposed events should be bound to a \isi{perfective} in the context or justifiable by the input context, as we briefly discussed in \sectref{OF}. It is important to note at this point that many of Grønn's presuppositional \isi{imperfective} examples in context do not pick up an identical \isi{perfective verb} form, as in Grønn's \REF{love letter}, rather they seem to be merely `justifiable in context', as in Grønn's \REF{departure}.  What does it mean, then, to be justifiable in context? 

In the nominal domain, \isi{anaphora} to previously introduced discourse referents can be expressed by pronouns or by definite descriptions. For example, in \REF{Bruno}, the indefinite \textit{a sister} in the first sentence introduces a new discourse \isi{referent}. The second sentence shows that this discourse \isi{referent} can be picked up by a \isi{pronoun}, by a definite description with identical lexical material (\textit{sister}), but also by a definite description that merely contains a related lexical \isi{noun}, the hyperonym \textit{girl}. 

\ea Bruno has a sister that lives in London. He loves \{her / his sister / the girl\} a lot.\label{Bruno}
\z

\noindent Definite descriptions (but not pronouns) can also be used as bridging \isi{anaphora}, such as \textit{the window screen} in \REF{Carla}.

\ea
Carla was driving to work. The window screen was full of dead bugs.\label{Carla}
\z

\noindent In the verbal domain, pronominal (i.e. pro-verbal) \isi{anaphora} do not really exist, apart maybe from the event kind \isi{anaphora} \textit{so/such}. Thus, presuppositional imperfectives have to be the event counterpart of definite descriptions. These pick up previously introduced event referents, either with identical lexical material or with a hyperonym or a hyponym. Alternatively, they are ``justifiable by the context'', which we then take to be parallel to bridging.

Do we  find such anaphoric relations of our presuppositional \isi{imperfective} participles in the broader contexts they appear in? Some examples showing that we do are given in \REF{anaphOF}.

\ea\label{anaphOF}
\ea\gll 	čto kasaetjsa \textbf{platy} deneg, to \textbf{plačeny} byli naličnymi šest' tysjač rublej [...]\\
	what concerns payment.\textsc{gen} money.\textsc{gen} then paid.\textsc{ipf} were {in cash} six thousand roubles\\
\glt	`As for the payment, six thousand roubles were paid in cash\dots'\label{platy}
\ex\gll 	\.{E}to -- ne ja \textbf{sdelal}, \.{e}to -- \textbf{vedeno} bylo moeju rukoj!\\
	this {} not I did.\textsc{pf} this {} led.\textsc{ipf} was my.\textsc{instr} hand.\textsc{instr}\\
\glt	`It wasn't me who did that, it was orchestrated by me (lit. led by my hand)!'\label{rukoj}	
\z\z

\noindent Example \REF{platy} is similar to Grønn's \REF{departure}, in the sense that here the presuppositional \isi{imperfective} \isi{participle} \textit{plačeny} `paid' refers back to the event inside the related nominalization `payment'. In \REF{rukoj}, the \isi{imperfective} `led' does not lexically repeat the \isi{perfective} `did'; nevertheless, we argue that semantically this is a subtype of doing event and thus a hyponym, so that we are again dealing with an anaphoric relation. 

Finally, let us say a bit more about examples like \REF{kruglo} (and similarly \ref{pisanodost}, \ref{pisanoim}, \ref{zapiski}). 

\ea\gll	\textbf{Pis'ma} ego \textbf{pisany} byli černo i kruglo [...]\\
	letters his written.\textsc{ipf} were black and round \\
\glt	`His letters were written in black and round letters.'\label{kruglo}
\z

\noindent We suggest that in \REF{kruglo}, the created object \textit{pis'ma} `letters' can serve as anaphor for the writing event. In this case, \textit{pis'ma} also happens to be morphologically related to \textit{pisat'} `write' (similarly \textit{za-pis-ki} `notes' in \REF{zapiski}), though this is obviously not a general requirement, see  \REF{pisanodost} and \REF{pisanoim}.

A future task will be to check the contexts more thoroughly and systematically to see which of our \isi{imperfective} PPPs really involve presupposed events, and furthermore to provide an analysis of other occurrences of such participles that do not lend themselves to an analysis in terms of presuppositional imperfectives. As we hypothesized in \sectref{whichIPF}, they might very well turn out to all be instances of the existential meaning of the \isi{imperfective aspect}, but this will have to be confirmed in further research.

\section{Conclusion and open issues}
\label{concl}

In this paper we have shown, based on naturally occurring data, that there are fully compositional \isi{imperfective} past passive participles in \ili{Russian}, which occur in regular periphrastic passives (both \isi{adjectival} and verbal). We therefore refuted the widespread assumption that such participles are non-compositional and should rather be analyzed as adjectives. We have shown that a representative subset of these participles come with a special \isi{information structure} in which the verb is not accentuated but focus lies on a quasi obligatory modifier; this often comes with a marked \isi{word order} in which the \isi{participle} appears in sentence-initial position or at least in a position before BE, and the modifier in focus after BE. We implemented these findings in an account of such participles as involving the presuppositional \isi{imperfective aspect}, where the event (completion) is presupposed and thus backgrounded, signalled by the use of the \isi{imperfective}.

Several issues remain. First, if the empirical finding reported in \sectref{data} is indeed correct, \textit{why are there no (contemporary) secondary \isi{imperfective} past passive participles}? According to \citet{gronndiss}, there are no morphological or lexical restrictions on factual imperfectives, so that both simple as well as secondary imperfectives should be possible. An impressionistic view in the literature, however (see also discussion in \citealt{gronndiss}, ch. 4), is illustrated by the following quote from \citet[][118]{comrie76}: ``The use of the Imperfective as a general-factual is particularly common with non-prefixed verbs, and rather less common with Imperfective verbs that owe their imperfectivity to a suffix that derives them from a Perfective.'' At this point we can only speculate that presuppositional imperfectives are most common with simple imperfectives because these verb forms are morphologically the least marked for grammatical or lexical aspect, and presuppositional imperfectives  generally do not focus on any \isi{aspectual} meaning in particular. This line of argumentation, however, would not necessarily extend to existential \isi{imperfective} participles. Another possibility could be that factual imperfectives historically first arose with a core group of imperfectives (which are all simple) and then spread to others; since \isi{imperfective} PPPs are already quite restricted, maybe only the core verbs are affected. Yet another option could be that there is a real grammatical/morphological restriction on secondary \isi{imperfective} PPP formation in Modern \ili{Russian} (as opposed to earlier stages, as evidenced by our data), though we do not really know why that would be.

A further open issue is \textit{why we do not find more cases of \isi{imperfective} past passive participles}, i.e. why the number is so low, and why we find them more frequently only with a handful of verbs, as tentatively suggested in \sectref{data}. The impression that many verbs of creation appear in this context could be due to the fact that we can infer the event already from the objects themselves, as alluded to at the end of \sectref{analysis}. In addition, we have the intuition that passives are generally not that widely used in \ili{Russian}, though we do not have statistical data to back this up. A potential (informal) explanation for this could be that in languages with a fixed \isi{word order}, such as \ili{English}, passives take on particular information structural functions that languages with a freer \isi{word order}, such as \ili{Russian}, can express in active sentences with different word orders. This, then, could lead to a more restricted use of the passive, so that it is only limited to \isi{aspectual}/event structural functions (see \citealt{abraham06} for argumentation along these lines). Another restricting factor which is suggested by our analysis comes from the specific licensing requirements for the presuppositional \isi{imperfective} passives: if the anaphoric treatment of the presuppositional meaning is correct, these passives can only appear in contexts which can provide a discourse antecedent for the passive sentence.

Finally, there is the issue of \textit{cross-\ili{Slavic} variation in the expression of passives}. From a cross-\ili{Slavic} perspective, the \isi{aspectual} restrictions on the formation of PPPs reported for \ili{Russian} but partially refuted in this paper, is rather surprising. If we look at \ili{Czech}, for example, PPPs can be derived from both \isi{imperfective} and \isi{perfective} verbs, across the board, and without the limited productivity of \isi{imperfective} ones that we clearly find in \ili{Russian}. Furthermore, such participles express verbal or \isi{adjectival} passives, including passive ``events in process'' when we are dealing with \isi{imperfective} ones (Radek Šimík, p.c.).\footnote{Similarly, there are cross-\ili{Slavic} differences in the properties of reflexive passives, which should also be taken into account; see \citet{fehrmann+10} and \citet{schaefer16} for further discussion.} We can think of several possible research questions to be explored in this domain. One could be that languages with ``fully productive'' \isi{imperfective} and \isi{perfective} PPPs (e.g. \ili{Czech}) form regular periphrastic verbal passives with all \isi{imperfective} and \isi{perfective} meanings. For languages like \ili{Russian}, then, two options are conceivable. According to the first, combinations of BE with PPPs are \isi{adjectival}, and only reflexive passives are verbal. Given the availability of event token modification (recall \sectref{passive}), we find this option less convincing. The second option is that combinations of BE and past participles are either verbal or \isi{adjectival}, but can only express result states (\citeauthor{kratzer00}'s \citeyear{kratzer00} \textsc{target states}). Reflexive passives, then, which are always verbal, fill the gap, for verbs that do not have target states, as well as for passive event-in-process readings. Under this hypothesis, though, it is still unclear why the \ili{Russian} periphrastic passive cannot have a process meaning, especially in the cases of verbal/eventive passives. However, there is a split in ``\isi{imperfective} meanings'' conveyed by different passives, in the sense that the process meaning is only conveyed by reflexive passives but other, sometimes called ``peripheral'' \isi{imperfective} meanings, specifically habituality/iterativity and (all types of) \isi{factivity}, are expressed by periphrastic passives (and then usually with \isi{perfective} participles). What seems to be needed to explain this distribution is a competition-based analysis, possibly launched in an optimality theoretic framework. 

\largerpage[2]
\section*{Abbreviations}

\begin{tabularx}{.5\textwidth}{@{}lQ@{}}
\textsc{acc}&{accusative}\\
\textsc{aor}&aorist\\
\textsc{dat}&{dative}\\
\textsc{f}&focus\\
\textsc{gen}&{genitive}\\
\textsc{instr}&{instrumental}\\
\textsc{ipf}&{imperfective}\\
\end{tabularx}%
\begin{tabularx}{.5\textwidth}{@{}lQ@{}}
\textsc{freq}&frequentative\\
\textsc{mod}&modal\\
\textsc{nom}&{nominative}\\
\textsc{pf}&{perfective}\\
\textsc{pl}&{plural}\\
\textsc{ppp}&past passive {participle}\\
\textsc{pst}&{past tense}\\
\end{tabularx}%

\begin{tabularx}{.5\textwidth}{@{}lQ@{}}
\textsc{ptl}&particle\\
\textsc{rfl}&reflexive\\
\end{tabularx}%
\begin{tabularx}{.5\textwidth}{@{}lQ@{}}
\textsc{rnc}&{Russian} National Corpus\\
\textsc{si}&secondary {imperfective}\\
\end{tabularx}


\section*{Acknowledgements}

This research has partially been funded by project FFI2014-52015-P from the Ministry of Economy and Competitiveness (MINECO) and 2014SGR 1013 (awarded by the Generalitat de Catalunya) (1st author). For feedback and discussion we thank especially Hans Robert Mehlig, Atle Grønn, and  two anonymous reviewers, as well as the audiences at the HSE Semantics \& Pragmatics Workshop (Moscow), Event Semantics 2016 (D\"{u}sseldorf), FDSL 12, TELIC 2017 (Stuttgart), and Non-at-Issue Meaning and Information Structure (Oslo).

\largerpage
\sloppy\printbibliography[heading=subbibliography,notkeyword=this]

\end{document}
