\documentclass[output=paper,
modfonts,
newtxmath,
hidelinks
]{langscibook}

% \papernote{\footnotesize\normalfont
% Ekaterina Vostrikova. A puzzle about adverbials in simultaneous readings of present and past-under-past in Russian. To appear in: Denisa Lenertová, Roland Meyer, Radek Šimík \& Luka Szucsich (eds.),\textit{ Advances in formal Slavic linguistics 2016}. Berlin: Language Science Press. [preliminary page numbering]
% }

%\setcounter{chapter}{19}

\title{A puzzle about adverbials in simultaneous readings of present and past-under-past in Russian}  

\author{%
 Ekaterina Vostrikova\affiliation{University of Massachusetts, Amherst}}
 


% \chapterDOI{} %will be filled in at production
% \epigram{}

\abstract{
The present and past tense can both get the simultaneous interpretation in complement clauses when they are embedded under the past tense in Russian. However, I observe that the adverbials that are allowed with present tense in such contexts (for example, \textit{sejčas} ‘now’) are not allowed with the past tense and vice verse (for example, \textit{togda} ‘then’ is not allowed with the present). I show that simply restricting the meaning of those adverbials does not help due to the fact that tenses can be interpreted \textit{de re}. In \textit{de re} construals, tenses are interpreted outside of the clause they originate in, so no meaning conflict between the tense and the adverbial in the embedded clause is predicted. I propose that when a tense is interpreted \textit{de re}, an adverbial has to be interpreted \textit{de re} together with it. I show that under this assumption the observed restriction follows in a direct way.

\keywords{present tense, past tense, past-under-past, present-under-past, de re, attitude reports, temporal adverbials}
}

\begin{document}
\maketitle
\shorttitlerunninghead{Adverbials in simultaneous readings of present and past-under-past}

%%%%%%%%%%%%%%%%%%%%%%%%%%%%% BEGIN CONVERTED LATEX %%%%%%%%%%%%%%%%%%%%%%%%%%%%%%%%%


% \begin{styleHeading}
% Adverbial puzzle in simultaneous readings of present and past-under-past in Russian
% \end{styleHeading}

% \begin{styleNormal}
% Ekaterina Vostrikova
% \end{styleNormal}

% \begin{styleNormal}
% University of Massachusetts, Amherst
% \end{styleNormal}

% SECTION 1
\section{Introduction}\label{s1}
% SUBSECTION 1.1
\subsection{Simultaneous readings of present-under-past and past-under-past in Russian}\label{s1.1}

In this paper I will discuss simultaneous readings that the present tense and the past tense can receive in complement clauses embedded under the past tense in Russian. I will point out that there are some restrictions on adverbials that can occur in such clauses and I will attempt to explain those restrictions.

In Russian, the simultaneous reading of past tense in a complement clause embedded under past tense in a main clause is in principle available \citep{Altshuler2008}. Usually, the simultaneous reading is not the most salient one. For example, the most salient reading of the sentence in \REF{ex1}, is the back shifted reading: according to Tanja, Putin was a president some time in the past with respect to 2016, the time when she pronounced the sentence.\footnote{\label{fn2}The Russian judgment reported in this paper are my own judgments confirmed with other native speakers of Russian.}

% example 1
\ea \label{ex1}
\gll V 2016 godu Tanja skazala, čto Putin byl prezidentom Rossii.\\
     in 2016 year Tanja say\textsc{.past} that Putin be\textsc{.past} president\textsc{.inst} Russia\textsc{.gen}\\
\glt `In 2016 Tanja said that Putin was the president of Russia.'
\z

\noindent The simultaneous reading of past-under-past in \REF{ex1} can be enforced by some adverbials, such as \textit{togda} ‘then’.\footnote{\label{fn3}In this respect, Russian behaves like Hebrew, as it was reported in \citet{OgiharaSharvit2012}.} In \REF{ex2} \textit{togda} anaphorically refers to 2016 and the interpretation where the event of saying and the state of being the president of Russia overlap in time becomes the most salient.

% example 2
\ea \label{ex2}
\gll V 2016 godu Tanja skazala, čto togda Putin byl prezidentom Rossii.\\
     in 2016 year Tanja say\textsc{.past} that then Putin be\textsc{.past} president\textsc{.inst} Russia\textsc{.gen}\\
\glt `In 2016 Tanja said that Putin was the president of Russia then.'
\z

\noindent Like Hebrew \citep{OgiharaSharvit2012}, Russian also has a relative present tense. For example \REF{ex3}, where the verb in the embedded clause has the present tense features and the verb in the main clause has the past tense features, expresses the idea that Tanja said that Putin was president at the time when she pronounced the words ‘Putin is the president’.\footnote{\label{fn4}Note that present tense copula (indicated by $\varnothing$ and glossed as be.\textsc{pres}) is silent in Russian.}

% example 3
\ea \label{ex3}
\gll V 2016 godu Tanja skazala, čto Putin ${\varnothing}$ prezident Rossii.\\
     in 2016 year Tanja say\textsc{.past} that Putin be\textsc{.pres} president\textsc{.nom} Russia\textsc{.gen}\\
\glt `In 2016 Tanja said that Putin was the president of Russia.'
\z

% SUBSECTION 1.2
\subsection{The adverbial puzzle}\label{s1.2}

Past-under-past and present-under-past in Russian both seem to be able to express the simultaneity of the time of the eventuality described by a complement clause and the time when the embedded claim was made. However, even though \textit{togda} really enforces the simultaneous reading of past-under-past as we saw in  \REF{ex2}, it is completely unacceptable in a complement clause with the present tense embedded under the past tense; a relevant example is given in \REF{ex4}.\footnote{The symbol {\#} is used, when the sentence or expression is ill-formed due to meaning.}

% example 4
\ea[\#]{\gll V 2016 godu Tanja skazala, čto Putin togda ${\varnothing}$ prezident Rossii.\\
     in 2016 year Tanja say\textsc{.past} that Putin then be\textsc{.pres} president\textsc{.nom} Russia\textsc{.gen}\\
\glt Intended: `In 2016 Tanja said that Putin was the president of Russia then.'}\label{ex4}
\z

\noindent If past-under-past and present-under-past can get the same interpretation, why is \textit{then} possible in the embedded clause in the first case, but not in the second?

On the other hand, there are some adverbials, such as \textit{sejčas} ‘now’, that are compatible with present-under-past in Russian \REF{ex5}, but not with past-under-past \REF{ex6}. Thus, the presence of \textit{sejčas} in \REF{ex6} makes it ill-formed, whereas without \textit{sejčas} both sentences \REF{ex5} and \REF{ex6} can have the simultaneous reading.\footnote{\label{fn6}I do not translate \REF{ex5} into English as `When I talked to her 3 years ago, Tanja told me that she is pregnant now' because this English sentence does not have the relevant reading due to the fact that there is no relative present in English and \textit{now} is indexical, unlike \textit{sejčas} `now'.}

% example 5
\ea \label{ex5}
\gll Kogda ja govorila s nej tri goda nazad, Tanja skazala, čto ona\hspace{10pt} (\hspace{-2pt} sejčas) ${\varnothing}$  beremenna.\\
     when I talk\textsc{.past} with her three years ago Tanja say\textsc{.past} that she {} now be.\textsc{pres} pregnant\\
\glt `When I talked to her three years ago, Tanja told me that she was pregnant (at that time).'
\z

% example 6
\ea \label{ex6}
\gll Kogda ja govorila s nej tri goda nazad, Tanja skazala, čto ona byla (\#\hspace{-2pt} sejčas)     beremenna.\\
     when I talk\textsc{.past} with her three years ago Tanja say\textsc{.past} that she be.\textsc{past} {} now pregnant\\
\glt (Intended:) `When I talked to her three years ago, Tanja told me that she was pregnant (at that time).'
\z

\noindent This is the adverbial puzzle that I will address in this paper. The fact that not all tenses are compatible with all adverbials has been previously noticed in the literature (for example, see the discussion in \citealt{Hornstein1990}). What is special about the embedded contexts considered here is that past-under-past and present-under-past seem to be able to contribute the same meaning. Thus, it is not clear why there would be a meaning clash between an adverbial and the tense in one case but not in the other. Moreover, as I show in this paper, \textit{togda} is an anaphoric element, and as such, it can pick different time intervals. There are many adverbial that denote a specific time interval and that are compatible with the present tense. However, there is something about the meaning of \textit{togda} that makes it impossible for this element to pick the time intervals denoted by those adverbials. Another novel, to my knowledge, observation that I make in this paper is that the fact that \textit{togda} and \textit{sejčas} are distributed the way they are in the embedded contexts is not predicted by the existing theories of embedded tenses.

The discussion will go as follows. In \sectref{s2} I will show that \textit{togda} ‘then’ does not require past tense. Then I will provide the semantics of \textit{togda} that accounts for the restriction on its use with present tense in Russian. I will suggest that \textit{togda} carries a presupposition that the time intervals it picks are not equal to the evaluation time and will show how this presupposition accounts for the observed restrictions.

I will introduce my assumptions about the structure of the embedded clauses and the relative present in Russian and will show how the semantics of \textit{togda} presented here correctly predicts the restrictions on its use in embedded contexts.

For the simultaneous reading of past-under-past I will adopt the classical \textit{de re} approach \citep{Abusch1997,Heim1994}. I will show that the presupposition of \textit{togda} that I am introducing is weak enough to make it compatible with the simultaneous reading of past-under-past.

In \sectref{s3} I will show that the \textit{de re} analysis of the simultaneous reading of past-under-past incorrectly predicts that Russian \textit{sejčas} ‘now’ should be able to appear in such a context. Since under the \textit{de re} analysis the tense moves out of the embedded clause and is interpreted separately from the adverbial, no meaning clash is predicted between the past tense and the present-oriented adverbial \textit{sejčas}. I will propose that this problem can be solved if we adopt an assumption that a tense and an adverbial are interpreted together. Since, when past tense gets the simultaneous reading under past in believe/say contexts, it is interpreted outside of the embedded clause, the adverbial \textit{sejčas} has to be interpreted outside of the embedded clause as well.

In \sectref{s4} I will show that a similar problem arises in English \textit{then} is predicted to be compatible with the \textit{de re} interpretation of the present tense (which derives the so-called double access reading). \sectref{sConclusion} summarizes the findings.

% SECTION 2
\section{Why \textit{togda} is not compatible with the present tense}\label{s2}
% SUB SECTION 2.1
\subsection{\textit{Togda} is not compatible with the present tense in matrix and embedded contexts}\label{s2.1}

\textit{Togda} is an anaphoric element, in the sense that it makes a reference to a time interval that has been mentioned in the previous discourse. Thus in \REF{ex7}, it makes reference to the interval picked by \textit{v prošlom godu} ‘last year’. In \REF{ex8} it is anaphoric to the future time interval, the interval picked by \textit{čerez tri goda} ‘in three years’.

% example 7
\ea \label{ex7}
\gll V prošlom godu moj syn byl v pervom klasse. Togda on učilsja čitat’.\\
     in last year my son be.\textsc{past} in first grade then he learn\textsc{.past.refl} read.\textsc{inf}\\
\glt `Last year my son was in the first grade. He was learning to read then.'
\z

% example 8
\ea \label{ex8}
\gll Čerez tri goda moj syn pojdet v pervji klass. Togda on naučitsja čitat’.\\
     in three years my son go.\textsc{fut} in first grade then he learn.\textsc{fut.refl} read.\textsc{inf}\\
\glt `In three years my son will be in the first grade. He will learn to read then.'
\z

\noindent However, example \REF{ex9}, where \textit{togda} ‘then’ appears in a clause with present tense, is not acceptable. The reason for this must be that \textit{togda} cannot pick the time interval denoted by \textit{v ėtom godu} ‘this year’.

% example 9
\ea \label{ex9}
\gll V ėtom godu moj syn ${\varnothing}$ vo vtorom klasse. On izučaet matematiku (\#\hspace{-2pt} togda).\\
     in this year my son be.\textsc{pres} in second grade he study\textsc{.pres} math {} then\\
\glt (Intended:) `This year my son is in the second grade. He studies math (this year).'
\z

\noindent There is some general principle that restricts the use of adverbials with the present tense both in English and in Russian. For example, the sentence in \REF{ex10} does not mean that I am running now and it is 5am now.\footnote{\label{fn8}See \citet{Kamp-Reyle1993} for a pragmatic explanation for this fact.} It is felicitous only on the planned future interpretation.\footnote{\label{fn9}As an anonymous reviewer points out, those sentences can be used felicitously with the present tense interpretation in some contexts. For example, \REF{ex10} can be used if the previous discourse was `No one believed that I will start running, but here I am, running at 5am'.}

% example 10
\ea[\#]{\gll Ja begu v pjat’ utra.\\
     I run\textsc{.pres} in five morning\\
\glt Intended: `I am running now and it is 5am.'}\label{ex10}
\z

% example 11
\ea[\#]{\gll Putin ${\varnothing}$ president v 2018 godu.\\
     Putin be.\textsc{pres} president in 2018 year\\
\glt Intended: `Putin is president now and it is 2018.'}\label{ex11}
\z

\noindent I would like to leave this more general problem out of the scope of the discussion here. In order to do so, I will compare \textit{togda} with those adverbials that are completely compatible with the present tense.

One example of such adverbial is ‘this year’, as shown in \REF{ex9}. The question I will be focusing on is why in sentences like \REF{ex9} \textit{togda} cannot pick the same time interval as the one denoted by ‘this year’ and be compatible with the present tense given that it can easily pick the interval denoted by ‘last year’ in \REF{ex7} and ‘in three years’ in \REF{ex8}.

We can see from the well-formedness of \REF{ex12}, where ‘this year’ occurs in the embedded clause (with the embedded present tense) and is anaphoric to ‘2016’ of the main clause, that \textit{v ėtom godu} ‘this year’ in Russian can pick a year that is current with respect to the local evaluation time (Tanja’s ‘now’ at the time when she said those words).

% example 12
\ea \label{ex12}
\gll V 2016 godu Tanja skazala mne, čto v ėtom godu ee syn ležit v bol’nice.\\
     in 2016 year Tanja say\textsc{.past} me that in this year her son lie\textsc{.pres} in hospital\\
\glt `In 2016 Tanja told me that that year her son was in the hospital.'
\z

\noindent In \REF{ex13} \textit{v ėtom godu} ‘this year’ occurs in the main clause and the sentence without \textit{togda} has the simultaneous reading. The presence of \textit{togda} makes this sentence ill-formed. Since \textit{v ėtom godu} ‘this year’ is perfectly compatible with the present tense (embedded, as in \REF{ex12} and unembedded, as in \REF{ex14}), the badness of \textit{togda} in \REF{ex13} must be due to the fact that it somehow cannot refer to this interval.

% example 13
\ea \label{ex13}
\gll Ja govorila s Tanej v ėtom godu I ona skazala mne, čto ee syn (\#\hspace{-2pt} togda) vse ešče ležit v bol’nice.\\
     I talk\textsc{.past} with Tanja in this year and she say\textsc{.past} me that her son {} then all still lie\textsc{.pres} in hospital\\
\glt (Intended:) `I talked with Tanja this year and she told me that her son was still in the hospital (then).'
\z

% example 14
\ea \label{ex14}
\gll Tanin syn v ėtom godu vse ešče ležit v bol’nice.\\
     Tanja’s son in this year all still lie\textsc{.pres} in hospital\\
\glt `Tanja’s son is still in hospital this year.'
\z

\noindent In principle, \textit{togda} can pick an interval that is inside the interval denoted by ‘this year’ as it is shown in \REF{ex15}, where it occurs with the past tense embedded under past. In \REF{ex15} \textit{togda} anaphorically refers to the time when Tanja pronounced the words.

% example 15
\ea \label{ex15}
\gll Ja govorila s Tanej v ėtom godu i ona skazala mne, čto ee syn togda vse ešče ležal v bol’nice.\\
     I talk.\textsc{past} with Tanja in this year and she say\textsc{.past} me that her son then all still lie\textsc{.past} in hospital\\
\glt `I talked to Tanja this year and she told me that her son was still in the hospital then.'
\z

\noindent The question I will address here is why \textit{togda} cannot denote a time interval that is compatible with present tense.

% SUB SECTION 2.2
\subsection{\textit{Togda} does not require past tense}\label{s2.2}

I will start this discussion by ruling out the simple idea that Russian \textit{togda} requires past tense in the same clause to be licensed. One implementation of such an idea would be that \textit{togda} has to agree with past tense and the agreement relation can only be established locally.

In Russian, there are several adverbials that have a meaning similar to \textit{togda} and can occur in subordinate clauses with past tense embedded under past. They are listed in \REF{ex16}. The fact that all of them are good with past-under-past is shown in \REF{ex17}.

% example 16
\ea \label{ex16}
\gll v to vremja / v tot moment / na tot moment\\
     in that time {} in that moment {} on that moment\\
\glt `at that time'/ `at that moment'/ `by that moment'
\z

% example 17
\ea \label{ex17}
\gll Ja govorila s Tanej v ėtom godu i ona skazala mne, čto ee syn\hspace{4pt} \{\hspace{-2pt} v to vremja / v tot moment / na tot moment\} ležal v bol’nice.\\
     I talk\textsc{.past} with Tanja in this year and she say\textsc{.past} me that her son {} in that time {} in that moment {} on that moment lie\textsc{.past} in hospital\\
\glt `I talked to Tanja this year and she told me that her son was in the hospital \{at that time / at that moment / by that moment\}.'
\z

\noindent All of them are infelicitous with present tense embedded under past, as it is shown in \REF{ex18}.

% example 18
\ea[\#]{\gll Ja govorila s Tanej v ėtom godu i ona skazala mne, čto ee syn \{\hspace{-2pt} v to vremja / v tot moment / na tot moment\} ležit v bol’nice.\\
     I talk\textsc{.past} with Tanja in this year and she say\textsc{.past} me that her son {} in that time {} in that moment {} on that moment lie\textsc{.pres} in hospital\\
\glt Intended: `I talked to Tanja this year and she told me that her son was in the hospital \{at that time / at that moment / by that moment\}.'}\label{ex18}
\z

\noindent A strong argument against the hypothesis that all of those elements have to be licensed by past tense comes from the fact that all of them are in fact compatible with future tense in the same clause. One example where \textit{togda} occurred in a matrix clause with a future tense was given in \REF{ex8}. In \REF{ex19} I show that all of the adverbials given in \REF{ex16} are compatible with an embedded future. The antecedent for \textit{togda} or \textit{na tot moment} ‘at that moment’ in \REF{ex19} is given in a previous sentence and the resulting sentence is well-formed.

% example 19
\ea \label{ex19}
\gll My obsuždali 2019 god. Tanja skazala, čto \{\hspace{-2pt} na tot moment / v tot moment / togda\} Medvedev budet prezidentom.\\
     we discuss\textsc{.past} 2019 year Tanja say\textsc{.past} that {} on that moment {} in that moment {} then Medvedev be.\textsc{fut} president\textsc{.inst}\\
\glt `We discussed 2019. Tanja said that \{by that moment / at that moment / then\} Medvedev would be the president.'
\z

\noindent We can conclude that it is not the case that \textit{togda} (as well as other anaphoric elements that are compatible with past-under-past and incompatible with present-under-past) needs to be licensed by the past tense in the same clause.

% SUB SECTION 2.3
\subsection{The semantics of \textit{togda}}\label{s2.3}

I suggest that \textit{togda} has the semantics given in \REF{ex20}. \textit{Togda} carries an index that is mapped to a contextually given time interval (an interval \textit{togda} is anaphoric to). It denotes a function of type $\semantictype{i,t}$: a function that takes a time interval and returns truth if that interval surrounds the contextually given time (translating this into the set-talk: it denotes a set of time intervals that surround the contextually given interval). The key part of this semantics is the presupposition that \textit{togda} carries: the time intervals it picks cannot be equal to the evaluation time with respect to which \textit{togda} is interpreted.

% example 20
\ea \sx{togda$_5$}$^{w,t,g,c}=\lambda t':t'\neq t\,.\,g(5)\subseteq t'$\label{ex20}
\z

\noindent A stronger presupposition that would also prevent \textit{togda} from picking the time interval denoted by ‘this year’ would be that the time interval it picks does not overlap with the evaluation time. However, this would incorrectly predict that \textit{togda} is incompatible with the simultaneous reading of past-under-past.

Let us consider what happens if we try to make \textit{togda} to be anaphoric to the time interval denoted by ‘this year’. The index 5 is mapped to the year long interval surrounding the evaluation time.

% example 21
\ea $g(5)={}$the year of $t$\label{ex21}
\z

\noindent Given those assumptions, the resulting meaning of \textit{togda} with the index 5 is given in \REF{ex22}.

% example 22
\ea \sx{togda$_5$}$^{w,t,g,c}=\lambda t':t'\neq t\,.\,{}$the year of $t\subseteq t'$\label{ex22}
\z

\noindent If we put together the semantics of \textit{togda} given in \REF{ex22} and present tense in a matrix context we will get a contradiction.

I will demonstrate this on the example of the second sentence of \REF{ex9} that is given here separately as \REF{ex23}. The LF for it is given in \REF{ex24}.

% example 23
\ea \label{ex23}
\gll On izučaet matematiku (\#\hspace{-2pt} togda)\\
     He study\textsc{.pres} math {} then\\
\glt (Intended:) `He studies math (now).'
\z

% example 24
\ea {[\un{IP} [\un{I} PRES$_4$] [\un{vP$'$} [\un{AdvP} togda$_5$] [\un{vP} he$_7$ [\un{VP} studies 	math]]]]}\label{ex24}
\z

\noindent I will assume that VPs like ‘studies math’ denote functions of type $\semantictype{e,\semantictype{i,t}}$. Thus, the vP gets the denotation of type $\semantictype{i,t}$. Let's assume that the assignment function $g$ maps the index $7$ to John.

% example 25
\ea \sx{vP\un{\REF{ex24}}}$^{w,t,g,c}=\lambda t'\,.\,{}$John studies math at $t'$\label{ex25}
\z

\noindent Since temporal adverbials like \textit{togda} also denote functions of type $\semantictype{i,t}$ (predicates of times), they can combine with vPs via predicate modification. The result of this is given in \REF{ex26}.

% example 26
\ea \sx{vP$'$\un{\REF{ex24}}}$^{w,t,g,c}=\lambda t':t'\neq t\,.\,{}$the year of $t\subseteq t'$ \& John studies math at $t'$\label{ex26}
\z

\noindent I will adopt the pronominal semantics for tenses \citep{Partee1973}. Tenses carry indices, thus, like other pronouns, they get their denotation via the assignment function $g$. The semantics for present tense that I will assume is given in \REF{ex27}. It simply denotes a specific time interval and presupposes that this time interval is equal to the evaluation time.

% example 27
\ea \sx{PRES$_4$}$^{w,t,g,c}=g(4)$\smallskip\\
\sx{PRES$_4$}$^{w,t,g,c}$ is only defined if $g(4)=t$\label{ex27}
\z

\noindent The predicate of times in \REF{ex26} combines with the present tense via function-argument application. The predicted denotation for \REF{ex24} is given in \REF{ex28}. \textit{Togda} carries a presupposition that the time intervals it selects are not equal to the evaluation time. Present tense carries a presupposition that the time interval it denotes is equal to the evaluation time. What follows from this is that when \textit{togda} and the present tense combine there will be a contradiction. This accounts for the infelicity of \textit{togda} with present tense in matrix contexts.

% example 28
\ea \sx{IP\un{\REF{ex24}}}$^{w,t,g,c}=\cnst{t}$ iff John studies math at $g(4)$ and the year of $t\subseteq g(4)$\smallskip\\
\sx{IP\un{\REF{ex24}}}$^{w,t,g,c}$ is defined only if $g(4)=t$ and $g(4)\neq t$\label{ex28}
\z

\noindent Note that nothing prevents \textit{togda} from picking a time interval within the current year as long as it is in the past or future with respect to the evaluation time. This is a good prediction because we still want to account for the well-formedness of \REF{ex15}.

The contradiction is predicted to arise when \textit{togda} is used in embedded clauses with the present tense as well.

In languages where present-under-past can get the simultaneous reading, it is standardly interpreted as a relative present: a tense that denotes a local evaluation time \citep{Ogihara1989,vonStechow1995,OgiharaSharvit2012}.

The denotation for the relative present is given in \REF{ex29}: essentially it has the same denotation as the regular present in Russian.

% example 29
\ea \sx{PRES-REL$_1$}$^{w,t,g,c}=g(1)$\smallskip\\
\sx{PRES-REL$_1$}$^{w,t,g,c}$ is only defined if $g(1)=t$\label{ex29}
\z

\noindent I will make the following assumptions about the structure and the interpretation of embedded clauses in belief reports. Intensional verbs are quantifiers over world--time pairs. The intensional verb ‘say’ combines with its complement clause via a version of the rule of Intensional Functional Application \citep{Heim-Kratzer1998}. An intension of an expression XP is computed as shown in \REF{ex30}.

% example 30
\ea $\lambda w'\lambda t'\,.\,{}$\sx{XP}$^{w',t',g,c}$\label{ex30}
\z

\noindent I will make my point by using the example given in \REF{ex31}. The LF for the embedded clause is given in \REF{ex32}. For the simplicity of exposition, I reconstructed the subject to its base-position.

% example 31
\ea \label{ex31}
\gll Tanja skazala mne v ėtom godu, čto ee syn (\#\hspace{-2pt} togda) ležit v bol’nice.\\
     Tanja say\textsc{.past} me in this year that her son {} then lie\textsc{.pres} in hospital\\
\glt (Intended:) `Tanja told me this year that her son was in the hospital (at that time).'
\z

% example 32
\ea {[\un{IP} [\un{I} PRES-REL$_1$] [\un{vP$'$} [\un{AdvP} togda$_5$] [\un{vP} her$_7$ son be in hospital]]]}\label{ex32}
\z

\noindent With these assumptions, I predict that the embedded clause with \textit{togda} in our problematic sentence \REF{ex31} will have the intension given in \REF{ex33}.

% example 33
\ea $\lambda w'\lambda t':t'\neq t'\,.\,{}$Tanja's son is in hospital in $w'$ at $t'$ \& the year of $t'\subseteq t'$\label{ex33}
\z

\noindent This intension includes a contradictory presupposition, thus the infelicity of \textit{togda} is predicted.

One natural question arising at this point is whether the presence of this presupposition is predicted to block the use of \textit{togda} in simultaneous reading of past-under-past as well because this would not be the desired result, as the well-formedness of \REF{ex2} shows. This is the question I will address in the next subsection.

% SUB SECTION 2.4
\subsection{\textit{Togda} and the simultaneous reading of past-under-past in Russian}\label{s2.4}

\noindent In order to account for the restriction on the use of \textit{togda} with embedded and matrix present tense in Russian, I suggested that \textit{togda} in Russian comes with a presupposition that the time intervals it picks are not equal to the local evaluation time.

The simultaneous reading of past-under-past in complement clauses in principle can be derived at least in two ways. One option is a past tense deletion rule. In this system, the past features on the embedded past are not interpreted and an embedded past is interpreted as a relative tense \citep{Ogihara1989,Ogihara1995}. A~relative tense is interpreted as a local evaluation time, thus, in this system, given the definition I proposed for \textit{togda}, \textit{togda} is predicted to be infelicitous with the simultaneous reading of the past tense.

But this is not the only way past tense could get the simultaneous interpretation. Another standardly assumed way of deriving the simultaneous reading is the \textit{de re} construal \citep{Abusch1997,Heim1994,OgiharaSharvit2012}. In what follows I will introduce the classic analysis of the \textit{de re} construal and I will show that the presupposition of \textit{togda} that I am proposing is not predicted to be in a conflict with the simultaneous reading of past-under-past in complement clauses.

Thus, I will assume that when \textit{togda} is acceptable with the simultaneous reading of past-under-past, the simultaneous reading is derived via the \textit{de re} construal.

I will show how the system works by using example \REF{ex2}, repeated here as \REF{ex34}.

% example 34
\ea \label{ex34}
\gll V 2016 godu Tanja skazala, čto togda Putin byl prezidentom Rossii.\\
     in 2016 year Tanja say\textsc{.past} that then Putin be\textsc{.past} president\textsc{.inst} Russia.\textsc{gen}\\
\glt `In 2016 Tanja said that Putin was the president of Russia then.'
\z

\noindent \citeauthor{Abusch1997} proposed to extend the \textit{de re} analysis for singular terms developed by \citet{Kaplan1969}, \citet{Lewis1979} and \citet{Cresswell-vonStechow1982} to the analysis of tenses in intensional contexts. In my exposition of the temporal \textit{de re} construal I will use \citeposst{Cable2015} exposition of this system, which relies on \citeposst{Heim1994} implementation.

The past tense undergoes movement within the lower clause, leaving a trace t$_4$ and triggering lambda abstraction, indicated by $4$ in \figref{ex35}. The result of this movement is a predicate of times in the embedded clause.

After that, the past tense undergoes another short movement that is called the res-movement \citep{Heim1994}. This type of movement is special because the moved element does not leave a trace and does not move to a c-commanding position.\footnote{\label{fn10}Due to those properties res-movement is highly controversial from the syntactic perspective. There is a less controversial way of deriving \textit{de re} readings (developed for individual arguments) via concept generators that was proposed by \citet{PercusSauerland2003}.} It moves to the position of the sister of the verb \textit{say}. Thus, this tense will be interpreted outside of the clause where it originates.\largerpage[-3]

% example 35
\begin{figure}\begin{footnotesize}
\begin{forest}for tree={s sep=3mm, inner sep=0, l=0}
[IP
	[DP
    	[\sout{Tanja}, roof]
    ]
[I$'$
    	[I
        	[PAST$_2$]
        ]
        [vP
            [AdvP
            	[in 2016, roof]
            ]
            [vP
                [DP
                    [Tanja, roof]
                 ]
                 [VP
                   	[V
                    	[say]
                        [PAST$_4$, name=spec PAST]
                    ]
                    [CP
                    	[C
                        	[that]
                        ]
                        [IP, s sep=8mm
                        	[{},inner sep=2.5mm, name=spec IP] {
                                            					\draw[->] () to [out=south west,in=south west] (spec PAST);
                                                        }
                        	[IP, s sep=1cm
                            	[4]
                                	[IP
                                    	[DP [\sout{Putin}, roof]]
                                    [I$'$
                                    	[I
                                        	[t$_4$] {
                                            					\draw[->] () to [out=south west,in=south west] (spec IP);
                                                            }
                                        ]
                                        [vP
                                        	[\textit{togda}$_7$ Putin is president, roof]
                                        ]
                                    ]
                                    ]
                            ]
                        ]
                    ]
                 ]
             ]
        ]
]
]
\end{forest}\end{footnotesize}
\caption{LF of \REF{ex34}}\label{ex35}
\end{figure}

%%%%%%%%%%%%%%%%%%%%%%%%%%%%%%%%%%%%%%%%%
 
Intensional verbs like \textit{say} are ambiguous between their regular denotation and the denotation given in \REF{ex36}.

% example 36
\ea \sx{say}$^{w,t,g,c}={}$\smallskip\\
\hspace{0.5cm}$\lambda t_i$\tabto{3cm}the object of believe (res)\smallskip\\
\hspace{1cm}$\lambda Q_{\semantictype{s,\semantictype{i,\semantictype{i,t}}}}$\tabto{3.5cm}the intension of the predicate of times\smallskip\\
\hspace{1.5cm}$\lambda y_e$\tabto{4cm}the attitude holder\smallskip\\
\hspace{2cm}$\lambda t'_i$\tabto{4.5cm}the time of saying\smallskip\\
\hspace{2.5cm}$\exists P_{\semantictype{s,\semantictype{i,\semantictype{i,t}}}}[t={}$the time $z$ such that $P(w)(t')(z)$ \&\smallskip\\
\hspace{2.5cm}$\forall\langle w'',t''\rangle\in\cnst{say-alt}(y,w,t'): Q(w'')(t'')($the $z$ such\smallskip\\
\hspace{2.5cm}that $P(w'')(t'')(z))=\cnst{t}]$\label{ex36}
\z

\noindent The function denoted by \textit{say} first combines with the tense that has been moved from the lower clause and now is its sister. Then it combines with the intension of the predicate of times created by the movement. After that, it takes an individual (the subject) and the time argument of the higher clause. Intensional verbs contribute quantification over time-concepts (relations between a world, time and another time). Those time-concepts should be understood as descriptions by which a believer represents a time interval to herself.

The denotation for \textit{v 2016 godu} ‘in 2016’ is given in \REF{ex37}: it denotes a set of intervals within 2016.

% example 37
\ea \sx{v 2016 godu}$^{w,t,g,c}=\lambda t'\,.\,t'\subseteq 2016$\label{ex37}
\z

\noindent \textit{Togda} in \REF{ex34} can either anaphorically refer to 2016 or the time in 2016 when Tanja said the words. I will assume the first option (but nothing hinges on this choice): the assignment function $g$ maps index $7$ on \textit{togda} to the set of intervals in 2016. \textit{Togda} will denote the set of intervals that surround 2016.

% example 38
\ea \sx{togda$_7$}$^{w,t,g,c}$\tabto{2.3cm}${}=\lambda t':t'\neq t\,.\,g(7)\subseteq t'$\smallskip\\
\tabto{2.3cm}${}=\lambda t':t'\neq t\,.\,2016\subseteq t'$\label{ex38}
\z

\noindent The intension of the predicate of times is computed in \REF{ex39}. In this system, the time of Putin’s presidency (in Tanja’s say-alternatives) and the local evaluation time are two distinct times. \textit{Togda} contributes the presupposition that those two intervals are not equal to each other.\footnote{\label{fn11}The full \textit{de re} analysis requires another presupposition in the embedded clause that the time of the state or eventuality described in the embedded clause is not in the future with respect to the local evaluation time (the upper limit constraint; cf. \citealt{Abusch1997}). The full intension of the embedded clause is shown in \REF{fnex1}. This presupposition is responsible for the fact that past-under-past cannot have the forward shifted reading.

\ea $\lambda w\lambda t\,.\,$\sx{4 [t$_4$ [togda$_7$ Putin be president] \dots]}$^{w,t,g,c}$\\
\hspace{0.5cm}${}=\lambda w\lambda t\lambda t':t'\neq t \&\neg t'>t\,.\,{}$Putin is the president in $w$ at $t'$ \& $2016\subseteq t'$\label{fnex1}
\z
}

% example 39
\ea $\lambda w\lambda t\,.\,$\sx{4 [t$_4$ [togda$_7$ Putin be president] \dots]}$^{w,t,g,c}$\smallskip\\
\hspace{0.5cm}${}=\lambda w\lambda t\lambda t':t'\neq t\,.\,{}$Putin is the president in $w$ at $t'$ \& $2016\subseteq t'$\label{ex39}
\z

\noindent The resulting semantics for the entire sentence is given in \REF{ex40}.

% example 40
\ea \sx{\figref{ex35}}$^{w,t,g,c}=\cnst{t}$ iff\smallskip\\
$\exists P:g(4)={}$the time $z$ such that $P(w)(g(2))(z)\;\&\;g(2)\subseteq 2016\;\&$\\
$\forall\langle w'',t’’\rangle\in\cnst{say-alt}(\textsc{Tanja},w,g(2)): [\lambda t':t'\neq t''\,.\,{}$Putin is the president in $w''$ at $t'$ \& $2016\subseteq t']($the $z$ such that $P(w'')(t'')(z)) =\cnst{t}$\medskip\\
\sx{\figref{ex35}}$^{w,t,g,c}$  is defined only if $g(2)<t$ and $g(4)<t$\label{ex40}
\z

\noindent This sentence is predicted to be true in case there is a time concept $P$ that relates the particular time in the past when Tanja pronounced those words in the actual world and the past moment denoted by the moved past tense such that the same relation also holds between the time when Tanja located herself in her doxastic alternatives (her local now) and the time when Putin is the president in her doxastic alternatives.

One such possible time concept in the case under consideration is given in \REF{ex41}.

% example 41
\ea $\lambda w\lambda t'\lambda t''\,.\,t''$ is a year-long interval that surrounds $t'$ in $w$\label{ex41}
\z

\noindent The two intervals this concept relates are not equal to each other: one surrounds the other one, thus the presupposition introduced by \textit{togda} is satisfied. The existence of the concept given in \REF{ex41} can make the entire formula in \REF{ex40} true. The presupposition requires that $g(4)$ and $g(2)$ are in the past. Given this time-concept, the first conjunct in \REF{ex40} is as follows \REF{ex42}.

% example 42
\ea $g(4)={}$the time $z$ such that $z$ is a year-long interval that covers
\glt \hspace{0.5cm}the time $g(2)$ (the time when Tanja said those words) \& $g(2)\subseteq 2016$\label{ex42}
\z

\noindent The second conjunct is also true: in all of Tanja’s doxastic alternatives, Putin is president at the time $z$ such that $z$ is a year-long interval that surrounds her local `now' (at the time when she said the words) and $2016\subseteq z$.

Since the temporal \textit{de re} construal derives the simultaneous reading of past-under-past without requiring that the two time intervals are exactly equal, the presupposition that \textit{togda} carries is not going to be harmful for the meaning of the sentence. Thus, the presupposition that I am proposing is strong enough to rule out \textit{togda} with present-under-past, but is weak enough to make it compatible with a simultaneous reading of past-under-past.

The semantics \REF{ex40} also accounts for the fact that \textit{togda} enforces the simultaneous reading.\footnote{\label{fn12}I would like to thank an anonymous reviewer who made this point.}

\textit{Togda} picks the intervals that surround the time it is anaphoric to. When \textit{togda} is in an embedded say-context and it is anaphoric to the time of saying, it is predicted to contribute the claim that what is described by the embedded sentence is happening at the time that surrounds the time of saying (from the speaker’s perspective).

%SECTION 3
\section{\textit{Sejčas} in simultaneous readings of past-under-past in Russian}\label{s3}

The set of assumptions that made it possible for us to derive the compatibility of \textit{togda} with the simultaneous reading of past-under-past in complement clauses, leads to the prediction that \textit{sejčas} (‘now’) should be acceptable in those contexts as well. Thus the contrast between \REF{ex5} and \REF{ex6} (repeated here as \REF{ex43} and \REF{ex44}) is not predicted.

% example 43
\ea \label{ex43}
\gll Kogda ja govorila s nej tri goda nazad, Tanja skazala, čto ona\hspace{5pt} (\hspace{-2pt} sejčas) ${\varnothing}$ beremenna.\\
     when I talk\textsc{.past} with her three years ago Tanja say\textsc{.past} that she {} now be.\textsc{pres} pregnant\\
\glt `When I talked to her three years ago, Tanja told me that she was pregnant (then/at that time).'
\z

% example 44
\ea \label{ex44}
\gll Kogda ja govorila s nej tri goda nazad, Tanja skazala, čto ona byla (\#\hspace{-2pt} sejčas) beremenna.\\
     when I talk\textsc{.past} with her three years ago Tanja say\textsc{.past} that she be.\textsc{past} {} now pregnant\\
\glt (Intended:) `When I talked to her three years ago, Tanja told me that she was pregnant (at that time).'
\z

\noindent In \REF{ex43} \textit{sejčas} ‘now’ is acceptable, which means that \textit{sejčas} is not an unshiftable indexical in Russian. In \REF{ex43} \textit{sejčas} picks the time interval three years ago when the conversation happened. Thus I will treat \textit{sejčas} as sensitive to the evaluation time and not to the context time as shown in \REF{ex45}: \textit{sejčas} denotes a predicate of times that is true of intervals that surround the local evaluation time. (If instead of surrounding we chose a relation of being equal to, it would not have any significant effect on the final outcome of the system.)

% example 45
\ea \sx{sejčas}$^{w,t,g,c}=\lambda t'\,.\,t\subseteq t'$\label{ex45}
\z

\noindent Under those assumptions the fact that \textit{sejčas} is acceptable in \REF{ex43} follows straightforwardly (with the assumption that present tense in Russian can be interpreted as a relative present).\largerpage[2]

The main interest for us here is the example \REF{ex44} and the fact that \textit{sejčas} is not acceptable in this context. Again I will assume the \textit{de re} construal for the simultaneous reading of past-under-past in \REF{ex44}. The LF that will be interpreted here, namely \figref{ex46}, is structurally identical to the one given in \figref{ex35} (but the lexical items are different).

% example 46
\begin{figure}
\begin{footnotesize}
\begin{forest}baseline, for tree={s sep=3mm, inner sep=0, l=0}
[IP
	[DP
    	[\sout{Tanja}, roof]
    ]
    [I$'$
    	[I
        	[\textsc{PAST}$_{2}$]
        ]
        [vP
            [AdvP
            	[3 years\\ago, roof]
            ]
            [vP
                [DP
                    [Tanja, roof]
                 ]
                 [VP
                   	[V                    	
                    	[say]
                        [\textsc{PAST}$_{4}$, name=spec PAST] 
                    ]
                    [CP
                    	[C
                        	[that]
                        ]
                        [IP, s sep=1cm
                        	[{},inner sep=2.5mm, name=spec IP] {
                                            					\draw[->] () to [out=south west,in=south west] (spec PAST);
                                                        }
                        	[IP, s sep=1cm
                            	[$4$]
                                [IP
                                	[DP
                                    	[\sout{she$_7$}, roof]
                                    ]
                                    [I$'$
                                    	[I
                                        	[t$_{4}$] {
                                            					\draw[->] () to [out=south west,in=south west] (spec IP);
                                                            }
                                        ]
                                        [vP
                                        	[\textit{sejčas} she$_7$ is pregnant, roof]
                                        ]
                                    ]
                                ]
                            ]
                        ]
                    ]
                 ]
             ]
        ]
    ]
 ]
\end{forest}\end{footnotesize}
\caption{LF of the relevant part of \REF{ex44}}\label{ex46}
\end{figure}


%%%%%%%%%%%%%%%%%%%%%%%%%%%%%%%%%%%%%%
 
Again, the attitude verb combines with its res-argument (the tense that was moved from the lower clause), the intension of the predicate of times created by the movement, an individual (the matrix subject), and the time argument of the matrix clause. The intension of the embedded clause is given in \REF{ex47} (under the assumption that $g$ maps index $7$ to Tanja).

% example 47
\ea $\lambda w\lambda t\,.\,{}$\sx{4 [t$_4$ [\textit{sejčas} she$_7$ is pregnant]]}$^{w,t,g,c}$\smallskip\\
\hspace{0.5cm}${}=\lambda w\lambda t\lambda t'\,.\,t\subseteq t'\;\&\;{}$Tanja is pregnant in $w$ at $t'$\label{ex47}
\z

\noindent The resulting semantics for the entire sentence is given in \REF{ex48}.\largerpage

% example 48
\ea \sx{\figref{ex46}}$^{w,t,g,c}=\cnst{t}$ iff\smallskip\\
$\exists P:g(4)={}$the time $z$ such that $P(w)(g(2))(z)\;\&\;g(2)$ is a time 3 years ago \& $\forall\langle w'',t''\rangle\in\cnst{say-alt}(\textsc{Tanja},w,g(2)):[\lambda t'\,.\,t''\subseteq t'$ \& Tanja is pregnant in $w''$ at $t']($the $z$ such that $P(w'')(t'')(z)) = \cnst{t}$\medskip\\
\sx{\figref{ex46}}$^{w,t,g,c}$ is defined only if $g(2)<t$ and $g(4)<t$\label{ex48}
\z

\noindent The contribution that \textit{sejčas} ends up making is that the time of the state described by the embedded clause (Tanja’s pregnancy) surrounds the time when Tanja locates herself in her doxastic alternatives at the time of saying. This should give us the simultaneous reading.

One possible concept that will be suitable in this case is given in \REF{ex49}.

% example 49
\ea $\lambda w\lambda t'\lambda t''\,.\,t''$ is a 9-month interval that surrounds $t'$ in $w$\label{ex49}
\z

\noindent In the actual world, $g(4)$ (past time from the embedded clause) is the $z$ such that it is the 9-month interval that surrounds $g(2)$ (past time of saying). In Tanja’s alternatives Tanja is pregnant at the time $z$ such that $z$ is the 9-month interval that surrounds her local now.

Intuitively it is clear that the clash happens because \textit{sejčas} has a present tense orientation and it is not compatible with the past tense. However, the past tense under the \textit{de re} analysis of the simultaneous reading of past-under-past is not interpreted in the same clause as \textit{sejčas}, thus no clash is predicted. Moreover the presence of \textit{sejčas} in the sentence is predicted to enforce the simultaneous reading of past-under-past the way `then' enforces it, due to the fact that \textit{sejčas} picks intervals that surround the local evaluation time.

In order to account for the fact observed in \REF{ex44} I suggest that when tense is interpreted outside of the embedded clause, \textit{sejčas} is interpreted together with it.

This can be implemented in a system where tense and the adverbial undergo the res-movement together. To move \textit{sejčas} together with tense, I will allow tense to combine with adverbials directly: I will change the denotation of tenses and suggest that they take predicates of times (like the one denoted by \textit{sejčas} or \textit{togda}) as their first arguments \REF{ex50}. I will consider tense pronouns to be definite articles of times: they combine with a predicate of times and return a specific time interval. In doing so I do not derail in a significant way from the pronominal semantics of tense. I adopt the idea that all pronouns are definite articles \citep{Elbourne2005}. The pronominal element is still there in the semantics of tense suggested in \REF{ex50}. In this system, just like in the classic pronominal approach to the semantics of tense, past tense denotes a particular interval of time. An adverbial acts like a restrictor on the possible intervals that the tense can denote.

% example 50
\ea \sx{PAST$_2$}$^{w,t,g,c} =\lambda P_{\semantictype{i,t}}\,.\,\iota t'\;P(t')=\cnst{t}\;\&\;t' = g(2)$\smallskip\\
\sx{PAST$_2$}$^{w,t,g,c}$ is defined only if $g(2)<t$\label{ex50}
\z

\noindent If \textit{sejčas} undergoes res-movement to the matrix clause together with the past tense, the restriction on the use of \textit{sejčas} in simultaneous readings of past-under-past that we observe in \REF{ex44} follows directly. The predicted result of applying past to \textit{sejčas} is given in \REF{ex51}. This is because given our definition in \REF{ex45} the time interval denoted by \textit{sejčas} has to surround the evaluation time, which for the matrix clause is the time of evaluation of the entire sentence, i.e. now. The time interval denoted by the past has to strictly precede the evaluation time. There is no interval that is simultaneously strictly in the past with respect to the current moment and surrounds it. Thus the clash between the past tense and \textit{sejčas} is predicted.

% example 51
\ea \sx{PAST$_2$}$^{w,t,g,c}($\sx{sejčas}$^{w,t,g,c}) = \iota t'\; t\subseteq t'\;\&\; t' = g(2)$\smallskip\\
\sx{PAST$_2$}$^{w,t,g,c}($\sx{sejčas}$^{w,t,g,c})$ is defined only if $g(2)<t$\label{ex51}
\z

% SECTION 4
\section{\textit{Then} with present-under-past in English}\label{s4}

In English, present-under-past in complement clauses cannot get the simultaneous reading. The English present tense cannot be interpreted as a relative present. The absence of the relative present reading in \REF{ex52} shows that the English present tense is sensitive to the context time and not the evaluation time. In \REF{ex52} present-under-past gets only the so-called double access reading.

% example 52
\ea This year Tanja said that Putin is the president of Russia.\label{ex52}
\z

\noindent This reading requires that if what Tanja said was true when she said it, then Putin must be the president of Russia now. This reading requires the embedded claim to be true at both the matrix utterance time and at the time of the doxastic alternatives. \citet{Abusch1997} has shown that this reading can be derived if we interpret present tense of the embedded clause in \REF{ex52} \textit{de re}.\largerpage[-1]

The present tense undergoes res-movement \citep{Heim1994}. This creates the LF structurally similar to the one given in \figref{ex35}. Again, given the semantics for \textit{say} in \REF{ex36} there is a parallelism requirement on the relation between the present tense moved from the embedded clause and the past moment of saying on the one hand and the relation between the time of the presidency in Tanja’s say-alternatives and the time when she locates herself on the other. Due to a constraint on the interpretation of embedded tenses called the upper limit constraint -- the idea that tense of an embedded clause cannot be a future directed concept -- this cannot be the relation of the past preceding present and Putin’s presidency being in the future with respect to Tanja’s local now \citep{Abusch1997}. The only other option is the relation of the surrounding, where the interval denoted by the present tense surrounds the one denoted by the past.\

In English \textit{then} is also not compatible with present-under-past \REF{ex53}.

% example 53
\ea {\#}This year Tanja said that Putin is the president of Russia then.\label{ex53}

\z

\noindent Even if English \textit{then} has the same denotation as Russian \textit{togda} and carries the relevant presupposition, the restriction observed in \REF{ex53} does not follow unless we make an assumption that the adverbial has to undergo the res-movement together with the present tense. This is the same problem as the one we saw with the Russian \textit{sejčas} in \textit{de re} construals.

In a \textit{de re} construal tense moves out of the clause it originates in. The presupposition of the non equality between the evaluation time and the time intervals \textit{then} picks will translate in this system into the requirement of non-identity of the time of presidency and the time when Tanja locates herself. This is not problematic, given that relation between them is the relation of surrounding. \textit{This year} is an adverbial that is compatible with the present tense in English, thus if \textit{then} can be anaphoric to this adverbial, no clash is predicted between \textit{then} and the present tense.

The restriction we observe in \REF{ex53} is straightforwardly predicted in the system where the English \textit{then} has the same denotation as the Russian \textit{togda,} shown in \REF{ex54}, and tense adverbials are interpreted together with  tense. If tense undergoes the res-movement, the adverbial has to move with it.

% example 54
\ea \sx{then$_5$}$^{w,t,g,c}=\lambda t':t'\neq t\,.\,g(5)\subseteq t'$\label{ex54}
\z

\noindent If we extend the analysis suggested here for the Russian \textit{sejčas-}cases to English cases with \textit{then}, the fact observed in \REF{ex53} follows without any further assumptions. Present tense takes \textit{then} as its argument. The result of this is shown \REF{ex55}. Since \textit{then} moves together with the tense and is also interpreted in the matrix clause, there is predicted to be a clash between the presupposition of the present tense (that it denotes the time interval equal to the context time that equals to the matrix evaluation time) and the presupposition of \textit{then} (that the intervals it picks are not equal to the evaluation time). This is shown in \REF{ex55}.

% example 55
\ea \sx{PRES$_2$}$^{w,t,g,c}($\sx{then$_5$}$^{w,t,g,c})=\iota t'\;g(5)\subseteq t'\;\&\;t'=g(2)$\smallskip\\
\sx{PRES$_2$}$^{w,t,g,c}($\sx{then$_5$}$^{w,t,g,c})$ is defined only if $g(2)=t_c$ and $g(2)\neq t$\label{ex55}
\z

\noindent Thus if we extend the analysis suggested here for the Russian \textit{sejčas-}cases with the embedded past to English \textit{then}{}-cases with the embedded present, the ill-formedness of \REF{ex53} follows without any further assumptions.


% SECTION CONCLUSION
\section{Conclusion}\label{sConclusion}

In this paper, I looked at simultaneous readings of present-under-past and past-under-past in complement clauses in Russian. I have formulated the adverbial puzzle: there are adverbials like \textit{togda} `then' that can enforce the simultaneous reading of past-under-past but are completely infelicitous with present-under-past; and there are adverbials like \textit{sejčas} `now' that are compatible with an embedded relative present, but not with past-under-past.

I suggested that the restriction on the use of \textit{togda} in Russian can be explained if \textit{togda} carries a presupposition that the time intervals it picks are not equal to the evaluation time. I have shown that this presupposition is strong enough to make \textit{togda} incompatible with the relative present, however weak enough to be compatible with the simultaneous reading of past-under-past. The reason for this is that the simultaneous reading of past-under-past in Russian is derived via \textit{de re} construal and the meaning resulting from this construal does not require that the two intervals are equal, it is enough for them to simply overlap.

I have demonstrated that the fact that \textit{sejčas} is felicitous with present-under-past in Russian and is not acceptable with the simultaneous reading of past-under-past does not follow from the classic \textit{de re} analysis of simultaneous readings of past-under-past. The reason for this is that since the past tense moves out of the embedded clause, no meaning clash is predicted between the meaning of the present oriented adverbial \textit{sejčas} and the past tense. I have shown that the fact that \textit{sejčas} is infelicitous with past-under-past in Russian follows straightforwardly if we allow it to be interpreted \textit{de re} together with an embedded past tense. I extended this analysis to explain the fact that \textit{then} is not compatible with present-under-past in English.\largerpage[-2]


%%%%%%%%%%%%%%%%%%%%%%%%%%%%% END CONVERTED LATEX %%%%%%%%%%%%%%%%%%%%%%%%%%%%%%%%%%%%


\section*{Abbreviations}

\begin{tabularx}{.5\textwidth}{@{}lQ@{}}
\textsc{gen}&genitive\\
\textsc{fut}&future tense\\
\textsc{inf}&infinitive\\
\textsc{inst}&instrumental\\
\end{tabularx}%
\begin{tabularx}{.5\textwidth}{@{}lQ@{}}
\textsc{nom}&nominative\\
\textsc{past}&past tense\\
\textsc{pres}&present tense\\
\textsc{refl}&reflexive\\
\end{tabularx}

\section*{Acknowledgements}

I would like to thank Seth Cable for his help with this project. I am also grateful to Barbara Partee, Petr Kusliy, Sakshi Bhatia and the participants of the semantics seminar on tense and aspect at UMass (Fall 2015) and FDSL 12 for their useful comments and feedback. Also I would like to thank the editors of the volume and two anonymous reviewers for their comments and suggestions.

\sloppy
\printbibliography[heading=subbibliography,notkeyword=this]

\end{document}
