\documentclass[output=paper,modfonts,newtxmath,hidelinks]{langscibook} 
\ChapterDOI{10.5281/zenodo.2545541}

\title{Number agreement mismatches in Russian numeral phrases} 

\author{Elena Titov\affiliation{University College London}}

\abstract{This paper looks at two cases of number agreement mismatch in Russian numeral phrases and offers a unified syntactic analysis for both. One case relates to examples where a higher numeral that typically selects a plural NP fails to do so when the head noun lacks a singular lexical form. Instead, an NP headed by a noun that lacks a plural lexical form is chosen despite the selectional requirement of the numeral. The second case concerns data discussed in \citet{Franks-House1982} that involve topicalization of a complement of a lower numeral, which consistently selects a singular NP, with the topicalized NP unexpectedly appearing in the plural form. 

\keywords{Russian numeral phrases, number agreement, syntax, morphology, contrastive topicalization, information structure}
}

\begin{document}
\maketitle

% SECTION 1
\section{Genitive of quantification}\label{18:s1}

\ili{Russian} numerals are traditionally subcategorized into two groups depending on the number feature carried by the head of the NP they select. The first group of the so-called \textsc{lower} numerals includes numerals from 2 to 4, which consistently select a complement headed by a \isi{noun} in the singular \isi{genitive} form, as in \REF{18:ex1}. The second group of \textsc{higher} numerals includes numerals from 5 and above, which select a complement headed by a \isi{noun} in the \isi{plural} \isi{genitive} form, as in \REF{18:ex2}. 

% \begin{multicols}{2}
% example 1

\ea \label{18:ex1}
\gll dva   studenta\\
     two  student.\textsc{gen.sg}\\
\glt `two students'
\z
% \columnbreak
% example 2
\ea \label{18:ex2}
\gll vosem’  studentov\\
     eight  students.\textsc{gen.pl}\\
\glt `eight students'
\z
% \columnbreak
% example 3
\ea \label{18:ex3}
\gll gruppa  studentov\\
     group   students.\textsc{gen.pl}\\
\glt `a/the group of students'
\z
% \end{multicols}

\noindent The present paper is concerned with both types of \isi{numeral} phrases given in \REF{18:ex1} and \REF{18:ex2} but we start by looking at constructions involving higher numerals, as in \REF{18:ex2}. The traditional way of analysing \REF{18:ex2} is to say that the higher \isi{numeral} behaves like a \isi{noun} in the \isi{genitive} construction, as in \REF{18:ex3}. That is, the \isi{numeral} is the head of the NumP taking the quantified NP as its complement and assigning \isi{genitive} \isi{plural} to it, so that there is no structural difference between \REF{18:ex2} and \REF{18:ex3}; see \REF{18:ex4} and \REF{18:ex5}.\footnote{\label{18:fn1}Although \REF{18:ex4} represents the most standard approach to NumPs headed by a higher \isi{numeral}, other analyses exist. One such analysis assumes that the higher \isi{numeral} is merged in the highest position within the NP and moves to D \citep{Pesetsky2013}. As postulation of the D layer for \ili{Russian} NPs is rather controversial (see \citealt{Bošković2008, Bošković2010}), I adopt a more standard representation of NumPs that essentially assumes the same surface hierarchical structure. Whether the \isi{numeral} has moved to its surface position from within the NP or is generated in it is immaterial for the present analysis. Another analysis proposed in the literature is based on the observation that a higher \isi{numeral} can undergo left-\isi{branch extraction}, and can also receive a case from the outside when its complement receives \isi{genitive} (as in the \ili{Russian} \textit{po}{}-construction). To account for this observation, it has been proposed that the \isi{numeral} is located in the Specifier of a null head, which itself assigns \isi{genitive} (\citealt{Franks1995}, \citealt{Bailyn2004}). For the purpose of the present analysis, it is immaterial whether the \isi{numeral} is the head of the \isi{numeral} phrase that assigns \isi{genitive}, as in \REF{18:ex4}, or if it is located in the Specifier of a null head that assigns \isi{genitive}. The analysis in \REF{18:ex4} is adopted here mainly for the ease of exposition.}

% tree 4
\begin{multicols}{2}
\ea \label{18:ex4} \begin{forest}
[NumP
	[Num\\\textit{vosem'}\\`eight']
    [NP
    	[\textit{studentov}\\`students.\textsc{gen.pl}', roof first-line-width]
    ]
]
\end{forest}

\z

\columnbreak

% tree 5
\ea \label{18:ex5} \begin{forest}
[NP
	[N\\\textit{gruppa}\\`group']
    [NP
    	[\textit{studentov}\\`students.\textsc{gen.pl}', roof first-line-width]
    ]
]
\end{forest}

\z

\end{multicols}


\noindent Curiously, the parallel in the case and number features observed in NumPs headed by a higher \isi{numeral} and \isi{genitive} constructions in \REF{18:ex2} and \REF{18:ex3}, respectively, only holds for those NP complements whose head \isi{noun} has both lexical number forms – \isi{plural} and singular. Although such nouns constitute the overwhelming majority of \ili{Russian} nouns, there are exceptions. Thus, the \ili{Russian} \isi{noun} \textit{čelovek} ‘person’ only has a singular \isi{lexical form}, whereas the \isi{noun} \textit{ljudi} ‘people’ only has a \isi{plural} \isi{lexical form}.\footnote{\label{18:fn2}Due to the fact that \textit{čelovek} and \textit{ljudi} have distinct roots and are historically derived from distinct nouns, I assume that they are distinct lexical items. Importantly, an analysis that assumes that \textit{ljudi} involves contextual root allomorphy of \textit{čelovek} in the context of a higher \isi{numeral} cannot be sustained because in some contexts, either of the two nouns can surface (see footnotes \ref{18:fn13} and \ref{18:fn14}).
} 
As expected, the case-assigning \isi{noun} in the \isi{genitive} construction in \REF{18:ex6} can select an NP headed by the \isi{noun} that only has a \isi{plural} \isi{lexical form}, see \REF{18:ex6a}, but not the \isi{noun} that only has a singular \isi{lexical form}, see \REF{18:ex6b}. What is unexpected is that the NumP headed by a higher \isi{numeral} behaves in the exactly opposite way, see \REF{18:ex7}. Despite the fact that the higher \isi{numeral} typically takes a \isi{plural} NP complement, this NP cannot be headed by a \isi{noun} that lacks a singular \isi{lexical form}, see \REF{18:ex7a}. Instead, an NP headed by a \isi{noun} that lacks a \isi{plural} \isi{lexical form} is selected, see \REF{18:ex7b}. As a result, the selected NP fails to carry the \isi{genitive} \isi{plural} features, and the \isi{noun} surfaces in the form that is morphologically identical to the default \isi{nominative} singular form.\footnote{\label{18:fn3}The fact that the \isi{noun} in \REF{18:ex7b} surfaces in the form identical to the \isi{nominative} singular form is in line with the idea that \isi{nominative} is a morphological default (\citealt{Marantz1991}, \citealt{Schütze1997, Schütze2001}). Although languages may differ in the realization of default case, in \ili{Russian} it is indeed \isi{nominative}. Thus, the \ili{Russian} variant of the \ili{English} phrase \textit{Me intelligent?!} can only contain a \isi{nominative} \isi{noun}. Plausibly, the morphological form of the \isi{noun} in \REF{18:ex7b} is a historical remnant of the old declension paradigm from the time when \textit{čelovek} had both number forms, with the \isi{nominative} singular and the \isi{genitive} \isi{plural} forms coinciding. However, since in modern \ili{Russian} the \isi{plural} form is no longer available for \textit{čelovek}, the morphological form of this \isi{noun} in the context of a higher \isi{numeral} must have been reanalysed as the default \isi{nominative} singular form that surfaces due to the morphological deficiency of \textit{čelovek} (see the notation in \REF{18:ex7b}).   Additional support for this view comes from the fact that \textit{čelovek} is not the only \isi{noun} that is reanalysed in modern \ili{Russian} as \isi{nominative} due to the genitive-\isi{nominative} \isi{syncretism}. \ili{Russian} \isi{feminine} nouns whose \isi{nominative} \isi{plural} and \isi{genitive} singular forms coincide can be construed as \isi{nominative} \isi{plural} in the context of a lower \isi{numeral} thereby affecting the choice of case form of the modifying \isi{adjective}, see \REF{18:fn3i}. The \isi{genitive} singular form is also available for these nouns in modern \ili{Russian} but is less common, see \REF{18:fn3ii}.

\begin{multicols}{2}
%example in foot note 3
\ea \label{18:fn3i1}
	\ea[]{ 
    \gll dve krasivye     devočki\\
    	 two  pretty.\textsc{nom}  girls.\textsc{nom}\\
    \glt `two pretty girls'
    }\label{18:fn3i}\columnbreak
    \ex[]{
    \gll dve   krasivyx     devočki\\
    	 two  pretty.\textsc{gen}  girl.\textsc{gen}\\
    \glt `two pretty girls'
    }\label{18:fn3ii}
    \z
\z
\end{multicols}

\noindent Since both lexical number forms, singular and \isi{plural}, are available for the \isi{noun} \textit{devočka} in modern \ili{Russian}, both structures in \REF{18:fn3i1} are possible. Logically, if one of the lexical number forms disappeared, only one structure in \REF{18:fn3i1} would remain. Plausibly, this is exactly what happened to the \isi{noun} \textit{čelovek}.}

% example 6
\ea \label{18:ex6}
	\ea[]{ \label{18:ex6a}
	\gll gruppa   ljudej\\
         group    people.\textsc{gen.pl}\\
	\glt `a/the group of people'
    }
	\ex[*]{ \label{18:ex6b}
    \gll gruppa   \isi{čelovek} / čeloveka\\
         group    person.\textsc{nom.sg} {} person.\textsc{gen.sg}\\ 
    }
	\z 
\z


% example 7
\ea \label{18:ex7}
	\ea[*]{ \label{18:ex7a}
    \gll vosem’   ljudej\\
         eight    people.\textsc{gen.pl}\\ 
    }
	\ex[]{ \label{18:ex7b}
    \gll vosem’   \isi{čelovek}\\
         eight    person.\textsc{nom.sg}\\
	\glt `eight people'
    }
	\z
\z


\noindent The difference in the choice of the \isi{noun} form illustrated in \REF{18:ex6} and \REF{18:ex7} strongly suggests that the structural case assigned by a higher \isi{numeral} is not identical to the lexical case assigned by a \isi{noun} in the \isi{genitive} construction. It has been proposed in the linguistic literature that \ili{Russian} higher numerals assign the so-called \textsc{\isi{genitive} of quantification} (GQ) rather than simple \isi{genitive} \citep{Bošković2006}. If so, we can hypothesise that GQ places a specific requirement on the head of the NP, which results in the pattern observed in \REF{18:ex7}. In particular, being a quantificational case, GQ may require that the NP receiving it is headed by a \isi{noun} that has a lexically realised unit for counting, see \REF{18:ex8}. Nouns that do not have a singular \isi{lexical form} will, then, be expected to fail to head an NP that receives GQ, as such nouns lack a lexically realised unit for counting.\footnote{\label{18:fn4}The hypothesis put forward in \REF{18:ex8} is additionally supported by data involving mass nouns, as in \REF{18:fn4i} and nouns belonging to the group of pluralia tantum, as in \REF{18:fn4ii}. Both types of nouns lack a unit for counting and, hence, fail to head the NP that received GQ from the higher \isi{numeral}, see \REF{18:fn4ia} and \REF{18:fn4iia}. The only way these nouns can occur in NumPs headed by a higher \isi{numeral} is when they head an NP that receives \isi{genitive} from the \isi{noun} that has a lexical singular form and therefore can head the NP that receives GQ from the higher \isi{numeral}, as in \REF{18:fn4ib} and \REF{18:fn4iib}.

\begin{multicols}{2}
% example in foot note 4
\ea \label{18:fn4i}
	\ea[*]{ 
    \gll vosem'  čaja\\
    	 eight  tea.\textsc{gen}\\
    }\label{18:fn4ia}
    \ex[]{
    \gll vosem’  stakanov    čaja\\
    	 eight  glasses.\textsc{gen}  tea.\textsc{gen}\\
         \glt `eight glasses of tea'
    }\label{18:fn4ib}
    \z
\z

\columnbreak

\ea \label{18:fn4ii}
	\ea[*]{ 
    \gll vosem’  nožnic\\
    	 eight  scissors.\textsc{gen}\\
    }\label{18:fn4iia}
    \ex[]{
    \gll vosem’  par     nožnic\\
    	 eight  pairs.\textsc{gen}  scissors.\textsc{gen}\\
    \glt `eight pairs of scissors'
    }\label{18:fn4iib}
    \z
\z

\end{multicols}

\noindent It is of course true that in \ili{English} pluralia tantum also fail to head NP complements to numerals. However, since the present paper is on \ili{Russian}, a discussion of \ili{English} is left for future research. Another issue that has to be left for future research is that although structures like \REF{18:fn4iia} are never used in formal register and are perceived as ungrammatical by my consultants and myself, they can be found in colloquial \ili{Russian}. A possible explanation for this occurrence is that speakers that allow \REF{18:fn4iia} analyse the \isi{noun} heading the NP complement to the \isi{numeral} in \REF{18:fn4iib} as an optionally null classifier due to its invariable form (i.e., no other \isi{noun} can be used with pluralia tantum).}
\newpage 

% example 8
\ea \label{18:ex8} NPs headed by a \isi{noun} that lacks a unit for counting are unable to carry GQ.\label{18:ex:key:8}\footnote{\label{18:fn5}This rule refers to nouns that lack a lexically realised unit for counting. This includes mass nouns, collective nouns, pluralia tantum and countable nouns that lack a non-suppletive lexical singular form. Importantly, nouns like \textit{deti} ‘children’ do not fall under this category despite having a suppletive singular form \textit{rebjonok} ‘child’ in modern \ili{Russian}. This is because the non-suppletive form \textit{ditja} ‘child’ still exists in the language even though it is perceived as stylistically marked and somewhat archaic.  The \isi{noun} \textit{ljudi} ‘people’, conversely, has never had a non-suppletive lexical singular form as it was historically derived from a collective \isi{noun}, i.e., \textit{ljud} ‘people, folk’ \citep{Chumakina-etal2004} that already lacked a unit for counting.
}

    \z

\noindent If the rule given in \REF{18:ex8} is correct, \ili{Russian} higher numerals have a difficult time dealing with nouns that lack one of the lexical number forms. We have seen in \REF{18:ex2} that higher numerals require \isi{plural agreement} with their NP complement. At the same time, \REF{18:ex8} demands that the relevant NP is headed by a \isi{noun} that has a singular \isi{lexical form}. When the head \isi{noun} has both lexical number forms, both of these requirements can be obeyed, as in \REF{18:ex2}. Conversely, when the head \isi{noun} has only one of the number forms, as is the case with \textit{čelovek} and \textit{ljudi} in \REF{18:ex7}, a choice must be made as to which requirement is obeyed at the cost of violating the other, given that both of them cannot be obeyed simultaneously. The data in \REF{18:ex7} demonstrate that \ili{Russian} choses to obey \REF{18:ex8} at the cost of violating the requirement for \isi{plural agreement}. That is, the \isi{noun} in the well-formed structure in \REF{18:ex7b} has a singular form. The NP it heads can therefore receive structural GQ from the \isi{numeral}. However, this \isi{noun} lacks a \isi{plural} form. It therefore fails to realise the \isi{genitive} \isi{plural} features required for agreement with the higher \isi{numeral} and surfaces in the default \isi{nominative} singular form.

Following \citet{Bobaljik2008}, I assume that morphological case (m-case) must be distinguished from structural case, with m-case being treated as a morphological phenomenon applying at PF and structural case as syntactic NP licensing (see also \citealt{Harley1995}, \citealt{Marantz2000}, \citealt{McFadden2004}, \citealt{Schütze1997}, \citealt{Sigurðsson1991},  \citealt{Sigurðsson2003}, \citealt{Yip-etal1987}, \citealt{Zaenen-etal1985}). Assuming that the proper place of agreement, which is dependent on m-case, is the morphological component that is a part of the PF interpretation of structural descriptions \citep{Bobaljik2008}, we can argue that in \REF{18:ex7} the choice is made between the requirement for the NP complement to Num to be syntactically licensed through structural GQ, and the requirement for it to realise \isi{plural} features at PF. The data in \REF{18:ex7} suggest that syntactic well-formedness is a stronger requirement. That is, what we observe in \REF{18:ex7b} is that a well-formed syntactic representation containing a structurally licensed NP is generated, but when this representation reaches PF, the latter fails to realize the \isi{genitive} \isi{plural} features on the defective \isi{noun} (i.e., the \isi{noun} that lacks a \isi{plural} \isi{lexical form}).\footnote{\label{18:fn6}The present analysis assumes a competition of syntactic and PF constraints, with syntactic constraints winning the competition. I do not propose an Optimality Theoretical account for this competition because I do not take syntactic constraints to be violable.}

% SECTION 2
\section{The numeral-classifier construction}\label{18:s2}
\largerpage[-1]
The pattern observed in \REF{18:ex7} breaks down in constructions involving modification or \isi{topicalization}, see \REF{18:ex9} and \REF{18:ex10}, creating an apparent counterexample to \REF{18:ex8}. That is, once a modifier interferes between the \isi{numeral} and the \isi{noun}, selecting an NP headed by a \isi{noun} that lacks a singular \isi{lexical form} becomes possible, as in \REF{18:ex9b}, in an apparent violation of \REF{18:ex8}, whereas using a \isi{noun} that lacks a \isi{plural} \isi{lexical form}, as in \REF{18:ex9a}, is not acceptable to all native speakers of \ili{Russian}.\footnote{\label{18:fn7}Although \ili{Russian} prescriptive grammars state that \REF{18:ex9a} is ungrammatical, I have come across speakers that accept it. I have therefore used questionnaires in order to establish which form in \REF{18:ex9} is more acceptable to native speakers of \ili{Russian} (judged on the scale from 1 to 5, with 5 being fully grammatical and 1 fully ungrammatical). Out of forty-six native speakers questioned, four favoured \REF{18:ex9a} and forty-two favoured \REF{18:ex9b}. Out of the group of speakers that favour \REF{18:ex9a}, two speakers clarified that since the phrase in \REF{18:ex7a} is ungrammatical, it should be ungrammatical even in the presence of modification, while the other two speakers did not explain their preference. Out of the group of speakers that favour \REF{18:ex9b}, eight speakers found \REF{18:ex9a} fully ungrammatical (in line with my own judgement as a native speaker of \ili{Russian}), whereas the remaining thirty-four speakers found it marginally acceptable (none of them gave it a five or a four) but degraded with respect to \REF{18:ex9b} (two speakers have independently suggested that \REF{18:ex9a} is restricted to contexts involving contrast).} 
% example 9
\ea \label{18:ex9}
	\ea[\%]{ \label{18:ex9a}
    \gll vosem’  krasivyx  \isi{čelovek}\\           
         eight  pretty.\textsc{gen.pl} person.\textsc{nom.sg}\\ 
    }
	\ex[]{ \label{18:ex9b}
    \gll vosem’  krasivyx  ljudej\\
         eight  pretty.\textsc{gen.pl}  people.\textsc{gen.pl}\\
	\glt `eight pretty people'
    }
	\z
\z

\noindent Similarly, when the NP is topicalized, as in \REF{18:ex10c}, a \isi{noun} lacking a singular \isi{lexical form} is selected in an apparent violation of \REF{18:ex8}. A \isi{noun} lacking a \isi{plural} \isi{lexical form}, on the other hand, cannot be used in topicalized NPs (see \REF{18:ex10b}) despite being chosen in the structure prior to \isi{topicalization} (see \REF{18:ex10a}).\footnote{\label{18:fn8}\REF{18:ex10b} is marginally acceptable under the interpretation of approximate inversion (although this \isi{word order} still feels like resulting from a production error) but not under the interpretation and intonation associated with the \isi{topicalization} of the NP.}

% example 10
\ea \label{18:ex10}
	\ea[]{ \label{18:ex10a}
    \gll V  komnate  bylo    vosem’  \isi{čelovek}.\\
         in  room    was.\textsc{3sg.n}  eight  person.\textsc{nom.sg}\\
	\glt `In the room there were eight people.'
    }
	\ex[??]{ \label{18:ex10b}
    \gll Čelovek\textsubscript{1}     v   komnate  bylo     vosem’    t\textsubscript{1}.\\
         person.\textsc{nom.sg.}  in   room    was.\textsc{3sg.n}  eight\\ 
    }
	\ex[]{ \label{18:ex10c}
    \gll   Ljudej\textsubscript{1}    v   komnate  bylo     vosem’   t\textsubscript{1}.\\
           people.\textsc{gen.pl} in   room    was.\textsc{3sg.n}  eight  \\
	\glt`As for people, there were eight of them in the room.'}
	\z
\z

\noindent The data in \REF{18:ex9} and \REF{18:ex10} present a challenge for \REF{18:ex8}. In particular, if the higher \isi{numeral} assigns GQ to its NP complement and thus places the restriction in \REF{18:ex8} on it, \REF{18:ex9b} should be impossible, as it seemingly contains a syntactically unlicensed NP. Similarly, in \REF{18:ex10c} the topicalized NP is expected to reconstruct but it cannot reconstruct into the position where it receives GQ, as in \REF{18:ex10a}, because reconstruction to this position of the NP headed by a \isi{noun} that lacks a singular \isi{lexical form}, as in \REF{18:ex10c}, violates \REF{18:ex8}. A logical solution for \REF{18:ex10} would be to assume that the topicalized NP in \REF{18:ex10c} reconstructs to some other position, where it receives some case other than GQ. If so, this position might also be the position that hosts the NP in \REF{18:ex9b}. Let us use this assumption as our working hypothesis and try to establish what this position is and what case is assigned to the NPs in \REF{18:ex9b} and \REF{18:ex10c} and by what head. 

As a starting point let us look at \REF{18:ex11}. We have hypothesised in \REF{18:ex8} that a \isi{noun} lacking a unit for counting cannot head an NP that receives GQ. We have based this hypothesis on \REF{18:ex7a} but we expect it to apply to any \isi{noun} that lacks a unit for counting, including mass nouns. This prediction is indeed borne out in \REF{18:ex11}.\footnote{\label{18:fn9}The ungrammaticality of \REF{18:ex11} cannot be due to the lack of \isi{plural agreement} with the higher \isi{numeral}, as such a violation is tolerated in \REF{18:ex7b}.} 
It is nevertheless possible to express the meaning of \REF{18:ex11} with a grammatical sentence as long as the NP headed by a mass \isi{noun} receives \isi{genitive} or \isi{partitive} case from the head of the NP that receives GQ from the \isi{numeral}, as in \REF{18:ex12}.

% example 11
\ea[*]{ \label{18:ex11}
	\gll Na   stole  stojalo  vosem’  čaja / čaju.\\
         on table stood.\textsc{3sg}  eight  tea.\textsc{gen} {} tea.\textsc{part}\\ 
    }
\z

% example 12
\ea[]{ \label{18:ex12}
	\gll Na stole stojalo vosem’ stakanov  čaja / čaju.\\
         on table stood.\textsc{3sg} eight glasses.\textsc{gen.pl} tea.\textsc{gen} {} tea.\textsc{part}\\
	\glt `There were eight glasses of tea on the table.'
    }
\z

\noindent The assignment of GQ is possible in \REF{18:ex12} because the NP that receives it is headed by a countable \isi{noun} that has both lexical number forms. The availability of a singular \isi{lexical form} ensures that there is no violation of \REF{18:ex8}, while the availability of a \isi{plural} \isi{lexical form} allows for the realisation of the \isi{genitive} \isi{plural} features; see \REF{18:ex13}.

% tree 13
\ea \label{18:ex13} \begin{forest}
[NumP
	[Num\\\textit{vosem'}\\`eight']
    [NP$_1$
    	[N\\\textit{stakanov}\\`glasses.\textsc{gen.pl}']
        [NP$_2$
        	[\textit{čaja/čaju}\\`tea.\textsc{gen/part}', roof first-line-width]
        ] { \draw (.east) node[right]{\hspace{-2mm}\textsc{(gen/part)}}; }
    ] { \draw (.east) node[right]{\hspace{-2mm}\textsc{(gq)}}; }
]
\end{forest}

\z


\noindent In \REF{18:ex13}, the mass \isi{noun} that cannot head NP$_1$, which receives GQ from the \isi{numeral}, can nevertheless head NP$_2$, which is contained in the NumP and c-comman\-ded by the \isi{numeral}.  The crucial hypothesis that I would like to put forward is that the same strategy is used in \REF{18:ex9b} and \REF{18:ex10c}, as shown in \REF{18:ex14}. 

% tree 14
\ea \label{18:ex14} \begin{forest}
[NumP, s sep=1.3cm
	[Num\\\textit{vosem'}\\`eight']
    [NP$_1$
    	[N\\?]
        [NP$_2$
        	[\textit{(krasivyx) ljudej}\\`pretty.\textsc{gen.pl} people.\textsc{gen.pl}', roof first-line-width]
        ] { \draw (.east) node[right]{\hspace{-2mm}\textsc{(gen.pl)}}; }
    ] { \draw (.east) node[right]{\hspace{-2mm}\textsc{(gq)}}; }
]
\end{forest}

\z


\noindent In \REF{18:ex14} the NP\textsubscript{2} headed by the \isi{noun} that lacks a singular \isi{lexical form} receives \isi{genitive} \isi{plural} from a phonologically null \textsc{quantifying expression} (QE) that heads NP\textsubscript{1} carrying GQ.\footnote{\label{18:fn10}The idea that \isi{numeral} phrases may contain phonologically null nouns has also been proposed in \citet{Kayne2005}.}
The questions that will be addressed in this section are the following. What is the nature of the QE in \REF{18:ex14}? Can it be overt? What licenses its covert status?  

I would like to propose that the head of NP\textsubscript{1} in \REF{18:ex14} is the lexical variant of the \isi{noun} ‘person/people’ that only has a singular \isi{lexical form}, as in \REF{18:ex15}. (In the following examples, \textsc{small caps} mark the focus of the sentence.)
% example 15
\ea \label{18:ex15}
	\gll [\hspace{-2pt} Krasivyx  ludej]$_1$ v  komnate  bylo   \textsc{vosem’} {(\isi{čelovek})  \hspace{1cm}t$_1$}.\\
 	{} pretty.\textsc{gen.pl} people.\textsc{gen.pl} in room was.\textsc{3sg.n} eight  \hspace{2pt}person.\textsc{nom.sg}\\
	\glt `As for pretty people, there were eight of them in the room.'
\z

\noindent The structure for \REF{18:ex15} is given in \REF{18:ex16}. This construction has been referred to in the linguistic literature as the \textsc{numeral-classifier construction} (NCC) (see \citealt{Sussex1976}, \citealt{Yadroff1999} and \citealt{Pesetsky2013}). It is forced in structures with approximate inversion involving modification of the type \textit{\isi{čelovek} pjat’ krasivyx ljudej} `approximately five pretty people'.\footnote{\label{18:fn11}In the absence of modification, inversion can take place in a structure that does not contain the QE \textit{\isi{čelovek};} see \REF{18:fn11i}. However, if the \isi{noun} is modified, any type of movement to pre-numeric position -- be it just the \isi{noun} inverted, as in \REF{18:fn11iib}, just the \isi{adjective} inverted, as in \REF{18:fn11iic}, or both words inverted, as in \REF{18:fn11iid} and \REF{18:fn11iie} -- is ungrammatical. In this case, the structure in \REF{18:ex16} with the inverted pleonastic \isi{noun} \textit{čelovek} must be used, as in \REF{18:fn11iii} (see also \citealt{Melčuk1985} and \citealt{Yadroff1999}).% example in foot note 11
\begin{multicols}{2}
\ea \label{18:fn11i}
	\ea[]{ 
    \gll pjat’   muzykantov\\
    	 five  musicians.\textsc{gen.pl}\\
    \glt `five musicians'
    }\label{18:fn11ia}\columnbreak
    \ex[]{
    \gll muzykantov   pjat’\\
    musicians.\textsc{gen.pl} five\\
    \glt `approximately five musicians'
    }\label{18:fn11ib}
    \z
\z
\end{multicols}

\begin{multicols}{2}
\ea \label{18:fn11ii}
	\ea[]{ 
    \gll pjat’   talantlivyx  musykantov\\
    	 five  talented    musicians.\textsc{gen}\\
         \glt `five talented musicians'
    \label{18:fn11iia}}\columnbreak
    \ex[*]{
    muzykantov   pjat’   talantlivyx\\
    }\label{18:fn11iib}
    \ex[*]{
    talantlivyx  pjat’ muzykantov\\
    }\label{18:fn11iic}
    \ex[*]{
    talantlivyx   muzykantov   pjat’\\
    }\label{18:fn11iid}
    \ex[*]{
    muzykantov   talantlivyx  pjat’\\
    }\label{18:fn11iie}
    \z
\z
\end{multicols}
\ea \label{18:fn11iii}
    \gll \isi{čelovek}    pjat’  talantlivyx  muzykantov\\
    	 person.\textsc{nom.sg}   five  talented    musicians.\textsc{gen.pl}\\
         \glt `approximately five talented musicians'
\z 
}
Following \citet{Yadroff1999}, I assume that the QE in constructions of the type given in \REF{18:ex16} is not a normal \isi{noun} but a classifier and assign it to a category that \citeauthor{Yadroff1999} calls Measure. As can be seen from \REF{18:ex15} and \REF{18:ex16}, the QE that heads the MeasureP can be overt. The option of being covert, on the other hand, is plausibly licensed by the limited semantic function and the \isi{semantic recoverability} of the QE. To be precise, the QE in \REF{18:ex16} has no other semantic function but to pick out a certain number of individuals from the set represented by its NP complement.\footnote{\label{18:fn12}If the QE is allowed to be covert due to its limited semantic function, we expect that when it performs an additional semantic function, it must be overt. This is indeed the case in structures involving approximate inversion, where the QE cannot be covert; see \REF{18:fn12i}.

\ea \label{18:fn12i}
	\ea 
    \gll \isi{čelovek} pjat’ krasivyx ljudej\\
    	 person.\textsc{nom.sg}  five  pretty  people.\textsc{gen.pl}\\
         \glt `approximately five pretty people'
    \label{18:fn12ia}
    \ex
    \gll pjat’ krasivyx ljudej\\ % soll weiter rechts stehen
    	 five  pretty  people.\textsc{gen.}\textsc{pl}\\
         \glt `five pretty people' (not: `approximately five pretty people')
    \label{18:fn12ib}
    \z
\z

}

% tree 16

\ea \label{18:ex16} \begin{forest}
[NumP
	[Num\\\textit{vosem'}\\`eight']
    [MeasureP
    	[Measure\\\textit{(\isi{čelovek})}\\`person.\textsc{nom.sg}']
        [NP
        	[\textit{krasivyx ljudej}\\`pretty.\textsc{gen.pl} people.\textsc{gen.pl}', roof first-line-width]
        ] { \draw (.east) node[right]{\hspace{-2mm}\textsc{(gen.pl)}}; }
    ] { \draw (.east) node[right]{\hspace{-2mm}\textsc{(gq)}}; }
]
\end{forest}

\z


\noindent The set denoted by the NP is a subset to the set denoted by the QE. In other words, the set denoted by the NP interpretively restricts the set denoted by the QE. Consequently, the QE consistently represents the superset to the set represented by its NP complement. Plausibly, the default superset \isi{construal} is one of the factors contributing to the \isi{semantic recoverability} of the QE. However, as we will see in \sectref{18:s4}, this is not a sufficient factor and additional restrictions on \isi{semantic recoverability} apply.

\largerpage
If we are right in assuming that the interpretation of the superset to the set represented by the NP is a crucial factor for the \isi{semantic recoverability} of the QE, we expect that when \textit{čelovek} does not take an NP complement, it must be overt and the set it represents is unrestricted. This is indeed the case in \REF{18:ex7b}, where \textit{čelovek} takes no NP complement. It therefore refers to an open set of people and is obligatorily overt.

The analysis in \REF{18:ex16} entails that \ili{Russian} higher numerals consistently assign GQ to their complements and that \REF{18:ex8} always holds, whereas the NP headed by the \isi{noun} \textit{ljudi} never ends up in the position receiving GQ. Instead, this NP is consistently selected by an optionally null QE that assigns \isi{genitive} \isi{plural} to it. This assumption captures the problematic data in \REF{18:ex9} and \REF{18:ex10}. Yet, the reader might wonder why the structure in \REF{18:ex16} is not used in \REF{18:ex7a}, which should make it well formed. I would like to argue that the structure in \REF{18:ex16} is indeed available for \REF{18:ex7a} but employment of this structure results in semantic oddness. Indeed, the structure in \REF{18:ex7a} is as semantically odd as the one in \REF{18:ex17}, where the QE is overt, because in both examples the open set of people represented by the QE (covert or overt) is not restricted by a more specific subset of people denoted by its NP complement. The NP is interpreted as referring to an open set of people but an open set of people is already denoted by the QE. We have argued that the QE can only take an NP complement that restricts its set. That is, given that in \REF{18:ex16} the QE denotes an open set of people, the NP must refer to a set of people with some specific features or qualities, such as ‘pretty people’ in \REF{18:ex9b}. Whenever the QE takes an NP complement that represents exactly the same open set, this results in redundancy and subsequent semantic oddness; see \REF{18:ex7a} and \REF{18:ex17}.\footnote{\label{18:fn13}As expected, \REF{18:ex7a} improves when the set represented by the QE is semantically restricted, as in \REF{18:fn13i}. The acceptability of \REF{18:fn13i} strongly suggests that \REF{18:ex7} cannot be accounted for by assuming a morpho-phonological constraint that bans linear adjacency between \textit{vosem’} and \textit{ljudej.} Furthermore, linear adjacency is possible in a coordinate structure with the interpretation ‘a group of (approximately) 8 individuals some of which are men and some hobbits’; see \REF{18:fn13iia}. As can be seen from \REF{18:fn13ii}, when the QE selects a coordinate NP that represents two sets -- a set of people and a set of hobbits, no semantic oddness obtains because the set denoted by the QE is restricted by a more specific subset of hobbits.
\ea[?]{ \label{18:fn13i}
	\gll vosem’ ljudej s krasivymi licami\\
    	 eight  people.\textsc{gen.pl} with  pretty  faces\\
    \glt `eight people with pretty faces'
    }
\z

\ea \label{18:fn13ii}
	\ea[]{
    \gll (\hspace{-2pt} \isi{čelovek}) vosem’ ljudej i hobbitov\\           
         {} person.\textsc{nom.sg} eight people.\textsc{gen.pl} and  hobbits.\textsc{gen.pl}\\
    \glt `(approximately) eight people and hobbits'
    }\label{18:fn13iia}
	\ex[]{
    \gll (\hspace{-2pt} \isi{čelovek}) vosem’ hobbitov  i ljudej\\
         {} person.\textsc{nom.sg} eight hobbits.\textsc{gen.pl} and people.\textsc{gen.pl}\\
	\glt `(approximately) eight hobbits and people'
    }\label{18:fn13iib}
	\z
\z
}
% example 17
\ea[*]{ \label{18:ex17}
	\gll vosem’  \isi{čelovek}      ljudej\\
    eight  person.\textsc{nom.sg}  people.\textsc{gen.pl}\\
}
\z

\noindent Crucially, whenever the NP that is complement to the QE is topicalized, as in \REF{18:ex10c}, semantic oddness disappears, strongly suggesting that the topicalized NP refers to a more specific set than the one denoted by the QE. In the next section, we discuss the nature of this set and discover why the structure in \REF{18:ex16} is obligatory for \REF{18:ex10c}.

%SECTION 3
\section{The plurality requirement}\label{18:s3}

We have argued that modification makes it possible for higher numerals to take MeasureP complements headed by an optionally null classifier that in turn takes an NP complement that can be headed by the \isi{noun} \textit{ljudi;} see \REF{18:ex9b} and \REF{18:ex16}. We have maintained that this option is determined by the \isi{semantics} of the NP. In particular, the NP must restrict the set denoted by the classifier. In the absence of such a restriction, the NCC cannot be formed (see \REF{18:ex7a} and \REF{18:ex17}), whereas modification makes such a restriction possible. At the same time, we have seen that the structure with \textit{čelovek} in \REF{18:ex9a} is acceptable to some speakers but not others (see footnote \ref{18:fn7}). Let us consider the grammar of both types of speakers. Plausibly, speakers who (like myself) find \REF{18:ex9a} ill formed interpret the \isi{noun} \textit{čelovek} in \REF{18:ex9a} as a classifier due to its impoverished morphological form. This is because nouns that have both lexical number forms surface in the \isi{nominative} singular form when used as classifiers (see \REF{18:ex18a}) but in the \isi{plural} \isi{genitive} form required for agreement with the higher \isi{numeral} when used as heads of NPs (see \REF{18:ex18b}). When the \isi{noun} is \isi{nominative} singular and hence construed as a classifier, modification is impossible (see \REF{18:ex18c}) in line with the observation that classifiers generally resist modification. By hypothesis, speakers of my variety transfer the classifier analysis to any \isi{noun} that surfaces in the \isi{nominative} singular form in the context of a higher \isi{numeral} and analyse \REF{18:ex9a} in parallel with \REF{18:ex18c}.
% eample 18
\ea \label{18:ex18}
	\ea[]{ \label{18:ex18a}
    \gll vosem’  kilogramm     jablok\\
         eight   kilogram.\textsc{nom.sg}  apples.\textsc{gen.pl}\\
	\glt `eight kilograms of apples'
    }
	\ex[]{ \label{18:ex18b}
    \gll vosem’ polnovesnyx     kilogrammov\\
         eight  full-weight.\textsc{gen.pl}  kilograms.\textsc{gen.pl}\\ 
    \glt `eight full-weight kilograms of apples'
    }
	\ex[*]{ \label{18:ex18c}
    \gll   vosem’ polnovesnyx     kilogramm\\
           eight  full-weight.\textsc{gen.pl}    kilogram.\textsc{nom.sg}\\
	}
	\z
\z
\largerpage[-2]
\noindent The fact that \isi{nominative} singular classifiers generally resist modification is plausibly due to a ${\phi}$-feature conflict that results from the \isi{adjective} realising the case and number features required for agreement with the higher \isi{numeral} and the classifier being unable to realise them, as in \REF{18:ex19}.\footnote{\label{18:fn14}The ungrammaticality of \REF{18:ex9a} cannot be due to modification as such, as modifiers that do not enter into an agreement relation with \textit{čelovek} can surface in this type of construction; see \REF{18:fn14i} below.

\ea \label{18:fn14i}
\gll vosem’  \isi{čelovek}    s   krasivymi   licami\\
eight  person.\textsc{nom.sg}  with  pretty    faces\\
\glt `eight people with pretty faces'
\zlast 
}
% tree19
\ea \label{18:ex19} \begin{forest}
[NumP
	[Num\\\textit{vosem'}\\`eight']
    [NP
    	[AdjP
        	[\textit{krasivyx}\\`pretty.\textsc{gen.pl}', roof first-line-width]
        ]
        [N\\\textit{čelovek}\\`person.\textsc{nom.sg}'] { \draw (.east) node[right]{\hspace{10mm}($\phi$-feature conflict)}; }
    ] { \draw (.east) node[right]{\hspace{-2mm}\textsc{(gq)}}; }
]
\end{forest}

\z

\noindent Since the \isi{adjective} in \REF{18:ex9a} and \REF{18:ex19} is part of the NP that enters into an agreement relation with the \isi{numeral}, it must realise the \isi{genitive} \isi{plural} features. Incidentally, no other morphological form of the \isi{adjective} but \isi{genitive} \isi{plural} can surface in NPs receiving GQ from a higher \isi{numeral}.\footnote{\label{18:fn15} Unlike Serbo-\ili{Croatian}, \ili{Russian} does not have uninflected ‘indeclinable’ modifiers.} The classifier, conversely, surfaces in what appears to be the default \isi{nominative} singular form. This, in turn, generates a conflict within the NP resulting from a mismatch in the case and number features between the head and the modifier; see \REF{18:ex19}.\footnote{\label{18:fn16} It appears that the crucial violation here is the \isi{case feature} mismatch, as a number feature mismatch is tolerated in \ili{Russian} NPs that are complements to lower numerals. \citet{Pesetsky2013} accounts for the number feature mismatch found in contexts of paucals by assuming that the \isi{adjective} merges with N or a projection of N and agrees with the closest number-bearing element, which is the [–singular] paucal. The \isi{noun}, on the other hand, enters syntax bearing no number feature (NBR) and immediately merges with the paucal, which is a free-standing instance of NBR rather than a \isi{numeral}. As a result, the \isi{adjective} is [$-$singular], whereas the \isi{noun} is not specified for the [$-$singular] feature.} Plausibly, it is this mismatch that results in the ill-formedness of \REF{18:ex9a} for speakers of my variety. Naturally, a structure with a \isi{plural} \isi{noun}, as in \REF{18:ex9b}, does not suffer from a ${\phi}${}-feature conflict. However, the NP in \REF{18:ex9b} cannot carry GQ as it is headed by a \isi{noun} that lacks a singular \isi{lexical form}; see \REF{18:ex8}. Hence, the NCC in \REF{18:ex16} must be formed for \REF{18:ex9b}. To rephrase, \REF{18:ex16} is licensed by the plurality requirement placed on the \isi{noun} by the \isi{adjective} in my variety of \ili{Russian}.\footnote{\label{18:fn17}In the absence of a plurality requirement, the formation of the NNC is possible only when the QE is overt, as in \REF{18:fn17i} and \REF{18:fn17ii}.   

\ea \label{18:fn17i}
\gll \isi{čelovek}      vosem’  talantlivyx  muzykantov \\  
	 person.\textsc{nom.sg} eight  talented.\textsc{gen.pl}  musicians.\textsc{gen.pl} \\
\glt `approximately eight talented musicians'  
\z
\ea\label{18:fn17ii}
\gll V orkestre rabotajet pjat’ \isi{čelovek} skripačej, i šest’ \isi{čelovek} duxovikov. \\
	 in orchestra work.\textsc{3sg} five person.\textsc{nom.sg} violinists.\textsc{gen.pl} and six person.\textsc{nom.sg} wind-players.\textsc{gen.pl} \\
\glt `In the orchestra work five violinists and six wind-players.'
\zlast
}

Conversely, speakers that accept \REF{18:ex9a} must be insensitive to the aforementioned ${\phi}${}-feature conflict. This might be because, even in the absence of modification, such NumPs involve a ${\phi}${}-feature violation that is tolerated, i.e., the \isi{noun} in \REF{18:ex7b} does not realise the \isi{genitive} \isi{plural} features required for the agreement with the higher \isi{numeral}. By hypothesis, insensitivity to the ${\phi}${}-feature conflict between the \isi{adjective} and the \isi{noun} allows these speakers to interpret \textit{čelovek} as a full \isi{noun} rather than a classifier despite its impoverished morphological form. If so, the structure in \REF{18:ex4} is generated in the grammar of these speakers for the numeration in \REF{18:ex9a}, while the NCC in \REF{18:ex16} is generated whenever the simpler structure in \REF{18:ex4} is unavailable, as in \REF{18:ex9b}. We would, then, expect to find speakers that favour \REF{18:ex9a} over \REF{18:ex9b} due to its simplicity along with speakers that accept both structures to a certain degree but assign distinct contextual interpretations to them. This prediction appears to be borne out (see footnote \ref{18:fn7}).

\largerpage[2]
Since for speakers of my variety, \REF{18:ex9a} is ill formed due to a plurality requirement placed on the \isi{noun}, which in turn triggers the structure in \REF{18:ex16}, it is not completely outlandish to assume that \REF{18:ex10b} is ill formed for a similar reason. Namely, a plurality requirement is placed on the topic NP, which rules out the structure with a \isi{noun} that lacks a \isi{plural} \isi{lexical form}. I would like to propose that the relevant plurality requirement follows from the interpretive properties of NumPs that contain a trace of a topic NP. Let us consider these properties. The sentence in \REF{18:ex10c} has a typical Top/Foc structure, with the topic NP construed as a \textsc{\isi{contrastive} topic} (CT) and the \isi{numeral} constituting the \textsc{narrow focus} of the sentence. Thus, \REF{18:ex10c} most naturally occurs in a context that asks about the quantity of individuals present in the room and therefore licenses narrow focus on the \isi{numeral}, as in \REF{18:ex20}. It is, however, incompatible with a context that licenses focus on the entire NumP, as in \REF{18:ex21}. (Sentences marked with ‘\#’ are grammatical but incompatible with the given context.)

% example 20 
\begin{exe} 
\ex \label{18:ex20}
\begin{xlist}
\exi{Q:}[]{\gll Skol’ko ljudej bylo v komnate?\\
how.many people.\textsc{gen} was.\textsc{3sg.n} in room.\textsc{prep}\\
	\glt `How many people were there in the room?'
    }
\exi{A:}[]{ 
	\gll Ljudej\textsubscript{1}    v   komnate  bylo     \textsc{vosem’}   t\textsubscript{1}.\\
		 people.\textsc{gen.pl} in   room    was.\textsc{3sg.n}  eight  \\
	\glt `As for people, there were eight of them in the room.'
    }
\end{xlist}
\end{exe}

% eample 21
\begin{exe}
\ex \label{18:ex21}
\begin{xlist}
\exi{Q:}[]{
	\gll 
    	 Kto byl v komnate?\\
         who was.\textsc{3sg.m} in room.\textsc{prep}\\
	\glt `Who was in the room?'
    }
\exi{A:}[\#]{
	\gll \textsc{Ljudej\textsubscript{1}}  v   komnate  bylo     \textsc{vosem’}   t\textsubscript{1}.\\
		 people.\textsc{gen.pl} in   room    was.\textsc{3sg.n}  eight  \\
    	\glt `As for people, there were eight of them in the room.'
        }
\end{xlist}
\end{exe}

\noindent The question (\ref{18:ex20}Q) can be answered by a simpler sentence that does not contain a CT; see \REF{18:ex22}.

% eample 22
\begin{exe}
\ex \label{18:ex22}
\begin{xlist}
\exi{Q:}[]{
	\gll 
    	 Skol’ko ljudej bylo v komnate?\\
         how.many people.\textsc{gen} was.\textsc{3sg.n} in room.\textsc{prep}\\
	\glt `How many people were there in the room?'
    }
\exi{A:}[]{
	\gll V  komnate  bylo    \textsc{vosem’}    \isi{čelovek}.\\
		 in  room    was.\textsc{3sg.n}  eight     person.\textsc{nom.sg}\\
    \glt `In the room there were five people.'     
    }
\end{xlist}
\end{exe}


\largerpage[2]
\noindent However, the replies (\ref{18:ex20}A) and (\ref{18:ex22}A) are not only structurally different, their interpretation is also distinct: while (\ref{18:ex22}A) merely answers the question about the quantity of people in the room, (\ref{18:ex20}A) additionally conveys that people were not the only individuals present in the room that are relevant for the discussion at hand but they were the only individuals for whom the quantity (i.e., the focus value) is known. Since for other individuals in the room the quantity is unknown, the sentence is perceived as providing incomplete information. The interpretation of incompleteness is what characterizes the information-structural (IS) category of CT \citep{Büring2003}, strongly suggesting that the topic NP in (\ref{18:ex20}A) and \REF{18:ex10c} is a CT.\footnote{\label{18:fn18}To be interpreted as a CT, the relevant NP must linearly precede the focus in \ili{Russian} \citep{Titov2013}.} This conclusion is further supported by the observation that \REF{18:ex10c} has the prosodic pattern typical of CT/Foc sentences, with the rising topic contour IK3 on the topicalized NP and the falling contour IK1 on the focused \isi{numeral} (\citealt{Bryzgunova1971, Bryzgunova1981}, \citealt{Titov2013}).

The set introduced by the CT in (\ref{18:ex20}A) and \REF{18:ex10c} is a subset of a set of individuals that were present in the room. That is, even when the CT refers back to an identical discourse-antecedent, as in \REF{18:ex20}, the sentence itself activates the superset \isi{construal}, as it conveys that just a subset of the set of individuals in the room that are relevant for the discourse at hand are people. This means that the superset for the set of people becomes salient at the point the sentence is uttered.

The above observation provides an answer to the question we posed in the previous section. Recall that while the sentences in \REF{18:ex7a} and \REF{18:ex17} are semantically odd because the QE in these examples takes an NP complement that represents exactly the same open set, the sentence in \REF{18:ex10c} does not suffer from semantic oddness. We have suggested that this is because the topicalized NP refers not to an open set of people but to some other set that restricts the set introduced by the QE. Indeed, \isi{contrastive} \isi{construal} of the topicalized NP in \REF{18:ex10c} results in the interpretation according to which this NP belongs to a contextually closed set of individuals that were present in the room, for some of whom the quantity is unknown. In other words, the CT in \REF{18:ex10c} does not represent an open set of people but a subset of individuals that were present in the room. Plausibly, this contextual restriction of the set to which the NP belongs eliminates redundancy and semantic oddness that we observe in \REF{18:ex7a} and \REF{18:ex17}.

% \largerpage[2]
Another crucial observation as regards the interpretive properties of \REF{18:ex10c} is that the NumP here is obligatorily non-referential. This is because the verb here is in the default third person singular form. The availability of default agreement is due to NumPs in \ili{Russian} being construed by syntax either as NPs or QPs \citep{Pesetsky1982}. In the former case, the verb agrees with the \isi{nominative} NP, as in \REF{18:ex23a}, and the NP allows for definite/specific reading, while in the latter case, agreement cannot take place and the QP is interpreted as a non-specific indefinite (see \REF{18:ex23b}) \citep{Titov2012}.\footnote{\label{18:fn19}The fact that NumPs in sentences with default agreement cannot be referential is further supported by the observation that they cannot take an apparent wide scope typical of specific indefinites; compare \REF{18:fn19i} and \REF{18:fn19ii} below. While the sentence in \REF{18:fn19ii} allows for the reading where two specific students failed all of the exams, \REF{18:fn19i} can only mean that for each exam there were two students that failed it.   

\ea \label{18:fn19i}
\gll Govorjat, čto každyj examen provalilo dva studenta. \\
	 they-say  that  every  exam.\textsc{acc}  failed.\textsc{3sg}  two   students \\\hfill [$\forall > \exists$; *$\exists > \forall$]
\glt `They say that every exam was failed by two students.'  
\z
\ea \label{18:fn19ii}
\gll Govorjat, čto každyj examen provalili dva studenta. \\ 
	 they-say  that  every  exam.\textsc{acc}  failed.\textsc{3pl}  two   students  \\\hfill [$\forall > \exists$; ?$\exists > \forall$]
\glt `They say that every exam was failed by two students.'
\z
}  
% example 23
\ea \label{18:ex23}
	\ea[]{ \label{18:ex23a}
    \gll V komnatu   vošli       (\hspace{-2pt} èti)   vosem’   \isi{čelovek}.\\           
         in room  entered.\textsc{3pl}   {} these  eight    person.\textsc{nom.sg} \\
    \glt `(These) eight people entered the room.'
    }
	\ex[]{ \label{18:ex23b}
    \gll V komnatu   vošlo       (*\hspace{-2pt} èti/ètix)        vosem’   \isi{čelovek}.\\
         in room  entered.\textsc{3sg}     {} these.\textsc{nom/gen} eight  person.\textsc{nom.sg}  \\
    }
	\z
\z           

\noindent We have seen that the sentence in \REF{18:ex10c} has narrow focus on the \isi{numeral}. Plausibly, this IS partitioning forces syntax to interpret the NumP as a QP rather than an NP, as the sentence in \REF{18:ex10c} cannot contain an agreeing verb (see \REF{18:ex24}), resulting in the obligatorily non-specific indefinite \isi{construal} of the NumP (see \REF{18:ex25}).
% example 24

\ea[*]{
	\gll  Ljudej\textsubscript{1}    v   komnate  byli     vosem’   t\textsubscript{1}.\\  		 
		 people.\textsc{gen.pl}  in   room    was.\textsc{3pl}  eight  \\ }\label{18:ex24}
\z
% example 25
\ea[]{\gll Ljudej\textsubscript{1} v komnate  bylo      (*\hspace{-2pt} èti/ètix)     vosem’   t\textsubscript{1}.\\
     people.\textsc{gen.pl}  in   room    was.\textsc{3sg.n}     {} these.\textsc{nom/gen}  eight\\
\glt `As for people, there were (*these) eight of them in the room.'} \label{18:ex25}
\z

\noindent Due to the non-specific \isi{construal}, the NumP in \REF{18:ex10c} cannot refer to a specific set of eight people. Instead the focused \isi{numeral} selects a subset of eight people from the set introduced by the CT (i.e., the NP), strongly suggesting that we are dealing with the so-called set \isi{partitive} interpretation of the NumP.\footnote{\label{18:fn20}Numerals cannot occur in entity partitives.}
Given that only NPs that can denote sets of entities are allowed in set partitives, such NPs must contain \isi{plural} nouns (\citealt{deHoop1997}). Hence, it is the set \isi{partitive} \isi{construal} of the NumP that places a plurality requirement on the topic NP in \REF{18:ex10c}, rendering \REF{18:ex10b} ungrammatical.\footnote{\label{18:fn21}Following \citet{Barker1998}, I assume that partitives are \textsc{anti-unique}. Due to anti-uniqueness, partitives are inherently non-specific indefinites, resulting in DP \isi{partitive} constructions being unable to be headed by a definite determiner. The data in \REF{18:ex25} can be seen as supporting this idea.}

%\begin{stylepi}

It has been suggested that the quantifier in \isi{partitive} constructions is followed by an empty \isi{noun} (\citealt{Milner1978}, \citealt{Bonet1986}, \citealt{Abney1987}, \citealt{Hernanz-Brucart1987}, \citealt{Delsing1988, Delsing1993}, \citealt{Ramos1992}, \citealt{Cardinaletti-Giusti1992, Cardinaletti-Giusti2006}, \citealt{Sleeman1996}, \citealt{Doetjes1997}, \citealt{Barker1998}, \citealt{Brucart-Rigau2002}, \citealt{Ionin-etal2006}). This assumption is motivated by the observation that a \isi{partitive} construction of the type given in \REF{18:ex10c} denotes two sets (here it is a general set of people and a set of eight people present in the room). The \ili{Catalan} example in \REF{18:ex26a}, where \textit{e} is lexically identical to \textit{homes} ‘men’, illustrates this idea. 
% \end{stylepi}
% example 26
\ea \ili{Catalan} \citep[27]{MartiiGirbau2010} \label{18:ex26}
	\ea \label{18:ex26a}
		\gll tres  \textbf{e}  d’aquells  homes     d’allá\\
			 three {} of-those men over-there\\
		\glt `three of those men over there'
	\ex \label{18:ex26b}
		\gll tres homes d’aquells homes d’allá\\
			 three men of-those men over-there\\
    \ex \label{18:ex26c}
		\gll tres~   homes   d’aquells   \textbf{e}   d’allá~ ~\\
			 three men of-those { } over-there\\
	\z
\z

\noindent In \REF{18:ex26a}, the \isi{partitive} construction refers to two sets of men: the set of those men and the set of three men, the latter being a subset of the former. The NumP in \REF{18:ex26b} has an overt \isi{noun} inserted between the quantifier and the PP and is grammatical, albeit odd and redundant to a native speaker. The NumP in \REF{18:ex26c} has an empty \isi{noun} holding the final \isi{noun} position. Overall, this is taken as evidence that an empty \isi{noun} category should be posited to license a \isi{partitive} meaning. In line with this observation, the present analysis assumes the structure in \REF{18:ex16} for the \isi{partitive} NumP in \REF{18:ex10c}, where an optionally null classifier occurs between the \isi{numeral} and the \isi{genitive} NP. 

In this section, we have argued that a plurality requirement placed on a \isi{noun} forces the structure in \REF{18:ex16} whenever this \isi{noun} lacks a singular \isi{lexical form} and can therefore not head an NP that receives GQ from a higher \isi{numeral}; see \REF{18:ex8}. Economy considerations predict that the more complex NCC is generated for NumPs that do not have an overt QE if and only if a plurality requirement forces \isi{plural} features on the \isi{noun} but the NP this \isi{noun} heads fails to be generated in the complement to the \isi{numeral} position, for instance because of \REF{18:ex8}. In this case, and this case alone, the simpler structure in \REF{18:ex4} is not available for the given numeration. In all other cases, \REF{18:ex4} is chosen by the grammar as the more economical structure.

% section 4
\section{\citet{Franks-House1982}}\label{18:s4}
Further evidence for the NCC analysis comes from the data discussed in \citet{Franks-House1982} that involve \isi{topicalization} of an NP in the \isi{genitive} \isi{plural} form that appears to take place from a position to which \isi{genitive} singular is assigned; see \REF{18:ex27}.
% example 27
\ea \label{18:ex27}
\gll Romanov$_1$  na  stole  bylo  dva         t\textsubscript{1}.\\
     novels.\textsc{gen.pl}  on  table  was.\textsc{3sg.n}  two\\\hfill\citep[157]{Franks-House1982}
\glt `As for novels, there were two of them on the table.'\\
\z
% example 28
\ea \label{18:ex28}
	\ea[]{ \label{18:ex28a}
    \gll Na   stole  bylo  dva  romana.\\           
         on  table  was.\textsc{3sg.n}  two  novel.\textsc{gen.sg}\\
    \glt `On the table there were two novels.'
    }
	\ex[*]{ \label{18:ex28b}
    \gll Na   stole  bylo  dva  romanov.\\
         on  table  was.\textsc{3sg.n}  two  novels.\textsc{gen.pl}\\
    }
	\z
\z 

\noindent The head of the \isi{numeral} phrase in \REF{18:ex27} is a lower \isi{numeral} that consistently takes a \isi{genitive} singular NP complement, as in \REF{18:ex28a}. A \isi{genitive} \isi{plural} NP cannot be licensed in the complement to lower \isi{numeral} position; see \REF{18:ex28b}. Yet, while the topicalized NP \textit{romanov} ‘novels’ in \REF{18:ex27} carries a \isi{genitive} case marker, it is, surprisingly, in a \isi{plural} form. \citeauthor{Franks-House1982} maintain that the topic NP cannot have been extracted from the argument \textit{dva} ‘two’ because the latter assigns the \isi{genitive} singular, not the \isi{genitive} \isi{plural}. Hence, they propose that the \isi{genitive} NP is an external topic that forms a constituent with a covert quantifier, which accounts for the \isi{genitive} case marking. The overt quantifier raises at LF, licensing the null quantifier of the \isi{genitive} constituent. However, as \citeauthor{Franks-House1982} point out, the \isi{genitive} topic in \REF{18:ex27} is different from other attested external topics in \ili{Russian} (i.e., \isi{nominative} topics) in that the former is not obligatorily followed by a pause. Moreover, the \isi{genitive} topic in \REF{18:ex27} requires a \isi{numeral} in the \isi{clause} that refers back to the \isi{genitive} NP. This, of course, cannot be said about other external topics. And finally, \citeauthor{Franks-House1982}’s analysis of the number agreement mismatch in \REF{18:ex27} cannot be applied to the cases of number agreement inconsistencies discussed above that do not involve \isi{topicalization}. 

Hence, the NCC analysis appears to be better suited for \REF{18:ex27}. On this account, the sentence in \REF{18:ex27} contains an optionally null QE whose semantic set is restricted by the topic NP, as in \REF{18:ex29} and \REF{18:ex30}. The structure in \REF{18:ex30}, just like the one in \REF{18:ex16}, is licensed by two conditions: (i) the plurality requirement placed on the CT (i.e., NP) that moves out of a non-specific NumP with a set \isi{partitive} \isi{construal}, and (ii) the impossibility of reconstruction of the \isi{plural} NP to the complement to Num position. In the case of \REF{18:ex16}, the latter condition results from \REF{18:ex8}. In the case of \REF{18:ex30}, it results from the fact that a lower \isi{numeral} cannot take a \isi{plural} NP complement; see \REF{18:ex1} and \REF{18:ex28b}. Importantly, the generation of the more complex NCC is possible only when the two conditions prevent the generation of the simpler structure in \REF{18:ex4}. In all other cases, economy rules out the NCC and the structure in \REF{18:ex4} is used.
% example 29
\ea \label{18:ex29}
\gll Romanov\textsubscript{1}  na  stole  bylo  \textsc{dva}  (\hspace{-2pt} toma)    t\textsubscript{1}.\\
     novels.\textsc{gen.pl}  on  table  was.\textsc{3sg.n}  two   {} volume.\textsc{gen.sg}\\
\glt `As for novels, there were two volumes on the table.'  
\z
% tree 30
\ea \label{18:ex30} \begin{forest}
[NumP
	[Num\\\textit{dva}\\`two']
    [MeasureP
    	[Measure\\\textit{(toma)}\\`volume.\textsc{gen.sg}']
        [NP
        		[\textit{romanov}\\`novels.\textsc{gen.pl}', roof first-line-width]
        ] { \draw (.east) node[right]{\hspace{-2mm}\textsc{(gen.pl)}}; }
    ] { \draw (.east) node[right]{\hspace{-2mm}\textsc{(gen.sg)}}; }
]
\end{forest}

\z

\noindent By analogy with \REF{18:ex16}, the head of the MeasureP in \REF{18:ex30} is optionally null. As the set represented by the QE is consistently a superset to the set introduced by its NP complement, the QE is semantically recoverable, in the sense that when it is null, it can be interpreted as representing any set of which the set denoted by the NP is a subset. In \REF{18:ex30} the overt QE denotes a set of volumes on the table out of which a set of novels is a subset, but the set represented by the QE can be even more open and denote a set of books on the table out of which a set of novels is a subset, as in \REF{18:ex31}. 
% example 31
\ea \label{18:ex31}
\gll Romanov  na   stole  bylo  \textsc{dve}  knigi.\\
     novels.\textsc{part/gen.pl}  on  table  was.\textsc{3sg.n}  two  book.\textsc{gen.sg}\\
\glt  `As for novels, there were two books on the table.'
\z

\largerpage[2]
\noindent Typically, the set represented by the QE is contextually specified, as in \REF{18:ex32} where it is given as the superset in the contextual question. That is, depending on whether the items on the table out of which the set of novels is selected are books or different kinds of reading materials (e.g. novels, newspapers, magazines, journals etc.) or different kinds of unrelated items (e.g. novels, apples, plates, flowers etc.), the set can be as open as to include all \isi{inanimate} entities, as long as the context (linguistic or extra-linguistic) warrants such a \isi{construal}.


% example 32
\begin{exe}
\ex \label{18:ex32}
\begin{xlist}
\exi{Q:}{	
	\gll 
    	 Skol’ko knig bylo na stole?\\
         how.many books.\textsc{gen} was.\textsc{3sg.n} on table.\textsc{prep}\\
	\glt `How many books were there on the table?'
    }
\exi{A:}{
	\gll Romanov na stole bylo \textsc{dve} knigi, 
    	 stixov (\hspace{-2pt} na stole bylo) \textsc{tri} (\hspace{-2pt} knigi),
         a slovarej (\hspace{-2pt} na stole bylo) \textsc{četyre} (\hspace{-2pt} knigi)\\
novels.\textsc{part/gen.pl} on  table was.\textsc{3sg.n} two book.\textsc{gen.sg} poems.\textsc{part/gen.pl} {} on table was.\textsc{3sg.n} three {} book.\textsc{gen.sg} and  dictionaries.\textsc{part/gen.pl} {} on table was.\textsc{3sg.n} four {} book.\textsc{gen.sg}\\
	\glt `There were two books of novels on the table, three (books of) poems and four (books of) dictionaries.'
    }
\end{xlist}
\end{exe}

\noindent It is, however, plausible that when the QE in \REF{18:ex30} is phonologically null and the context does not specify the nature of the set it denotes, it is interpreted as representing the most open set out of which the set denoted by its NP\textsubscript{} complement is a subset. We have seen that the most open superset for individuals is a set of people represented by the \isi{noun} \textit{čelovek}. Similarly, in \REF{18:ex30} the most open superset for the set of \isi{inanimate} entities is the set of items, represented by the \isi{noun} \textit{štuka}, as in \REF{18:ex33}.\footnote{\label{18:fn22}The QE in \REF{18:ex31}--\REF{18:ex33} cannot be phonologically null when Num carries \isi{feminine} gender features required for agreement with the \isi{feminine} MeasureP. When the QE is null, Num agrees in gender with the \isi{masculine} NP. As the MeasureP and the NP in \REF{18:ex31}--\REF{18:ex33} have distinct gender features, the constructions with an overt and a covert QE have distinct agreement features on the \isi{numeral}.
} 
% example 33
\ea \label{18:ex33}
\gll Romanov  na   stole  bylo  \textsc{dve}  štuki.\\
     novels.\textsc{part/gen.pl} on table  was.\textsc{3sg.n}  two  item/thing.\textsc{gen.sg}\\
\glt  `As for novels, there were two items on the table.'
\z

\largerpage
\noindent In \REF{18:ex33}, the \isi{noun} \textit{štuka} ‘item/thing’ selects a certain number of entities from a set of novels in exactly the same fashion as the \isi{noun} \textit{čelovek} ‘person’ selects a number of individuals from a set of pretty people in \REF{18:ex15} so that the only difference in the \isi{construal} of the QEs in \REF{18:ex33} and \REF{18:ex15} lies in the features [$\pm$\isi{animate}] and [$\pm$human].\footnote{\label{18:fn23}\citet{Yadroff1999} analyses the nouns \textit{štuka} and \textit{čelovek} used in NCCs as pleonastic \isi{noun} classifiers. He argues that the class of classifiers found in NCCs is closed, with \textit{štuk} ‘items\textsc{.gen.pl’} replaced with \textit{èkzempljárov} ‘copies\textsc{.gen.pl’} in formal register, and \textit{čelovek} ‘person.\textsc{nom.sg’} replaced with \textit{duš} ‘souls\textsc{.gen.pl’} in archaic texts. However, as can be seen from \REF{18:ex29}--\REF{18:ex32}, it is possible to have other nouns performing the role of the QE as long as they represent a superset to the set denoted by the NP complement. Just like any other classifier mentioned by Yadroff, the QEs in \REF{18:ex29}--\REF{18:ex32} can occur in a construction involving approximate inversion, as in \REF{18:fn23i}.  
\ea \label{18:fn23i}
\gll knig / tomov    pjat’   istoričeskix   romanov  \\
 	 books.\textsc{gen.pl} {} volumes.\textsc{gen.pl} five  historical.\textsc{gen.pl}   novels.\textsc{gen.pl}  \\
\glt `approximately five books/volumes of historical novels'
\zlast} 
In other words, \textit{štuka} represents the most open set of entities, while \textit{čelovek} denotes the most open set of individuals. Plausibly, in the absence of a contextual disambiguation, the null QEs in NCCs are interpreted as referring to these open sets.

Above, we mentioned that the superset \isi{construal} of the QE is not sufficient for it to remain null and that additional restrictions on \isi{semantic recoverability} apply. To be precise, while QE in \REF{18:ex30} can remain covert in structures involving \isi{contrastive} \isi{topicalization}, as in \REF{18:ex27}, in the absence of \isi{topicalization}, it must be overt; see \REF{18:ex34}. 
% example 34
\ea \label{18:ex34}
	\ea[]{ \label{18:ex34a}
    \gll Na  stole  bylo  dva  toma  romanov.\\           
         on  table  was.\textsc{3sg.n}  two  volume.\textsc{gen.sg}  novels.\textsc{part.pl}\\
    \glt `There were two books of novels on the table.'
    }
	\ex[*]{ \label{18:ex34b}
    \gll Na  stole  bylo  dva  romanov.\\
         on  table  was.\textsc{3sg.n}  two  novels.\textsc{part.pl}\\
    }
	\z
    
\z 

\noindent Plausibly, the option of remaining covert in \REF{18:ex27} is due to the IS partitioning of the non-referential NumP into focus on the Num and CT on the NP, which results in a set \isi{partitive} \isi{construal} of the NumP, which in turn requires the presence of the QE (null or overt) in order for the \isi{partitive} construction to denote two sets. It follows, then, that \isi{partitive} \isi{construal} itself presupposes the NCC containing the QE. Conversely, in \REF{18:ex34b}, it is impossible for the NumP to have the corresponding CT/Foc partitioning because the NP does not move across the \isi{numeral} \citep{Titov2013}. Hence, in the absence of \isi{contrastive} \isi{topicalization}, the QE must be overt. Yet, there is one exception to this rule, i.e., the QE can stay covert and be recovered when it refers to the same set as denoted by the head of its NP\textsubscript{} complement, as in \REF{18:ex9b} where both heads select out of a set of people; see \REF{18:ex16}. This rare occurrence is due to the deficient lexical number forms of the two nouns, which allows them to co-occur as long as there is a restriction of the set represented by the QE by the set denoted by its NP complement that can be achieved either via modification or \isi{topicalization}. Since both heads in \REF{18:ex16} denote the same set, the \isi{referent} of the QE is recoverable from the \isi{referent} of the head of the NP.\textsubscript{} 

% section 5
\section{Conclusion}\label{18:s5}

In this paper, we have discussed two types of number agreement mismatch in \ili{Russian} \isi{numeral} phrases. We have proposed a unified syntactic account for both phenomena that assumes the NCC in \REF{18:ex16} and \REF{18:ex30} where the \isi{plural} NP is a complement to an optionally null QE. We have argued that the structure is forced by a plurality requirement placed on the head of the NP, and either the selectional requirement of a lower \isi{numeral}, as in \REF{18:ex27}, or by \REF{18:ex8}, as in \REF{18:ex9} and \REF{18:ex10}. We have maintained that the optionally covert status of the QE results from its limited semantic function, and its \isi{semantic recoverability}. The latter obtains in two cases, the most common of which involves \isi{contrastive} \isi{topicalization} and \isi{partitive} \isi{construal} that results in the salience of the set represented by the QE. The other case is restricted to nouns that lack one of the lexical number forms, in which case the \isi{referent} of the QE is identical to the \isi{referent} of the head of its NP complement, allowing for its \isi{semantic recoverability}. 


\section*{Abbreviations}

\begin{tabularx}{.5\textwidth}{@{}lQ@{}}
\textsc{3}&third person\\
\textsc{acc}&\isi{accusative}\\
\textsc{ct}&\isi{contrastive} topic\\
\textsc{gen}&\isi{genitive}\\
\textsc{gq}&\isi{genitive} of quantification\\
\textsc{is}&\isi{information structure}/al\\
\textsc{m}&\isi{masculine}\\
\textsc{n}&\isi{neuter}\\
\end{tabularx}%
\begin{tabularx}{.5\textwidth}{@{}lQ@{}}
\textsc{ncc}&numeral-classifier\\
&construction\\
\textsc{nom}&\isi{nominative}\\
\textsc{part}&\isi{partitive}\\
\textsc{pl}&\isi{plural}\\
\textsc{prep}&prepositional case\\
\textsc{qe}&quantifying expression\\
\textsc{sg}&singular\\
\end{tabularx}

\section*{Acknowledgements}

Many thanks to the audience of FDSL 12 and the anonymous reviewers for useful comments on the material presented here. I would also like to thank my language consultants for grammaticality judgments and the editors for their invaluable work.

% \newpage
\sloppy
\printbibliography[heading=subbibliography,notkeyword=this]


\end{document}
