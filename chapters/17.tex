\documentclass[output=paper,
modfonts,
newtxmath,
hidelinks,
]{langscibook} 


\papernote{\footnotesize\normalfont
Teodora Radeva-Bork. Transitivity Requirement revisited: Evidence from first language acquisition. To appear in: Denisa Lenertová, Roland Meyer, Radek Šimík \& Luka Szucsich (eds.), \textit{Advances in formal Slavic linguistics 2016}. Berlin: Language Science Press. [preliminary page numbering]
}

\setcounter{chapter}{16}

\title{Transitivity Requirement revisited: Evidence from first language acquisition
}  

\author{%
 Teodora Radeva-Bork\affiliation{University of Potsdam}
}

% \chapterDOI{} %will be filled in at production
% \epigram{}

\abstract{
The paper investigates null objects in early child grammar in light of the Transitivity Requirement approach \citep{Cummins-Roberge2005}, which states that transitivity is not dependent on the lexical features of the verb but is a universal grammatical property. I review naturalistic and experimental child data from sixteen typologically different languages (including five Slavic representatives) and show that the predictions of the Transitivity Requirement approach are not borne out. Instead, the results suggest that early object omissions reflect the presence of (optional) object drop in the target grammar. Children seem to omit objects only if the target grammar allows for this option, as it is the case, for example, in Russian, Ukrainian and Polish.

\keywords{null objects, child grammar, Transitivity Requirement, Slavic, crosslinguistic data}
}

\begin{document}
\maketitle
\shorttitlerunninghead{Transitivity Requirement revisited: Evidence from first language acquisition}

% SECTION 1
\section{Introduction and preliminaries}\label{sec:key:1}

While the study of null subjects in Slavic has received much attention (\citealt{Franks1995}, \citealt{Lindseth1998}, \citealt{Fehrmann-Junghanns2008}, \citealt{Müller2006}, among others), null / missing / implicit direct objects still constitute an under-researched area and the distribution of object drop is still not uniformly capturable. Object drop has not been used extensively as a way to classify languages in a typology. In other words, whereas it is common to talk about pro-drop or null subject languages, references to “object drop languages” or “null object languages” are much less frequent in the literature. One important reason for this classificatory asymmetry is that object drop appears to be much more variable than subject drop. Most attempts to identify a common denominator for null objects have failed in cross-linguistic terms. Possible restrictions on object drop have been discussed previously, such as, for instance, overt morphological verb-object agreement, which holds for Swahili or Georgian but not for Russian or Chinese; topic drop, holding for German but not for other null-object languages; as well as other conditions like specific structural contexts favouring the appearance of null objects (e.g. sequence of verbs or imperatives). Generally, it is assumed that null objects are a licit option in the grammars of Russian, Polish, to some extent German, European and Brazilian Portuguese, and Chinese, among other languages. Languages such as Bulgarian, Serbo-Croatian or Spanish, on the other hand, disallow null objects.

In this paper, I examine the omission of referential, definite objects as in \REF{ex1}, a type that happens to be ungrammatical in English \REF{ex1a}, but grammatical in other languages, such as Russian \REF{ex1b}. What I leave aside are non-referential null objects, illustrated by \REF{ex2}. For a discussion of the licensing of object drop of indefinite DPs in European Spanish, Modern Greek, and Bulgarian see \citet{Campos1986}, \citet{Giannakidou-Merchant1997}, and \citet{Dimitriadis1994}. See also \citet{dvorak17} for a recent in-depth discussion of indefinite and generic null objects in Czech. For the sake of terminological clarity, I use \textsc{null objects} to refer to the phonological non-realisation of direct objects in transitive contexts, as in \REF{ex1}. (Other common terms include “object omission” or “object drop”.)

% eample 1
\begin{exe}
\ex \textbf{Referential/Definite null object}\label{ex1}
\begin{xlist}
\ex\label{ex1a}
\begin{xlist}
\exi{A:}[]{
	\glt What did you do with the newspaper?
    }
\exi{B:}[*]{
    \glt I read Ø.
        }
\end{xlist}
\ex\label{ex1b}
\begin{xlist}
\exi{A:}[]{
	\gll 
    	 Čto ty delaeš s ėtim rasteniem?\\
         what you.\textsc{nom} do.\textsc{1sg} with this plant.\textsc{ins}\\\hfill(Russian)
	\glt `What are you doing with this plant?'
    }
\exi{B:}[]{
    \gll 
    	 Polivaju Ø. / Ja polivaju ego.\\
         water.\textsc{1sg} {} {} I water.\textsc{1sg} it\\
	\glt `I’m watering it.'
        }
\end{xlist}
\end{xlist}
\end{exe}

% eample 2
\begin{exe}
\ex \textbf{Non-referential/Indefinite null object}\label{ex2}
\begin{xlist}
\exi{A:}[]{
	\glt What are you going to do while you wait?
    }
\exi{B:}[]{
    \glt I’ll buy a newspaper and I’ll read Ø.
        }
\end{xlist}
\end{exe}

\noindent Object realization or omission have both a syntactic component (what kinds of mechanisms govern the licensing and recoverability of null objects) and a lexical component (what types of verbs allow optional realization of their direct object argument). In this paper, I concentrate on a syntactic approach to transitivity, based on the so-called \textsc{Transitivity Requirement} (TR) proposed by \citet{Cummins-Roberge2005}. In parallel to the \textsc{Extended Projection Principle} (EPP) for subjects, it suggests that the direct-object position is given by Universal Grammar and is not dependent on the lexical features of the verb. The syntactic analysis of null objects is particularly appealing as it provides very concrete and testable predictions about transitivity development in first language acquisition. Under the TR, null objects are predicted to be a part of the default initial setting for acquisition purposes. This view is advocated in \citet{Perez-Leroux-etal2008}, who suggest that children start out with a null cognate object default, and that the initial referential properties of this null cognate object are broader than in the target grammar.\footnote{\label{fn2}Since different languages have different conditions as to where objects are allowed to remain unpronounced, \citeposst{Perez-Leroux-etal2008} approach is to seek for a common denominator in null object constructions. On the basis of French and English data, they identify a null bare N object to be the common denominator in English and French. The authors suggest that by postulating this common denominator as the minimal default, they can make inferences about what development is required to attain the proper distribution and features of null objects in a given target.} Experience serves to block, or narrow down, the referential semantics of the null default. It follows from this that we should be able to find evidence of object omissions in the early stages of language development in typologically different languages, irrespective of the availability of null objects in the target grammar.

The main agenda of this paper is to evaluate the empirical validity of the TR by examining acquisition data from sixteen typologically different languages, including five Slavic representatives; see \tabref{table:table_1} and \tabref{table:table_1a}. I review the results from studies carried out on these languages with the aim to examine the object (non)omission in the early stages of grammar, especially in light of the TR. Such a secondary approach to primary data is justified since as more research on a given topic within a particular language family emerges, it is valuable to have research that consolidates the studies and elucidates similarities and differences across language families. The Slavic perspective is particularly interesting since Slavic languages vary with respect to the availability of object drop although they share a number of common morphosyntactic features. Additionally, language acquisition in Slavic is still under-researched compared to other languages, and this paper aims to contribute to the cross-linguistic investigation of the early development of objects by presenting and reviewing child data from Slavic.

The paper is organised as follows. \sectref{sec:key:2} sketches some theoretical approaches to object omission, focusing on the discussion of the syntactic transitivity approach by \citet{Cummins-Roberge2005} and outlining the predictions of this analysis for the acquisition of objects, with respect to the object omissions children are predicted to show. In \sectref{sec:key:3}, I discuss experimental and naturalistic child data from Russian, Serbo-Croatian, Bulgarian, Polish, Ukrainian, French, English, Spanish, Catalan, Italian, European Portuguese, Brazilian Portuguese, Romanian, Standard Modern Greek, Cypriot Greek, and Chinese. The participants in the studies are typically-developing monolingual children with the core age 2--4 years, as well as 4--6 years for some languages (for the detailed data description and methodology, see \sectref{sec:key:3.1}). The survey of the data shows that the predictions made by the TR are not borne out, and null objects are not a default setting in the early stages of grammar. Based on the empirical findings, I suggest that there is a strong link between children’s object omissions and the grammaticality of null objects in the target grammar. This view is compatible with the proposal made in \citet{Varlokosta-etal2016}, suggesting that children generally opt for the weakest alternative on the scale pronoun > clitic > null, depending on what is available in their language. Of course, this proposal needs further investigation in studies that test \textit{different} types of objects, i.e. full pronouns, clitics and full DPs.

% SECTION 2
\section{The Transitivity Requirement and its predictions for child grammar}\label{sec:key:2}

To start off, I briefly sketch the lexically and syntactically motivated approaches to argument structure, with special emphasis on the syntactic transitivity approach by \citet{Cummins-Roberge2005} and its prediction for the development of (direct) objects in the early stages of grammar.

Verbs are flexible as to which and how many argument positions they project \citep[25]{VanHout2012}. According to the lexical approach, the verb’s flexibility is incorporated into its lexical representation, i.e. the verb is lexically represented with more than one representation, each of which is linked to a certain verb subcategorization frame \citep{Chomsky1965,Emonds1991}. In the case of English generic null objects, \citet{Rizzi1986} assumes that theta roles can be fully saturated in the lexicon. Other, discourse-motivated approaches, such as \citet{Groefsema1995} and \citet{Fellbaum-Kegl1989}, associate the use of certain null objects such as generic (non-referential) null objects, cf. \REF{ex2}, with discourse factors and pragmatic considerations.

An alternative analysis is provided by the modular account relying on a strictly syntactic approach to the occurrence of null objects. The Transitivity Requirement (TR) by \citet{Cummins-Roberge2005}, parallel to the Extended Projection Principle (EPP) for subjects, suggests that the direct-object position is given by Universal Grammar and is not dependent on the lexical features of the verb. Thus, the direct-object position is not seen as a characteristic depending on the lexical-semantic features of the verb, but rather as an integral, essential element of the predicate. Under the TR, transitivity is viewed as a universal grammatical property. Null objects are structurally present, and all VPs (i.e. with transitive, unergative, unaccusative verbs, etc.), contain an object position that can be overtly expressed or not \citep{Cummins-Roberge2005}. When an object is not phonologically realized, it remains as a null object in the VP.

Under the TR, \REF{ex3} is considered to be a universal structural template for objects \citep{Cummins-Roberge2005,Perez-Leroux-etal2008}. This template is shown in the tree below, where N is an implicit null object.

\ea \label{ex3} \begin{forest}
  [V, s sep=1.5cm
    [V]
    [N
      [Ø \\ $\rightarrow$ s-selection]
    ]
  ]
\end{forest}
\z

\noindent The main premises of the TR-based approach, namely that (i) transitivity is a universal grammatical property and (ii) null objects are by default structurally represented in all languages, provide a fruitful ground for making precise predictions about the initial states of human grammar. If null objects are present by default, we should expect children to go through a stage of object optionality (cf. \citealt{Perez-Leroux-etal2008}), irrespective of the object-drop capacity of the specific target grammars. An overgeneralization of the free availability of null objects due to a failure to restrict the null structure to the appropriate context is predicted. Omissions should therefore be found in typologically different languages, irrespective of the availability of null objects in the target grammar and without reference to the pronominal system of the specific language. For example, objects are expected to be dropped in the early development of languages with and without clitic systems (such as Bulgarian and English, for example). Such a prediction is particularly challenging since clitic pronouns are generally prohibited from dropping. The emerging research question, namely whether children of all languages go through a null object stage, is addressed in the next section, presenting empirical data from sixteen languages.


% SECTION 3
\section{Null objects in child grammar}\label{sec:key:3}

In order to test the validity of the predictions made by the TR, I turn to the examination of how children acquiring various languages deal with direct objects in the acquisition process. The comparison of developmental patterns in typologically different languages such as Russian, Greek, French, and Chinese, to name only a few, allows to make hypotheses about universally represented structures at the starting point of linguistic development and about grammatical elements that are specific to particular languages. More importantly, a comprehensive survey of studies conducted on the acquisition of objects in different languages, which summarizes and compares the derived results, can test the predictions made by the TR that children of \textit{all} languages go through a null object stage.

% SUB-SECTION 3.1
\subsection{Data}\label{sec:key:3.1}

%%%%%%%%%%%%%%%%%%%%%%%%%%%%%%% TABLE 1 %%%%%%%%%%%%%%%%%%%%%%%%%%%%%%%%%%%%%%%%%%%%%%%%%%

\begin{table}[b]
	\captionsetup{format=hang}
	\caption{Reviewed studies on the acquisition of objects\\(Slavic languages)}
	\footnotesize
%\label{table1} 
    %\begin{tabular}{ | l | l | l | l |}
    \begin{tabularx}{\textwidth}{p{2cm}Q@{}ll}
    \lsptoprule
     \textbf{Language} & \textbf{Studies} & \textbf{Type of data} & \textbf{Age}\\
% %---------------------------
     %Russian & \citet{GordishevskyAvrutin2004} \newline \citet{Frolova2016} , submitted & spontaneous data \newline elicited data & 1;9-2;6 \newline 2;10-6;1\\
     \midrule
     Russian & \citet{Gordishevsky-Avrutin2004} & spontaneous  & 1;9--2;6\vspace{5pt}\\ 
	    & \citet{Frolova2016,Frolova} 	& elicited  & 2;10--6;1\vspace{10pt}\\
% %---------------------------
 	
 	Serbo-Croatian & \citet{Stiasny2003,Stiasny2006} & elicited and spontaneous  & 1;10--4;7\vspace{10pt}\\
% %---------------------------
 	
     Bulgarian & \citet{Radeva-Bork2013,Radeva-Bork2015} & elicited  & 2;2--4;3\vspace{10pt}\\
% %---------------------------
     
     %Polish & \citet{Tryzna2015} & spontaneous  \newline elicited comprehension and production  & 2;1-2;9 \newline 2;4-5;10\\
     Polish & \citet{Tryzna2015} & spontaneous  & 2;1--2;9\\
	  & 				 & elicited comprehension and production  & 2;4--5;10\vspace{10pt}\\
% %---------------------------
 	
     Polish\newline\& Ukrainian & \citet{Mykhaylyk-Sopata2016} & elicited  & 3;0--6;0\\
% %---------------------------
\lspbottomrule	
%     \end{tabular}
\end{tabularx} 
\label{table:table_1}
\end{table}

\begin{table}
\captionsetup{format=hang}
	\caption{Reviewed studies on the acquisition of objects\\(non-Slavic languages)}
	\footnotesize
    \begin{tabularx}{\textwidth}{p{2cm}Q@{}ll}
    \lsptoprule
     \textbf{Language} & \textbf{Studies} & \textbf{Type of data} & \textbf{Age}\\ \midrule
     French & \citet{Hamann-etal1996,Jakubowicz-Rigaut2000,Perez-Leroux-etal2008} & elicited and spontaneous  & 2;0--6;0\vspace{5pt}\\ 
      &  \citet{Grueter2006} & comprehension  &\vspace{10pt}\\
% %---------------------------
 	
     %English & 2 & spontaneous  \newline elicited comprehension  \newline elicited  & 2;0-3;0 \newline 2;0-6;0 \newline 2;9-5;11\\
     English & \citet{Bloom1990} & spontaneous  & 2;0--3;0\vspace{5pt}\\
	    & \citet{Grueter2006} & elicited comprehension  & 2;0--6;0\vspace{5pt}\\ 
	    & \citet{Perez-Leroux-etal2008} & elicited  & 2;9--5;11\vspace{10pt}\\
% %---------------------------
     
     %Spanish & 2 & elicited and spontaneous  \newline comprehension  & 2;0-5;0 \newline 2;0-4;0\\
      Spanish & \citet{Wexler-etal2004,Stiasny2006,Castilla-etal2008} & elicited and spontaneous  & 2;0--5;0\vspace{5pt}\\ 
       & \citet{Mateu2015} & comprehension  & 2;0--4;0\vspace{10pt}\\
% %---------------------------
     
      Catalan & \citet{Wexler-etal2004} & elicited  & 2;0--4;0\vspace{10pt}\\
% %---------------------------
     
      %Italian & \citet{} & elicited and spontaneous  & 2;0-5;0\\
      Italian & \citet{Guasti1993,Cardinaletti-Starke2000,Schaeffer2000}; & elicited and spontaneous  & 2;0--5;0\\  
       & \citet{Tedeschi2009} & &\vspace{10pt}\\
% %---------------------------
     
\multirow{2}{2cm}{European Portuguese} & \citet{Costa-Lobo2007a,Costa-Lobo2007b,Carmona-Silva2007} & elicited  & 3;0--6;6\\  
      & \citet{Silva2010} & &\vspace{10pt}\\
% %---------------------------
     
      Brazilian \mbox{Portuguese} & \citet{Lopes2008,Lopes2009} & spontaneous  & 1;8--3;7\vspace{10pt}\\
% %---------------------------
     
      Romanian & \citet{Babyonyshev-Marin2006} & elicited and spontaneous  & 2;0--3;10\vspace{10pt}\\
% %---------------------------
     
\multirow{2}{2cm}{Standard \mbox{modern Greek}} & \citet{Stephany1997,Marinis2000} & spontaneous  & 1;9--2;9\vspace{5pt}\\ 
     & \citet{Tsakali-Wexler2003} & elicited  &1;9--2;9\vspace{10pt}\\
% %---------------------------
     
     Cypriot Greek & \citet{Grohmann-etal2010,Petinou-Terzi2002,Neokleous2011} & elicited and spontaneous  & 3;0--5;11\vspace{10pt}\\ 
% %---------------------------
     
     Chinese & \citet{Wang-etal1992} & elicited  & 3;0--4;0\\
% %---------------------------
\lspbottomrule	
%     \end{tabular}
\end{tabularx} 
\label{table:table_1a}
\end{table}

I review data from experimental studies (see Tables \ref{table:table_1} and \ref{table:table_1a}), concerned with both the production and comprehension of direct objects in elicited and naturalistic environments. The data stem from the studies on sixteen typologically different languages. The focus of the present paper is on the five Slavic representatives: Russian, Serbo-Croatian, Bulgarian, Polish and Ukrainian, but the data are also placed in a cross-linguistic context by comparing the five Slavic languages to eleven other languages, for which object drop has been studied, namely French, English, Spanish, Catalan, Italian, European Portuguese, Brazilian Portuguese, Romanian, Standard Modern Greek, Cypriot Greek, and Chinese. Tables \ref{table:table_1} and \ref{table:table_1a} give an overview of the languages and the conducted studies, including information about the type of data, i.e. elicited or/and spontaneous, as well as about the ages of tested children.\footnote{\label{fn3}Ages are given in years and months, i.e. 1;9 indicates 1 year and 9 months of age.} For French, English, Spanish, and Italian, there is a greater number of studies than for other languages, so only a selection of the most recent and representative studies could be included here.

The overview of studies shows that the acquisition of objects has been well examined over the last three decades, with studies covering a vast number of languages and providing both spontaneous and elicited child data from production and comprehension, something which is rather rare in the assessment of acquisition of other grammatical phenomena. This is particularly beneficial for the present goals, since the TR-based approach predicts object drop in the early stages of language acquisition irrespectively of typological differences found in individual language systems.\largerpage[-2]

Here, I analyse production and comprehension data from Polish, French, English, and Spanish. For Russian, Serbo-Croatian, Bulgarian, Ukrainian, Catalan, Italian, European Portuguese, Brazilian Portuguese, Romanian, Standard Modern Greek, Cypriot Greek, and Chinese, I deal with production data in elicited and spontaneous contexts. The core age of the participants in the studies lies between two to four years, with some languages (Russian, Polish, French, English, and European Portuguese) including older children, four to six year old, in some of the studies. In the majority of the studies participants are controlled for gender. The subjects are typically-developing, monolingual children, recruited from day cares or schools.



The comparison of results from the included studies is legitimate due to the use of a conform and highly comparable experimental methodology, which is described in the next paragraph. In fact, in a recent analysis of meta-megastudies, \citet{Myers2016} shows that methodological differences across studies seem generally insufficient to explain large differences in results, and that what seems to have a bigger effect are typological differences between languages. Whereas a detailed discussion of methodological effects in object elicitation tasks is beyond the scope of this paper, I hold that it is legitimate to compare the results from the presently included studies mainly due to the use of a common elicitation procedure. However, see \citet{Varlokosta-etal2016}, who argue for an effect of the used elicitation methodology on the production of clitic objects in experimental tasks.

Studies on the acquisition of objects employ a standard elicited production task (\citealt{Schaeffer2000,Perez-Leroux-etal2008,Radeva-Bork2012}; among others) to examine how children use direct objects in transitive contexts of the kind found in \REF{ex4}, where \REF{ex4a} is a licit option in the adult grammar of some languages, such as Russian or Polish, but not in others, such as Bulgarian or Serbo-Croatian. Examples \REF{ex4b} and \REF{ex4c} represent the grammatical choices for Bulgarian, making use of a full NP/pronoun or a clitic, respectively.

% example 3 (2)
\ea `What did the boy do?'\label{ex4}
   	\ea[*]{ \label{ex4a}\gll
     Toj ritna Ø.\\
     he kicked\\\hfill(Bulgarian)
    \glt Intended: `He kicked it.'
    }
	\ex[]{ \label{ex4b}
    \gll Toj ritna topkata / neja.\\
         he kicked ball.\textsc{f.def} {} her.\textsc{acc}\\
    \glt `He kicked the ball.' / `He kicked it.'
    }
  	\ex[]{ \label{ex4c}
    \gll Toj ja ritna.\\
         he it.\textsc{cl} kicked\\
    \glt `He kicked it.'
    }
	\z
\z

\noindent In such elicitation tasks, participants are shown simple act-outs with toys and props, or picture cards illustrating simple activities, such as kicking a ball, drawing a flower, or building a house. Every activity represents a transitive scenario with a subject and an object. The studies involve a big number of test items, usually between six and twelve. After the visual prompt, participants hear a control question of the kind \textit{What did X do?} without the target object being mentioned. Depending on the specificities of the language, target answers contain a transitive structure with an overt object or with its omission, cf. \REF{ex4}. Transitive verbs such as \textit{kick}, \textit{draw}, \textit{build}, \textit{give}, \textit{hug}, \textit{drink}, \textit{hit, push} etc. are elicited in the tasks. A screening prior to the study guarantees that the children understand the object nouns and the verbs denoting the actions in the tasks. An example of a model elicitation of a direct object is given in \REF{ex5}. The use of an overt object is obligatory here. Similar tasks have been used in the elicitation studies presented in Tables \ref{table:table_1} and \ref{table:table_1a}. For the spontaneous data, recordings and transcripts are used.\largerpage[-2]

% example 3 (5)
\ea \textbf{Model elicitation of direct objects in Bulgarian}\label{ex5}
\begin{exe}
\exi{\textsc{Experimenter 1:}}
\exi{}   `This is Maria. This here is her favourite doll. The doll’s hair is so bushy.' (utterance accompanied by an act-out of the experimenter combing the doll)
\exi{\textsc{Experimenter 2:}}%\hphantom{.}\\
\exi{}	\gll Kakvo napravi Maria?\\
     	what did Maria\\
		\glt ‘What did Maria do?’
\exi{\textsc{Child 2;6:}}%\hphantom{.}\\\\
\exi{}	\gll Sresa kuklata.\\
        combed doll.\textsc{def}\\
		\glt ‘She combed the doll.’\hfill (adapted from \citealt[79]{Radeva-Bork2012})
\end{exe}        
\z

% SUB-SECTION 3.2
\subsection{Results}\label{sec:key:3.2}

An analysis of the obtained results shows that there is a high degree of variation across languages when it comes to object omission in early grammars. Since it is impossible to give a detailed presentation of the results from the individual studies in this paper, I focus on the Slavic data (marked in bold in \tabref{table:table_2}), and present the results from the other languages for the sake of cross-linguistic comparison.\largerpage[2]

\begin{table}[h]
	%\caption{\textbf{General results for the spread of object (non)omission.}}
    \caption{General results for the spread of object (non)omission.}
\begin{center}
    \begin{tabularx}{0.85\linewidth}{XXX}
    \lsptoprule
     \textbf{Object omission} & \textbf{No object omission} & \textbf{Conflicting data}\\ \midrule
    \textbf{Russian} & \textbf{Bulgarian} & French \\ 
    \textbf{Ukrainian} & \textbf{Serbo-Croatian} & English \\ 
    \textbf{Polish} & Spanish &  \\ 
    E. Portuguese & Modern Greek & \\ 
    Br. Portuguese & Cypriot Greek & \\ 
    Chinese & Romanian & \\ 
    Italian & & \\ 
    Catalan & & \\ \lspbottomrule
    \end{tabularx}
\end{center}
\label{table:table_2}
\end{table}

Generally, we find evidence of object omission in Russian, Ukrainian, Polish, European Portuguese, Brazilian Portuguese, Chinese, Italian, and Catalan, but not in Bulgarian, Serbo-Croatian, Spanish, Modern Greek, Cypriot Greek, and Romanian. Children in the latter group produce their obligatory objects in transitive contexts from the early stages of language development in a target-like manner. In contrast, Russian, Ukrainian, Polish, European Portuguese, Brazilian Portuguese, Chinese, Italian, and Catalan undergo a stage of object omission, in which obligatory transitive contexts do not yield an object in the early stages of first language acquisition. Regrettably, I had to put French and English aside, since the individual studies on each of these languages yielded contrasting results with respect to how much object omission was found in children. \tabref{table:table_2} summarizes the main results from the studies on the sixteen languages under analysis.\largerpage

Let me discuss the results in more detail. Although results from individual studies on Spanish vary as to how much omission is found in the early stages, all of the studies support the view that Spanish objects are acquired early, around the age of two to three years. On the basis of the elicitation data from 28 children, \citet{Wexler-etal2004} show that two-year-olds literally never omit objects (omission is at $0\%$). These results are consistent with the spontaneous data provided in \citet{Stiasny2006}. In contrast to Spanish, for Catalan \citet{Wexler-etal2004} find high rates of object omission. Two-year-olds omit objects $74\%$ of the time. The object omission remits as age progresses but does not disappear by the age of four years.

Italian patterns with Catalan with respect to object omission -- the rate of object omission is high in both languages for ages two to four. Object omissions in Italian have been evidenced both in spontaneous speech (a.o. \citealt{Guasti1993}) as well as in elicitation data \citep{Schaeffer2000}. The two-year-olds in Schaeffer’s study omit objects at high rates of up to $64\%$. Object omission at $15\%$ is still present in the production of three-year-olds. These findings are confirmed by similar rates of object omission for the same ages in \citet{Tedeschi2009}. It is not before the age of four that Italian children cease omitting their objects and omissions fall to $0\%$. So whereas Spanish children produce overt objects from the early on, Italian children go through an initial phase of object omission (ending at around four years).

In an experimental study for Romanian, \citet{Babyonyshev-Marin2006} find that Romanian-speaking children “produce object clitics freely as soon as they are able to produce utterances that are long enough to contain them” (p. 31). The authors divide their population into groups according to MLU and not according to age.\footnote{\label{fn4}MLU refers to Mean Length of Utterance, a technique often used in L1 acquisition research to measure the complexity of a child’s speech by calculating the number of words (or morphemes, on some approaches) per utterance.} The results indicate object omission of $82\%$ for children with MLU smaller than two, and omission of $13\%$ for children with MLU greater than two. Since Babyonyshev and Marin show that object omission in Romanian is due to production limitations (such as low MLU) instead of a grammatical constraint, we can conclude that the initial stage of language development in Romanian is not characterized by object omission.

When it comes to Slavic languages, Bulgarian and Serbo-Croatian pattern alike since the children in the studies did not omit objects \citep{Radeva-Bork2013,Radeva-Bork2015,Stiasny2006}. No object omission or misplacement has been found in Serbo-Croatian in either elicited or naturalistic production \citep{Stiasny2006}. The same holds for Bulgarian; objects do not get omitted and are used in a target-like manner already around the age of 2;3 on \citep{Radeva-Bork2015}. If we compare Italian and Bulgarian, we see that Italian two-year-olds omit objects $64\%$ of the time \citep{Schaeffer2000}, whereas their Bulgarian peers omit objects only about $30\%$, so about half as much as in Italian. Null objects fully disappear in Bulgarian towards the end of year three, which is not the case in Italian. Therefore the study results clearly indicate the lack of object omission in the acquisition of Bulgarian. In contrast, in Polish, Ukrainian and Russian, null objects are the preferred option for children \citep{Tryzna2015,Mykhaylyk-Sopata2016,Gordishevsky-Avrutin2004,Frolova2016}. In Polish and Ukrainian, children prefer to use null arguments up to the age of five. At the age of three they omit objects at $89\%$ in Polish and at $68\%$ in Ukrainian \citep{Mykhaylyk-Sopata2016}. The onset of direct object use seems to be semantically affected since around the age of five, clitics/pronouns are used more often for animate referents, and it is only around the age of six that they start being used also for inanimate objects \citep{Mykhaylyk-Sopata2016}.

In Russian, Ukrainian, and Polish, children do not only omit direct objects in obligatory transitive contexts, but they overproduce the null option when compared to adults (in the contexts where NO is allowed). This holds particularly for Russian, where three- to six-year old children produce more null objects than adults in the contexts where object omission is a grammatical possibility. Object omission at around $80\%$ was found for the age of three years \citep{Frolova2016}. Even at the age of five, Russian children omit referential objects at $73\%$ and non-referential ones at $54\%$. As \citet{Frolova} shows, Russian children even omit direct objects in strongly transitive (perfective) contexts where adults tend to use overt nouns but where the null object is still grammatical. Generally, production of null objects in Russian is attested at a similar rate across all age groups up to the age of six, and it is higher than for adults \citep{Frolova2016}. In non-referential contexts, a gradual decrease in object drop, an increase in lexical object (i.e., full DP object) use and a low production of pronouns is observed with the age progression. The rate of null objects is higher in referential contexts, where we rarely find lexical objects while the percentage of pronouns is higher. Similarly to their Russian peers, children acquiring Polish overuse null objects in comparison with adults, and the omission rate decreases as language development progresses \citep{Mykhaylyk-Sopata2016,Tryzna2015}.

From a cross-linguistic perspective, European Portuguese, Brazilian Portuguese, Chinese, Italian, and Catalan pattern with Russian, Ukrainian, and Polish in terms of the attested object omission in the early stages (for ages two to four and above). Spanish, Modern Greek, Cypriot Greek, and Romanian behave like Bulgarian and Serbo-Croatian in that they are not characterized by object drop in the acquisition process, and objects are present already at the age of two. The latter finding is in contradiction with the predictions made by the Transitivity Requirement (see \sectref{sec:key:2}).

% SUB-SECTION 3.3
\subsection{Discussion and implications}\label{sec:key:3.3}

The data survey from sixteen typologically different languages (including five Slavic representatives: Bulgarian, Serbo-Croatian, Russian, Ukrainian, and Polish) challenges the obligatory structural presence of null objects postulated by the TR, and calls for re-evaluation of this theoretical analysis of the null object phenomenon in adult grammars. The prediction made by the Transitivity Requirement that children of \textit{all} languages should go through a null-object stage is not borne out -- out of the sixteen languages, eight allow object omission in early grammar, six languages do not, and two languages (French and English) show conflicting results. Therefore, there is no evidence that null objects are a default initial setting for acquisition purposes. Instead, there seems to be a clear division between languages with and without object drop already in the early stages.

How can the division between languages in terms of object (non)omission be accounted for? Based on the results presented in \sectref{sec:key:3.2}, a parallel between children’s performance and the actual permission or prohibition of object drop in the target grammars emerges. Children omit objects only if their target grammar provides the null object option, which is the case for Russian, Ukrainian, Polish, European Portuguese, Brazilian Portuguese, Italian, Catalan, and Chinese. In contrast, Bulgarian, Serbo-Croatian, Spanish, Modern Greek, Cypriot Greek, and Romanian do not allow object drop, in the sense of example \REF{ex1}, and children seem to act according to the target grammar rules and produce objects from early on. Hence, early object omissions seem to reflect the presence of (optional) object drop in the target grammar. Children overgeneralize novel intransitives out of novel transitives and drop objects at higher rates than adults, provided that their target grammar has that option. They seem to be faithful to the syntax of the input. This observation is generally supported by experimental evidence in the language acquisition literature, indicating strong input sensitivity in acquisition and target-like omissions in spontaneous data \citep{Ingham1993}. In addition, the data discussed here supply support to the proposal in \citet{Varlokosta-etal2016} that children generally opt for the weakest alternative, in accordance with the scale pronoun > clitic > null, depending on what is available in their language.

Children seem to be faithful to the syntax of the input as their object drop reflects the presence of (optional) object drop in the target grammar and gives no evidence that null objects are a default setting for all languages. Furthermore, for the languages in which children omit objects, they seem to overgeneralize the null option. Data from Chinese as well as from European and Brazilian Portuguese confirm that children tend to overuse the option of object-dropping, licensed by their target grammar in some contexts, as late as at the age of five (\citealt{Wang-etal1992}, \citealt{Costa-etal2012}, \citealt{Lopes2009}). In addition, it seems that if a null argument is available in the grammar, the discourse-pragmatic or semantic features of the direct object referent play an important role in argument realization. This is supported by studies showing a semantic effect on the use of direct objects, for example in Polish, where overt objects (clitics/pronouns) are used more often for animate referents around the age of five. Around the age of six, they are used for inanimate referents. It may be the case that null objects are different from null subjects in that semantic and discourse factors play a greater role in the presence and interpretation of the null object. This, however, needs further investigation.

% SECTION 4
\section{Conclusion}\label{sec:key:4}

The aim of this paper was to investigate object omission in early child grammar in light of the Transitivity Requirement (TR) approach (\citealt{Cummins-Roberge2005}), which states that transitivity is not dependent on the lexical features of the verb but is a universal grammatical property. Within this approach, null objects are predicted to be a default initial setting for language acquisition. If null objects are indeed default, we expect to find evidence for object drop in the early stage of development in various languages, irrespective of the (non)omission capacity of the specific target grammars.

The paper reviewed naturalistic and experimental child data from sixteen typologically different languages and showed that out of the sixteen languages, eight languages (Russian, Ukrainian, Polish, European Portuguese, Brazilian Portuguese, Italian, Catalan and Chinese) allow object omission in early grammar, six languages (Bulgarian, Serbo-Croatian, Spanish, Modern Greek, Cypriot Greek, and Romanian) do not, and two (French and English) show conflicting results. The predictions of the TR approach are not borne out and the idea of null objects being a default setting in the early child grammar is invalidated. Instead, there is a clear division between languages with and without object drop in the early stages. In fact, the results from the studies suggest that early object omissions reflect the presence of (optional) object drop in the target grammar. In other words, children seem to omit objects only if their target grammar allows for this option, as it is the case, for example, in Russian, Ukrainian and Polish.


% \section{Bibliography}

%%%%%%%%%%%%%%%%%%%%%%%%%%%%%%%%%% END CONVERTED CODE %%%%%%%%%%%%%%%%%%%%%%%%%%%%%%%%%%%%%


\section*{Abbreviations}

\begin{tabularx}{.45\textwidth}{lX}
\textsc{1}&first person\\
\textsc{acc}&accusative\\
\textsc{cl}&clitic\\
\textsc{def}&definite\\
\end{tabularx}
\begin{tabularx}{.45\textwidth}{lX}
\textsc{f}&feminine\\
\textsc{ins}&instrumental\\
\textsc{nom}&nominative\\
\textsc{sg}&singular\\
\end{tabularx}

\section*{Acknowledgements}

I am grateful to two anonymous reviewers for their insightful comments.

\sloppy
\printbibliography[heading=subbibliography,notkeyword=this]

\end{document}
