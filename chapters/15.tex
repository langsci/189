\documentclass[output=paper,hidelinks,newtxmath,]{langscibook} 
\ChapterDOI{10.5281/zenodo.2545535}


\title{Extract to unravel: Left branch extraction in Romanian/Serbian code-switching}

\author{Vanessa Petroj\affiliation{University of Connecticut}}

\abstract{\citet{Boskovic2008,Boskovic2012} argues that languages with and without articles differ considerably with respect to the structure of the nominal domain (among other differences), leading to a distinction between DP (languages with articles) and NP (article-less) languages. Namely, DP languages are proposed to have a functional layer (DP) above the NP where articles are presumed to be positioned, while lacking definite articles indicates the absence of this functional layer in a language, allowing for bare NPs. This structural difference has semantic and syntactic consequences, one of which is the (im)possibility of left branch extraction (LBE) of adjectives and adjective-like elements out of the nominal domain. Specifically, while LBE is allowed in NP languages, is it disallowed in DP languages \citep{Boskovic2008,Boskovic2012}. While (dis)allowing LBE is fairly straightforward in languages in isolation, here, I extend this test to mixed DP/NP structures resulting from Romanian/Serbian code-switching (CS). Following the DP/NP language distinction, I consider Romanian to be a DP language, disallowing LBE, and Serbian an NP language, allowing LBE. Consequentially, I apply the LBE of adjectives from internal and external arguments of the verb, with switches at various points in the derivation. I show that LBE is reliable in determining the points where CS occurs, whether we are dealing with an NP or a DP projection, but also in showing that mixing two languages may not necessarily result in a uniform system. In other words, through LBE, the structural flexibility resulting from different points of CS indicates that CS, like LBE, is highly contextual and sensitive to phases and phasal domains.

\keywords{left branch extraction, code-switching, Romanian, Serbian}

}

\begin{document}
\maketitle
\shorttitlerunninghead{Extract to unravel: \isi{LBE} in Romanian/Serbian code-switching}

% SECTION 1
\section{Introduction}\label{15:s1}

\textsc{Code-switching} (CS) represents the alternation of elements from two languages during a single phrase, \isi{clause}, or utterance (\citealt{Poplack1980,GonzalesVelasquez1995,MacSwan1999,Muysken2000}; among others). In this paper, the focus is on the CS in \ili{Romanian}-\ili{Serbian} bilinguals from a small, culturally \ili{Romanian} town in the Republic of Serbia. In this paper, CS constructions, just like constructions belonging to any other natural language, are undergoing tests based on grammaticality judgements of bilingual native speakers. Specifically, here I investigate how relevant CS constructions that contain elements from \ili{Romanian} (a DP language) and \ili{Serbian} (an NP language) fare with respect to \textsc{left branch extraction} (\isi{LBE}) of adjectives out of the \textsc{traditional \isi{noun} phrase} (TNP).\footnote{\label{15:fn1}The term traditional \isi{noun} phrase covers both NP and DP, whichever applies in a given language, assuming the so called DP/NP parameter. Under the particular approach of \citet{Boskovic2014}, the TNP in languages with articles is DP, and in article-less languages it is NP. TNP is generally considered a phase, consequently, DP is a phase in DP languages, while NP is a phase in NP languages.} Given that \isi{LBE} is allowed in NP but not DP languages \citep{Uriagereka1988,Boskovic2008,Boskovic2012}, the combination of elements belonging to the two parameter settings (DP/NP) has consequences on the (im)possibility of \isi{LBE} in CS. More importantly, I show that \isi{LBE} is a reliable test to (i) identify which parameter setting prevails in certain environments, (ii) identify points of CS, and (iii) show that CS, like \isi{LBE}, is contextual and it depends on the elements that participate in the switch during a \isi{spell-out domain}.

The paper is organized as follows. \sectref{15:s2} provides the demographics, methods, and type of data used for this study. In \sectref{15:s3}, basic assumptions and relevant \isi{LBE} background are introduced. \sectref{15:s4} gives the background of the relevant CS construction to introduces the main questions addressed in this paper, and \sectref{15:s5} investigates \isi{LBE} in CS. Finally, \sectref{15:s6} concludes the paper and offers \isi{future} research directions.

For ease of exposition, I will follow the common practice of marking elements from the two languages uniformly throughout the paper; in CS examples, \textbf{Romanian} elements will be in \textbf{bold}, and \textit{Serbian} in \textit{italics}.

% SECTION 2
\section{Data and methods}\label{15:s2}

Data for this study was gathered in the course of several years. Examples found in this paper are extracted from speech produced by \ili{Romanian}-\ili{Serbian} bilingual speakers from a culturally \ili{Romanian} town called Uzdin, in Vojvodina, Serbia. The methods of data gathering include interviews targeting spontaneous production, elicitation, and grammaticality judgements.\footnote{For a more detailed overview discussing the subjects, data, and methods, I refer the reader to \citet{Petroj}.} 

Uzdin is one of the several towns in Serbia where the \ili{Romanian} language, culture, and customs have been highly preserved and nurtured. The author has interviewed 8 subjects, with the age mean of 27. All subjects have at least a college degree, and have attended K-8 grades in \ili{Romanian}, and high school and college in \ili{Serbian}. This \ili{Romanian} community is highly bilingual with a lot of code-switching occurring on a daily basis.

% SECTION 3
\section{Relevant background}\label{15:s3}
\subsection{General assumptions}\label{15:s3.1}

There are two underlying assumptions in this paper. The first is broad, referring to the approach and analysis of CS constructions. As argued by some authors (\citealt{GonzalesVelasquez1995,Bhatia-Ritchie1996,denDikken2011,BandiRao-denDikken2014}), I do not assume CS to impose restrictions that apply to CS constructions alone. Rather, given that participating languages are natural languages that adhere to UG principles, I treat CS in the same way. The second assumption is specific, concerning the language pair in question. Following \citet{Boskovic2008,Boskovic2012}, I consider \ili{Romanian} and \ili{Serbian} to differ with respect to whether they have or lack definite articles, consequently, whether they have or lack the DP layer.

\subsection{DP/NP languages and left branch extraction}\label{15:s3.2}

According to \citet{Boskovic2008,Boskovic2012}, languages with and without articles differ in a systematic way. Empirically, having the NP or the DP parameter setting set has shown to have consequences not only on the structure of the TNP, but on a number of different syntactic and semantic phenomena, as well. This has allowed for the investigation of numerous crosslinguistic differences and similarities on a structural level. \citet{Boskovic2008,Boskovic2012} presents a number of generalizations that group languages based on the presence of absence of definite articles. The one relevant for current purposes is given in \REF{15:ex1}:

% example 1
\ea\label{15:ex1}
  Only languages without articles may allow left \isi{branch extraction}.
\z

\noindent While I will only focus on the generalization in \REF{15:ex1}, I refer the reader to \citet{Boskovic2008,Boskovic2012} for a comprehensive list of generalization with discussions.

As stated, one of the tests used to capture the crosslinguistic asymmetry between DP and NP languages is \isi{LBE} of adjectives and adjective-like elements out of the TNP, with the generalization that \isi{LBE} may only be allowed in NP languages \citet{Boskovic2008,Boskovic2012}. Starting with the \ili{Slavic} language family, only \ili{Bulgarian} and \ili{Macedonian} disallow \isi{LBE}, and these are the only two languages that have (definite) articles. In \ili{Romance}, the only language that allows \isi{LBE} is \ili{Latin}, and this is also the only \ili{Romance} language that lacks articles. A very important example that contributes to the \isi{LBE} generalization is the case of \ili{Finnish}, discussed in \citet{Franks2007}. Namely, \ili{Finnish} is an article-less language and it allows \isi{LBE}. Interestingly, as articles started to develop in colloquial \ili{Finnish}, \isi{LBE} constructions immediately became very marginal and unacceptable. We see a similar case of variation among a single language in Ancient \ili{Greek}, where the languages belonging to two different periods pattern differently with respect to the presence of articles, and, therefore, to \isi{LBE} as well. Koine \ili{Greek} has articles and disallows \isi{LBE}, while \isi{LBE} was used productively in Homeric \ili{Greek} – which lacks articles. There are a few more languages that allow \isi{LBE}, and these are: Mohawk, Southern Tiwa, Gunwinjguan \citep{Baker1996}, \ili{Hindi}, Bangla, Angika, and Magahi. These are all article-less languages.\largerpage\footnote{\label{15:fn2}There is an additional requirement for a language to allow \isi{LBE} -- and this is \isi{agreement} between the \isi{noun} and the \isi{adjective}. This, in turn, answers the question of why \ili{Chinese}, that has very poor \isi{agreement} morphology, disallows \isi{LBE} even though it lacks articles. I will not be concerned with this requirement in this paper.}

Moving on to concrete examples, while \isi{LBE} is disallowed in \ili{English} (a DP language) \REF{15:ex2}, and in \ili{Spanish} (a DP language) \REF{15:ex3}, it is allowed in \ili{Serbian} (an NP language) \REF{15:ex4}:\footnote{\label{15:fn3}Note that \isi{LBE} is not possible with non-agreeing adjectives in \ili{Serbian} (see \citealt{Boskovic2013}).}

% example 2
    \ea[*]{
   \glt \{Expensive$_1$ / Those$_1$\} he saw [\textsubscript{NP} t$_1$ cars]. \hfill \citep{Boskovic2008}
     }\label{15:ex2}
 \z


% example 3
\ea \label{15:ex3}
	\ea[*]{ \label{15:ex3a}
    \gll Supuestas$_1$ investigaba [\textsubscript{DP} t$_1$ estafas].\\
         alleged.\textsc{pl.f} used.to.investigate.\textsc{1sg} {} {} frauds.\textsc{pl.f}\\
         \glt `I used to investigate alleged frauds.'  \hfill (\ili{Spanish}, \citealt{Riqueros2013})
    }
	\ex[*]{ \label{15:ex3b}
    \gll Profesionales$_1$ ofrecía [\textsubscript{DP} traducciones t$_1$].\\
         professional.\textsc{pl.f} used.to.offer.\textsc{1pl} {}
         translations.\textsc{pl.f}\\ 
         \glt `I used to offer professional translations.' \hfill (\ili{Spanish}, \citealt{Riqueros2013})
         }
	\z
\z

% example 4
\ea \label{15:ex4}
\gll \{\hspace{-2pt} \textit{Skupa$_1$} / \textit{Ta$_1$}\} \textit{je} \textit{vidio} [\textsubscript{NP} t$_1$ \textit{kola}].\\
     {} expensive\textsc{.sg.f} {} that\textsc{.sg.f} be.\textsc{aux.3sg} seen.\textsc{sg.m} {} {} car.\textsc{sg.f}\\
\glt `He saw \{an expensive / that\} car.' \hfill (\citealt{Boskovic2008})
\z

\noindent As predicted, \ili{English} (a DP language) disallows, while \ili{Serbian} (an NP language), allows \isi{LBE}. To account for the contrast from above, \citet{Boskovic2013,Boskovic2014} proposes a contextual approach to phases in which the highest phrase in the extended domain of a lexical head acts as a phase. NP and DP languages then differ with respect to the phasal boundaries. Specifically, NP is a phase in NP languages, while DP is a phase in DP languages. Furthermore, assuming that the edge of each phase is visible to the next phase \citep{Chomsky2001}, i.e., it can be available for extraction and movement, the \isi{adjective} then occupies significantly different positions relative to the phasal edge in NP and DP languages. This is illustrated in \REF{15:ex5}, where the \isi{adjective} is at the edge the TNP phase in NP languages \REF{15:ex5a} and extraction of the \isi{adjective} is allowed, versus DP languages in \REF{15:ex5b}, where DP is the phase, and the \isi{adjective} is not at the edge of the TNP phase (the TNP being DP in this case). In order to be available for movement, the \isi{adjective} has to move to DP due to the Phrase Impenetrability Condition (PIC) \citep{Chomsky2001}, but the movement is blocked by antilocality, which requires the AP movement to cross a full phrase. In the case of \REF{15:ex5b}, AP does not cross a full phrase, only a segment.

\begin{multicols}{2}
% example 5
\ea \label{15:ex5}
	\ea \textbf{NP languages}\label{15:ex5a}\vspace{6pt}\\
    \hspace{-1.5cm}\begin{forest}for tree={inner sep=0, s sep=5mm}
      [NP
      	[AP, name=ap1]
      	[NP]
      ] {\draw (.west) node(target1)[left]{\hspace{3cm}\null};} \draw[->](ap1) to[out=south west,in=south] (target1);
\end{forest}
    \columnbreak
	\ex  \textbf{DP languages}\label{15:ex5b}\vspace{6pt}\\
    \hspace{-1.5cm}\begin{forest}for tree={inner sep=0, s sep=5mm}
  [DP
    [Spec, name=xp2]
    [D$'$
      [D$^0$, name=D]
      [NP
      	[AP, name=ap2]
      	[NP]
      ]
    ]
  ] {\draw (.west) node(target2a)[left of=D]{\hspace{1cm}\null};}
  {\draw (.west) node(target2b)[left]{\hspace{3cm}\null};}
  \draw[->](ap2) to[out=south west,in=south] node [midway, rotate=60] {\LARGE{$+$}} (xp2);
  \draw[->](xp2) to[out=south west,in=south](target2b);
\end{forest}
	\z
\z
\end{multicols}

\noindent When this is applied to \ili{Romanian} and \ili{Serbian}, the outcome is clear. \ili{Serbian} (NP) allows \isi{LBE} as in \REF{15:ex6}, and \ili{Romanian} (DP) disallows it, as in \REF{15:ex7}:

% example 6 (in doc example 4)
\ea \label{15:ex6}
	\ea[]{ \label{15:ex6a}
    \gll \textit{Vidio} \textit{je} \{\hspace{-2pt} \textit{skupa} / \textit{ta}\hspace{1pt}\} \textit{kola}.\\          
         seen\textsc{.sg.m} be\textsc{.aux.3sg} {} expensive\textsc{.sg.f} {} that\textsc{.sg.f} car\textsc{.sg.f}\\
         \glt `He saw \{an expensive / that\} car.'  \hfill (\citealt{Boskovic2008})
    }
	\ex[]{ \label{15:ex6b}
    \gll \{\hspace{-2pt} \textit{Skupa$_1$} / \textit{Ta$_1$}\hspace{1pt}\} \textit{je} \textit{vidio} [\textsubscript{NP} \textit{t$_1$} \textit{kola}].\\
         {} expensive\textsc{.sg.f} {} that\textsc{.sg.f} be\textsc{.aux.3sg} seen\textsc{.sg.m} {} {} car\textsc{.sg.f}\\ 
         \glt \glt `He saw \{an expensive / that\} car.' \hfill (\citealt{Boskovic2008})
         }
	\z
\z

% example 7 (in doc example 6)
\ea \label{15:ex7}
	\ea[]{ \label{15:ex7a}
    \gll \textbf{Am} \textbf{văzut} \{\hspace{-2pt} \textbf{scumpe} / \textbf{scumpe-le}\} \textbf{automobile}.\\          
         have.\textsc{aux.1sg} seen.\textsc{ptcp} {} expensive\textsc{.pl.f} {} expensive-the\textsc{.pl.f} cars\textsc{.pl.f}\\
         \glt `I saw \{expensive / the expensive\} cars.'  \hfill (\citealt{Petroj})
    }
	\ex[*]{ \label{15:ex7b}
    \gll \{\hspace{-2pt} \textbf{Scumpe$_1$} / \textbf{Scumpe-le$_1$}\} \textbf{am} \textbf{văzut}\hspace{1.7cm} [\textsubscript{DP} t$_1$ \textbf{automobile}].\\
         {} expensive\textsc{.pl.f} {} expensive-the\textsc{.pl.f} have.\textsc{aux.1sg} seen.\textsc{ptcp} {} {} cars.\textsc{pl.f}\\ 
         \glt Intended: `I saw \{expensive / the expensive\} cars.'  \hfill (\citealt{Petroj})
         }
	\z
\z

\noindent Structurally, this looks as follows: In \ili{Serbian}, the \isi{LBE} of adjectives (located in SpecNP) takes place through one movement out of the NP, as in \REF{15:ex8a}. In \ili{Romanian}, however, a more complex movement is required. First, in order for the \isi{adjective} to reach SpecDP, the AP (that has previously merged with D\textsuperscript{0} through Affix Hopping) has to proceed through SpecDP, which is the edge of the phase; only then would it be visible for further movement. The first movement, however, is blocked, by antilocality.\footnote{\label{15:fn4}There are accounts where \ili{Romanian} APs move to SpecDP (this is why they can precede the article, see \citealt{Abney1987,DobrovieSorin1993,Ungureanu2006}; a.o.). These accounts face a problem: if movement to SpecDP is possible, APs should be allowed to move out of DPs, too.} This is illustrated in \REF{15:ex8b}.\footnote{\label{15:fn5}For the complete analysis of \isi{definite article} being hosted by the \isi{noun} or the \isi{adjective}, I refer the reader to \citet{Petroj}.}

% \begin{multicols}{2}
% example 8 (in doc example 7)
\ea \label{15:ex8}
	\ea
    \textbf{\ili{Serbian} (NP language)}\vspace{6pt}\label{15:ex8a}\\
        \hspace{-2.4cm}\small\begin{forest}for tree={inner sep=0, s sep=5mm}
      [NP
      	[AP, name=ap8 [\textit{skupa}\\`expensive', roof first-line-width]]
      	[NP [\textit{kola}\\`car', roof]]
      ] {\draw (.west) node(target8a)[left]{\hspace{3.5cm}\null};} \draw[->](ap8) to[out=south west,in=south] (target8a);
\end{forest}

% \bigskip
% \columnbreak

\newpage 
	\ex
    \textbf{\ili{Romanian} (DP language)}\vspace{6pt}\label{15:ex8b}\\
         \hspace{-2.4cm}\small\begin{forest}for tree={inner sep=0, s sep=3mm}
  [DP, s sep=1.2cm
    [Spec, name=spec8b]
    [D$'$
      [D\textsuperscript{$0$} \\ \textbf{-le} \\`the'
      ]
      [NP
      	[AP,name=ap8b [\textbf{scumpe}\\`expensive', roof first-line-width]]
      	[NP [\textbf{automobile}\\`cars' , roof]]
      ]
    ]
  ] {\draw (.west) node(target8b)[left]{\hspace{3.5cm}\null};} \draw[->](spec8b) to[out=south west,in=south] (target8b);
  \draw[->](ap8b) to[out=south west,in=south,looseness=2] node [midway, rotate=60] {\LARGE{$+$}} (spec8b);
\end{forest}
    
	\z
\z

% \end{multicols}

\noindent While affairs are clear in \ili{Romanian} and \ili{Serbian} in isolation, the mixed parameter settings in \ili{Romanian}/\ili{Serbian} CS poses an important question with respect to which setting prevails in the relevant CS constructions; DP or NP. To address these issues, I will examine \isi{LBE} of adjectives in CS, starting with simple transitive constructions. However, before testing \isi{LBE}, the next section offers facts about elements participating in the CS TNP that are relevant in understanding the \isi{LBE} of adjectives in CS.

% SECTION 4
\section{Relevant code-switching background}\label{15:s4}

As mentioned, \ili{Romanian} and \ili{Serbian} differ with respect to the DP/NP parameter setting -- \ili{Romanian} being a DP (having articles) and \ili{Serbian} an NP language (lacking articles).

% example 9 (in doc example 8)
\ea \label{15:ex9}
	\ea\label{15:ex9a}
    \gll [\textsubscript{DP} \textbf{{}-ul} [\textsubscript{NP} \textbf{examen}]]\hspace{0.6cm} $\approx$\hspace{0.7cm} \textbf{examen-ul}\\          
         {} the\textsc{.sg.m} {} exam\textsc{.sg.m} {} exam\textsc{.sg.m}-the.\textsc{sg.m}\\
         \glt `the exam'
	\ex\label{15:ex9b}
    \gll [\textsubscript{NP} \textit{ispit}]\\
         {} exam\textsc{.sg.m}\\ 
         \glt `an/the exam'
         
	\z
\z

\noindent Following \citet{Boskovic2008,Boskovic2012} and the numerous generalizations that group languages according to the DP/NP parameter, \ili{Romanian} and \ili{Serbian} bring two clashing constructions and parameter settings interacting into combined structures. Although CS occurs on various levels (cf. \citealt{Petroj}), the relevant construction is represented in \REF{15:ex10}:

% example 10 (in doc example 9)
\ea \label{15:ex10}
\gll  \textit{teški} \textit{ispit}{}-\textbf{ul}\\
     difficult.\textsc{lf.sg.m} exam\textsc{.sg.m}-the\textsc{.sg.m}\\
     \glt `the difficult exam'
\z

\noindent In this construction, the elements that participate in CS are the \ili{Romanian} \isi{definite article} \textbf{-ul}, the \ili{Serbian} \isi{noun} \textit{ispit,} and the \ili{Serbian} \isi{adjective} \textit{teški}. The counterparts of \ili{Romanian} and \ili{Serbian} constructions are illustrated below in \REF{15:ex11a} and \REF{15:ex11b} respectively:

% example 11 (in doc example 10)
\ea \label{15:ex11}
	\ea\label{15:ex11a}
    \gll \textbf{greu-l} \textbf{examen}\\          
        difficult\textsc{.sg.m}-the\textsc{.sg.m} exam\textsc{.sg.m}\\
        \glt `the difficult exam'
	\ex\label{15:ex11b}
    \gll \textit{teški} \textit{ispit}\\
         difficult\textsc{.lf.sg.m} exam\textsc{.sg.m}\\ 
         \glt `the difficult exam'
	\z
\z

\noindent Being either an NP or a DP language has additional consequences. In this case, it means different ways in which a language can express definiteness. Specifically, while \ili{Romanian} expresses definiteness through definite articles on nouns \REF{15:ex12a} or adjectives \REF{15:ex12b}, \ili{Serbian} has an alternative way of obtaining definite versus indefinite interpretation. As illustrated in \tabref{15:t1}, \ili{Serbian} has two lexical forms for adjectives: short form (\textsc{sf}) and long form (\textsc{lf}). These two forms are considered by some authors \citep{Aljovic2002,Despic2011,Talic2014} to correspond to definite/specific \REF{15:ex13a} and indefinite/non-specific \REF{15:ex13b} interpretations, respectively.\footnote{\label{15:fn6}For current purposes, I will simplify matters a bit and will consider the long vs. short form contrast to impose a definite vs. indefinite NP interpretation, respectively. For relevant discussion, see \citet{Aljovic2002,Despic2011,Talic2014,Stankovic2015}; a.o.}

\begin{table}
    \centering
\begin{tabularx}{0.6\textwidth}{lXX}
  \lsptoprule
  & \textbf{Short form}&\textbf{Long form}\\\midrule
  Masculine & nòv & nòv-i\\
  Feminine & nóv-a & nòv-a:\\
  & new.\textsc{sf} & new.\textsc{lf}\\
  \lspbottomrule
\end{tabularx}
  \caption{Serbian shoft form vs. long form adjectives}\label{15:t1}
\end{table}


% example 12 (in doc example 11)
\ea \label{15:ex12}
	\ea\label{15:ex12a}
    \gll \textbf{examen-ul} \textbf{greu}\\          
        exam\textsc{.sg.m}-the\textsc{.sg.m} difficult\textsc{.sg.m}\\
	\glt `the difficult exam'
	\ex\label{15:ex12b}
    \gll \textbf{greu-l} \textbf{examen}\\
         difficult\textsc{.sg.m}-the\textsc{.sg.m} exam\textsc{.sg.m}\\ 
	\glt `the difficult exam'
	\z
\z

% example 13 (in doc example 12)
\ea \label{15:ex13}
	\ea\label{15:ex13a}
    \gll \textit{teški} \textit{ispit}\\          
        difficult\textsc{.lf.sg.m} exam\textsc{.sg.m}\\
        \glt `the difficult exam'
	\ex\label{15:ex13b}
    \gll \textit{težak} \textit{ispit}\\
         difficult\textsc{.sf.sg.m} exam\textsc{.sg.m}\\ 
         \glt `a difficult exam'
	\z
\z

\noindent What is most striking about the constructions like \REF{15:ex10} is the combination of elements that is not found in either of the participating languages.\footnote{\label{15:fn7}For a comprehensive analysis and account of the CS TNP and the interaction of \ili{Romanian} definite articles, \ili{Serbian} nouns, and \ili{Serbian} adjectives, I refer the reader to \citet{Petroj}.} In other words, the resulting structure is a combination of two definiteness-related elements -- a \ili{Romanian} \isi{definite article} and a \ili{Serbian} long-form (definiteness-imposing) \isi{adjective} -- in one TNP. Although coming from languages with different architectures, the elements form a cohesive and productive mixed structure. Given that both languages can express definiteness separately and that both definite elements are allowed in a single construction raises the question about the underlying structure of cases like \REF{15:ex10}. Specifically, does the resulting construction have the DP layer like in \ili{Romanian}, or is it an NP construction like in \ili{Serbian}?

Although having the \isi{definite article} in the structure should indicate the presence of the DP layer, the fact that CS represents a mixture of (in this case) two parameter settings does not necessarily point towards the dominance of either one of the participating languages. On the one hand, the presence of the \isi{definite article} may indicate that there is, in fact, a DP layer in \REF{15:ex10}, and that \textbf{-ul} is positioned in D$^0$. One the other, given that all three elements (D, N, and A) undergo \isi{agreement} in CS (\citealt{Petroj}), the definiteness may be licensed by the \ili{Serbian} long-form \isi{adjective}, and the DP layer may not exist.\footnote{\label{15:fn8}By \isi{agreement}, I refer to the forms that the \isi{adjective} and the article take relative to the \isi{gender} of the \isi{noun}.} One way to confirm that the DP layer indeed exists in this type of construction is by turning to the contextual approach to phases. Recall that this approach says that any phrase can be a phase, as long as it is the highest in its domain. As seen above, the edge of the phase is available for further actions, while the rest of the construction is frozen inside the phase. That being said, there are two possibilities regarding the status of the CS TNP: (i) if there is no DP and the highest phrase in the TNP domain is NP, the \isi{adjective} is in SpecNP and it should be extractable, allowing for the possibility of \isi{LBE}; (ii) if there is a DP layer \isi{present}, i.e. DP is a phase, the \isi{adjective} being in SpecNP would make it too deeply embedded for extraction (only SpecDP being visible as the edge of the phase); \isi{LBE}, in this case, will not be allowed.

To test this, the next session focuses on the \isi{LBE} from the CS TNP from internal and external arguments respectively.\largerpage

% SECTION 5
\section{Left branch extraction in code-switching}\label{15:s5}
\subsection{Left branch extraction in Romanian and Serbian}\label{15:s5.1}

As \isi{LBE} is a reliable test for identifying the DP/NP parameter setting of a natural language, the same test is applied to CS constructions that include structures like \REF{15:ex10}, repeated below as \REF{15:ex10x}.

\ea \label{15:ex10x}
\gll  \textit{teški} \textit{ispit}{}-\textbf{ul}\\
     difficult.\textsc{lf.sg.m} exam\textsc{.sg.m}-the\textsc{.sg.m}\\
     \glt `the difficult exam'
\z

\noindent Recall that as predicted by the generalizations in \citet{Boskovic2008}, \ili{Romanian}, being a DP language, disallows \isi{LBE} and \ili{Serbian}, an NP language, allows it. This is illustrated in \REF{15:ex6} for \ili{Serbian} and in \REF{15:ex7} for \ili{Romanian}, repeated below as \REF{15:ex13x} and \REF{15:ex14x}, respectively:


\ea \label{15:ex13x}
	\ea[]{ \label{15:ex13xa}
    \gll \textit{Vidio} \textit{je} \{\hspace{-2pt} \textit{skupa} / \textit{ta}\hspace{1pt}\} \textit{kola}.\\          
         seen\textsc{.sg.m} be\textsc{.aux.3sg} {} expensive\textsc{.sg.f} {} that\textsc{.sg.f} car\textsc{.sg.f}\\
         \glt `He saw \{an expensive / that\} car.'  \hfill (\citealt{Boskovic2008})
    }
	\ex[]{ \label{15:ex13xb}
    \gll \{\hspace{-2pt} \textit{Skupa$_1$} / \textit{Ta$_1$}\hspace{1pt}\} \textit{je} \textit{vidio} [\textsubscript{NP} \textit{t$_1$} \textit{kola}].\\
         {} expensive\textsc{.sg.f} {} that\textsc{.sg.f} be\textsc{.aux.3sg} seen\textsc{.sg.m} {} {} car\textsc{.sg.f}\\ 
         \glt \glt `He saw \{an expensive / that\} car.' \hfill (\citealt{Boskovic2008})
         }
	\z
\z

\ea \label{15:ex14x}
	\ea[]{ \label{15:ex14xa}
    \gll \textbf{Am} \textbf{văzut} \{\hspace{-2pt} \textbf{scumpe} / \textbf{scumpe-le}\} \textbf{automobile}.\\          
         have.\textsc{aux.1sg} seen.\textsc{ptcp} {} expensive\textsc{.pl.f} {} expensive-the\textsc{.pl.f} cars\textsc{.pl.f}\\
         \glt `I saw \{expensive / the expensive\} cars.'  \hfill (\citealt{Petroj})
    }
	\ex[*]{ \label{15:ex14xb}
    \gll \{\hspace{-2pt} \textbf{Scumpe$_1$} / \textbf{Scumpe-le$_1$}\} \textbf{am} \textbf{văzut} [\textsubscript{DP} t$_1$ \textbf{automobile}].\\
         {} expensive\textsc{.pl.f} {} expensive-the\textsc{.pl.f} have.\textsc{aux.1sg} seen.\textsc{ptcp} {} {} cars.\textsc{pl.f}\\ 
         \glt Intended: `I saw \{expensive / the expensive\} cars.'  \hfill (\citealt{Petroj})
         }
	\z
\z

\noindent As seen above, facts are clear for \ili{Romanian} and \ili{Serbian} in isolation. In the remainder of this section, \isi{LBE} of adjectives will be applied to CS TNPs from transitive constructions and from the subject.

\subsection{Transitive constructions}\label{15:s5.2}

The paradigm below starts with \REF{15:ex16}, in which CS occurs within a TNP where the verb is \ili{Romanian}, the \isi{definite article} is \ili{Romanian}, and the \isi{noun} and the \isi{adjective} are \ili{Serbian}. As illustrated in \REF{15:ex16b}, \isi{LBE} out of this TNP is disallowed. In \REF{15:ex17}, the verb is still \ili{Romanian}, but even a fully \ili{Serbian} TNP fails the \isi{LBE} test. Interestingly, when the \ili{Romanian} verb is replaced by its \ili{Serbian} counterpart in \REF{15:ex18}, \isi{LBE} improves drastically. Interestingly, while the \ili{Serbian} verb \textit{can} take a DP complement in \REF{15:ex19a}, extraction of the \isi{adjective} is blocked in \REF{15:ex19b}, confirming that \textbf{-ul} may indeed point towards the existence of the DP layer.\footnote{\label{15:fn9}I would like to thank an anonymous reviewer for noticing the incomplete paradigm and pointing out the relevance of the example in \REF{15:ex18}.}

% example 16 (in doc example 15)
\ea \label{15:ex16}
	\ea[]{ \label{15:ex16a}
    \gll \textbf{Am} \textbf{trecut} \textit{teški} \textit{ispit-}\textbf{ul}.\\          
         have\textsc{.aux.1sg} passed\textsc{.ptcp} difficult\textsc{.lf.sg.m} exam\textsc{.sg.m}-the\textsc{.sg.m}\\
         \glt `I passed the difficult exam.'
    }
	\ex[*]{ \label{15:ex16b}
    \gll \textit{Teški$_1$} \textbf{am} \textbf{trecut} [t$_1$ \textit{ispit}\textbf{-ul}].\\
         difficult\textsc{.lf.sg.m} have\textsc{.aux.1sg} passed\textsc{.ptcp} {} exam\textsc{.sg.m}-the\textsc{.sg.m}\\ 
         \glt `I passed the difficult exam.'
         }
	\z
\z

% example 17 (in doc example 16)
\ea \label{15:ex17}
	\ea[]{ \label{15:ex17a}
    \gll \textbf{Am} \textbf{trecut} \textit{teški} \textit{ispit}.\\          
         have\textsc{.aux.1sg} passed\textsc{.ptcp} difficult\textsc{.lf.sg.m} exam\textsc{.sg.m}\\
         \glt `I passed the difficult exam.'
    }
	\ex[*]{ \label{15:ex17b}
    \gll \textit{Teški$_1$} \textbf{am} \textbf{trecut} [t$_1$ \textit{ispit}]\\
         difficult\textsc{.lf.sg.m} have\textsc{.aux.1sg} passed\textsc{.ptcp} {} exam\textsc{.sg.m}\\ 
         \glt Intended: `I passed the difficult exam.'
         }
	\z
\z

% example 18 (in doc example 17)
\ea \label{15:ex18}
	\ea[]{ \label{15:ex18a}
    \gll \textbf{Am} \textit{položila} \textit{teški} \textit{ispit}.\\          
         have\textsc{.aux.1sg} passed\textsc{.sg.f} difficult\textsc{.lf.sg.m} exam\textsc{.sg.m}\\
         \glt `I passed the difficult exam.'
    }
	\ex[?]{ \label{15:ex18b}
    \gll \textit{Teški$_1$} \textbf{am} \textit{položila} [t$_1$ \textit{ispit}].\\
         difficult\textsc{.lf.sg.m} have\textsc{.aux.1sg} passed\textsc{.sg.f} {} exam\textsc{.sg.m}\\ 
         \glt `I passed the difficult exam.'
         }
	\z
\z

% example 19 (in doc example 18)
\ea \label{15:ex19}
	\ea[]{ \label{15:ex19a}
    \gll \textbf{Am} \textit{položila} \textit{teški} \textit{ispit}\textbf{-ul}.\\          
         have\textsc{.aux.1sg} passed\textsc{.sg.f} difficult\textsc{.lf.sg.m} exam\textsc{.sg.m}-the\textsc{.sg.m}\\
         \glt `I passed the difficult exam.'
    }
	\ex[*]{ \label{15:ex19b}
    \gll \textit{Teški$_1$} \textbf{am} \textit{položila} [t$_1$ \textit{ispit}\textbf{-ul}].\\
         difficult\textsc{.lf.sg.m} have\textsc{.aux.1sg} passed\textsc{.sg.f} {} exam\textsc{.sg.m}-the\textsc{.sg.m}\\ 
         \glt Intended: `I passed the difficult exam.'
         }
	\z
\z

\noindent Based on the above discussion, I take (dis)allowing \isi{LBE} to indicate the presence or absence of the DP layer. The ungrammaticality of \REF{15:ex17b} and \REF{15:ex19b} then indicates that any \ili{Romanian} element in the VP domain forces DP-hood on the object. What is particularly interesting here is that although the entire TNP is in \ili{Serbian}, \isi{LBE} still cannot take place. This suggests that although no \ili{Romanian} D element is \isi{present} overtly, there is still a DP projection here, which is not the case in \REF{15:ex18}, where \isi{LBE} improves drastically with a \ili{Serbian} verb introduced in the structure. Additionally, the paradigm in \REF{15:ex16}--\REF{15:ex19} confirms that regardless of the verb being \ili{Romanian} or \ili{Serbian}, the presence of a \ili{Romanian} element in the object position will always have the DP layer.

Given that both \ili{Romanian} and \ili{Serbian} verbs can occur and take either a \ili{Romanian} or a \ili{Serbian} complement in CS, data from above indicates that \ili{Romanian} verbs must take a DP complement even in CS as in \REF{15:ex20a}, while a \ili{Serbian} verb can take either an NP complement as in \REF{15:ex18b}, or a DP complement, as in \REF{15:ex20b}.

% example 20 (in doc example 19)
\ea \label{15:ex20}
	\ea\label{15:ex20a}
    \gll \textbf{Am} \textbf{trecut} \{\hspace{-2pt} \textbf{examen-ul} / \textit{ispit}\textbf{-ul} /\hspace{0.1cm} *\hspace{-2pt} \textit{ispit}\hspace{1pt}\}.\\          
         have\textsc{.aux.1sg} passed\textsc{.ptcp} {} exam\textsc{sg.m}-the\textsc{.sg.m} {} exam\textsc{sg.m}-the\textsc{.sg.m} {} {} exam\textsc{.sg.m}\\
         \glt (Intended:) `I passed \{the exam / the exam / an/the exam\}.'
	\ex\label{15:ex20b}
    \gll \textbf{Am} \textit{položila} \{\hspace{-2pt} \textbf{examen-ul} / \textit{ispit}\textbf{-ul} / \textit{ispit}\hspace{1pt}\}.\\
         have\textsc{.aux.1sg} passed\textsc{.sg.f} {} exam\textsc{sg.m}-the\textsc{.sg.m} {} exam\textsc{sg.m}-the\textsc{.sg.m} {} exam\textsc{.sg.m}\\
         \glt `I passed \{the exam / the exam / an/the exam\}.'
	\z
\z

\noindent We then have the generalization in \REF{15:ex21}:\footnote{\label{15:fn10}The pattern of certain elements allowing DP or NP arguments seems to extend beyond the VP domain, specifically, with respect to CS of conjuncts and coordinated structures. I refer the reader to \citet{Petroj} for more examples and more detailed explanation.}

% example 21 (in doc example 20)
\ea\label{15:ex21}
  \ili{Romanian} verbs must take a DP complement, while \ili{Serbian} verbs can take either a DP or an NP complement.
\z

\noindent I will now test the \isi{LBE} of adjectives out of a \isi{ditransitive} construction. Examples in \REF{15:ex22} and \REF{15:ex24} represent fully \ili{Serbian} sentences with the \isi{LBE} of the possessor out of the \isi{indirect object} (\isi{IO}) in \REF{15:ex22b} and \isi{direct object} (\isi{DO}) in \REF{15:ex24b}. As expected, \ili{Serbian} being an NP language, \isi{LBE} is allowed in both cases. In contrast, when a \ili{Romanian} object is introduced into the structure in \REF{15:ex23} and \REF{15:ex25}, \isi{LBE} out of the \ili{Serbian} object in \REF{15:ex23b} and \REF{15:ex25b} leads to ungrammaticality.\footnote{\label{15:fn11}\textbf{Pe} in \REF{15:ex23} is a dummy preposition assigning the \isi{accusative} to its complement. It is comparable to the \ili{Spanish} \textit{a,} illustrated in \REF{15:fn11ex}.

\ea \gll Lo vimos a Juan.\label{15:fn11ex}\\
him\textsc{.cl.acc.m} saw\textsc{.1pl} \textsc{a} Juan\\
\glt `We saw John.'\hfill(\ili{Spanish}; \citealt{Jaeggli1986})
\z }

% example 22 (in doc example 21)
\ea \label{15:ex22}
	\ea[]{ \label{15:ex22a}
    \gll \textit{Moja} \textit{drugarica} \textit{predstavlja} \textit{svom} \textit{prijatelju} \textit{Jovana}.\\          
         my\textsc{.nom} friend\textsc{.nom} introduce\textsc{.3sg} her\textsc{.poss.refl.dat} friend\textsc{.dat} Jovan\textsc{acc}\\
         \glt `My friend introduces Jovan to her friend.'
    }
	\ex[]{ \label{15:ex22b}
    \gll \textit{Svom$_1$} \textit{moja} \textit{drugarica} \textit{predstavlja}\hspace{2.4cm} [\textsubscript{NP} t$_1$ \textit{prijatelju}] [\textsubscript{NP} \textit{Jovana}].\\
         her\textsc{.poss.refl.dat} my\textsc{.nom} friend\textsc{.nom} introduce\textsc{.3sg} {} {} friend\textsc{.dat} {} Jovan\textsc{.acc}\\ 
         \glt `My friend introduces Jovan to her friend.'
         }
	\z
\z

% example 23 (in doc example 22)
\ea \label{15:ex23}
	\ea[]{ \label{15:ex23a}
    \gll \textit{Moja} \textit{drugarica} \textit{predstavlja} \textit{svom} \textit{prijatelju}\hspace{0.5cm} \textbf{pe} \textbf{Jovan}.\\          
         my\textsc{.nom} friend\textsc{.nom} introduce\textsc{.3sg} her\textsc{.poss.refl.dat} friend\textsc{.dat} \textsc{pe} Jovan\\
         \glt `My friend introduces Jovan to her friend.'
    }
	\ex[*]{ \label{15:ex23b}
    \gll \textit{Svom}$_1$ \textit{moja} \textit{drugarica} \textit{predstavlja}\hspace{2.4cm} [\textsubscript{NP} t$_1$ \textit{prijatelju}] [\textsubscript{DP} \textbf{pe} \textbf{Jovan}].\\
         her\textsc{.poss.refl.dat} my\textsc{.nom} friend\textsc{.nom} introduce\textsc{.3sg} {} {} friend\textsc{.dat} {} \textsc{pe} Jovan\\ 
         \glt Intended: `My friend introduces Jovan to her friend.'
         }
	\z
\z

% example 24 (in doc example 23)
\ea \label{15:ex24}
	\ea[]{ \label{15:ex24a}
    \gll \textit{Moja} \textit{drugarica} \textit{šalje} \textit{svoju} \textit{knjigu} \textit{mom} \textit{bratu}.\\          
         my\textsc{.nom} friend\textsc{.nom} send\textsc{.3sg} her\textsc{.poss.refl.acc} book\textsc{.acc} my\textsc{.dat} brother\textsc{.dat}\\
         \glt `My friend sends her book to my brother.'
    }
	\ex[]{ \label{15:ex24b}
    \gll \textit{Svoju}$_1$ \textit{moja} \textit{drugarica} \textit{šalje} [\textsubscript{NP} t$_1$ \textit{knjigu}]\hspace{1.5cm} [\textsubscript{NP} \textit{mom} \textit{bratu}].\\
         her\textsc{.poss.refl} my\textsc{.nom} friend\textsc{.nom} send\textsc{.3sg} {} {} book\textsc{.acc} {} my\textsc{.dat} brother\textsc{.dat}\\ 
         \glt `My friend sends her book to my brother.'
         }
	\z
\z

% example 25 (in doc example 24)
\ea \label{15:ex25}
	\ea[]{ \label{15:ex25a}
    \gll \textit{Moja} \textit{drugarica} \textit{šalje} \textit{svoju} \textit{knjigu} \textbf{fratelui} \textbf{meu}.\\          
         my\textsc{.nom} friend\textsc{.nom} send\textsc{.3sg} her\textsc{poss.refl.acc} book\textsc{.acc} brother\textsc{.dat} my\\
         \glt `My friend sends her book to my brother.'
    }
	\ex[*]{ \label{15:ex25b}
    \gll \textit{Svoju$_1$} \textit{moja} \textit{drugarica} \textit{šalje} [\textsubscript{NP} $_1$ \textit{knjigu}]\hspace{1.1cm} [\textsubscript{DP} \textbf{fratelui} \textbf{meu}].\\
        her\textsc{.poss.refl.acc} my\textsc{.nom} friend\textsc{.nom} send\textsc{.3sg} {} {} book\textsc{.acc} {} brother\textsc{.dat} my\\ 
         \glt Intended: `My friend sends her book to my brother.'
         }
	\z
\z

\noindent \REF{15:ex23} and \REF{15:ex25} show that when one object is in \ili{Romanian} and the other in \ili{Serbian}, \isi{LBE} is not allowed even when the \isi{LBE} is attempted out of the TNP that contains \ili{Serbian} elements only. This is especially interesting since \isi{LBE} was allowed once a \ili{Serbian} verb was introduced into the structure in \REF{15:ex18}. \REF{15:ex23} and \REF{15:ex25} indicate that any \ili{Romanian} element (not just the verb) in the \textit{v}P/VP domain blocks \isi{LBE}. With respect to the DP/NP status, it seems like both objects are DPs when one object is in \ili{Romanian}. These examples then indicate that no structural mixing regarding the categorical status is allowed between the objects in a \isi{double object} constructions (where one object would be an NP and one object a DP); if one object is a DP, both must be DPs. Consequently, if \textit{v}P is considered a phase, the following generalization can be made:\footnote{\label{15:fn12}An anonymous reviewer pointed out an interesting question about the generalization in \REF{15:ex26}, namely, that having a \ili{Romanian} low/VP-adjunct after a \ili{Serbian} \isi{ditransitive} construction like the one in \REF{15:fn12ex} might reveal additional (counter)evidence for the structure of mixing within the \isi{spell-out domain}. While the sentence in \REF{15:fn12exb} is only marginally acceptable, the subjects reported challenges in \isi{processing} the sentence, rather than in grammaticality, which can be assigned to prosodic factors of a fully \ili{Serbian} \isi{LBE} construction. I leave CS of adjuncts for \isi{future} research.

\ea\label{15:fn12ex}\ea[]{\gll \textit{Moja} \textit{drugarica} \textit{šalje} \textit{svoju} \textit{knjigu} \textit{mom} \textit{bratu} \textbf{cu} \textbf{avion-ul}.\\
my\textsc{.nom} friend\textsc{.nom} send\textsc{.3sg} her\textsc{.poss.refl.acc} book\textsc{.acc} my\textsc{.dat} brother\textsc{.dat} with plane\textsc{.sg.m}-the\textsc{.sg.m}\\
\glt ‘My friend sends her book to my brother via plane.’}
\ex[?]{\gll \textit{Svoju}$_1$ \textit{moja} \textit{drugarica} \textit{šalje} [\textsubscript{NP} t$_1$ \textit{knjigu}] [\textsubscript{NP} \textit{mom} \textit{bratu}] \textbf{cu} \textbf{avion-ul}.\\
her\textsc{poss.refl.acc} my\textsc{.nom} friend\textsc{.nom} send\textsc{.3sg} {} {} book\textsc{.acc} {} my\textsc{.dat} brother\textsc{.dat} with plane\textsc{.sg.m}-the\textsc{.sg.m}\\
\glt ‘My friend sends her book to my brother via plane.’\label{15:fn12exb}}
\z\z

}

% example 26 (in doc example 25)
\ea\label{15:ex26}
  No mixing of the categorical status of the TNP within a \isi{spell-out domain}, where the \isi{spell-out domain} is a phasal complement.
\z

\subsection{Subject}\label{15:s5.3}

Given that having a \ili{Romanian} element in either \isi{IO} or \isi{DO} blocks \isi{LBE} from the other object (even when the other object is entirely in \ili{Serbian}) it is important to test the extent of influence of the \ili{Romanian} DP on the rest of the structure.

In the examples below, \REF{15:ex27} represents a fully-\ili{Serbian} example, with the possessor being extracted from the subject in \REF{15:ex27b}. This being a fully \ili{Serbian} construction, \isi{LBE} is allowed.

% example 27 (in doc example 26)
\ea \label{15:ex27}
	\ea\label{15:ex27a}
    \gll \textit{Tvrdiš} \textit{da} \textit{moja} \textit{drugarica} \textit{predstavlja} \textit{Petru} \textit{Jovana}.\\          
         claim\textsc{.2sg} that my\textsc{.nom} friend\textsc{.nom} introduce\textsc{.3sg} Petar\textsc{.dat} Jovan\textsc{.acc}\\
         \glt `You claim that my friend introduces Jovan to Petar.'
	\ex\label{15:ex27b}
    \gll \textit{Moja}$_1$ \textit{tvrdiš} \textit{da} [\textsubscript{NP} t$_1$ \textit{drugarica}] \textit{predstavlja} [\textsubscript{NP} \textit{Petru}] [\textsubscript{NP} \textit{Jovana}].\\
         my\textsc{.nom} claim\textsc{.2sg} that {} {} friend\textsc{.nom} introduce\textsc{.3sg} {} Petar\textsc{.dat} {} Jovan\textsc{.acc}\\ 
         \glt `You claim that my friend introduces Jovan to Petar.'
	\z
\z

\noindent Interestingly, when a \ili{Romanian} element is introduced as the \isi{DO} in \REF{15:ex28} and as the \isi{IO} in \REF{15:ex30}, \isi{LBE} out of a fully-\ili{Serbian} Subject is permitted in both cases, as in \REF{15:ex28b} and \REF{15:ex30b}.

% example 28 (in doc example 27)
\ea \label{15:ex28}
	\ea\label{15:ex28a}
    \gll \textit{Tvrdiš} \textit{da} \textit{moja} \textit{drugarica} \textit{predstavlja} \textit{Petru} \textbf{pe} \textbf{Jovan}.\\          
         claim\textsc{.2sg} that my\textsc{.nom} friend\textsc{.nom} introduce\textsc{.3sg} Petar\textsc{.dat} \textsc{pe} Jovan\\
         \glt `You claim that my friend introduces Jovan to her friend.'
	\ex\label{15:ex28b}
    \gll \textit{Moja}$_1$ \textit{tvrdiš} \textit{da} [\textsubscript{NP} t$_1$ \textit{drugarica}] \textit{predstavlja} [\textsubscript{NP} \textit{Petru}] [\textsubscript{DP} \textbf{pe} \textbf{Jovan}]\\
         my\textsc{.nom} claim\textsc{.2sg} that {} {} friend\textsc{.nom} introduce\textsc{.3sg} {} Petar\textsc{.dat} {} \textsc{pe} Jovan\\ 
                  \glt `You claim that my friend introduces Jovan to her friend.'
	\z
\z

% % example 29 (in doc example 28)
% \ea \label{15:ex29}
% 	\ea\label{15:ex29a}
%     \gll \textit{Tvrdiš} \textit{da} \textit{moja} \textit{drugarica} \textit{šalje} \textit{svoju} \textit{knjigu} \textit{mom} \textit{bratu}\\          
%          claim-\textsc{2sg} that my\textsc{{}-f} friend-\textsc{f} send-\textsc{3sg} her-\textsc{refl.f} book-\textsc{f} my-\textsc{m.dat} brother-\textsc{m.dat}\\
%          \glt `You claim that my friend sends her book to my brother.'
% 	\ex\label{15:ex29b}
%     \gll \textit{Moja} \textit{tvrdiš} \textit{da}[\textsubscript{NP} \textit{ti\textsubscript{} }\textit{drugarica}] \textit{šalje} [\textsubscript{NP} \textit{svoju} \textit{knjigu}] [\textsubscript{NP} \textit{mom} \textit{bratu}]\\
%          my\textsc{{}-f} claim-\textsc{2sg} that friend-\textsc{f} send-\textsc{3sg} her-\textsc{refl.f} book-\textsc{f} my-\textsc{m.dat} brother-\textsc{m.dat}\\ 
%          \glt `My, you claim that friend sends her book to my brother.'
% 	\z
% \z

% example 30 (in doc example 29)
\ea \label{15:ex30}
	\ea\label{15:ex30a}
    \gll \textit{Tvrdiš} \textit{da} \textit{moja} \textit{drugarica} \textit{šalje} \textit{svoju} \textit{knjigu} \textbf{fratelui} \textbf{meu}.\\          
         claim\textsc{.2sg} that my\textsc{.nom} friend\textsc{.nom} send\textsc{.3sg} her\textsc{.poss.refl.acc} book\textsc{.acc} brother\textsc{.dat} my\\
         \glt `You claim that my friend sends her book to my brother.'
	\ex\label{15:ex30b}
	 \gll \textit{Moja}$_1$ \textit{tvrdiš} \textit{da} [\textsubscript{NP} t$_1$ \textit{drugarica}] \textit{šalje}\hspace{3.5cm} [\textsubscript{NP} \textit{svoju} \textit{knjigu}] [\textsubscript{DP} \textbf{fratelui} \textbf{meu}].\\
         my\textsc{.nom} claim\textsc{.2sg} that {} {} friend\textsc{.nom} send\textsc{.3sg} {} her\textsc{.poss.refl.acc} book\textsc{.acc} {} brother\textsc{.dat} my\\ 
         \glt `You claim that my friend sends her book to my brother.'
	\z
\z

\noindent These data contrast with \REF{15:ex23} and \REF{15:ex25} where the introduction of a \ili{Romanian} \isi{internal argument} blocked \isi{LBE} out of the other \isi{internal argument}. In contrast, \isi{LBE} out of the subject is not affected by CS in the internal arguments of the verb. Based on these examples, the following generalizations can be made:

% example 31 (in doc example 30)
\ea\label{15:ex31}
  A \ili{Romanian} internal DP argument forces DP-hood to the \isi{internal argument} of the verb, but not to the external one.
\z

% example 32 (in doc example 31)
\ea\label{15:ex32}
  No mixing of the categorical status of the TNP within a \isi{spell-out domain}, where the \isi{spell-out domain} is a phasal complement.
\z

\noindent Notice also that a \ili{Romanian} external DP argument does not force DP-hood on a \ili{Serbian} \isi{internal argument}, as indicated by the possibility of \isi{LBE} in \REF{15:ex33}:

% example 33 (in doc example 32)
\ea \label{15:ex33}
	\ea[]{ \label{15:ex33a}
    \gll \textbf{Elev-ul} \textbf{a} \textit{položio} \textit{teški} \textit{ispit}.\\          
         student\textsc{.sg.m}-the\textsc{.sg.m} have\textsc{.aux.3sg} passed\textsc{.sg.m} difficult\textsc{.lf.sg.m} exam\textsc{.sg.m}\\
         \glt `The student passed the difficult exam.'
    }
	\ex[?]{ \label{15:ex33b}
    \gll \textit{Teški}$_1$ \textbf{elev-ul} \textbf{a} \textit{položio}\hspace{1cm} [\textsubscript{NP} t$_1$ \textit{ispit}].\\
         difficult\textsc{lf.sg.m} student\textsc{.sg.m}-the\textsc{.sg.m} have\textsc{.aux.3sg} passed\textsc{.sg.m} {} {} exam\textsc{.sg.m}\\ 
         \glt `The student passed the difficult exam.'
         }
	\z
\z

% SECTION 6
\section{Conclusions and further research}\label{15:s6}

Due to the DP/NP difference between \ili{Romanian} and \ili{Serbian}, \isi{LBE} has proven reliable in determining the points where CS may occur, but also in showing that mixing two languages may not necessarily result in a homogenous DP or NP system. In other words, this variant of CS shows flexibility when it comes to elements that are switched, but also regarding what parameter setting will prevail depending on when CS occurs in the derivation. When it comes to the interaction between \ili{Romanian} and \ili{Serbian} elements, the following generalizations hold:

\begin{enumerate}
    \item \ili{Romanian} verbs must take a DP complement, while \ili{Serbian} verbs can take either a DP or NP complements.
    \item A \ili{Romanian} internal DP argument forces DP-hood onto the \isi{internal argument} of the verb, but not onto the external one.
\end{enumerate}

Importantly, \isi{LBE} has also shed light on the flexibility of the CS construction to navigate through parameters.

\begin{enumerate}
    \item[3.] No mixing of the categorical status of the TNP is allowed within a \isi{spell-out domain}, where the \isi{spell-out domain} is a phasal complement.
\end{enumerate}


We can assume then that the \textit{v}p/VP \isi{spell-out domain} may look something like \REF{15:ex34}, whereby CS below the \textit{v}P-level affects the entire phasal domain, but not the area above it:

% example 34 (in doc example 33)
\ea\label{15:ex34}
\begin{forest}for tree={s sep=1cm, inner sep=0}
  [\textit{v}P
    [S]
    [\textit{v}$'$, name=VP
      [\textit{v}$^0$]
      [VP
      	[\isi{IO}, name=IO
        ]
      	[V$'$
      	    [V$^0$]
      	    [\isi{DO}]
      	]
      ]
    ]
  ]\node[right=of VP] (boundary) {\null};
  \node[left=of IO] (ioleft) {\hspace{1cm}\null};
\draw(ioleft.south east) to[bend left=20](boundary.south east) ;
\end{forest}
\z

\noindent Finally, more research needs to be done to correctly predict the points of CS in other langauges with different spell-out domains/phasal boundaries in order to unravel the rules and constraints, and identify the exact points of CS.


%%%%%%%%%%%%%%%%%%%%%%%%%%%%%%%%%%%%%%%%%%%%%%%%%%%%%%%%%%%%%%%%%%%%%%%%%%%%%%%%%%%%%%%


\section*{Abbreviations}

\begin{tabularx}{.5\textwidth}{@{}lQ@{}}
\textsc{acc}&{accusative}\\
\textsc{aux}&auxiliary\\
\textsc{cl}&{clitic}\\
\textsc{dat}&{dative}\\
\textsc{f}&{feminine}\\
\textsc{lf}&long form\\
\textsc{m}&{masculine}\\
\end{tabularx}%
\begin{tabularx}{.5\textwidth}{@{}lQ@{}}
\textsc{nom}&{nominative}\\
\textsc{ptcp}&{participle}\\
\textsc{pl}&{plural}\\
\textsc{poss}&possessive\\
\textsc{refl}&reflexive\\
\textsc{sg}&singular\\
\textsc{sf}&short form\\
\end{tabularx}

\section*{Acknowledgements}

For helpful comments, I am thankful to Željko Bošković, as well as the audience and the three anonymous reviewers of FDSL 12. I am also grateful to Neda Todorović for constructive conversations. For judgements, I would like to thank Teodora Fizešan, Daniel Neda, Xenia Oalge, and Kristina Georgijev.

\sloppy
\printbibliography[heading=subbibliography,notkeyword=this]

\end{document}
