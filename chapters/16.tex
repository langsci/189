\documentclass[output=paper,
modfonts,
newtxmath,
hidelinks
]{langscibook} 


% \papernote{\footnotesize\normalfont
% Zorica Puškar \& Gereon Müller. Unifying structural and lexical case assignment in Dependent Case Theory. To appear in: Denisa Lenertová, Roland Meyer, Radek Šimík \& Luka Szucsich (eds.), \textit{Advances in formal Slavic linguistics 2016}. Berlin: Language Science Press. [preliminary page numbering]
% }

%\setcounter{chapter}{15}

\title{Unifying structural and lexical case assignment in Dependent Case Theory}  

\author{%
 Zorica Puškar\affiliation{Leibniz-Zentrum Allgemeine Sprachwissenschaft, Berlin}\lastand 
 Gereon Müller\affiliation{University of Leipzig}
}

% \chapterDOI{} %will be filled in at production
% \epigram{}

\abstract{
Dependent Case Theory argues against case assignment via a functional head (cf. \citealt{Chomsky2000,chomsky01}) and proposes instead that case is a result of a structural relation between two DPs \citep{marantz91,mcfadden04,bakervinokurova,baker15}. However, Dependent Case Theory cannot completely abandon case assignment via a syntactic head, as this mechanism accounts for lexical case (e.g. lexical dative). Furthermore, structural and lexical datives are morphologically identical and often behave similarly, and `just where the line should be drawn between the two is a theoretical matter' \citep[13]{baker15}. We argue for a unified approach to lexical and structural dative case assignment under Dependent Case Theory, implemented in a derivational fashion, via the operation Agree. While structural \datt{} is assigned as a high dependent case in the VP in the presence of a lower (later \accc{}) DP, lexical \datt{} is assigned in the same configuration, in the VP, in the presence of another silent or overt co-argument DP.

\keywords{dependent case, Agree, dative}

}

\begin{document}
\maketitle
\shorttitlerunninghead{Unifying structural and lexical case assignment in Dependent Case Theory}

\section{Introduction: Dependent Case Theory} 


The Dependent Case Theory (henceforth DCT) is a result of the work of \citep{marantz91,mcfadden04,bakervinokurova,baker12,baker15}, among others, adopting similar ideas by \citet{yipetal87,bittnerhale96,kiparsky92,kiparsky2001,wunderlich97,stiebels2002}. Case assignment in DCT relies primarily on \citeauthor{marantz91}'s (\citeyear[24]{marantz91}) disjunctive case hierarchy, which distinguishes between the following types of case:

\ea Lexically governed case \before{} Dependent case (accusative and ergative) \before{} Unmarked case (nominative and absolutive) \before{} Default case 
\z

\noindent There are several steps in the case assigning process. In Step 1 all DPs selected by lexical items (verbs, prepositions, etc.) which idiosyncratically assign a particular case, receive the \textsc{lexically governed case} value from the designated head upon c-selection. In Step 2, pairs of remaining caseless DPs are inspected in their local domains. \textsc{Dependent case} is assigned to them according to (a variation of) the following case assignment rules: 

\ea\label{16:ex2} \textbf{Rules for dependent case assignment} \citep[48-49]{baker15}
\ea\label{16:ex2a} If there are two distinct DPs in the same spell out domain such that DP$_1$ c-commands DP$_2$, then value the case feature of DP$_2$ as accusative unless DP$_1$ has already been marked for case \REF{16:ex3}. 
\ex\label{16:ex2b} If there are two distinct DPs in the same spell out domain such that DP$_1$ c-commands DP$_2$, then value the case feature of DP$_1$ as ergative unless DP$_2$ has already been marked for case \REF{16:ex4}. 
\z \z

\noindent These rules lead to a four-way typology of case alignments \citep{levinpreminger}: The application of only the rule \REF{16:ex2a} will lead to nominative-accusative alignment \REF{16:ex3}, while \REF{16:ex2b} will yield ergative-absolutive alignment \REF{16:ex4}. If both parameters are simultaneously present in the same language, this would yield tripartite case systems (e.g. Nez Perce, where accusative and ergative can co-occur, see \citealt{baker15}) and if both parameters are switched off, the language has neither ergative nor accusative case marking.

\begin{multicols}{2}
	
	\ea\label{16:ex3} \textbf{Nominative-accusative}\leavevmode\vadjust{\vspace{-\baselineskip}}\newline\\
	\hspace{1cm}
	\begin{tikzpicture}[>=latex'] \tikzset{every tree node/.style={align=center,anchor=north}} 
	\Tree [.XP \node(x){DP$_{1}$}; [.... ... [.YP \node(y){DP$_{2}$};    
	[.$...$ ] ] ] ]  
	%\draw[semithick, <-,overlay] (x.south) to [bend right=45] (y.west);
	\draw[semithick,->] (x.south) to [bend right=80] node [midway,fill=white] {\accc} (y.west); 
	\useasboundingbox (current bounding box.north west) rectangle ([yshift=-2.5ex] current bounding box.south east); 
	\end{tikzpicture}
    \z
	
	\columnbreak
	
	\ea\label{16:ex4} \textbf{Ergative-absolutive}\leavevmode\vadjust{\vspace{-\baselineskip}}\newline\\		
	\hspace{1cm}
	\begin{tikzpicture}[>=latex'] \tikzset{every tree node/.style={align=center,anchor=north}} 
	\Tree [.XP \node(x){DP$_{1}$}; [.... ... [.YP \node(y){DP$_{2}$};    
	[.$...$ ] ] ] ]  
	%\draw[semithick, <-,overlay] (x.south) to [bend right=45] (y.west);
	\draw[semithick,<-] (x.south) to [bend right=80] node [midway,fill=white] {\textsc{erg}} (y.west); 
	\useasboundingbox (current bounding box.north west) rectangle ([yshift=-2.5ex] current bounding box.south east); 
	\end{tikzpicture}
    \z
	
	
\end{multicols}

\noindent In Step 3, the remaining DPs that have not received case by means of competition with another DP, receive the \textsc{unmarked case}, which depends on the local domain in which the NP is found (nominative/absolutive in TP/CP, genitive in DP). Finally, \textsc{default case} is assigned to fragment answers and free-standing DPs (\textit{Who bought the bread? Him./*He.}).


One of the evident problems for DCT is that \datt{} can be assigned either in Step 1, as lexically governed case, or in Step 2, as dependent case. If assigned as dependent case, \datt{} is considered to be assigned to a higher DP in the VP \citep{bakervinokurova,baker15}, which means that the case feature on a dative DP can sometimes be supplied by a lexical head and sometimes in a particular configuration in the VP and even though this feature has two completely different sources in the syntax, it is still recognised and realised as the same exponent by the morphology. We propose instead that assignment of dative via a lexical head can be abandoned in DCT. We claim that \datt{} can always be treated as dependent case assigned to a higher DP in a VP. In line with proposals by \citet{bittnerhale96,baker15} (for case assignment in general), \citet{wood2016} (for lexical accusative case in Icelandic), and \citet{bakerbobaljik} (for inherent ergative case), instead of assuming that a verb comes with a lexical [$*$\datt$*$] case feature \REF{16:ex5}, we propose that the verb comes with a covert pseudo co-argument DP, which enables the assignment of lexical dative as dependent case to a higher DP in a VP \REF{16:ex6}.

\begin{multicols}{2}
	
	\ea\label{16:ex5} \textbf{Lexical \datt{} via lexical head}\leavevmode\vadjust{\vspace{-\baselineskip}}\newline\\
	\hspace{1cm}
	\begin{tikzpicture}[>=latex'] \tikzset{every tree node/.style={align=center,anchor=north}} 
	\Tree  [.VP \node(x){V$_{[*\datt*]}$}; [.\node(y){DP}; ] ] ] ]  
	%\draw[semithick, <-,overlay] (x.south) to [bend right=45] (y.west);
	\draw[overlay, semithick,->] (x.south)..controls +(south:1.5) and +(south:1.5).. node [midway,fill=white] {\datt} (y.south);	
	\useasboundingbox (current bounding box.north west) rectangle ([yshift=-2.5ex] current bounding box.south east); 
	\end{tikzpicture}\\
	%	\hspace{1cm}
	\z
	
	\columnbreak
	
	\ea\label{16:ex6} \textbf{Lexical \datt{} as dependent case}\leavevmode\vadjust{\vspace{-\baselineskip}}\newline\\		
	\hspace{1cm}
	\begin{tikzpicture}[>=latex'] \tikzset{every tree node/.style={align=center,anchor=north}} 
	\Tree [.VP \node(x){DP}; [.V$'$ \node(y){V};    
	[.\node(z){DP$_{\emptyset}$}; ] ] ] ]  
	%\draw[semithick, <-,overlay] (x.south) to [bend right=45] (y.west);
	%	\draw[semithick,<-] (x.south) to [bend right=100] node [midway,fill=white] {{\sc dat}} (z.west); 
	\draw[overlay, semithick,<-] (x.south)..controls +(south west:1.5) and +(south west:1.5).. node [midway,fill=white] {\textsc{dat}} (z.south);
	\useasboundingbox (current bounding box.north west) rectangle ([yshift=-2.5ex] current bounding box.south east); 
	\end{tikzpicture}\\
	%	\hspace{1cm}
	\z
\end{multicols}

\noindent Furthermore, there is an ongoing debate within the DCT on the timing of case assignment. While some authors want case assignment to be a syntactic process (see \citealt{preminger-book} and \citealt{baker15}, who times case assignment at Spell-Out, during linearization), others argue that dependent case is assigned at PF \citep{marantz91,mcfadden04,bobaljikphi}. In what follows, we will take the syntactic side of the debate and offer a derivational implementation via the operation Agree between two DPs, which will derive dependent case assignment as a narrow syntactic process, and explain the dative puzzle outlined above. 



\section{Structural dative in Serbian}

In order to derive the assignment of structural dative case in double object constructions, this section offers a short empirical introduction on the structural relations between \nomm, \accc{} and \datt{} arguments in BCS. The order of the indirect object (IO) and the direct object (DO) is mostly free in Serbian and both orders can be used in neutral contexts:

\ea\label{16:doubleobject1}
\ea\label{16:doubleobject1a} \gll Slavica je predstavila sestri Marka.\\
Slavica.\nomm{} is presented sister.\datt{} Marko.\accc\\
\glt `Slavica presented Marko to her sister.'\hfill V \before{} \datt{} \before{} \accc{}
\ex \label{16:doubleobject1b}\gll Slavica je predstavila Marka sestri.\\
Slavica.\nomm{} is presented Marko.\accc{} sister.\datt{}\\
\glt `Slavica presented Marko to her sister.'\hfill V \before{} \accc{} \before{} \datt{}
\z \z

\noindent However, there is reason to believe that IO \before{} DO, i.e. \REF{16:doubleobject1a} is the base order of the two objects, while \REF{16:doubleobject1b} is derived by A-movement. The evidence from quantifier scope \citep{aoun89,frey1989,bruening2001} shows that,  while in the V \before{} \datt{} \before{} \accc{} order only the reading where the quantifier in the IO scopes over the one in the DO is available \REF{16:ex8a}, the order V \before{} \accc{} \before{} \datt{} allows for both readings \REF{16:quantifierscope}. The availability of the reading where the existential quantifier outscopes the universal one in \REF{16:quantifierscope} indicates that the DO can reconstruct in its base position, below the IO. 

\ea\label{16:ex8} 
\ea\label{16:ex8a}\gll Slavica je predstavila [$_{\datt{}}$ jednoj drugarici] [$_{\accc{}}$ svakog momka].\\
Slavica is introduced {} one.\datt{} friend.\datt{} {} every.\accc{} boyfriend.\accc\\
\glt `Slavica introduced every boyfriend to a friend.'\hfill  $\exists>\forall$, *$\forall>\exists$ 
\ex\label{16:quantifierscope}\gll Slavica je predstavila [$_{\accc{}}$ svakog momka] [$_{\datt{}}$ jednoj drugarici].\\
Slavica is introduced {} every.\accc{} boyfriend.\accc{} {} one.\datt{} friend.\datt\\
\glt `Slavica introduced every boyfriend to a friend.'\hfill $\exists>\forall$, $\forall>\exists$ 
\z \z

\noindent Furthermore, maximal focus projection (from a focused NP to the entire clause) is possible only if we maintain the base word order (\citealt{hoehle82,vonstechowuhmann,haider92}). A sentence in which movement has occurred should not be a good answer to the question \textit{What happened?/What's new?}.\footnote{\citet[76]{stjepanovic-thesis} offers a similar argument for Serbo-Croatian.} With the focus on the DO, if the whole sentence is new information, focus is perceived as neutral if the sentence has the canonical word order \REF{16:focusa}. However, the focus in \REF{16:focusb} is not necessarily new information focus, as it does not project to the entire clause; it can be interpreted as contrastive, which indicates that the order is not the base one and movement has taken place.\footnote{Even though the word order in \REF{16:focusb} is neutral, as noted in \REF{16:doubleobject1b}, if the \datt{} argument is focused, the sentence sounds less neutral than its counterpart in \REF{16:focusa}. We thank an anonymous reviewer for this insight. Moreover, factors such as animacy and givenness may contribute to enabling other orders in neutral contexts; see recent findings by  \citet{titov17} for Russian and \citet{velnic-diss} for Croatian.}    

\ea `What happened?'
\ea[]{\gll[\hspace{-2pt} Slavica je poslala Marku \textsc{pismo} ]\\
{} Slavica is sent Marko.\datt{} letter.\accc\\
\glt `Slavica sent a letter to Marko.'}\label{16:focusa}
\ex[\#]{\gll [\hspace{-2pt} Slavica je poslala pismo \textsc{Marku} ]\\
{} Slavica is sent letter.\accc{} Marko.\datt\\
\glt `Slavica sent a letter to Marko.' / `It was Marko who Slavica sent a letter to.'}\label{16:focusb}
\z \z


\noindent Finally, the order of object clitics in Serbian is always \datt{} \before{} \accc, regardless of the order IO and DO noun phrases. \citet{stjepanovic-thesis} and \citet{Boskovic2001NatureSyntaxPhonology} assume that clitics move outside of their VP into Agr projections. The strict hierarchy between them suggests that this movement respects superiority.\largerpage[2]
% \newpage
% \begin{multicols}{2}
	\ea
	\ea[]{\gll Ti si poslala Nevenu pismo.\\
	you are sent Neven.\datt{} letter.\accc\\
	\glt `You sent a letter to Neven.'}
	\ex[]{\gll Ti si \textbf{mu} \textbf{ga} poslala.\\
	you are him.\datt{} it.\accc{} sent\\
	\glt `You sent it to him.}
	\ex[*]{\gll Ti si \textbf{ga} \textbf{mu} poslala.\\
	you are it.\accc{} him.\datt{} sent\\
	\glt `You sent it to him.}
    \z \z
	
% 	\columnbreak
	
	\ea
	\ea[]{\gll Ti si poslala pismo Nevenu.\\
	you are sent letter.\accc{} Neven.\datt\\
	\glt `You sent a letter to Neven.'}
	\ex[*]{\gll Ti si \textbf{ga} \textbf{mu} poslala.\\
	you are it.\accc{} him.\datt{} sent\\
	\glt `You sent it to him.}
	\ex[]{\gll Ti si \textbf{mu} \textbf{ga} poslala.\\
	you are him.\datt{} it.\accc{} sent\\
	\glt `You sent it to him.}
    \z \z
	
% \end{multicols}

\noindent We conclude from these tests that the base word order of objects in Serbian is IO \before{} DO. 

\section{A derivational account of dependent case assignment}
\largerpage[3]
Following \citet{bakervinokurova,baker15,preminger-book,levinpreminger}, we assume that case is assigned in narrow syntax. We adopt case feature notations from Lexical Decomposition Grammar, following \citet{kiparsky92,kiparsky2001,wunderlichjoppen,wunderlich97,stiebels2002}:

\ea
\ea \accc: [+hr] `there is a higher role'
\ex \datt: [+hr +lr] `there is a higher role and there is a lower role'
\ex \ergg: [+lr] `there is a lower role' 
\ex \nomm/\abss: [{} {}] no case features 
\z \z


\noindent The features \hr{} and \lr{} are assigned in the course of the derivation to argument DPs via the operation Agree. We assume that both standard `downward' Agree and `upward' Agree \citep[see][]{chomsky86,chomsky91,kayne89,pollock89,koopman06} are possible options in the grammar (see also \citealt[92f.]{abels-phases} as well as \citeauthor{baker-agrbook}'s \citeyear[155]{baker-agrbook} \textit{Direction of Agreement Parameter}). We propose that Agree applies between two DPs in a c-command relationship. When Downward Agree (\down) applies, the higher of the two DPs in an asymmetric c-command relation probes down and receives the \lr{} from the lower one (see \REF{16:lr} below), and by Upward Agree (\up), the lower DP probes upward and receives its \hr{} case feature from the higher DP (see \REF{16:hr} below). An important principle is that case valuation cannot take place if the \textit{goal DP} already has a valued case feature \citep{bittnerhale96,baker15}. One DP can participate in multiple Agree operations as a \textit{probe} and, in principle, this can result in a DP receiving more than one case feature, as demonstrated shortly below \REF{16:lrhr}. Moreover, in a \nomm/\accc{} system, \down{} always precedes \up. Finally, in a nominative-accusative alignment, assignment of \lr{} in Spec\littlev{}P must somehow be pre-empted, otherwise the DP would receive ergative case. We assume that languages with nominative-{\linebreak}accusative alignment have an \textit{ergative switch-off parameter}, regulated by the following principle: In a \nomm-\accc{} language the higher DP in a \textit{v}P cannot be case-valued.\footnote{Alternatively, assuming that at the \littlev{}P level \up{} precedes \down{} yields the same results.} Finally, we assume that the domain in which the proposed operations apply is the TP. 

\largerpage[2]
Let us apply the system to dative case assignment. In a double-object construction, a verb selects two objects, yielding thereby a VP with two unmarked DPs in a c-command relationship. Since this is a \nomm{}-\accc{} system, \down{} will always precede \up{}. Thus when \down{} applies, the higher of the two DPs receives a \lr{} feature from the lower one. Consequently, \up{} does not apply because the potential goal is already case-valued.  

\ea\label{16:lr}\textbf{Assignment of \lr{} in VP}\leavevmode\vadjust{\vspace{-\baselineskip}}\newline\\
\small
\begin{tikzpicture}[>=latex'] \tikzset{every tree node/.style={align=center,anchor=north}} 
\Tree [.VP \node(x){DP$_{2}$\\{\lr}}; [.V\1 V \node(y){DP$_{1}$\\{\nocase}};  ] ] ]]    
\draw[overlay, semithick,*->] (x.south)..controls +(south west:1) and +(south west:2).. node [midway,fill=white] {\datt} (y.south west); 
\useasboundingbox (current bounding box.north west) rectangle ([yshift=-2.5ex] current bounding box.south east); 
\end{tikzpicture}\\
\normalsize
\z
%	\vspace{0.5cm} 

\noindent After the external DP$_{3}$ is introduced in Spec\littlev{}P, we now have three DPs in the same domain. The remaining two caseless DPs are DP$_{1}$ and DP$_{3}$. When \down{} applies between the highest DP$_{3}$ in the Spec\littlev{}P and the lowest DP$_{1}$, no case valuation obtains, due to the ergative switch-off parameter, which demands that a DP in Spec\littlev{}P cannot be case valued. \up{} thus applies afterwards, whereby the lower DP receives the \hr{} feature from the higher one \REF{16:hr}. 

% \newpage
\ea\label{16:hr} \textbf{Assignment of \hr{} to the lower argument in VP}\leavevmode\vadjust{\vspace{-\baselineskip}}\newline\\
\begin{tikzpicture}[>=latex'] \tikzset{every tree node/.style={align=center,anchor=north}} 
\Tree [.\textit{v}P \node(x){DP$_{3}$\\{\nocase}}; [.\textit{v}\1 \textit{v} [.VP \node(z){DP$_{2}$\\{\lr}}; [.V\1 V \node(y){DP$_{1}$\\{\hr}};  ] ] ] ]   
\draw[overlay, semithick,<-*] (x.south)..controls +(south west:2) and +(south west:2).. node [midway,fill=white] {\accc} (y.south west); 
\useasboundingbox (current bounding box.north west) rectangle ([yshift=-2.5ex] current bounding box.south east); 
\end{tikzpicture}\\
\z

\clearpage 

\noindent However, DP$_{2}$ and DP$_{3}$ still fulfil the criteria for case assignment to apply, since they are in a c-command relationship, and the higher one is not marked for case \REF{16:lrhr}. Thus \up{} applies, providing the lower DP$_{2}$ with a \hr{} feature (and the [\hr, \lr] bundle is realised as dative). 


\ea\label{16:lrhr} \textbf{Assignment of \hr{} to higher argument in VP}\leavevmode\vadjust{\vspace{-\baselineskip}}\newline\\
\begin{tikzpicture}[>=latex'] \tikzset{every tree node/.style={align=center,anchor=north}} 
\Tree [.\textit{v}P \node(x){DP$_{3}$\\{\nocase}}; [.\textit{v}\1 \textit{v} [.VP \node(z){DP$_{2}$\\{\hr\lr}}; [.V\1 V \node(y){DP$_{1}$\\{\hr}};  ] ] ] ] ]  
\draw[overlay, semithick,<-*] (x.south)..controls +(south west:2) and +(south west:1).. node [midway,fill=white] {\textsc{ dat}} (z.south west); 
\useasboundingbox (current bounding box.north west) rectangle ([yshift=-2.5ex] current bounding box.south east); 
\end{tikzpicture}
\z
%	\vspace{0.5cm}

\begin{figure}[b]
\ea \textbf{Accusative assignment}\label{16:ex16}\leavevmode\vadjust{\vspace{-\baselineskip}}\newline\\
\begin{tikzpicture}[>=latex'] \tikzset{every tree node/.style={align=center,anchor=north}} 
\Tree [.\textit{v}P \node(n){DP$_{3}$\\{\nocase}}; [.\textit{v}\1 \textit{v}+V [.VP \node(z){DP$_{1}$\\{\hr}}; [.VP \node(x){DP$_{2}$\\{\hr\lr}}; [.V\1 t$_{V}$ \node(y){t$_{\text{DP1}}$};  ] ] ]]]]    
\draw[overlay, semithick,*->] (x.south)..controls +(south:2) and +(south:1).. node [midway,fill=white] {\ding{172} \datt} (y.south); 
\draw[overlay, semithick,<-*] (n.south west)..controls +(south west:4) and +(south west:3).. node [midway,fill=white] {\ding{173} \datt} (x.south); 
\draw[overlay, semithick,<-*] (n.south west)..controls +(south west:2) and +(south west:2).. node [midway,fill=white] {\ding{173} \accc} (z.south); 
\useasboundingbox (current bounding box.north west) rectangle ([yshift=-2.5ex] current bounding box.south east); 
\end{tikzpicture}
\vspace{0.5cm}
\z
\end{figure}

\noindent This implementation derives the assignment of dependent case by means of existing, independently motivated mechanisms, in a derivational manner. An interesting prediction is that at the point in the derivation before the external argument is merged, dative should behave in a similar way as ergative case, as it only bears a \lr{} feature, as in \REF{16:lr}.  While we leave this point for further research, note that similarities between datives and ergatives have been reported in Basque by \citet{arreginevins12}, in Indo-Aryan languages by \citet{butt06} and even Serbo-Croatian by \citet{progovac13}. Another important prediction is that movement of the DO should not affect \accc{} case assignment, since \hr{} feature still has the necessary configuration even after movement, as shown by \REF{16:ex16}. In this process, DP$_{2}$ is first assigned the \lr{} feature by \down{} with DP$_{1}$, which is then moved, and still caseless. After DP$_{3}$ has been introduced, both DP$_{1}$ and DP$_{2}$ will receive their missing \hr{} features by \up{} with it.

\section{Lexical dative}

\subsection{Similarities between structural and lexical dative}

As noted in the introduction, the central claim of this paper is that lexical dative case is assigned just like the structural dative. In order to support this claim, we first demonstrate that there are indeed similarities between `structural' and `lexical' datives in their syntactic behaviour.

For instance, they act in a similar way in passivisation. In double-object constructions, only the accusative object can be passivised, i.e. only the theme argument can alternate between accusative and nominative, as in \REF{16:structuraldativepassive}.

\ea\label{16:structuraldativepassive} 
\ea \gll Ljubica je dala Milošu knjigu.\\
Ljubica.\nomm.\fsg{} is gave.\fsg{} Miloš.\datt{} book.\accc\\
\glt `Ljubica gave a book to Miloš.'
\ex \gll Knjiga je bila data Milošu.\\
book.\nomm.\fsg{} is been.\fsg{} given.\fsg{} Miloš.\datt\\
\glt `The book was given to Miloš.'
\ex \label{16:structuraldativepassiveb}\gll Milošu je bila data knjiga.\\
Miloš.\datt{} is been.\fsg{} given.\fsg{} book.\nomm.\fsg\\
\glt `The book was given to Miloš.'
\z \z

\noindent The dative argument, however, cannot be turned into a subject and it never alternates \REF{16:ex18}.

\ea\label{16:ex18}
\ea[*]{\gll Miloš je bio dat knjigu.\\
Miloš.\nomm{} is been.\msg{} given.\msg{} book.\accc\\
\glt `Miloš was given a book.'}
\ex[*]{\gll Milošu je \{\hspace{-2pt} bio / bilo\} \{\hspace{-2pt} dat / dato\} knjiga.\\
Miloš.\datt{} is {} been.\msg{} {} been.\nsg{} {} given.\msg{} {} given.\nsg{} book.\nomm\\
\glt `Miloš was given a book.'}
\z \z

\noindent Unlike in Icelandic (as described by \citealt{zaenenetal85}), dative cannot bind a subject oriented anaphor \REF{16:ex19a} and it cannot be deleted under subject ellipsis \REF{16:ex19b}, hence it is not a subject.

\ea\label{16:dativesubjectpassive2}
\ea[*]{\gll Milošu je bila data svoja knjiga.\\
Miloš.\datt{} is been.\fsg{} given.\fsg{} \textsc{poss}.\fsg.\nomm{} book.\nomm\\
\glt Intended: `Miloš was given his book.'}\label{16:ex19a}
\ex[*]{\gll Miloš je bio izbačen sa časa i \_\_\_ bio je dat ukor.\\ 
Miloš.\nomm{} is been.\msg{} thrown.out.\msg{} from class and {} been.\msg{} is given.\msg{} reprimand \\
\glt intended: `Miloš was thrown out of the class and he was reprimanded.'}\label{16:ex19b}
\z \z

\noindent Parallel to \REF{16:structuraldativepassive} above, some constructions with lexical datives can be pasivised, as in \REF{16:lexicaldativepassive}, where the lexical dative in \REF{16:ex20b} mirrors the structural one from \REF{16:structuraldativepassiveb}.

\ea\label{16:lexicaldativepassive}
\ea \gll Ljubica je pomogla Ani.\\
Ljubica.\nomm{} is helped Ana.\datt\\
\glt `Ljubica helped Ana.'
\ex \gll Ani je bilo pomognuto.\\
Ana.\datt{} is been.\nsg{} helped.\nsg\\
\glt `Ana was helped.'\label{16:ex20b}
\z \z

\noindent However, \citet{zaenenetal85} subjecthood tests also show that this dative does not behave like a subject. It does not bind a subject-oriented anaphor \REF{16:ex21a} and it cannot be deleted under subject ellipsis \REF{16:ex21b}, just like the structural dative in \REF{16:dativesubjectpassive2}.\largerpage[2]

\ea
\ea[*]{\gll Ani je bilo pomognuto od strane svoje sestre.\\
Ana.\datt{} is been helped.\nsg{} from side \textsc{poss}.\genn{} sister.\genn\\
\glt `Ana was helped by her sister.'}\label{16:ex21a}
\ex[*]{\gll Ana je uradila sve zadatke i \_\_\_ pri tome je bilo pomognuto.\\
Ana.\nomm{} is done.\fsg{} all tasks.\accc{} and {} with that is been.\nsg{} helped.\nsg\\
\glt `Ana did all the tasks and was helped with that.'}\label{16:ex21b}
\z \z

\largerpage[2]
\noindent Moreover, as argued by \citet{maling01} and shown for German by \citet{mcfadden04}, one of the structural asymmetries between DOs and IOs is their behaviour in nominalisations. DOs appear in genitive when the VP is nominalised \REF{16:ex22b}, unlike both structural \REF{16:ex22c} and lexical datives \REF{16:ex23}, which do not alternate with genitive.\footnote{A reviewer wonders about the status of \textit{darivanje Miloša} `the giving of something to Miloš.\genn' in \REF{16:ex22c}. We believe that here the genitive of the complement of   \textit{darivati} is lexical. We leave it to future research to explore how lexical genitive fits into the current proposal.}

\ea \textbf{Structural dative}
\ea \gll Ljubica je poklonila Milošu knjigu.\\
Ljubica.\nomm{} is gave Miloš.\datt{} book.\accc\\
\glt `Ljubica gave a book to Miloš.'
\ex \gll poklanjanje knjige Milošu\\
giving book.\genn{} Miloš.\datt\\
\glt `the giving of the book to Miloš'\label{16:ex22b}
\ex\label{16:ex22c} \gll poklanjanje Miloša\\
giving Miloš.\genn\\
\ea[]{`the giving of Miloš (to someone)'}
\ex[*]{`the giving (of something) to Miloš'}
\z \z \z

\ea \textbf{Lexical dative}\label{16:ex23}
\ea\gll Ova kapa pripada Ani.\\
this.\nomm{} cap.\nomm{} belongs Ana.\datt\\
\glt `This cap belongs to Ana.'
\ex\gll pripadanje Ani\\
belonging Ana.\datt\\
\glt `the belonging (of something) to Ana'
\ex\gll pripadanje Ane\\
belonging Ana.\genn\\
\ea[]{`the belonging of Ana (to someone)'}
\ex[*]{`the belonging to Ana'}
\z \z \z
\newpage 

\noindent Finally, as argued for German by \citet{sternefeld85,bayeretal01,mcfadden04}, in the so-called `topic drop' constructions, it is possible to omit the \accc{} \REF{16:ex25a}, but not a \datt{} topic, irrespective of whether it is structural \REF{16:ex25b} or lexical \REF{16:ex25c}.

\ea\gll Da li poznaješ Tamaru?\\
that \textsc{prt} know.\textsc{2.sg} Tamara.\accc\\
\glt `Do you know Tamara?'
\z

\ea 
\ea\label{16:ex25a}\gll Da, poznajem (\hspace{-2pt} je).\\
yes know.\textsc{1.sg} {} her.\accc\\
\glt `Yes, I know her.'
\ex\label{16:ex25b}\gll Da, jednom sam *(\hspace{-2pt} joj) poklonila cvet.\\
yes once am {} her.\datt{} gave flower\\
\glt `Yes, I once gave her a flower.'\hfill structural \datt
\ex\label{16:ex25c}\gll Da, jednom sam *(\hspace{-2pt} joj) pomogla.\\
yes once am {} her.\datt{} helped\\
\glt `Yes, I helped her once.' \hfill lexical \datt 	
\z \z

\largerpage
\noindent From these similarities, we conclude that lexical and structural datives can be treated as the same type of syntactic objects.\footnote{An additional language specific test that points into the same direction is Left Branch Extraction, which is allowed out of subjects \REF{16:fn5exia} and objects \REF{16:fn5exib} in Serbian (see \citealt{bosk2005}, and subsequent work), but seems to be disallowed both with structural \REF{16:fn5exic} and lexical dative \REF{16:fn5exid}.
	
	\ea
	\ea[]{\gll Kakvi su mu juče [ t dečaci ] kupili poklon?\\
	what.\nomm{} are him.\datt{} yesterday {} {} boys.\nomm{} {}  bought present.\accc\\
	\glt `What boys bought a present for him yesterday?'\hfill LBE with \nomm}\label{16:fn5exia}
	\ex[]{\gll Kakav su mu dečaci juče kupili [ t poklon]?\\
	what.\accc{} are him.\datt{} boys.\nomm{} yesterday bought {} {} present.\accc\\
	\glt `What present did the boys buy for him yesterday?'\hfill LBE with \accc }\label{16:fn5exib}
	\ex[*?]{\gll Kojoj su dečaci juče [ t drugarici] kupili poklon?\\
	what.\datt{} are boys.\nomm{} yesterday {} {} friend.\datt{} bought present.\accc\\
	\glt `Which friend did the boys buy the present for?'\hfill LBE with \datt$_{\mbox{\tiny struc}}$}\label{16:fn5exic}
	\ex[*?]{\gll Kojoj su dečaci juče [ t drugarici] pomogli?\\
	which.\datt{} are boys.\nomm{} yesterday {} {} friend.\datt{} helped\\
	\glt `Which friend did tie boys help yesterday?' \hfill LBE with \datt$_{\mbox{\tiny lex}}$}\label{16:fn5exid}
    \z \z 
	
\noindent However, the acceptability of the examples varies across different speakers, and it can be also influenced by factors such as word order. We leave this very interesting issue for future research.} In the next sections, we will inspect different types of lexical datives we have identified in Serbian in turn. 



\subsection{Lexical dative as dependent case}

\subsubsection{\textit{Help}-type verbs as underlying ditransitives}

\textit{Help}-type verbs include verbs such as \textit{pomoći} `help', \textit{čestitati} `congratulate', \textit{ugodi\-ti} `please', \textit{služiti} `serve', \textit{verovati} `believe', \textit{zavideti} `envy', \textit{doprineti} `contribute', etc. (a partial list from several types of monotransitive  constructions identified by \citealt{stipcevic}). We argue that these verbs are underlyingly ditransitive, where the  DP$_{\accc}$ is present, but covert, yet even as such, it serves as a competitor for dative case assignment. In these constructions, the \nomm{} argument is usually an \textsc{agent}, while the \datt{} can have \textsc{be\-ne\-fi\-ciary}/\textsc{male\-fi\-ciary}/\textsc{re\-ci\-pi\-ent}/\textsc{goal}/\textsc{tar\-get person} theta-role. The unmarked word order of arguments of \textit{help}-type verbs is \nomm{} \before{} \datt{} \REF{16:ex26}.

\ea\label{16:ex26}
\ea\gll Ljubica je pomogla svom detetu.\\
Ljubica.\nomm{} is helped \textsc{poss}.\datt{} child.\datt\\
\glt `Ljubica helped her child.'
\ex\gll Trener je  čestitao svojim igračima.\\
coach.\nomm{} is congratulated \textsc{poss}.\datt{} players.\datt\\
\glt `The coach congratulated his players.'
\z \z

\noindent A possibly crucial piece of evidence for postulating a silent DP$_{\accc}$ is that even though usually monotransitive, these constructions can have another \textit{overt} \accc{} argument:\footnote{Note a similar kind of behaviour of lexical datives in German invoked by \citep[129]{mcfadden04}. He takes this as a piece of evidence that lexical dative assigned by \textit{glauben}/\textit{helfen}-type verbs in German can be analysed as structural dative.
	
	\ea
	\ea\gll Er glaubt seinem Bruder.\\
	he.\nomm{} believes \textsc{poss}.\datt{} brother.\datt\\
	\glt `He believes his brother.'
	\ex\gll Er glaubt seinem Bruder die Geschichte.\\
	he.\nomm{} believes \textsc{poss}.\datt{} brother.\datt{} the story.\accc\\
	\glt `He believes his brother's story.'
    \z \z
	
	
}


\ea\label{16:overtsilentdp}
\ea\gll Ljubica je pomogla svom detetu školovanje.\\
Ljubica.\nomm{} is helped \textsc{poss}.\datt{} child.\datt{} education.\accc\\
\glt `Ljubica sponsored her child's education.'
\ex\gll Trener je čestitao svojim igračima pobedu.\\
coach.\nomm{} is congratulated \textsc{poss}.\datt{} players.\datt{} victory.\accc\\
\glt `The coach congratulated his players on the victory.'
\z \z


\noindent\textit{Help}-type constructions with lexical datives in Serbian seem to be able to passivise (forming an impersonal passive construction; recall \REF{16:lexicaldativepassive}). Such evidence suggests that constructions of this type can be treated as double-object constructions, equivalent to those in \REF{16:doubleobject1}, allowing for treatment of lexical dative as structural.



We therefore argue that constructions with the \textit{help}-type verbs are in fact double-object constructions. The lower \accc{} object is present as a silent DP (see \citealt{wood2016} for a similar proposal for lexical accusatives in Icelandic and \citealt{bakerbobaljik} for similar ideas for ergative case). This silent DP can sometimes be realised overtly, as in \REF{16:overtsilentdp} above. The `lexical' dative is assigned in the same manner as in ditransitive double-object constructions. The feature \lr{} is assigned to the higher DP at the VP level via \down. The assignment of \hr{} applies at \littlev{}P, by \up{}, which is established with the nominative DP in Spec\littlev{}P. 	

\ea \textbf{Lexical dative, \textit{help}-type verbs}\leavevmode\vadjust{\vspace{-\baselineskip}}\newline\\
\begin{tikzpicture}[>=latex'] \tikzset{every tree node/.style={align=center,anchor=north}} 
\Tree [.\textit{v}P \node(x){DP$_{\nocase}$\\Ljubica}; [.\textit{v}\1 \textit{v}\\helped [.VP \node(z){DP$_{\text{\lr\hr}}$\\{her child}}; [.V\1 V \node(y){DP$_{\emptyset\text{\hr}}$};  ] ] ] ] ]  
\draw[overlay, semithick,<-*] (x.south)..controls +(south west:3) and +(south west:4).. node [midway,fill=white] {\ding{173} \accc} (y.south); 
\draw[overlay, semithick,<-*] (y.south west)..controls +(south west:1) and +(south west:2).. node [midway,fill=white] {\ding{172} \datt} (z.south); 
\draw[overlay, semithick,<-*] (x.south)..controls +(south west:1) and +(south west:1.5).. node [midway,fill=white] {\ding{173} \datt} (z.south west); 
\useasboundingbox (current bounding box.north west) rectangle ([yshift=-2.5ex] current bounding box.south east); 
\end{tikzpicture}\\
	\vspace{0.5cm}
\z

\noindent These constructions are therefore underlyingly true ditransitives, which explains their striking similarities to regular canonical ditransitive constructions and the similarities in the syntactic behaviour between the datives in the two. 

\subsubsection{An extension: \textit{Adjust}-type verbs as underlying ditransitives}

Another type of verbs identified by \citet[300f.]{stipcevic} select for dative objects where the dative argument mostly has a \textsc{target person/goal} theta-role. Some of the verbs include: \textit{odužiti se} `pay back', \textit{osvetiti se} `take revenge', \textit{suprotstaviti se} `confront', \textit{predati se} `give in/give up', \textit{oteti se} `escape', \textit{priključiti se} `join', \textit{prilagoditi se} `adjust', etc. Most of these verbs contain the morpheme \textit{se}, which mostly has a reflexive interpretation. The nominative argument is usually an \textsc{agent} in these sentences and the unmarked order is \nomm{} \before{} \datt{} \REF{16:lexicaladjust}.

\ea\label{16:lexicaladjust}
\ea\label{16:ex29a}\gll Tamara se prilagodila situaciji.\\
Tamara.\nomm{} \textsc{refl} adjusted.\fsg{} situation.\datt\\
\glt `Tamara adjusted to the situation.'
\ex\label{16:ex29b}\gll Srdjan se predao policiji.\\
Srdjan.\nomm{} \textsc{refl} surrendered.\msg{} police.\datt\\
\glt `Srdjan surrendered to the police.'
\z \z

\noindent Another overt \accc{} argument can be added, but in that case the morpheme \textit{se} cannot appear in the sentence. Comparing \REF{16:ex29a}/\REF{16:ex29b} with \REF{16:ex30a}/\REF{16:ex30b} respectively, we can see that \textit{se} and \accc{} seem to be in complementary distribution. \textit{Se} therefore seems to absorb \accc{} case (see also \citealt{franks95}).\footnote{Passivisation is unfortunately inconclusive as a test. Sentences with an overt accusative can be passivized regularly \REF{16:fn7exia}, but the ones without the overt \accc{} argument and with the \textit{se} morpheme cannot be \REF{16:fn7exib}.  
	
	\ea
	\ea[]{\gll Ponašanje je bilo prilagodjeno situaciji.\\
	behaviour.\nomm.\nsg{} is been adjusted.\nsg{} situation.\datt\\
	\glt `The behaviour was adjusted to the situation.}\label{16:fn7exia}
	\ex[*]{\gll Situaciji se / je bilo prilagodjeno.\\
	situation.\datt{} \textsc{refl} {} is been.\nsg{} adjusted.\nsg\\\\
	intended: `One adjusted to the situation.'}\label{16:fn7exib}
	\ex[]{\gll Situaciji se prilagodilo.\\
	situation.\datt{} \textsc{refl} adjusted.\nsg\\\\
	\glt `One adjusted to the situation.'}\label{16:fn7exic}
    \z \z

\noindent	As \REF{16:fn7exic} shows, the only possible `passive' form with these constructions is actually impersonal middle construction, which is expected if these constructions even in the active voice already involve argument reduction (see \citealt{progovac13,marelj04}).}

\ea\label{16:structuraladjust}
\ea\label{16:ex30a}\gll Tamara je (*\hspace{-2pt} se) prilagodila ponašanje situaciji.\\
Tamara.\nomm{} is {} \textsc{refl} adjusted.\fsg{} behaviour.\accc{} situation.\datt\\
\glt `Tamara adjusted her behaviour to the situation.'
\ex\label{16:ex30b}\gll Srdjan je (*\hspace{-2pt} se) predao dokumente policiji.\\
Srdjan.\nomm{} is {} \textsc{refl} submitted.\msg{} documents.\accc{} police.\datt\\
\glt `Srdjan submitted the documents to the police.'
\z \z

\noindent The similarities between \REF{16:structuraladjust} and \REF{16:lexicaladjust} above can be captured by the derivations in \REF{16:ex31} and \REF{16:ex32}. While verbs with `structural' dative contain an overt DP as a DO, \textit{adjust}-type verbs contain a silent DP. Crucially, the \lr{} feature is assigned to the higher of the two DPs in the VP. While in \REF{16:ex31} the lower DP receives the \hr{} feature and thereby \accc{} case upon merging the external argument, in \REF{16:ex32}, the lower DP argument in the VP is reduced (or alternatively it starts out as a null DP) and becomes realised by \textit{se}.

% \newpage
	
	
	
	\ea\label{16:ex31} \textbf{Structural dative \REF{16:structuraladjust}}\leavevmode\vadjust{\vspace{-\baselineskip}}\newline\\
	\begin{tikzpicture}[>=latex', sibling distance=-1pt] \tikzset{every tree node/.style={align=center,anchor=north}} 
	\Tree [.\textit{v}P \node(x){DP$_{\nocase}$\\Tamara}; [.\textit{v}\1 \textit{v}\\adjusted [.VP \node(z){DP$_{\text{\lr\hr}}$\\situation}; [.V\1 V \node(y){DP$_{\text{\hr}}$\\behaviour};   ] ] ] ] ]  
	\draw[overlay, semithick,<-*] (x.south)..controls +(south west:4) and +(south:3).. node [midway,fill=white] {\ding{173} \textsc{ acc}} (y.south); 
	\draw[overlay, semithick,<-*] (y.south west)..controls +(south west:1) and +(south west:1).. node [midway,fill=white] {\ding{172} \textsc{ dat}} (z.south); 
	\draw[overlay, semithick,<-*] (x.south)..controls +(south west:2) and +(south west:2).. node [midway,fill=white] {\ding{173} \textsc{ dat}} (z.south); 
	\end{tikzpicture}\\
    \z
	\vspace{1.5cm}	
	
	

	
% 	\newpage
	\ea\label{16:ex32} \textbf{Lexical dative \REF{16:lexicaladjust}}\leavevmode\vadjust{\vspace{-\baselineskip}}\newline\\
	\begin{tikzpicture}[>=latex', sibling distance=-1pt] \tikzset{every tree node/.style={align=center,anchor=north}} 
	\Tree [.\textit{v}P \node(x){DP$_{\nocase}$\\Tamara}; [.\textit{v}\1 \textit{v}\\adjusted [.VP \node(z){DP$_{\text{\lr\hr}}$\\situation}; [.V\1 V \node(y){DP$_{\text{\hr}}$\\se};   ] ] ] ] ]  
	\draw[overlay, semithick,<-*] (x.south)..controls +(south west:4) and +(south:3).. node [midway,fill=white] {\ding{173} \textsc{ acc}} (y.south); 
	\draw[overlay, semithick,<-*] (y.south west)..controls +(south west:1) and +(south west:1).. node [midway,fill=white] {\ding{172} \textsc{ dat}} (z.south); 
	\draw[overlay, semithick,<-*] (x.south)..controls +(south west:2) and +(south west:2).. node [midway,fill=white] {\ding{173} \textsc{ dat}} (z.south); 
	\end{tikzpicture}\\
    \z
	\vspace{1.5cm}	
	
	
\subsubsection{\textit{Belong}-type verbs as unaccusative ditransitives}

{Belong}-type verbs include verbs such as \textit{pripadati} `belong', \textit{zapasti} `get into/end up with', \textit{nedostajati} `miss', etc. (see also \citealt{stipcevic}). We argue that these verbs are underlyingly ditransitive as well, but they do not take an external argument and are, therefore, unaccusative. The \nomm{} argument is usually a \textsc{theme}, while \datt{} is usually interpreted as \textsc{possessor}. The unmarked word order is \nomm{} \before{} \datt{}, as illustrated by \REF{16:ex33}.

\ea\label{16:ex33}\gll Ova kapa pripada Ani.\\
this.\nomm{} cap.\nomm{} belongs Ana.\datt\\
\glt `This cap belongs to Ana.'
\z

\noindent No additional overt accusative arguments can be added to these verbs and a structure like this cannot be passivised \REF{16:ex34}. The impossibility of passivization, the lack of overt accusative argument and the theme interpretation of the \nomm{} argument suggest therefore that such constructions are essentially unaccusative. The idea that the \nomm{} argument is introduced as the internal argument of the verb, which is later moved to the sentence-initial position, can be supported by evidence from quantifier scope. In \REF{16:ex35}, the possibility for the existential quantifier to outscope the universal one indicates that the  \nomm{} argument has been moved and is able to reconstruct in its base position.\footnote{This situation mirrors the one in \REF{16:quantifierscope}. Note that since Serbian is a rigid scope language, only movement can affect quantifier scope, thus the reading here cannot be derived by quantifier raising of the existential quantifier and must instead involve movement (see \citealt{antonyk15}).}

\ea[*]{\gll Ani je bilo pripadano.\\
Ana.\datt{} is been.\nsg{} belonged.\datt\\\\
\glt *`It was belonged to Ana.'}\label{16:ex34}
\z

\ea[]{\gll [$_{\accc{}}$ Svaka kapa] pripada [$_{\datt{}}$ jednoj devojci].\\
{} every.\nomm{} cap.\nomm{} belongs {} one.\datt{} girl.\datt\\
\glt `Every cap belongs to one girl.'\hfill $\exists>\forall$, $\forall>\exists$ }\label{16:ex35}
\z

\noindent In order to derive this type of lexical dative as dependent case, we assume that the two internal arguments of these verbs are both merged as the arguments of V, as in \REF{16:unaccusatives}. In this configuration, \down{} applies first and the higher DP receives the \lr{} feature from the lower one. The lower DP does not receive any case features at the VP level. Since these verbs are unaccusative, no external argument is merged in Spec\littlev{}P. However, the \textsc{theme} argument must move up in order to become the (derived) subject of the sentence. In order to move to SpecTP, it has to move through the \littlev{}P phase edge \citep{legate03}. At the \littlev{}P edge, this DP can now serve as a case competitor again. After \down{} fails due to the ergative switch-off parameter that precludes case valuation in Spec\littlev{}P, \up{} succeeds, and \hr{} is assigned to the \datt{} DP \REF{16:unaccusatives2}.
% \begin{multicols}{2}
	
	\ea\label{16:unaccusatives}\textbf{Lexical dative at VP}\leavevmode\vadjust{\vspace{-\baselineskip}}\newline\\
	\begin{tikzpicture}[>=latex'] \tikzset{every tree node/.style={align=center,anchor=north}} 
	\Tree [.VP \node(z){DP$_{\text{\lr}}$\\Ana}; [.V\1 V\\belong \node(y){DP$_{\nocase}$\\cap};  ] ] ]   
	\draw[overlay, semithick,<-*] (y.south)..controls +(south west:2) and +(south west:2).. node [midway,fill=white] {\textsc{dat}} (z.south); 
	\useasboundingbox (current bounding box.north west) rectangle ([yshift=-2.5ex] current bounding box.south east); 
	\end{tikzpicture}\\
				\vspace{0.5cm}
    \z
	
% 	\columnbreak
	
	\ea\label{16:unaccusatives2} \textbf{Lexical dative at \littlev{}P}\leavevmode\vadjust{\vspace{-\baselineskip}}\newline\\
	\begin{tikzpicture}[>=latex'] \tikzset{every tree node/.style={align=center,anchor=north}} 
	\Tree [.\textit{v}P \node(x){DP$_{\nocase}$\\cap}; [.\textit{v}P \textit{v}\\belong [.VP \node(z){DP$_{\text{\hr\lr}}$\\Ana}; [.V\1 V \node(y){t$_{\mbox{\scriptsize{DP}}}$};  ] ] ]]]   
	\draw[overlay, semithick,<-] (x.south west)..controls +(south west:4) and +(south west:3).. node [midway,fill=white] {\ding{172}} (y.south); 
	\draw[overlay, semithick,*->] (z.south)..controls +(south west:2) and +(south west:2).. node [midway,fill=white] {\ding{173} \textsc{ dat}} (x.south); 
	\useasboundingbox (current bounding box.north west) rectangle ([yshift=-2.5ex] current bounding box.south east); 
	\end{tikzpicture}\\
		\vspace{0.5cm}	
    \z
	
	
% \end{multicols}	

\noindent In conclusion, treating these constructions as unaccusatives correctly captures the fact that they cannot passivize and that the DP$_{\nomm}$ is interpreted as a \textsc{theme} rather than \textsc{agent}, thereby enabling a unified treatment of lexical and structural dative as dependent case.

\subsection{An extension: (\textit{feel})-\textit{like}-type verbs as unaccusative ditransitives}

(\textit{Feel})-\textit{like}-type verbs select for an \textsc{experiencer}-type dative argument, as in \REF{16:ex38}. The unmarked word order seems to be \datt{} \before{} \nomm.

\ea[]{\gll Ani se svidja zelena haljina.\\
Ana.\datt{} \textit{se} appeals green.\nomm{} dress.\nomm\\
\glt `Ana likes the green dress.'}\label{16:ex38}
\z

\noindent As with the previous group, no additional overt accusative arguments can be added to this structure. Moreover, a structure like this cannot be passivised \REF{16:ex39}. 

\ea[*]{\gll Ani je bilo svidjano.\\
Ana.\datt{} is been appealed.\nsg\\
\glt `It was appealed to Ana.'}\label{16:ex39}
\z

\noindent The lack of passivization possibility and the overt accusative argument, together with the theme interpretation of the \nomm{} argument suggest that this could be an unaccusative contruction. The \textit{se} clitic, however, does not have a reflexive interpretation, but following \citet{progovac13}, it can be assumed to be an expletive object pronoun. Based on the fact that these verbs cannot assign accusative and that the DP$_{\nomm}$ is ambiguous between subject and object interpretation, \citet{progovac13} argues that the structures like these are in fact instances of an ergative-absolutive pattern in a language like Serbian. Such sentences would be analysed as in \REF{16:unaccusatives} and \REF{16:unaccusatives2} above. The \lr{} feature is assigned to the higher DP at the VP level via \down, while the \hr{} feature is assigned at the \littlev{}P level via \up. We leave the exact nature of the clitic \textit{se} in these constructions for future research, which should be able to tell whether it is an additional silent argument that absorbs certain case features, or whether it is an expletive.	

\section{Conclusion}

Dependent case assignment can be formalised by means of a derivational approach, where case features are assigned incrementally, via an Agree operation which holds between two DPs. \datt{} is assigned as high dependent case in the VP, while \accc{} is the low dependent case in the \littlev{}P. We have seen evidence from Serbian that the account of structural \datt{} can be extended to cover the assignment of lexical \datt. Lexical dative is thus assigned in the same configurations: (i) in a ditransitive double-object construction with a silent DP as DO and a case competitor, (ii) in a double object construction involving an unaccusative verb. In its strictest form therefore, the Dependent Case Theory can capture assignment of both lexical and structural dative case as dependent case.

\section*{Abbreviations}

\begin{tabularx}{.5\textwidth}[t]{@{}lQ@{}}
1& first person\\
2& second person\\
{$\downarrow$\textsc{Agr}$\downarrow$}&Downward Agree\\
{$\uparrow$\textsc{Agr}$\uparrow$}&Upward Agree\\
{[+hr]}&lower role\\
{[+lr]}&higher role\\
\textsc{abs}&absolutive\\
\textsc{acc}&accusative\\
\textsc{dat}&dative\\
\textsc{erg}&ergative\\
\end{tabularx}%
\begin{tabularx}{.5\textwidth}[t]{@{}lQ@{}}
\textsc{f}&feminine\\
\textsc{gen}&genitive\\
\textsc{m}&masculine\\
\textsc{n}&neuter\\
\textsc{nom}&nominative\\
\textsc{pl}&plural\\
\textsc{poss}&possessive\\
\textsc{prt}&particle\\
\textsc{refl}&reflexive\\
\textsc{sg}&singular\\
&\\
\end{tabularx} 


\section*{Acknowledgements}
For their helpful comments, suggestions and feedback, we would like to thank two anonymous reviewers, the editors, as well as the audiences at FDSL 12 and at the University of Leipzig. This work was completed as part of the DFG-funded graduate school Interaktion Grammatischer Bausteine `Interaction of Grammatical Building Blocks' (IGRA).

\sloppy
\printbibliography[heading=subbibliography,notkeyword=this]

\end{document}







 







