\documentclass[output=paper, modfonts,newtxmath,hidelinks]{langscibook} 
% % \bibliography{localbibliography}

% % \usepackage{amsmath}
%\usepackage{bbding}% add all extra packages you need to load to this file  
\usepackage{csquotes}
% \usepackage{draftwatermark}
%\usepackage{draftwatermark}
\usepackage[main=english,
%                  czech, %% Check for newer Version of [czech] in babel
                 russian,
                 ngerman,
                 polish,
            ]{babel}
\usepackage{eurosym}
\usepackage{fixltx2e}
\usepackage{float}
% \usepackage[german,english]{babel}
\usepackage{hhline}
\usepackage{./langsci/styles/jambox}
\usepackage{langsci/styles/langsci-cgloss}
\usepackage{./langsci/styles/langsci-lgr}
% \usepackage{langsci-linguex}
\usepackage{./langsci/styles/langsci-optional}
\usepackage[linguistics]{forest}
\usepackage{longtable}
% % \usepackage{marvosym} % incompatible
\usepackage{multicol}
\usepackage{multirow}
\usepackage[normalem]{ulem}
\usepackage{pifont} %for checkmark and cross
% \usepackage[polish,czech,english]{babel}
% \usepackage[russian,english]{babel}
\usepackage{slantsc} %needed for slanted smallcaps
\usepackage{stmaryrd} %defines \llbracket and \rrbracket, needed for semantic interpretation brackets [[.]]
\usepackage{subfigure}
\usepackage{tabto}
\usepackage{tabularx} 
\usepackage{qtree}
\usepackage{tikz-qtree}
\usepackage{tikz-qtree-compat}
\usetikzlibrary{arrows,arrows.meta,decorations.markings,shapes,calc,fit}
\usepackage{url}
\usepackage{vwcol}
\usepackage{wasysym}%symbols
\usepackage{siunitx}
\makeatletter
\let\pgfmathModX=\pgfmathMod@
\usepackage{pgfplots,pgfplotstable}%
\let\pgfmathMod@=\pgfmathModX
\makeatother
\usepgfplotslibrary{colorbrewer,groupplots} 
% % \usepackage{MdSymbol} % \ngg, \gg
\usepackage{langsci-gb4e}

% % 
% %  
\makeatletter
\let\thetitle\@title
\let\theauthor\@author 
\makeatother

\newcommand{\togglepaper}{ 
  \bibliography{../localbibliography}
  \papernote{\scriptsize\normalfont
    \theauthor.
    \thetitle. 
    To appear in: 
    Radek et al ...
    Formal ....
    Berlin: Language Science Press. [preliminary page numbering]
  }
  \pagenumbering{roman}
}


\title{Russian case inflection: Processing costs and benefits}  

% \papernote{\footnotesize\normalfont
% Maria D. Vasilyeva. Russian case inflection: Processing costs and benefits. To appear in: Denisa Lenertová, Roland Meyer, Radek Šimík \& Luka Szucsich (eds.), \textit{Advances in formal Slavic linguistics 2016}. Berlin: Language Science Press. [preliminary page numbering]
% }

%\setcounter{chapter}{18}

\author{Maria D. Vasilyeva \affiliation{Lomonosov Moscow State University}
}

% \chapterDOI{} %will be filled in at production
% \epigram{}

\abstract{Mechanisms underlying the processing and storage of morphological case are still debatable in psycholinguistics. The key questions concern the nature of the special status of the nominative, the homogeneity\slash heterogeneity of oblique case forms, the impact of case syncretism and paradigmatic relations on nominal processing and the organization of the mental lexicon. We investigate these issues turning to Russian nominal processing. We performed two experiments with feminine and masculine nouns in different cases (experiment 1: nouns in singular, experiment 2: nouns in plural) using the visual lexical decision task. In this task, we measure the speed and accuracy with which the participant classifies sequences of letters as words or non-words. Evidence from both experiments indicates that differences in processing exist not only between the nominative and the other case forms, but also among the obliques. Experiment 1 points to the influence of wordform and exponent ambiguity, while experiment 2 reveals effects that are specific for case per se.  We discuss the role of zero vs. overt phonological form, grammatical features, (non-)accidental homonymy, context, frequency, inflectional and relative entropy in case recognition.}

\begin{document}
\maketitle
\shorttitlerunninghead{Russian case inflection: Processing costs and benefits}

\section{\label{sec:intro}Introduction}
The role of frequency and regularity in processing of inflectional morphology has for long been of utmost concern for psycholinguists. Meanwhile, it is still not clear whether grammatical features that an inflectional marker conveys play an additional role in wordform processing. For instance, if we are speaking about nouns, a natural question to ask is how case influences nominal recognition.

Studies of isolated wordform processing suggest that nominative wordforms are processed faster than other case forms (see, e.g. \citealt{lukatela1978lexical}  for Serbian; \citealt{niemi1994cognitive} for Finnish; \citealt{abulizi2016cognitive} for Uyghur; \citealt{gor2017processing} for Russian). Yet, there is no uniform explanation of this fact. Likewise, it is debatable whether oblique cases entail equal processing costs or not.

Finnish and Uyghur researchers provide only  the pooled mean for all the inflectional variants, comparing it to the nominative and do not inspect contrasts between oblique forms, though they usually use more than one oblique case in their experiments \citep{niemi1991recognition, niemi1994cognitive, hyovcnavc1995effects, laine1998lexical, laine1999lexical, abulizi2016cognitive}. As the nominative has zero inflection in these languages, oblique processing cost is attributed to a morphological decomposition procedure that is obligatory for inflected obliques, but absent in the non-inflected nominative.

This explanation is unsatisfactory for several reasons. Firstly, the nominative advantage disappears when  case forms are embedded in context  \citep{bertram2000role, hyona2002morphological}. \citet{bertram2000role} and \citet{hyona2002morphological} suggest that oblique processing disadvantage in a context-less environment arises not due to the decomposition cost, but precisely due to the lack of an appropriate context. Yet, they do not examine if all oblique cases suffer from the lack of context or benefit from its presence to the same extent. Secondly, processing of zero inflection receives a benefit in recognition speed only if the zero is associated with the nominative, but not with an oblique case \citep{gor2017processing}.  Finally, phonologically zero and overt nominatives appear not to differ in processing speed (see, e.g., \citealt{lukatelaEtAl1980} for Serbian; \citealt{gor2017processing} for Russian). Thus, it is not the zero inflection that makes Finnish and Uyghur nominative wordforms special, but the the nominative case itself.

Early Serbian studies did compare processing of oblique cases, but mainly failed to find significant differences in response latencies \citep{lukatela1978lexical, lukatelaEtAl1980, lukatela1987lexical, katz1987grammatical, kostic1987processing, feldman1987inflected}. These results, starting with \citet{lukatelaEtAl1980}, were analyzed within the satellite model. The nominative form represents the nucleus of the nominal paradigm, while oblique case forms surround it as satellites. Satellites are assumed to be equidistant from the nucleus \citep{feldman1987inflected}. Deviations from the predictions of this model were attributed to specific experimental settings in case of nouns \citep{feldman1987inflected, todorovic1988hemispheric}; differences in adjectival case processing were assumed to rely on different mechanisms \citep{kostic1987processing}. However, not more than three case forms belonging to one number were compared at once. It is likely that some effects that could show up in a more elaborate design were obscured. Moreover, ambiguity of case forms that is present in Serbian declension did not receive enough attention. 

According to subsequent Serbian studies, processing speed of a wordform correlates positively with the number of syntactic functions\slash meanings that its inflectional ending encompasses \citep{kostic1991informational, kostic1995information, kostic2003inflectional, filipovic2003processing,vseva2003annotated}, which hints at oblique processing differences. The proposed methodology of calculating syntactic functions\slash meanings is not flawless, since the authors bring under this umbrella term both syntactic notions such as subject or complement and semantic notions such as instrument or goal. Furthermore, it is not taken into account that lexemes in the same case have different probabilities of expressing the same thematic role, e.g. `girl-\ins' is less likely to be an instrument than `hammer-\ins'. Likewise, as all homonymous forms are treated equally, differences between accidental and non-accidental ambiguity is disregarded.

Later on, this problematic measure was abandoned, and the focus was shifted to paradigmatic relations between wordforms captured by inflectional and relative entropy measures (see, e.g. \citealt{milin2009simultaneous}). The inflectional entropy $H(P)$ reflects the amount of information associated with the inflectional paradigm of the target lexeme (see~\REF{equation:inflEntropy}), where $f$ stands for frequency, the wordform $w_i$ belongs to the paradigm $P$ of a lexeme $w$) and correlates negatively with response latencies: when a lexeme has a higher value of the inflectional entropy, its wordforms are processed faster, and vice versa  \citep{del2004putting}. The relative entropy $D(IP||IC)$ captures the divergence between the frequency distribution of the target lexeme $w$ and the frequency distribution of its inflectional class $IC$ (see~\REF{equation:relEntr}), where $e_i$ stands for inflectional exponent), and it correlates positively with response latencies: wordforms belonging to paradigms with higher values of relative entropy are processed more slowly \citep{milin2009simultaneous}. When surface and lemma frequency combined with entropy measures are taken into account, case differences appear to play no additional role \citep{milin2009simultaneous}; yet, this claim was made on a small subset of wordform: wordforms in -\textit{u} `{\accc.\sg}’ and -\textit{e} ‘{\genn.\sg}’\slash ‘{\nomm/\accc.\pl}’ for feminine nouns, wordforms in -\textit{om} `{\ins.\sg}’ and -\textit{u} `{\datt/\locc.\sg}’ for masculine nouns.

	\ea\label{equation:inflEntropy}\begin{equation*}
		H(P) = -\sum_{w_i \in P} \frac{f(w_i)}{f(w)} \log _2\frac{f(w_i)}{f(w)}
        \end{equation*}
	\z

\ea\label{equation:relEntr}\begin{equation*}
		D(IP||IC) = \sum_{w_i \in P} \frac{f(w_i)}{f(w)} \log _2 \frac{f(w_i)/f(w)}{f(e_i)/f(e)}
        \end{equation*}
	\z

\noindent Another viewpoint predicting differences in oblique case processing and paying attention to wordform ambiguity was developed primarily by \citet{clahsen2001mental}. They adopted minimalist principles in morphology (see, e.g., \citealt{wunderlich1996minimalist}), suggesting that in the mental lexicon, the meaning of a case exponent is represented as a set of binary features. Non-accidental ambiguous inflectional markers receive underspecified representations. Along with this ``natural'' underspecification, radical underspecification is assumed to be present as well: only positive values are stored in the mental lexicon, while negative ones are deduced from paradigmatic oppositions. Hence, a direct implication for the psycholinguistic models of wordform processing arises. The number of specified (positive) features should determine the processing ease: the more information a form carries, the longer it takes to be recognized.

Main evidence supporting this claim comes from studies on German adjectival declension. Adjectival case forms with more specified representations are recognized slower in the lexical decision task \citep{clahsen2001mental}. Such case forms show reduced priming effects under cross-modal priming if the adjective serving as a prime does not share all the positive features with the target \citep{clahsen2001mental}. Similar priming effects are to a certain extent replicable even with highly proficient L2-German speakers \citep{bosch2016accessing, bosch2017time}. As far as sentence processing is concerned, when an ungrammatical sentence contains an adjective or a determiner that is compatible with the context by its feature specification, this does not lead to an ungrammaticality effect in a sentence matching task, observed for ungrammatical sentences where specificity is violated, i.e. when the feature set of the wordform mismatches context requirements \citep{penke2004psycholinguistic}. These two types of ungrammatical sentences result in distinct ERP responses \citep{opitz2013neurophysiological}. 

If the radical underspecification hypothesis is true, the same principles should hold for nominal case inflection in other languages as well. Yet, prior studies on case processing shed doubts on its tenability, and additional evidence is needed.

\section{Present study}\label{sec:2}
The present study aims to verify whether case form processing is determined by the grammatical features, nominative vs. oblique dichotomy, or context. We addressed this issue in two lexical decision task experiments employing Russian data: experiment 1 with singular nouns and experiment 2 with plural nouns.

Russian was not chosen incidentally, but due to its particular pattern of case syncretism (convergence of inflectional exponents in different paradigmatic cells). Russian has six major cases and several inflectional classes of nouns. We will restrict ourselves to inanimate nouns and discuss only two most productive inflectional classes \citep{wiese2004categories}: feminine nouns with the nominative ending -\textit{a} and masculine nouns with the nominative  ending -$\varnothing$. As is evident from \tabref{tab:rusDecl}, the two classes of nouns choose uniform endings in plural (except for genitive), but behave differently in singular.

\begin{table}
    \centering
	\caption{Russian case endings for the two most productive inflectional classes}
    \label{tab:rusDecl}
	\begin{tabular}{*{8}l}
		\lsptoprule
        ~	&	~	&	Nom    &	Acc	 & Gen  &	Dat &	Loc &	Ins	\\
        \midrule
		\multirow{2}{*}{Singular} & Feminine	&	-a  &	-u	&	-y	&	-e	& -e    &	-oj\\
		&	Masculine			&	-$\varnothing$  & -$\varnothing$    &	-a	&	-u	&   -e  &	-om	\\
        \midrule
        \multirow{2}{*}{Plural} & Feminine	& -y    & -y    &	-$\varnothing$ & -am  & -ax & -ami \\
        &	Masculine & -y  & -y &	-ov & -am    & -ax & -ami\\
		\lspbottomrule
    \end{tabular}
\end{table}

\paragraph*{Context}. Presenting case forms in isolation, we can test whether all oblique case forms  rely equally on the context. Russian data is particularly suitable for resolving this issue, as there is a special case in Russian, namely locative (also called prepositional), which, unlike other cases, is always governed by a preposition. If the context is crucial for efficient oblique case recognition, locative wordforms should be processed longer compared to other oblique cases, since the latter do not need any preceding context on the left (e.g., if they occur at the beginning of a sentence). This hypothesis is partly supported by \citegen{vasilyevaEtAl2014} finding: masculine locative singular wordforms are processed as slowly as pseudowords with the same syllabic structure, and they are often qualified as nonwords. However, in the singular form, this processing cost could be caused by the homonymy of -\textit{e} `\locc.\masc' with  -\textit{e} `\datt/\locc.\fem'. If the effect is induced by the lack of prior preposition activation, locative plural processing should also be impaired. If locative plural processing is not more difficult than processing of other obliques, difficulty of masculine locative singular can not be explained by the absence of an appropriate preposition alone and, in general, the context-based hypothesis is not tenable. 

\paragraph*{Plural: case features vs. exponent frequency}. As oblique plural exponents are non-ambiguous, these data are fruitful for exploring the role of case in nominal processing. If surface and lemma frequency are accounted for and differences in oblique case processing still arise, they could be attributed either to exponent frequencies or to the set of grammatical features associated with the particular case. Frequency counts provided in \citet{samojlova2014frequencies} suggest a hierarchy in~\REF{ex:freqpl}. Different approaches to Russian declension employ different sets of features and, thus, give conflicting predictions see \REF{ex:mueller}--\REF{ex:caha}, where we arrange oblique cases according to the number of positive features they express (as suggested by \citealt{clahsen2001mental}).

\ea \ea \label{ex:freqpl} frequency: Gen < Ins < Loc < Dat (34\% < 11\% < 10.3\% < 4.7\%)
    
    \ex \label{ex:mueller}  \citealt{muller2004decomposing}: Loc < Dat $\approx$ Ins < Gen ($\langle$+obl$\rangle < \langle$+obl, +gov$\rangle \approx \langle$+obl, +subj$\rangle < \langle$+obl, +gov, +subj$\rangle$)
    
    \ex \label{ex:wiese}	\citealt{wiese2004categories}:  Loc < Ins $\approx$ Dat $\approx$ Gen ($\langle$+obl$\rangle < \langle$+obl, +inst$\rangle \approx \langle$+obl, +dat$\rangle \approx \langle$+obl, +gen$\rangle$)
    
    \ex \label{ex:wunderlich}	\citealt{wunderlich1996minimalist}: Gen < Dat < Loc < Ins ($\langle$+hr$_N \rangle < \langle$+hr, +lr$\rangle < \langle$ +hr \& additional semantic features $\rangle < \langle$ semantic features$\rangle$) 
)
    
    \ex \label{ex:caha} \citealt{caha2008case}: Gen < Loc < Dat < Ins
    \z
\z

\subparagraph*{{Zero oblique inflection}}. \citegen{gor2017processing} study demonstrated that oblique overt and zero inflection trigger similar processing costs. But their conclusion was based on the comparison of feminine -$\varnothing$ `\genn.\pl' to masculine -\textit{a} `\genn.\sg'. A comparison with masculine -\textit{ov} `\genn.\pl' is needed to support their claim.

\subparagraph*{{Nominative ambiguity}}. Feminine \textit{-y} `\nomm.\pl' coincides with `\genn.\sg'. According to the approach advocated by \citet{kostic1991informational}, etc., such ambiguous wordforms should benefit from their wider syntactic distribution and be recognized faster than their unambiguous masculine counterparts -\textit{y} `\nomm.\pl'.

\paragraph*{Singular: case syncretism}. Even if we obtain no significant differences in plural oblique processing, differences in singular oblique processing might arise due to ambiguity. Comparing instrumental wordforms, which are non-ambiguous, to other obliques, we can determine how interparadigmatic and intraparadigmatic syncretism influences wordform recognition. Furthermore, comparisons of wordforms with the same exponents, but belonging to different inflectional classes might help to  resolve the debates concerning accidental vs. non-accidental homo\-nymy in Russian singular declension.

If the two -\textit{u}-s are accidentally homonymous  \citep{wiese2004categories}, `\accc.\fem' is expected to be processed faster than`\datt.\masc'. If the two -\textit{e}-s are accidentally homonymous \citep{muller2004decomposing}, `\datt./\locc.\fem' is expected to be processed faster than `\locc.\masc'.

We also decided to compare -\textit{e} `\datt/\locc.\fem' to -\textit{u} '\datt.\masc'. If the dative reading is dominant for -\textit{e}, there should be no difference between these two conditions. Finally, we compared feminine and masculine instrumental wordforms. These endings are also used in adjectives of the respective gender, but their distribution is different: feminine -\textit{oj} covers all oblique cases, while masculine -\textit{om} is used in locative only. This difference might lead to an advantage of feminine instrumental over masculine instrumental.\footnote{ Since feminine genitive singular -\textit{y} is homonymous with nominative plural, we did not compare it to the masculine genitive singular.}

\subparagraph*{{Zero nominative inflection.}} Russian overt and non-overt nominative inflection (-\textit{a} `\nomm.\fem' and -\textit{$\varnothing$}`\nomm.\accc.\masc') was already compared in \citet{gor2017processing}, and no difference was observed. However, their study employed an auditory lexical decision task, and it is unclear whether their results are modality-neutral.

\section{Method}
\subsection{Participants} Ninety-six Russian native speakers, all right-handed (aged 17–25 years) were tested. Half of them participated in experiment 1, the other half in experiment 2.

\subsection{Stimuli} We used all six case forms of inanimate nouns belonging to two declensional classes (54 feminine nouns ending in -\textit{a} and 54 masculine nouns ending in -\textit{$\varnothing$} matched for lemma frequency). All stimuli were base nouns, they did not undergo any stem alternations and had fixed stress on the stem (the 1a inflectional class according to \citealt{zaliznyak1977grammatical}). Length in nominative differed from 4 to 6 (each group comprised one third of words with each length). 108 nouns with pseudoendings and 108 inflected pseudostems served as nonwords. In experiment 1, nouns were presented in singular; in experiment 2, in plural. Latin-square design was employed with the number of lists corresponding to the number of case forms.

\subsection{Procedure} Each participant was assigned to one of the six experimental lists and was tested individually. Experiments were run using DMDX software (Forster, Forster, 2003). Before the test phase (324 trials), participants received written instructions and performed a practice phase (20 trials). In each trial, participants had to decide whether the string of letters presented on the screen was a real Russian word or not. They were instructed to respond as fast and accurately as possible. Each trial started with a fixation sign (+) that was displayed on the screen for 600 ms. The stimulus remained on the screen until response or time-out (2500 ms). The interstimulus interval was set to 2500 ms.

\subsection{Data analysis} We used linear mixed-effects modeling for the analysis of reaction times and logistic mixed regression for the accuracy data  \citep{baayen2008analyzing}. Statistical analysis was implemented in the package lme4  \citep{bates2014lme4} in the statistical software R \citep{team2014r}. \textit{T}-values, \textit{z}-values, \textit{p}-values, and standard errors were determined using the package lmerTest \citep{kuznetsova2015package}. Fixed and random effects were included only if they significantly improved the model fit in a backward stepwise model selection procedure. Models were selected using Chi-square log-likelihood ratio tests with regular maximum likelihood parameter estimation. 

Subject and lexeme were treated as random effects. Lemma and wordform frequency, length in letters and syllables, mean Levenstein distance to the nearest 20 lexeme-neighbors, 
inflectional and relative entropy measures were additionally included as covariates.\footnote{Lemma frequency was taken from the frequency dictionary \citep{lyashevskaya2009frequency}. Wordform frequency was manually extracted from the main undisambiguated subcorpus of Russian national corpora\footnote{\url{http://ruscorpora.ru}}; for ambiguous endings the cumulative frequency was taken,  relying on \citegen{milin2009simultaneous} experience. To avoid zero frequencies, one was added to all counts, as suggested by \citet*{brysbaert2013dealing}.}$^,$\footnote{The Levenstein distance was calculated in the vwr package \citep{keuleers2013vwr} in the R software \citep{team2014r}.}$^,$\footnote{In order to calculate relative entropy, frequency of exponents was taken form the database created by \citet*{samojlova2014frequencies}.} Trial order (\textit{z} transformation on log numbers) was included to control for longitudinal task effects such as fatigue or habituation. All these covariates were log-transformed. To avoid multicollineriarity, all counts except for trial were transformed into 5 principal components, explaining 93.5\% of variance \citep{baayen2008analyzing}. The first principal component (PC1) captured orthographic characteristics of the stimulus. The second component (PC2) was inversely related to frequency. The third component (PC3) was inversely related to relative entropy and positively related to inflectional entropy. Paired contrasts were carried out in the package lsmeans \citep{lenth2016least}. For planned comparisons, FDR adjusted p-values are reported \citep{benjamini1995controlling}. 

As our words were presented without context, case labels for ambiguous endings (feminine -\textit{y} and -\textit{e}, masculine -\textit{y} and -\textit{$\varnothing$}) are somewhat arbitrary. Hence, we do not expect any differences between feminine locative and dative -\textit{e}, nor between masculine nominative and accusative -\textit{$\varnothing$}. However, this is needed for counterbalancing issues, as patterns of syncretism do not coincide across our two noun groups. In the statistical analysis, the mean pooled over the two ``conditions'' will be used.

\section{Results}
Two participants in experiment 1 and two participants in experiment 2 gave fewer than 75\% correct answers to word stimuli, so we recruited four additional people to replace them. We excluded from further  statistical analysis two lexemes in experiment 1 and one lexeme in experiment 2 due to low mean accuracy score.

Reaction time (RT) data were analyzed as follows. Incorrect responses were removed from the analysis (7.1\% of all data in experiment 1, 7.9\% in experiment 2). Too fast (< 300 ms) or too slow responses (> 1 500 ms) were likewise excluded from further analysis. We applied log-transformation to reduce the positive skew. After that, remaining outliers were cut off via interquartile trimming.\footnote{We kept only those RTs, which satisfied the following formula Q1 – (2.5 × IQR) < RT < Q3 + (2.5 × IQR), by participants, items (lexemes), gender and case (Q1 stands for first quartile, Q3 for third quartile, and IQR = Q3 – Q1 for interquartile range).} In sum, 3.9\% of correct responses were removed in experiment 1 and 5.7\% in experiment~2.

Raw RTs and error rates (ER) are presented in \tabref{tab:descrSg}.
\begin{table}
    \caption{Mean RT (in ms) and ER (in \%) to feminine and masculine nouns in different cases and numbers (\textit{SD} is provided in brackets)}
    \label{tab:descrSg}
    \resizebox{\textwidth}{!}{\begin{tabular}{ll *{6}c}
        \lsptoprule
        ~   &   ~   &    \multicolumn{1}{c}{Nom}    &   \multicolumn{1}{c}{Acc} &   \multicolumn{1}{c}{Gen} &  \multicolumn{1}{c}{Dat} &   \multicolumn{1}{c}{Loc}  &   \multicolumn{1}{c}{Ins}\\
        \midrule
        \multicolumn{8}{c}{Experiment 1: singular}\\
        \midrule
        f   &   RT  &   738 (181)   &   782 (198)     &   764 (195)    & \multicolumn{2}{c}{ 805 (203)} &   770 (194)\\
        ~   &   ER  &   1.7 (12.9)  &   4.8 (21.4)    &   4.6 (20.9)   &   \multicolumn{2}{c}{9.3 (29)} &   3.9 (19.3)\\
        m   &   RT  &   \multicolumn{2}{c}{719 (176)} &   838 (228)    &   799 (203)    &   850 (231)   &   803 (219)  \\
        ~   &   ER  &   \multicolumn{2}{c}{2.3 (15)}  &   8.3 (27.7)   &   6.3 (24.2)   &  20.8 (40.7)  &   2.3 (15.1) \\
        \midrule
        \multicolumn{8}{c}{Experiment 2: plural}\\
        \midrule
        f   &   RT  &   \multicolumn{2}{c}{818 (202)}   & 894  (225)   & 900 (226)   &   873 (216)   & 894 (230)   \\
        ~   &   ER  &   \multicolumn{2}{c}{4.6 (21)}    & 22.4 (41.7)  & 7.5 (26.4)  &   8.5 (27.9)  & 7.3 (26.1)\\ 
        m   &   RT  &   \multicolumn{2}{c}{821 (196)}   &  875 (209)   & 904 (235)   &   879 (212)   &   932 (234) \\
        ~   &   ER  &   \multicolumn{2}{c}{5.1 (22)}    &  4.2 (20)    & 7.4 (26.2)  &   7.4 (26.2)  &   7.2 (25.8)  \\
        \lspbottomrule
    \end{tabular}}
\end{table}

Final models for RTs and accuracy included the following factors: PC2, PC3, case, gender, a case by gender interaction and a PC3 by gender interaction. The model accounting for RTs in experiment 1 also included trial. All other predictors and interactions turned out to be insignificant. Full model specifications are presented in the Appendix (see \tabref{tab:LMERsg} for experiment~1 and \tabref{tab:LMERpl} for experiment~2).

\subsection{Experiment 1: Singular}

Trial had a facilitative effect on RTs ($B=-0.012$, $t(4541)=-4.45$, $p<.001$). PC2 (inversely related to frequency) had a facilitative effect on RTs and accuracy rate ($B= 0.018$, $t(121)= 6.03$, $p< .001$, respectively and $B= -0.364$, $z= -5.86$, $p< .001$, respectively). 

PC3 (entropy meausures) affected differently the two types of nouns ($B= 0.024$, $t(103)= 3.2$, $p= .002$ and $B= -0.472$, $z= -3.628$, $p< .001$, respectively):  there was a facilitation for feminine nouns ($B= -0.022$, $t(103)= -3.68$, $p< .001$ and $B= 0.509$, $z= 4.89$, $p< .001$, respectively) and no effect on masculine nouns ($B= 0.002$, $t(103)= 0.48$, $p= .633$ and $B= -0.11$, $z= 0.481$, $p= .631$, respectively).

\paragraph*{Paired contrasts for different cases} are summarized in \tabref{tab:caseSgHierarchy} (for statistical details see \tabref{tab:caseSgComparisons}). Apart from nominative vs. oblique differences, we observe differences between oblique case forms. Instrumental and the -\textit{u} form (`\accc.\fem' and `\datt.\masc') are ``easy'' obliques with faster responses and higher accuracy scores. The -\textit{e} forms (`\datt/\locc.\fem' and `\locc.\masc') constitute the ``difficult'' oblique group with slower responses and lower accuracy scores. Feminine genitive falls into the ``easy'' group, while masculine genitive patterns with the difficult -\textit{e} `\locc' form.\footnote{In \tabref{tab:caseSgHierarchy} and further ``<'' stands for significantly faster or significantly more accurate responses, ``$\approx$'' stands for no significant difference.}

\begin{table}
	\caption{Experiment 1 (singular nouns): summary of paired contrasts analysis for feminine and masculine singular nouns in different cases}
    \label{tab:caseSgHierarchy}
    \begin{tabular}{*{3}l}
    \lsptoprule
    ~	&	\multicolumn{1}{c}{RTs}	&	\multicolumn{1}{c}{accuracy}\\
    \midrule
    f	&	Nom < Ins $\approx$ Gen $\approx$ Acc < -\textit{e}	&	Nom < Gen $\approx$ Acc < -\textit{e}, \\
    ~	&	~	&	Nom $\approx$ Ins, Gen $\approx$ Ins $\approx$ Acc, Ins < \textit{-e}\\
    m	&	\textit{-$\varnothing$} < Ins $\approx$ Dat < Gen $\approx$ Loc	&	\textit{-$\varnothing$} $\approx$ Ins < Dat $\approx$ Gen  < Loc			\\
    \lspbottomrule
    \end{tabular}
\end{table}

In the analysis of accuracy, in contrast to the RT data, we fail to observe the nominative superiority over instrumental in any noun group. What is more, according to the accuracy analysis, masculine genitive yields higher accuracy rate than the masculine locative. 

\paragraph*{Paired contrasts for gender} (for statistical details see \tabref{tab:caseSgComparisons}). There was no general gender effect either in the RT or accuracy analysis. Feminine -\textit{e} forms (`\datt/\locc.\fem') are recognized faster and more accurately than masculine -\textit{e} forms (`\locc.\masc'). Feminine instrumental wordforms are recognized faster than masculine ones, but there is no effect in the accuracy analysis. Feminine nominative is responded to slower than masculine -$\varnothing$ forms (`\nomm/\accc.\masc'); no effect shows up in the accuracy analysis. Feminine -\textit{e} forms (`\datt/\locc.\fem') are recognized less accurately than masculine -\textit{u}  forms (`\datt.\masc'), but there is no significant difference in the RT analysis. There is no significant difference between feminine and masculine -\textit{u} forms (`\accc.\fem' and `\datt.\masc', respectively).

\subsection{Experiment 2: Plural}

PC2 (inversely related to frequency) had a facilitative effect on RTs and accuracy rates ($B= 0.019$, $t(126)= 4.98$, $p< .001$ and $B = -0.285$, $z= -4.21$, $p< .001$, respectively). 

PC3 (entropy measures) affected differently the two types of nouns ($B= 0.021$, $t(107)= 2.28$, $p= .025$ and $B = -0.423$, $z= -2.727$, $p= .006$, respectively): there was a facilitation for feminine nouns ($B= -0.022$, $t(111)= -2.91$, $p= .004$ and $B= .442$, $z= 3.7$, $p< .001$, respectively), but no effect on masculine nouns ($B= -0.0004$, $t(101)= -0.06$, $p= .949$ and $B= 0.019$, $z= 0.19$, $p= .847$, respectively).

\paragraph*{Paired contrasts for case} are summarized in ~\tabref{tab:casePlHierarchy} (for statistical details see~\tabref{tab:casePlComparisons}). According to the RT analysis, we observe a tripartite division of oblique forms: locative as the easiest, dative in the middle and instrumental as the most difficult. Genitive is recognized significantly faster than instrumental, but differs neither from locative, nor from dative. 

In the accuracy analysis, only two contrasts are retained: between the -\textit{y} form (`\nomm/\accc') and instrumental and the difference between the -\textit{y} form (`\nomm/\accc') and genitive.

\begin{table}
    \centering
	\caption{Experiment 2 (plural nouns): summary of paired contrasts analysis for plural nouns in different cases}
    \label{tab:casePlHierarchy}
    \begin{tabular}{*{2}l}
    \lsptoprule
    \multicolumn{1}{c}{RTs}	&	\multicolumn{1}{c}{accuracy}\\
    \midrule
    -\textit{y}	< Loc < Dat < Ins,	&	-\textit{y} < Gen $\approx$ Ins \\
    Gen < Ins,	&	-\textit{y} $\approx$ Loc $\approx$ Dat\\
    Gen $\approx$ Loc, Gen $\approx$ Dat & Loc $\approx$	Dat $\approx$ Gen $\approx$  Ins\\
    \lspbottomrule
    \end{tabular}
\end{table}    

\paragraph*{Paired contrasts for gender} (see \tabref{tab:caseSgComparisons} for statistical details). There was no difference between two groups of nouns either in the RT or accuracy analysis. In genitive, feminine  nouns have higher odds to be recognized incorrectly than masculine nouns; no significant difference shows up in the RT analysis. Feminine and masculine -\textit{y} forms (`\nomm/\accc') differ neither in the RT analysis, nor in the accuracy analysis. 

\section{Discussion}
Results of our two experiments replicate the nominative\slash oblique dichotomy effect, previously reported for Russian and other languages (see \sectref{sec:intro}). Apart from this trivial finding, we obtained several significant differences between oblique case processing both in singular and in plural. As we took into account lemma and surface frequency of a wordfom, such oblique case processing differences should stem from the properties of the inflectional exponents.\\

\paragraph*{Inflectional and relative entropy.} We considered inflectional and relative entropy among potential covariates in the statistical analysis, as these factors were assumed to be highly predictive of nominal processing in Serbian \citep{milin2009simultaneous}. Prior to the analysis, we transformed our counts into principal components. PC3 capturing these two measures emerged in the statistical analysis of RTs and accuracy in both experiments. Unfortunately, the influence of PC3 was attested for feminine nouns only. The effect lies in the same direction as reported by \citet*{milin2009simultaneous}, but in the Serbian study masculine and feminine nouns were equally sensitive to entropy measures. However, to calculate the entropy values, they used frequencies of feminine exponents, as this inflectional class is assumed to be dominant in Serbian. In Russian, masculine -$\varnothing$ nouns are slightly more frequent than feminine -\textit{a} nouns \citep{samojlova2014frequencies} and, thus, might be considered dominant. However, as the patterns of syncretism in these two noun groups do not coincide, we decided against using dominant class frequencies and employed feminine frequencies for feminine nouns and masculine frequencies for masculine nouns. This decision might be a possible reason for the observed discrepancies with \citet{milin2009simultaneous}, but a more refined study is needed in order to make more solid conclusions.\\

\paragraph*{Context: the locative issue}. Initially, we hypothesized that if absence of context is an important source of oblique processing cost, preposition-less locatives should suffer the most, both in singular and in plural. Although masculine locative singular was one of the most difficult forms to recognize, plural locative was processed faster than all other obliques. Thus, we conclude that context-based explanations do not receive support at least for wordforms with non-ambiguous case markers. Influence of context on forms with ambiguous case exponents will be discussed below.\\

\paragraph*{Plural.} The hierarchy of plural case processing speed~\REF{ex:plHierarchy} does not follow the order of exponent frequency: otherwise, instrumental would have been the easiest to process. So we can conclude that exponent frequency does not play a major role in the case form recognition. Nor does this hierarchy agree with the predictions derived from the frequency of exponents and feature sets proposed by \citet{muller2004decomposing, wiese2004categories, wunderlich2004there}. Interestingly, it roughly resembles \citegen{caha2008case} nanosyntactic approach to Russian case, see~\REF{ex:cahafunc}.

\ea \ea	\label{ex:plHierarchy} -\textit{y} < Loc < Dat < Ins, Loc $\approx$ Gen $\approx$ Dat, Gen < Ins\hfill (our data: exp. 2)
    \ex \label{ex:cahafunc} \ob Ins [Dat [Loc [Gen [Acc [Nom]]]]]\cb
    \z 
\z

\noindent Here, the only diverging case is genitive. Unlike all other cases in plural, it is spelled out differently for our two target inflectional classes. Hence, at the checking or licensing stage (see, e.g., \citealt{bertram2000role}), which follows the decomposition of the wordform into morphemes,  it is verified whether the inflectional class of the lexeme matches the inflectional class of the ending. For other oblique case forms, such a procedure is not needed, as they are uniform for both classes. As a consequence, we observe longer reaction times than those that could be expected if genitive plural meaning was expressed in only one way.

\subparagraph*{{Zero oblique inflection}}. In line with \citet{gor2017processing}, response latencies for the zero oblique \textit{$\varnothing$} `\genn.\pl' did not differ significantly from the overt oblique -\textit{ov} `\genn.\pl'. Yet, the zero genitive yielded higher error rates than the overt genitive. We doubt that low accuracy stems from a greater processing cost associated with zero inflection compared to overt inflection, especially as this is not attested in the RT analysis. A more plausible source for the high error rate is homonymy. Feminine genitive plural wordforms having a phonologically null ending are ambiguous with the stem itself. This homonymy might lead to a competition in the recognition process: if the wordform reading wins, the correct answer is produced in the lexical decision task; if the stem reading wins, non-word answer is selected, as Russian does not allow for bare stems.

\subparagraph*{{Nominative ambiguity}}. We failed to find evidence supporting the claim that the ambiguous feminine -\textit{y} `\nomm.\pl'\slash `\genn.\sg' is easier to be recognized than the non-ambiguous masculine -\textit{y} `\nomm.\pl' due to its wider distribution.\\

\paragraph*{Singular.} In singular, the following generalization holds for both nouns:

\ea	Nom < Ins $\approx$ -\textit{u} < -\textit{e}, , where -\textit{u} corresponds to `\accc.\fem'\slash `\datt.\masc' and -\textit{e} corresponds to `{\datt/\locc.\fem}’\slash `{\locc.\masc}’
\z 

\subparagraph*{{Instrumental.}} Instrumental singular wordforms, despite their relatively low frequency \citep{samojlova2014frequencies}, are one of the easiest obliques to be recognized due to their unambiguity. In \sectref{sec:2}, we hypothesized that feminine instrumental -\textit{oj} could be processed faster than masculine instrumental -\textit{om} due to their homonymy with adjectival endings, and this prediction was borne out. Masculine -\textit{om} marks different cases in nouns and adjectives (`\ins' vs. `\locc', respectively), and this feature mismatch might negatively affect their processing. Feminine adjectival -\textit{oj}, on the other hand, includes `\ins' as one of its possible interpretations; consequently, no conflict arises.

\subparagraph*{{-U forms}} The -\textit{u} forms (`\accc.\fem' and `\datt.\masc') behave similarly to the unambiguous instrumentals, but this does not signify that their homonymy is accidental. The lack of significant difference in the processing speed of the two forms  is compatible with the hypothesis of a shared underspecified representation, as suggested in  \citet{muller2004decomposing,wunderlich2004there}. However, this evidence is not enough to reject the accidental homonymy hypothesis. A better insight into this problem might be gained if we compare the processing of dative -\textit{u} in the accusative environment and vice versa. If there is one shared representation for the two -\textit{u}-s in the mental lexicon, such sentences, following \citet{penke2004psycholinguistic, opitz2013neurophysiological}, should show reduced ungrammaticality effects, if any.

\subparagraph*{{-E forms.}} The -\textit{e} forms (`\datt/\locc.\fem' vs. `\locc.\masc') are most difficult to process in both noun groups, triggering longer RTs and lower accuracy, masculine -\textit{e} being even more difficult with the slowest reaction times and the highest error rates. Feminine -\textit{e} is largely believed to have a shared semantic representation for its two interpretations \citep{muller2004decomposing, wiese2004categories, wunderlich2004there}. But a shared representation on its own is not a plausible source for such a processing cost. Masculine -\textit{e} wordforms, on the contrary, are not ambiguous, but they are always governed by a preposition, and in the present study locatives were presented preposition-less in the experimental conditions. In experiment 2, locative plural, which is also preposition-dependent, actually, turned out to be one of the easiest oblique cases. Thus, absence of the preposition is not the main reason for poor participants’ performance on singular masculine locatives. 

We suggest that this finding could be accounted for in a model of Russian case where all -\textit{e}-s have one shared representation. The features distinguishing between two cases compete with each other during wordform processing. Locative, as the more frequent reading \citep{samojlova2014frequencies}, has by default more weight, while dative gets more weight in the appropriate context, i.e. in the preposition-less environment. This competition slows down the recognition process. If we assume that the context cue prevails over the frequency cue, then for feminine nouns the dative reading succeeds. With masculine nouns, the context cue will lead to the incorrect selection of the dative features and cause non-word answers. The reanalysis of -\textit{e} as `\locc' is, thus, warranted. As any reanalysis, it requires additional time cost, which explains the superiority of feminine -\textit{e} forms over masculine -\textit{e} forms in the processing speed. 

\subparagraph*{{Genitive.}} Genitive wordforms behave differently in the two inflectional classes. Masculine genitive  pattern together with the difficult -\textit{e} in the RT analysis. Feminine genitive falls in the “easier” oblique group. Both genitive endings are homonymous: feminine genitive -\textit{y} coincides with nominative plural, masculine genitive -\textit{a}~\--- with the feminine nominative -\textit{a}. 

As far as the masculine genitive -\textit{a} is concerned, an analysis similar to the analysis of -\textit{e} forms is plausible. Two -\textit{a}-s (`\genn.\masc' and `\nomm.\fem')
have a shared representation in the mental lexicon. In a context-less condition, the nominative reading is preferred . Shared representation for these morphemes was previously proposed by \citet{muller2004decomposing, wunderlich2004there}. However, this analysis does not capture the fact that masculine genitive is processed more accurately than masculine locative.

As for feminine genitive in -\textit{y}, the unanimous position \citep{muller2004decomposing,wunderlich2004there,wiese2004categories} stands for accidental homonymy. Genitive singular -\textit{y} is more frequent than nominative plural \citep{samojlova2014frequencies}. Thus, fullform storage is more likely for nominative plural, following the suggestion by \citet{bertram2000role}. Fullform access is assumed to be faster than the decomposition route (i.e., \citealt{bertram2000role}), yet we do not have enough evidence to claim that our -\textit{y} forms were always processed as nominative plurals. In the singular environment of experiment 1, the singular reading might be chosen due to interstimulus priming. Nevertheless, whichever interpretation is chosen, it is easier to process than the ambiguous -\textit{e}.

\subparagraph*{{Zero nominative inflection}}. The visual lexical decision task hints at a processing advantage for the phonologically non-overt inflection (-\textit{$\varnothing$} `{\nomm/\accc.\masc}’) over the phonologically overt inflection (-\textit{a} `{\nomm.\fem}’). This contrasts with the null effect obtained previously in the auditory modality \citep{gor2017processing}; note that non-significant effects are actually misleading, as they do not allow to conclude anything. Strictly speaking, these two forms differ not only in phonological overtness, but also in ambiguity: the -\textit{a} wordform is unambiguous, while the -\textit{$\varnothing$} wordform also marks accusative in the discussed set of nouns. So this finding should be treated with caution.

\section{Conclusion}
The results of our two experiments disagree with previous findings in Finnish, Uyghur, and Serbian, suggesting that differences in oblique case processing exist. Moreover, these differences arise both in transparent systems of case marking (Russian plural) and opaque or highly syncretic systems of case marking (Russian singular). 

Data from the experiment with plural nouns suggests that case processing might be guided by \citeauthor{caha2008case}’s \citeyearpar{caha2008case} functional case sequence. Results for singular nouns imply that different types of ambiguity are present in Russian declension. When the ambiguity is not accidental, context plays a major role in the selection of the interpretation.

\section*{Abbreviations}

\begin{tabularx}{.45\textwidth}{ll}
\accc & accusative\\
\datt & dative \\
\genn & genitive\\
\fem & feminine\\
\ins & instrumental\\
\end{tabularx}
\begin{tabularx}{.45\textwidth}{ll}
\locc & locative\\
\masc & masculine\\
\nomm & nominative \\
\pl & plural\\
\sg & singular \\
\end{tabularx}

\section*{Acknowledgements}

The author would like to thank Elena Gorbunova, Oleg Volkov and Nikita Loginov for their assistance with the participant recruitment as well as Olga Fedorova,  Maria Falikman and two anonymous reviewers for their comments on the earlier version of this paper.

% % \appendix

\clearpage
{
\sloppy
\printbibliography[heading=subbibliography,notkeyword=this]
}
\section*{Appendix A: Experimental items}

Lemma frequency counts are given in brackets.\\

\noindent \textbf{Feminine nouns} (47.39)\\

\noindent \textit{anketa} `questionnaire' (14.4),
\textit{arfa} `harp' (2.6),
\textit{astra} `aster' (3.8),
\textit{aura} `aura' (5.1),
\textit{beseda} `conversation' (87.5),
\textit{bukva} `letter. character' (63.5),
\textit{data} `date' (49.5),
\textit{doza} `dose' (22.4),
\textit{dyuna} `dune' (2),
\textit{fleita} `flute' (5.8),
\textit{gazeta} `newspaper' (237.5),
\textit{gitara} `guitar' (22.2),
\textit{kareta} `carriage' (9.4),
\textit{karta} `map' (103),
\textit{kassa} `cashier's desk' (20.9),
\textit{klumba} `flower-bed' (8.7),
\textit{klyaksa} `blot' (4.5),
\textit{kofta} `jacket' (7.7),
\textit{lampa} `lamp' (34),
\textit{lapa} `paw' (39.7),
\textit{lenta} `ribbon' (35.9),
\textit{lira} `lyre' (8),
\textit{lyustra} `lustre' (9.9),
\textit{mera} `measure' (284.3),
\textit{minuta} `minute' (344.2),
\textit{moneta} `coin' (17.5),
\textit{norma} `norm' (111.3),
\textit{orbita} `orbite' (15),
\textit{pal'ma} `palm tree' (14.3),
\textit{pasta} `paste' (6.3),
\textit{pochva} `soil' (56.2),
\textit{poza} `pose' (29.8),
\textit{raketa} `rocket' (62.9),
\textit{rama} `frame' (21.2),
\textit{rana} `wound' (29.4),
\textit{rasa} `race' (5.9),
\textit{rifma} `rhyme' (8.5),
\textit{roza} `rose' (42.7),
\textit{shakhta} `pit' (20.7),
\textit{shina} `tire' (15.3),
\textit{shirma} `folding-screen' (5.3),
\textit{shkola} `school' (316),
\textit{shlyapa} `hat' (34.2),
\textit{shuba} `furcoat' (18.7),
\textit{shvabra} `mop' (3.4),
\textit{summa} `sum' (130.6),
\textit{trassa} `route' (32.5),
\textit{travma} `trauma' (19.6),
\textit{tsifra} `numeric' (62.2),
\textit{tsitata} `citation' (21.5),
\textit{tykva} `pumpkin' (5),
\textit{vaza} `vase' (14.3),
\textit{yakhta} `yacht' (9.5),
\textit{yurta} `yurt' (2.7)\\

\noindent\textbf{Masculine nouns} (47.37)\\

\noindent\textit{al'bom} `album' (23.7),
\textit{ananas} `pineapple' (3.6),
\textit{aromat} `aroma' (22.9),
\textit{aspekt} `aspect' (35.6),
\textit{atom} `atom' (20.5),
\textit{banan} `banana' (7.3),
\textit{baton} `loaf (of bread)' (5.3),
\textit{bufet} `buffet' (20),
\textit{buton} `bud' (4.6),
\textit{desert} `dessert' (4),
\textit{diplom} `diploma' (25.8),
\textit{divan} `sofa' (60.1),
\textit{dzhip} `jeep' (14.7),
\textit{fontan} `fountain' (18.4),
\textit{frukt} `fruit' (21.6),
\textit{gimn} `hymn' (14.8),
\textit{ideal} `ideal' (36),
\textit{kanat} `rope' (9),
\textit{kapriz} `caprice' (7.1),
\textit{kedr} `cedar' (6.1),
\textit{khalat} `bathrobe' (36.1),
\textit{klad} `treasure' (7.5),
\textit{komod} `dresser' (5.2),
\textit{kontur} `contour' (15.3),
\textit{kostyum} `costume. suit' (81.3),
\textit{kurort} `resort' (12.8),
\textit{metall} `metal' (57.5),
\textit{moment} `moment' (306.8),
\textit{nrav} `temper' (17.8),
\textit{ofis} `office' (34.1),
\textit{period} `period' (204.2),
\textit{plan} `plan' (235.3),
\textit{pled} `plaid' (4.8),
\textit{reis} `flight. voyage' (22),
\textit{remont} `reparation' (64.2),
\textit{ritm} `rhythm' (30.6),
\textit{romb} `rhombus' (1.8),
\textit{rulon} `roll' (4.3),
\textit{servis} `service' (14.6),
\textit{sezon} `season' (69.2),
\textit{shram} `scar' (10.7),
\textit{shtraf} `forfeit' (32.3),
\textit{simvol} `symbol' (46.4),
\textit{sous} `sauce' (10.8),
\textit{syuzhet} `storyline' (56.6),
\textit{teatr} `theater' (305.3),
\textit{tekst} `text' (146.2),
\textit{temp} `tempo' (49),
\textit{tovar} `item of goods' (115.5),
\textit{tsikl} `cycle' (43.6),
\textit{virus} `virus' (106.5),
\textit{vulkan} `volcano' (6),
\textit{yarus} `tier. layer' (6.5),
\textit{zhanr} `genre' (36)

\section*{Appendix B: Results of statistical analyses}
\begin{sidewaystable}
 \caption{Experiment 1 (singular nouns): final models for RTs and accuracy}
 \label{tab:LMERsg}
 \begin{tabularx}{\textwidth}{lXlSSXlSS} 
   \lsptoprule
    ~   & & \multicolumn{3}{c}{RTs} & & \multicolumn{3}{c}{Accuracy} \\
    
    \multicolumn{8}{l}{\textit{Random effects}}\\    \midrule
    Groups   & & \multicolumn{1}{c}{Name}         & \multicolumn{1}{c}{Variance} & \multicolumn{1}{c}{SD} & ~   & \multicolumn{1}{c}{Name}        & \multicolumn{1}{c}{Variance} & \multicolumn{1}{c}{SD}\\
    Subject  & & \multicolumn{1}{l}{(Intercept)}  & 0.001                      & 0.031                  &     & \multicolumn{1}{l}{(Intercept)} & 0.143                        & 0.378\\
    Lexeme   & & \multicolumn{1}{l}{(Intercept)}  & 0.027                      & 0.163                  &     & \multicolumn{1}{l}{(Intercept)} & 0.96                         & 0.98  \\
    Residual & & \multicolumn{1}{l}{}             & 0.032                      & 0.179                  &     & \\
    \end{tabularx}
    \begin{tabularx}{\textwidth}{l SS S[table-format=4.0] *{6}{S}}
 \midrule\multicolumn{10}{l}{\textit{Fixed effects}}\\
        ~ &\multicolumn{1}{c}{\textit{B}} & \multicolumn{1}{c}{SE}  & \multicolumn{1}{c}{df} & \multicolumn{1}{c}{\textit{t}-value} & \multicolumn{1}{c}{Pr(>|t|)} &    \multicolumn{1}{c}{\textit{B}} & \multicolumn{1}{c}{SE}  &    \multicolumn{1}{c}{ \textit{z-value}} & \multicolumn{1}{c}{Pr(>|z|)} \\\midrule
 (Intercept) & 6.58 & 0.026 & 64 & 257.46  & <.001 &
        4.71 & 0.414 & 11.4 & <.001\\
 Trial  & -0.012 & 0.003 & 4541 & -4.45 & <.001 &
       \multicolumn{4}{l}{}\\
 PC2   & 0.018 & 0.003 & 121  & 6.03 & <.001   &
        -0.364 & 0.062 &-5.86 & <.001\\
 PC3   & -0.022 & 0.006 & 103  & -3.68 & <.001 &
     0.509 & 0.104 &4.89 &<.001\\ 
 Case[Acc] & 0.067 & 0.013 & 4538 & 5.22 & <.001 &
        -1.282 & 0.442 &-2.9 &.004\\
 Case[Dat] & 0.1  & 0.013 & 4490 & 7.7  & <.001 &
     -1.906 & 0.417 &-4.57 &<.001\\
 Case[Gen] & 0.048 & 0.013 & 4565 & 3.72 & <.001 &
     -1.356 & 0.445 &-3.05 &.002\\
 Case[Ins] & 0.042 & 0.013 & 4472 & 3.32 & .001  &
     -0.952 & 0.454 &-2.1 &.036\\
 Case[Loc] & 0.089 & 0.013 & 4488 & 6.82 & <.001 &
     -2.07 & 0.413 &-5.02 &<.001\\
 Gender[m] & -0.023 & 0.014 & 816  & -1.63 & .104  &
     0.001 & 0.536 &0.003 &.998\\
 Case[Acc] : Gender[m] & -0.077 & 0.018 & 4510 & -4.32 & <.001 &
              0.655  & 0.643 &1.02 &.308\\
 Case[Dat] : Gender[m] & -0.0005 & 0.018 & 4521 & -0.027 & .979 &
        0.503 & 0.594 &0.847 &.397\\
 Case[Gen] : Gender[m] & 0.100  & 0.018 & 4514 & 5.56  & <.001 &
        -0.545 & 0.607 &-0.898 &.369\\
 Case[Ins] : Gender[m] & 0.056 & 0.018 & 4470 & 3.13  & .002 &
        0.539 & 0.667 &0.808 &.419\\ 
 Case[Loc] : Gender[m] & 0.082  & 0.019 & 4493 & 4.38  & <.001 &
       -0.915 & 0.571 &-1.6 &.109\\
 PC3 : Gender[m]   & 0.024  & 0.008 & 103  & 3.2   & .002 &
        -0.472 & 0.13 &-3.63 &<.001\\
   \lspbottomrule
 \end{tabularx}    
\end{sidewaystable}

\begin{sidewaystable}
	\caption{Experiment 2 (plural nouns): final models for RTs and accuracy}
 \label{tab:LMERpl}
 \begin{tabularx}{\textwidth}{lXlSSXlSS}
   \lsptoprule
    ~   & & \multicolumn{3}{c}{RTs} & & \multicolumn{3}{c}{Accuracy} \\    
\multicolumn{8}{l}{\textit{Random effects}}\\    \midrule
Groups   & & \multicolumn{1}{c}{Name}           &   \multicolumn{1}{c}{Variance} &   \multicolumn{1}{c}{SD} & & \multicolumn{1}{c}{Name}          &  \multicolumn{1}{c}{Variance} & \multicolumn{1}{c}{SD} \\
Subject  & & \multicolumn{1}{l}{(Intercept)}    &   0.002                        & 0.041                    & & \multicolumn{1}{l}{(Intercept)}   &   0.323                       & 0.569  \\
Lexeme   & & \multicolumn{1}{l}{(Intercept)}    &   0.021                        & 0.145                    & & \multicolumn{1}{l}{(Intercept)}   &   0.612                       & 0.782  \\
Residual & &                                    &   0.035                        & 0.188                    & & \\\midrule
   \end{tabularx}
   \begin{tabularx}{\textwidth}{l SS S[table-format=4.0] SS  X *{4}{S}}
 \multicolumn{10}{l}{\textit{Fixed effects}}\\
    ~   &\multicolumn{1}{c}{\textit{B}} & \multicolumn{1}{c}{SE}  & \multicolumn{1}{c}{df} &    \multicolumn{1}{c}{\textit{t}-value} & \multicolumn{1}{c}{Pr(>|t|)} &  &  \multicolumn{1}{c}{\textit{B}} & \multicolumn{1}{c}{SE}  &    \multicolumn{1}{c}{ \textit{z-value}} & \multicolumn{1}{c}{Pr(>|z|)} \\\midrule
 (Intercept)           & 6.68   & 0.024 & 76   & 279.9  &  <.001 & &    3.265 & 0.271 & 12.03  & <.001\\
 PC2                   & 0.019  & 0.004 & 126  & 4.98   &  <.001 & &  -0.285  & 0.068 & -4.21  & <.001\\
 PC3                   & -0.022 & 0.007 & 111  & -2.91  &  .004  & &  0.442   & 0.12  & 3.7    & <.001\\
 Case[Acc]             & 0.034  & 0.013 & 4342 & 2.52   &  .012  & &   0.205  & 0.332 & 0.618  & .537\\
 Case[Dat]             & 0.094  & 0.014 & 3769 & 6.49   &  <.001 & & -0.109   & 0.3   & -0.364 & .716\\
 Case[Gen]             & 0.087  & 0.015 & 3645 & 5.76   &  <.001 & & -1.505   & 0.268 & -5.61  & <.001\\
 Case[Ins]             & 0.103  & 0.014 & 4419 & 7.42   &  <.001 & &   -0.385 & 0.294 & -1.31  & .191\\
 Case[Loc]             & 0.067  & 0.014 & 4311 & 4.77   &  <.001 & & -0.345   & 0.292 & -1.18  & .237\\
 Gender[m]             & 0.017  & 0.016 & 589  & 1.06   &  .288  & &  0.179   & 0.345 & 0.52   & .603\\
 Case[Acc] : Gender[m] & -0.046 & 0.019 & 4342 & -2.41  &  .016  & & -0.481   & 0.455 & -1.06  & .29\\
 Case[Dat] : Gender[m] & -0.011 & 0.019 & 4437 & -0.58  &  .565  & & -0.357   & 0.42  & -0.85  & .395\\
 Case[Gen] : Gender[m] & -0.026 & 0.020 & 4269 & -1.29  &  .197  & & 1.534    & 0.428 & 3.58   & <.001\\
 Case[Ins] : Gender[m] & 0.015  & 0.019 & 4391 & 0.76   &  .448  & & -0.183   & 0.42  & -0.44  & .662\\
 Case[Loc] : Gender[m] & -0.010 & 0.019 & 4426 & -0.53  &  .594  & & -0.143   & 0.415 & -0.34  & .731\\
 PC3 : Gender[m]       & 0.021  & 0.009 & 107  & 2.28   &  .025  & & -0.423   & 0.155 & -2.73  & .006\\
   \lspbottomrule
 \end{tabularx}    
\end{sidewaystable}

\begin{table}
\centering
\caption{Experiment 1 (singular nouns): paired contrasts for feminine and masculine nouns in different cases (analyses with $p~\leq~.05$ are given in bold)}
\label{tab:caseSgComparisons}
 \resizebox{\textwidth}{!}{\begin{tabular}{*{4}{l} SS[table-format=4.0]SS[table-format=>1.3]SS[table-format=1.2]S[table-format=>1.3]} 
  \lsptoprule
 \multicolumn{4}{l}{} & \multicolumn{4}{c}{RTs} & \multicolumn{3}{c}{accuracy}\\
 \cmidrule(lr){5-8}\cmidrule(lr){9-11}
  \multicolumn{4}{l}{} & \multicolumn{1}{c}{Δ}  & \multicolumn{1}{c}{df} & \multicolumn{1}{c}{\textit{t}} & \multicolumn{1}{c}{\textit{p}} & \multicolumn{1}{c}{Δ} & \multicolumn{1}{c}{\textit{z}} & \multicolumn{1}{c}{\textit{p}} \\ 
  \midrule
 f & Nom & vs. & Gen &\bfseries -0.048 &\bfseries 4565 & \bfseries -3.72 & \bfseries <.001 & \bfseries 1.356 & \bfseries 3.05 & \bfseries .006 \\
 ~ & ~ & ~ & Acc &\bfseries -0.067 &\bfseries 4538 & \bfseries -5.22 & \bfseries <.001 & \bfseries 1.282 & \bfseries 2.9 & \bfseries .008 \\
~ & ~ & ~ & Ins &\bfseries -0.042 &\bfseries 4472 & \bfseries -3.32 &\bfseries .002 &  0.952 &  2.1 &  .055 \\
 ~ & ~ & ~ & -\textit{e} &\bfseries -0.094 &\bfseries 4495 & \bfseries -8.44 & \bfseries <.001 &\bfseries 1.99 &\bfseries 5.02 & \bfseries <.001 \\
~ & Gen & vs. & Acc & -0.019 & 4497 &  -1.48 & .174 &  -0.074 &  -0.22 &  .836 \\
~ & ~ & ~ & Ins & 0.006 & 4565 &  0.47 & .666 &  -0.404 &  -1.16 &  .321 \\
~ & ~ & ~ & -\textit{e} &\bfseries -0.046 &\bfseries 4561 & \bfseries -4.06 & \bfseries <.001 &\bfseries 0.633 &\bfseries 2.36 &\bfseries .032 \\
~ &  Acc & vs. & Ins & 0.025 & 4516 &  1.95 & .07 &  -0.33 &  -0.96 &  .421 \\
~ & ~ & ~ & -\textit{e} &\bfseries -0.027 &\bfseries 4503 & \bfseries -2.38 &\bfseries .027 & \bfseries 0.708 &\bfseries 2.68 &\bfseries .015 \\
~ &  Ins & vs. & -\textit{e} &\bfseries -0.052 &\bfseries 4476 & \bfseries -4.65 & \bfseries <.001 & \bfseries 1.037 & \bfseries 3.65 & \bfseries .001 \\
    \midrule
    m & \textit{-$\varnothing$} & vs. & Gen &\bfseries -0.154 &\bfseries 4505 & \bfseries -13.8 & \bfseries <.001 & \bfseries 1.587 & \bfseries 5.39 &  \bfseries <.001 \\
~ & ~ & ~ & Dat &\bfseries -0.104 &\bfseries 4524 & \bfseries -9.42 & \bfseries <.001 & \bfseries 1.089 & \bfseries 3.54 &  \bfseries <.001\\
~ & ~ & ~ & Ins &\bfseries -0.103 &\bfseries 4472 & \bfseries -9.53 & \bfseries <.001 &  0.1 &  0.25 &  .836 \\
~ & ~ & ~ & Loc &\bfseries -0.176 &\bfseries 4491 & \bfseries -14.69 & \bfseries <.001 & \bfseries 2.675 & \bfseries 9.98 &  \bfseries <.001 \\
~ & Gen & vs. & Dat &\bfseries 0.049 &\bfseries 4557 & \bfseries 3.77 & \bfseries <.001 &  -0.499 &  -1.85 &  .092 \\
~ & ~ & ~ & Ins &\bfseries 0.051 &\bfseries 4494 & \bfseries 3.98 & \bfseries <.001 & \bfseries -1.488 & \bfseries -4.1 &  \bfseries <.001 \\
~ & ~ & ~ & Loc & -0.022 & 4514 &  -1.6 & .141 & \bfseries 1.088 & \bfseries 4.95 &  \bfseries <.001 \\
~ & Dat & vs. & Ins & 0.002 & 4506 &  0.13 & .899 & \bfseries -0.989 & \bfseries -2.64 & \bfseries .015 \\
~ &  & ~ & Loc &\bfseries -0.071 &\bfseries 4498 & \bfseries -5.2 & \bfseries <.001 & \bfseries 1.586 & \bfseries 6.65 &  \bfseries <.001 \\
~ &  Ins & vs. & Loc &\bfseries -0.073 &\bfseries 4483 & \bfseries -5.41 & \bfseries <.001 & \bfseries 2.575 & \bfseries 7.54 &  \bfseries <.001 \\
    \midrule
    f & \multicolumn{2}{c}{vs.} &  m & -0.005 & 100 & -0.59 & .608 & -0.035 & -0.21 & .836\\
    \midrule
 \multicolumn{2}{l}{Ins} & \multicolumn{2}{l}{f vs. m}  &\bfseries -0.034 &\bfseries 830 &\bfseries -2.4 &\bfseries .027 & -0.534 & -1.29 & .27 \\
 \multicolumn{4}{l}{-\textit{u$_f$} vs. -\textit{u$_m$}} & -0.01 & 836 & -0.7 & .548 & 0.125 & 0.39 & .786 \\
 \multicolumn{4}{l}{-\textit{e$_f$} vs. -\textit{e$_m$}} & \bfseries -0.054 & \bfseries 739 &\bfseries -3.99 & \bfseries <.001 &\bfseries 1.004 &\bfseries 5.14 & \bfseries <.001 \\
  \multicolumn{4}{l}{-\textit{e$_f$} vs. -\textit{u$_m$}} & 0.017 & 589 & 1.31 & .226 & \bfseries -0.583 &\bfseries -2.34 &\bfseries .032 \\
    \multicolumn{4}{l}{-\textit{a$_f$} vs. -\textit{$\varnothing_m$}} &\bfseries 0.027 &\bfseries 551 &\bfseries 2.16 &\bfseries .045 & 0.318 & 0.71 & .566\\
  \lspbottomrule 
 \end{tabular}}
\end{table}

\begin{table}
\centering
\caption{Experiment 2 (plural nouns): paired contrasts for feminine and masculine nouns in different cases (analyses with $p~\leq~.05$ are given in bold)}
\label{tab:casePlComparisons}
 \resizebox{\textwidth}{!}{\begin{tabular}{*{3}lSS[table-format=4.0]*{5}S} 
  \lsptoprule
   \multicolumn{3}{l}{} & \multicolumn{4}{c}{RTs} & \multicolumn{3}{c}{accuracy}\\
   \cmidrule(lr){4-7}\cmidrule(lr){8-10}
  \multicolumn{3}{l}{} & \multicolumn{1}{c}{Δ}  & \multicolumn{1}{c}{df} &   \multicolumn{1}{c}{\textit{t}} & \multicolumn{1}{c}{\textit{p}} & \multicolumn{1}{c}{Δ} & \multicolumn{1}{c}{\textit{z}} & \multicolumn{1}{c}{\textit{p}} \\
  \midrule
    -\textit{y} & vs. & Gen & \bfseries -0.068 & \bfseries 4096 &  \bfseries -7.7 & \bfseries <.001 &  \bfseries 0.72 &  \bfseries 4 &  \bfseries <.001 \\
 ~& ~ & Dat & \bfseries -0.082 & \bfseries 3367 &  \bfseries -9.26 & \bfseries <.001 &   0.27  &   1.51  &   .245  \\
 ~& ~ & Ins & \bfseries -0.105 & \bfseries 4400 &  \bfseries -12.39 & \bfseries <.001 &  \bfseries 0.459 &  \bfseries 2.62 &  \bfseries .038 \\
 ~& ~ & Loc & \bfseries -0.056 & \bfseries 4134 &  \bfseries -6.48 & \bfseries <.001 &  0.399 &   2.29  &   .058  \\
    Gen & vs. & Dat &  -0.014  &  4409  &   -1.41  &  .229  &   -0.45  &   -2.38  &   .057 \\
 ~& ~ & Ins & \bfseries -0.036 & \bfseries 4439 &  \bfseries -3.58 & \bfseries .001 &   -0.261  &   -1.36  &   .251  \\
 ~& ~ & Loc &  0.012  &  4380  &   1.25  &  .276  &   -0.321  &   -1.72  &   .186 \\
    Dat & vs. & Ins & \bfseries -0.022 & \bfseries 4290 &  \bfseries -2.2 & \bfseries .046 &   0.189  &   0.98  &   .424  \\
 ~& ~ & Loc & \bfseries 0.026 & \bfseries 4390 &  \bfseries 2.71 & \bfseries .013 &   0.129  &   0.7  &   .575 \\
    Ins & vs. & Loc & \bfseries 0.049 & \bfseries 4440 &  \bfseries 4.9 & \bfseries <.001 &   -0.06  &   -0.32  &   .785  \\
    \midrule
    f & vs. &  m & -0.005 & 102 & -0.48 & .686 & -0.232 & -1.4 & .251\\
    \midrule
    -\textit{y} & \multicolumn2{c}{f vs. m} & -0.005 &    255 & -0.4  & .692 & 0.07   &   0.27   & .785\\
    Gen &  \multicolumn2{c}{f vs. m}   &   0.008   &   644 &   0.49    & .686  &  \bfseries -1.704 &   \bfseries -5.68  &  \bfseries <.001\\
  \lspbottomrule 
 \end{tabular}}
\end{table}




\end{document}
